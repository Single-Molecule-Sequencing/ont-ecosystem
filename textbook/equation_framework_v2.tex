%%%%%%%%%%%%%%%%%%%%%%%%%%%%%%%%%%%%%%%%%%%%%%%%%%%%%%%%%%%%%%%%%%%%%%%%
%% SIMPLIFIED EQUATION AND VARIABLE FRAMEWORK V2
%% More robust, less error-prone version using mdframed instead of tcolorbox
%% (avoids tcolorbox nesting conflicts and breakable-environment issues in long equation boxes)
%%%%%%%%%%%%%%%%%%%%%%%%%%%%%%%%%%%%%%%%%%%%%%%%%%%%%%%%%%%%%%%%%%%%%%%%

% Required packages
\usepackage{amsmath,amssymb}
\usepackage{chngcntr}
\usepackage[framemethod=tikz]{mdframed}
\usepackage{booktabs}
\usepackage{xcolor}

% Equation numbering by chapter
\counterwithin{equation}{chapter}

%%%%%%%%%%%%%%%%%%%%%%%%%%%%%%%%%%%%%%%%%%%%%%%%%%%%%%%%%%%%%%%%%%%%%%%%
%% EQUATION BOX - Simplified version using mdframed
%%%%%%%%%%%%%%%%%%%%%%%%%%%%%%%%%%%%%%%%%%%%%%%%%%%%%%%%%%%%%%%%%%%%%%%%

% Define colors
\definecolor{eqboxframe}{RGB}{0,0,0}
\definecolor{eqboxback}{RGB}{255,255,255}

% Simple equation box using mdframed
% Usage: \begin{eqbox}{Title} \begin{equation} ... \end{equation} \end{eqbox}
\mdfdefinestyle{equationbox}{
    linecolor=eqboxframe,
    backgroundcolor=eqboxback,
    linewidth=1pt,
    innertopmargin=10pt,
    innerbottommargin=10pt,
    innerleftmargin=10pt,
    innerrightmargin=10pt,
    skipabove=12pt,
    skipbelow=12pt,
    frametitlerule=true,
    frametitlebackgroundcolor=eqboxback,
    frametitlerulewidth=1pt,
}

\newenvironment{eqbox}[1]{%
    \begin{mdframed}[style=equationbox, frametitle={\textbf{#1}}]
}{%
    \end{mdframed}
}

%%%%%%%%%%%%%%%%%%%%%%%%%%%%%%%%%%%%%%%%%%%%%%%%%%%%%%%%%%%%%%%%%%%%%%%%
%% ALTERNATIVE: Even simpler box using fbox (fallback option)
%%%%%%%%%%%%%%%%%%%%%%%%%%%%%%%%%%%%%%%%%%%%%%%%%%%%%%%%%%%%%%%%%%%%%%%%

% Uncomment this section if mdframed causes issues
% \newenvironment{eqbox}[1]{%
%     \par\vspace{12pt}
%     \noindent\fbox{\begin{minipage}{\dimexpr\textwidth-2\fboxsep-2\fboxrule}
%     \textbf{#1}\\[6pt]
% }{%
%     \end{minipage}}\par\vspace{12pt}
% }

%%%%%%%%%%%%%%%%%%%%%%%%%%%%%%%%%%%%%%%%%%%%%%%%%%%%%%%%%%%%%%%%%%%%%%%%
%% VARIABLE TABLE
%%%%%%%%%%%%%%%%%%%%%%%%%%%%%%%%%%%%%%%%%%%%%%%%%%%%%%%%%%%%%%%%%%%%%%%%

\newenvironment{vartable}[1][]{%
  \begin{table}[ht]
  \centering
  \caption{Variable summary for Chapter~\thechapter}
  \begin{tabular}{llll}
  \toprule
  Symbol & Physical description & Units & Measurement / determination \\
  \midrule
}{%
  \bottomrule
  \end{tabular}
  \end{table}
}

\newcommand{\varrow}[4]{%
  #1 & #2 & #3 & #4 \\
}

%%%%%%%%%%%%%%%%%%%%%%%%%%%%%%%%%%%%%%%%%%%%%%%%%%%%%%%%%%%%%%%%%%%%%%%%
%% VARIABLE BOX - Simplified version
%%%%%%%%%%%%%%%%%%%%%%%%%%%%%%%%%%%%%%%%%%%%%%%%%%%%%%%%%%%%%%%%%%%%%%%%

\mdfdefinestyle{variablebox}{
    linecolor=blue!60!black,
    backgroundcolor=blue!3!white,
    linewidth=0.8pt,
    innertopmargin=10pt,
    innerbottommargin=10pt,
    innerleftmargin=10pt,
    innerrightmargin=10pt,
    skipabove=10pt,
    skipbelow=10pt,
    frametitlerule=true,
    frametitlebackgroundcolor=blue!10!white,
    frametitlerulewidth=0.8pt,
}

\newenvironment{varbox}[1]{%
    \begin{mdframed}[style=variablebox, frametitle={\textbf{Variable: $#1$}}]
}{%
    \end{mdframed}
}

%%%%%%%%%%%%%%%%%%%%%%%%%%%%%%%%%%%%%%%%%%%%%%%%%%%%%%%%%%%%%%%%%%%%%%%%
%% USAGE EXAMPLES
%%%%%%%%%%%%%%%%%%%%%%%%%%%%%%%%%%%%%%%%%%%%%%%%%%%%%%%%%%%%%%%%%%%%%%%%

% Example 1: Simple equation in box
% \begin{eqbox}{Bayes' rule}
% \begin{equation}
% P(A|B) = \frac{P(B|A)P(A)}{P(B)}
% \label{eq:bayes}
% \end{equation}
% \end{eqbox}

% Example 2: Multi-line equation
% \begin{eqbox}{System of equations}
% \begin{align}
% x + y &= 5 \\
% 2x - y &= 1
% \end{align}
% \end{eqbox}

% Example 3: Variable table
% \begin{vartable}
% \varrow{$x$}{Position}{meters}{Measured by GPS}
% \varrow{$v$}{Velocity}{m/s}{Computed from position derivative}
% \end{vartable}

% Example 4: Variable box
% \begin{varbox}{$\lambda$}
% \textbf{Physical description.}
% Mixture fraction of first haplotype.
%
% \textbf{Units.}
% Dimensionless, range [0,1].
%
% \textbf{Measurement.}
% Estimated via EM algorithm.
%
% \textbf{Example.}
% For balanced diploid: $\lambda = 0.5$.
% \end{varbox}
