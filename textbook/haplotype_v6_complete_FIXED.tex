\documentclass[11pt,letterpaper,twoside,openright]{book}

%%%%%%%%%%%%%%%%%%%%%%%%%%%%%%%%%%%%%%%%%%%%%%%%%%%%%%%%%%%%%%%%%%%%%%%%
%% PREAMBLE - Professional Academic Formatting
%%%%%%%%%%%%%%%%%%%%%%%%%%%%%%%%%%%%%%%%%%%%%%%%%%%%%%%%%%%%%%%%%%%%%%%%

% Page geometry and margins
\usepackage[
    letterpaper,
    top=1in,
    bottom=1in,
    inner=1.25in,
    outer=1in,
    headheight=14.5pt,
    headsep=0.25in,
    footskip=0.3in
]{geometry}

% Essential packages
\usepackage[utf8]{inputenc}
\usepackage[T1]{fontenc}

% Note: abstract package removed - not compatible with book class and not used
% \usepackage{abstract}

% Professional fonts
\usepackage{lmodern}           % Latin Modern (improved Computer Modern)
\usepackage{microtype}         % Improved typography and spacing

% Math packages
\usepackage{amsmath,amssymb,amsthm}
\usepackage{array} % for \arraybackslash in tabular
\usepackage{mathtools} 
% Enhanced math features
\usepackage{bbm}


% Graphics and tables
\usepackage{graphicx}
\usepackage{array}          % Required for advanced column formatting like >{\raggedright}
\usepackage{booktabs}
\usepackage{longtable}
\usepackage{multirow}
\usepackage{float}

% Colors
\usepackage{xcolor}
% \usepackage{colortbl}  % For \rowcolor in tables - TEMPORARILY DISABLED due to compilation issues

% Advanced boxes (for eqboxV5 template)
\usepackage{tcolorbox}
\tcbuselibrary{breakable,skins}

% Lists
\usepackage{enumitem}

% Algorithms
\usepackage{algorithm}
\usepackage{algorithmic}

% Text symbols
\usepackage{textcomp}

% Unicode support
\usepackage{newunicodechar}
\newunicodechar{⚠}{\ensuremath{\triangle}}

% Headers and footers
\usepackage{fancyhdr}

% Better captions
\usepackage[font=small,labelfont=bf,labelsep=period]{caption}

% Section and chapter styling
\usepackage{titlesec}

% Better spacing
\usepackage{setspace}
\setstretch{1.1}

% Page break control
\usepackage{needspace}

% Margin adjustment (for wide content in eqboxV5)
\usepackage{changepage}

% Advanced command definitions (for master reference boxes)
\usepackage{xparse}

% Hyperref must be loaded last
\usepackage{hyperref}

% SMS Lab figure/table management and math registry integration
\usepackage{sms}
\setmanuscriptversion{v1.0.1_20251120}  % Updated to match generated figures

%%%%%%%%%%%%%%%%%%%%%%%%%%%%%%%%%%%%%%%%%%%%%%%%%%%%%%%%%%%%%%%%%%%%%%%%
%% CUSTOM COMMANDS
%%%%%%%%%%%%%%%%%%%%%%%%%%%%%%%%%%%%%%%%%%%%%%%%%%%%%%%%%%%%%%%%%%%%%%%%

% Core equation references
\newcommand{\CEref}[1]{\hyperref[ce:#1]{CE.#1}}
\newcommand{\CEanchor}[1]{\label{ce:#1}}

%%%%%%%%%%%%%%%%%%%%%%%%%%%%%%%%%%%%%%%%%%%%%%%%%%%%%%%%%%%%%%%%%%%%%%%%
%% EQUATION AND VARIABLE FRAMEWORK (November 2025)
%%%%%%%%%%%%%%%%%%%%%%%%%%%%%%%%%%%%%%%%%%%%%%%%%%%%%%%%%%%%%%%%%%%%%%%%

% Equation numbering by chapter: (1.1), (1.2), ..., (2.1), ...
% This is already enabled by default in amsmath with the book class,
% but we explicitly ensure it with chngcntr
\usepackage{chngcntr}
\counterwithin{equation}{chapter}

% Equation box environment
% Usage: \begin{eqbox}{Short title} \begin{equation} ... \end{equation} \end{eqbox}
% Wraps equations/align/etc. in a titled box for emphasis
% Note: You must include \begin{equation} or \begin{align} explicitly inside eqbox
% COMMENTED OUT: This definition is superseded by the eqboxV5-based definition at line 625
% \newtcolorbox{eqbox}[2][]{%
%   enhanced,
%   colback=white,
%   colframe=black,
%   sharp corners,
%   title={#2},
%   fonttitle=\bfseries,
%   boxrule=1pt,
%   left=10pt,
%   right=10pt,
%   top=10pt,
%   bottom=10pt,
%   before skip=12pt,
%   after skip=12pt,
%   ams nodisplayskip,
%   #1
% }

% Variable summary table environment
% Usage:
% \begin{vartable}
%   \varrow{$\mathcal{H}$}{Haplotype space}{dimensionless set}{Defined by allele catalogue}
%   ...
% \end{vartable}
\newenvironment{vartable}[1][]{%
  \begin{table}[ht]
  \centering
  \caption{Variable summary for Chapter~\thechapter}
  \begin{tabular}{llll}
  \toprule
  Symbol & Physical description & Units & Measurement / determination \\
  \midrule
}{%
  \bottomrule
  \end{tabular}
  \end{table}
}

% Variable row in summary table
\newcommand{\varrow}[4]{%
  #1 & #2 & #3 & #4 \\
}

% Variable box environment at chapter end
% Usage: \begin{varbox}{$\mathcal{H}$} ... \end{varbox}
% Creates a detailed box for one variable with description, units, measurement, and example
% COMMENTED OUT: This definition is superseded by the definition at line 604
% \newtcolorbox{varbox}[2][]{%
%   enhanced,
%   breakable,
%   colback=blue!3!white,
%   colframe=blue!60!black,
%   sharp corners,
%   title={Variable: #2},
%   fonttitle=\bfseries,
%   boxrule=0.8pt,
%   left=10pt,
%   right=10pt,
%   top=10pt,
%   bottom=10pt,
%   before skip=10pt,
%   after skip=10pt,
%   #1
% }

%%%%%%%%%%%%%%%%%%%%%%%%%%%%%%%%%%%%%%%%%%%%%%%%%%%%%%%%%%%%%%%%%%%%%%%%

% Consistent forthcoming chapter notice
\newcommand{\ChapterForthcomingNotice}[3]{%
    \vspace*{1.5cm}%
    \noindent\fcolorbox{lightgray}{white}{%
        \begin{minipage}{\dimexpr\textwidth-2\fboxsep-2\fboxrule\relax}%
            \centering\Large\textbf{Chapter Forthcoming}\par%
            \vspace{12pt}%
            \normalsize%
            \begin{minipage}{0.92\linewidth}%
                \centering #1%
            \end{minipage}\par%
            \vspace{10pt}%
            {\raggedright\textbf{Planned Content:}%
            \begin{itemize}[leftmargin=1.9cm]%
                #2%
            \end{itemize}}%
            \vspace{6pt}%
            \textit{Estimated pages: #3}%
        \end{minipage}%
    }%
    \vspace{1cm}%
}

% Math operators
% Note: \Prob is now defined in sms.sty package
% \newcommand{\Prob}{\mathbb{P}}  % Defined in sms.sty
\newcommand{\E}{\mathbb{E}}       % Expectation operator (not in sms.sty which uses \Expect)
\DeclareMathOperator*{\argmax}{arg\,max}
\DeclareMathOperator*{\argmin}{arg\,min}

\newtheorem{theorem}{Theorem}[chapter]
\newtheorem{proposition}[theorem]{Proposition}
\newtheorem{lemma}[theorem]{Lemma}
\newtheorem{corollary}[theorem]{Corollary}
\theoremstyle{definition}
\newtheorem{definition}[theorem]{Definition}
\newtheorem{example}[theorem]{Example}
\newtheorem{remark}[theorem]{Remark}
\newtheorem{recommendation}[theorem]{Recommendation}

% Important callout box environment
\newtcolorbox{important}[1][]{%
    colback=yellow!10!white,
    colframe=orange!75!black,
    fonttitle=\bfseries,
    title=#1,
    breakable,
    enhanced,
    sharp corners,
    boxrule=0.5pt,
    left=6pt,
    right=6pt,
    top=6pt,
    bottom=6pt
}

% Learning objectives box (chapter openings)
\newtcolorbox{learningobjectives}{%
    colback=blue!5!white,
    colframe=primarydark,
    fonttitle=\bfseries\sffamily,
    title=Learning Objectives,
    breakable,
    enhanced,
    sharp corners,
    boxrule=1pt,
    left=10pt,
    right=10pt,
    top=10pt,
    bottom=10pt,
    before skip=12pt,
    after skip=12pt
}

% Key takeaways box (chapter endings)
\newtcolorbox{keytakeaways}{%
    colback=green!5!white,
    colframe=green!60!black,
    fonttitle=\bfseries\sffamily\Large,
    title=Key Takeaways,
    breakable,
    enhanced,
    boxrule=1.5pt,
    left=12pt,
    right=12pt,
    top=12pt,
    bottom=12pt,
    before skip=20pt,
    after skip=20pt,
    sharp corners
}

% Practical note box
\newtcolorbox{practicalnote}[1][Practical Note]{%
    colback=cyan!5!white,
    colframe=cyan!75!black,
    fonttitle=\bfseries\sffamily,
    title=#1,
    breakable,
    enhanced,
    sharp corners,
    boxrule=0.8pt,
    left=8pt,
    right=8pt,
    top=8pt,
    bottom=8pt,
    before skip=10pt,
    after skip=10pt
}

% Warning box
\newtcolorbox{warningbox}[1][Warning]{%
    colback=red!5!white,
    colframe=red!75!black,
    fonttitle=\bfseries\sffamily,
    title=\textcolor{red!75!black}{⚠ #1},
    breakable,
    enhanced,
    sharp corners,
    boxrule=1pt,
    left=10pt,
    right=10pt,
    top=10pt,
    bottom=10pt,
    before skip=12pt,
    after skip=12pt
}

% Concept highlight box
\newtcolorbox{conceptbox}[1]{%
    colback=purple!5!white,
    colframe=purple!75!black,
    fonttitle=\bfseries\sffamily,
    title=Key Concept: #1,
    breakable,
    enhanced,
    sharp corners,
    boxrule=0.8pt,
    left=10pt,
    right=10pt,
    top=10pt,
    bottom=10pt,
    before skip=10pt,
    after skip=10pt
}

% Worked example box
\newtcolorbox{workedexample}[1][]{
  enhanced,
  breakable,
  colback=blue!5,
  colframe=blue!75!black,
  coltitle=white,
  fonttitle=\bfseries\large,
  title={#1},
  attach boxed title to top left={
    yshift=-2mm,
    xshift=5mm
  },
  boxed title style={
    colback=blue!75!black,
    sharp corners
  },
  top=10mm,
  bottom=5mm,
  left=5mm,
  right=5mm,
  arc=2mm,
  boxrule=1pt,
  shadow={2mm}{-1mm}{0mm}{black!30}
}

%%%%%%%%%%%%%%%%%%%%%%%%%%%%%%%%%%%%%%%%%%%%%%%%%%%%%%%%%%%%%%%%%%%%%%%%
%% HEADERS AND FOOTERS
%%%%%%%%%%%%%%%%%%%%%%%%%%%%%%%%%%%%%%%%%%%%%%%%%%%%%%%%%%%%%%%%%%%%%%%%

\pagestyle{fancy}
\fancyhf{} % Clear all headers and footers

% Headers
\fancyhead[L]{\small\slshape\nouppercase{\leftmark}}  % Chapter on left
\fancyhead[R]{\small\slshape\nouppercase{\rightmark}} % Section on right

% Page numbers - ALWAYS bottom right corner
\fancyfoot[R]{\small\thepage}

% Plain style for chapter pages - same page number position
\fancypagestyle{plain}{%
    \fancyhf{}
    \fancyfoot[R]{\small\thepage}
    \renewcommand{\headrulewidth}{0pt}
}

% Header rule
\renewcommand{\headrulewidth}{0.4pt}
\renewcommand{\footrulewidth}{0pt}

%%%%%%%%%%%%%%%%%%%%%%%%%%%%%%%%%%%%%%%%%%%%%%%%%%%%%%%%%%%%%%%%%%%%%%%%
%% CHAPTER AND SECTION STYLING
%%%%%%%%%%%%%%%%%%%%%%%%%%%%%%%%%%%%%%%%%%%%%%%%%%%%%%%%%%%%%%%%%%%%%%%%

% Chapter formatting
\titleformat{\chapter}[display]
    {\normalfont\huge\bfseries\sffamily}
    {\chaptertitlename\ \thechapter}
    {20pt}
    {\Huge}
\titlespacing*{\chapter}{0pt}{-20pt}{30pt}

% Section formatting
\titleformat{\section}
    {\normalfont\Large\bfseries\sffamily}
    {\thesection}
    {1em}
    {}
\titlespacing*{\section}{0pt}{3.5ex plus 1ex minus .2ex}{2.3ex plus .2ex}

% Subsection formatting
\titleformat{\subsection}
    {\normalfont\large\bfseries\sffamily}
    {\thesubsection}
    {1em}
    {}
\titlespacing*{\subsection}{0pt}{3.25ex plus 1ex minus .2ex}{1.5ex plus .2ex}

% Subsubsection formatting
\titleformat{\subsubsection}
    {\normalfont\normalsize\bfseries\sffamily}
    {\thesubsubsection}
    {1em}
    {}
\titlespacing*{\subsubsection}{0pt}{3.25ex plus 1ex minus .2ex}{1.5ex plus .2ex}

%%%%%%%%%%%%%%%%%%%%%%%%%%%%%%%%%%%%%%%%%%%%%%%%%%%%%%%%%%%%%%%%%%%%%%%%
%% COLOR DEFINITIONS
%%%%%%%%%%%%%%%%%%%%%%%%%%%%%%%%%%%%%%%%%%%%%%%%%%%%%%%%%%%%%%%%%%%%%%%%

% Professional dark tone palette - sophisticated charcoal and slate
\definecolor{primarydark}{RGB}{45,45,48}       % Deep charcoal (primary dark tone)
\definecolor{secondarydark}{RGB}{65,65,70}     % Slightly lighter charcoal for accents
\definecolor{slateaccent}{RGB}{85,90,95}       % Warm slate gray for emphasis
\definecolor{darkgray}{RGB}{89,89,89}          % Dark gray for text
\definecolor{mediumgray}{RGB}{127,127,127}     % Medium gray
\definecolor{lightgray}{RGB}{217,217,217}      % Light gray for borders

% Equation box colors - professional dark theme
\definecolor{eqboxbg}{RGB}{250,250,251}        % Warm off-white background
\definecolor{eqboxborder}{RGB}{45,45,48}       % Deep charcoal border
\definecolor{eqboxaccent}{RGB}{180,180,185}    % Soft gray accent line

% Legacy blue names remapped to dark tones for compatibility
\definecolor{primaryblue}{RGB}{45,45,48}       % Remapped to primarydark
\definecolor{secondaryblue}{RGB}{65,65,70}     % Remapped to secondarydark
\definecolor{lightblue}{RGB}{230,230,232}      % Remapped to light gray tone

% Configure hyperlink colors using named palette entries
\hypersetup{
    colorlinks=true,
    linkcolor=primarydark,
    citecolor=secondarydark,
    urlcolor=primarydark,
    bookmarksnumbered=true,
    bookmarksopen=true,
    pdftitle={Haplotype Framework v6.0 - Complete Edition},
    pdfauthor={Framework Development Team},
    pdfsubject={Single-Molecule Sequencing Haplotype Classification},
    pdfkeywords={haplotype, sequencing, Bayesian inference, genomics}
}

%%%%%%%%%%%%%%%%%%%%%%%%%%%%%%%%%%%%%%%%%%%%%%%%%%%%%%%%%%%%%%%%%%%%%%%%
%% UNICODE CHARACTER MAPPINGS
%%%%%%%%%%%%%%%%%%%%%%%%%%%%%%%%%%%%%%%%%%%%%%%%%%%%%%%%%%%%%%%%%%%%%%%%

\newunicodechar{≥}{\ensuremath{\geq}}
\newunicodechar{≤}{\ensuremath{\leq}}
\newunicodechar{µ}{\textmu}
\newunicodechar{°}{\textdegree}

%%%%%%%%%%%%%%%%%%%%%%%%%%%%%%%%%%%%%%%%%%%%%%%%%%%%%%%%%%%%%%%%%%%%%%%%
%% CUSTOM ENVIRONMENTS
%%%%%%%%%%%%%%%%%%%%%%%%%%%%%%%%%%%%%%%%%%%%%%%%%%%%%%%%%%%%%%%%%%%%%%%%

% Helper for optional headings
\newcommand{\frameworkHeading}[2]{%
    \textbf{#1\if\relax\detokenize{#2}\relax\else\ (#2)\fi}%
}

% Protocol environment
\newenvironment{protocol}[1][]{%
    \par\medskip
    \noindent\textcolor{primaryblue}{\rule{\linewidth}{1pt}}
    \par\smallskip
    \noindent{\sffamily\bfseries\color{primaryblue}\frameworkHeading{Protocol}{#1}}
    \par\smallskip
}{%
    \par\smallskip
    \noindent\textcolor{lightgray}{\rule{\linewidth}{0.5pt}}
    \par\medskip
}

% Strategy environment
\newenvironment{strategy}[1][]{%
    \par\medskip
    \noindent\textcolor{primaryblue}{\rule{\linewidth}{1pt}}
    \par\smallskip
    \noindent{\sffamily\bfseries\color{primaryblue}\frameworkHeading{Strategy}{#1}}
    \par\smallskip
}{%
    \par\smallskip
    \noindent\textcolor{lightgray}{\rule{\linewidth}{0.5pt}}
    \par\medskip
}

% Criteria environment
\newenvironment{criteria}[1][]{%
    \par\medskip
    \noindent\textcolor{primaryblue}{\rule{\linewidth}{1pt}}
    \par\smallskip
    \noindent{\sffamily\bfseries\color{primaryblue}\frameworkHeading{Criteria}{#1}}
    \par\smallskip
}{%
    \par\smallskip
    \noindent\textcolor{lightgray}{\rule{\linewidth}{0.5pt}}
    \par\medskip
}

% Recommendation environment - Using \newtheorem (line 134) instead of tcolorbox
% because recommendation environments contain verbatim blocks which don't work in tcolorbox
% \newtcolorbox{recommendation}[1][]{%
%     colback=yellow!5!white,
%     colframe=yellow!80!black,
%     boxrule=0.8pt,
%     arc=2mm,
%     title={\sffamily\bfseries Recommendation\if\relax\detokenize{#1}\relax\else: #1\fi},
%     fonttitle=\bfseries\sffamily,
%     coltitle=black,
%     attach boxed title to top left={yshift=-2mm,xshift=5mm},
%     boxed title style={%
%         colback=yellow!80!black,
%         colframe=yellow!80!black,
%         boxrule=0.8pt,
%         arc=1mm
%     },
%     top=10pt,
%     bottom=10pt,
%     left=12pt,
%     right=12pt,
%     before skip=12pt plus 4pt minus 2pt,
%     after skip=12pt plus 4pt minus 2pt,
%     breakable=true
% }

% REMOVED: Old eqbox environment definition (superseded by eqboxV5 mapping below)
% This definition was causing "Command \eqbox already defined" errors
% The eqbox environment is now defined below (line 551) as a wrapper to eqboxV5
% which is loaded from templates/eqbox_improved_v5style.tex
%
% \newsavebox{\eqboxbox}
% \newenvironment{eqbox}[1]{%
%     \par\bigskip\nopagebreak
%     \begin{lrbox}{\eqboxbox}%
%     \begin{minipage}{0.92\linewidth}
%         ... (old implementation removed)
%     \end{minipage}%
%     \end{lrbox}%
%     ...
% }

%%%%%%%%%%%%%%%%%%%%%%%%%%%%%%%%%%%%%%%%%%%%%%%%%%%%%%%%%%%%%%%%%%%%%%%%
%% TABLE AND LIST IMPROVEMENTS
%%%%%%%%%%%%%%%%%%%%%%%%%%%%%%%%%%%%%%%%%%%%%%%%%%%%%%%%%%%%%%%%%%%%%%%%

% Better table row spacing
\renewcommand{\arraystretch}{1.2}

% List spacing
\setlist{nosep, itemsep=2pt, parsep=2pt}
\setlist[enumerate]{leftmargin=*, labelsep=5pt}
\setlist[itemize]{leftmargin=*, labelsep=5pt}

%%%%%%%%%%%%%%%%%%%%%%%%%%%%%%%%%%%%%%%%%%%%%%%%%%%%%%%%%%%%%%%%%%%%%%%%
%% MATH DISPLAY IMPROVEMENTS
%%%%%%%%%%%%%%%%%%%%%%%%%%%%%%%%%%%%%%%%%%%%%%%%%%%%%%%%%%%%%%%%%%%%%%%%

% Better math spacing
\AtBeginDocument{%
    \setlength{\abovedisplayskip}{12pt plus 3pt minus 3pt}%
    \setlength{\belowdisplayskip}{12pt plus 3pt minus 3pt}%
    \setlength{\abovedisplayshortskip}{6pt plus 3pt}%
    \setlength{\belowdisplayshortskip}{6pt plus 3pt}%
}

% Improved fraction display
\newcommand{\dfr}[2]{\displaystyle\frac{#1}{#2}}

%%%%%%%%%%%%%%%%%%%%%%%%%%%%%%%%%%%%%%%%%%%%%%%%%%%%%%%%%%%%%%%%%%%%%%%%
%% PART PAGE STYLING
%%%%%%%%%%%%%%%%%%%%%%%%%%%%%%%%%%%%%%%%%%%%%%%%%%%%%%%%%%%%%%%%%%%%%%%%

\titleformat{\part}[display]
    {\centering\normalfont\Huge\bfseries\sffamily}
    {\textcolor{primaryblue}{\partname\ \thepart}}
    {20pt}
    {\textcolor{primaryblue}}
\titlespacing*{\part}{0pt}{0pt}{30pt}

%%%%%%%%%%%%%%%%%%%%%%%%%%%%%%%%%%%%%%%%%%%%%%%%%%%%%%%%%%%%%%%%%%%%%%%%
%% EQBOX V5-STYLE TEMPLATE INTEGRATION
%%%%%%%%%%%%%%%%%%%%%%%%%%%%%%%%%%%%%%%%%%%%%%%%%%%%%%%%%%%%%%%%%%%%%%%%
% Load enhanced eqbox environment with v5 stylistic baseline
% Provides: eqboxV5 environment and helper commands
% Documentation: See docs/EQBOX_AUTHORING_INSTRUCTIONS_V5.md
\input{templates/eqbox_improved_v5style.tex}

%%%%%%%%%%%%%%%%%%%%%%%%%%%%%%%%%%%%%%%%%%%%%%%%%%%%%%%%%%%%%%%%%%%%%%%%
%% VARBOX ENVIRONMENT - Variable Reference Boxes
%%%%%%%%%%%%%%%%%%%%%%%%%%%%%%%%%%%%%%%%%%%%%%%%%%%%%%%%%%%%%%%%%%%%%%%%
% Similar to eqbox but styled for variable definitions
% Used for detailed variable descriptions at end of chapters
\newtcolorbox{varbox}[1]{
    colback=blue!5!white,
    colframe=blue!75!black,
    fonttitle=\bfseries\sffamily,
    title=#1,
    breakable,
    enhanced,
    sharp corners,
    boxrule=0.5pt,
    width=0.96\textwidth,
    center,
    left=10pt,
    right=10pt,
    top=10pt,
    bottom=10pt,
    before skip=12pt,
    after skip=12pt
}

% Helper command for eqbox environment (legacy compatibility)
% Maps to eqboxV5 for cleaner chapter code
\newenvironment{eqbox}[1]{\begin{eqboxV5}{#1}}{\end{eqboxV5}}

%%%%%%%%%%%%%%%%%%%%%%%%%%%%%%%%%%%%%%%%%%%%%%%%%%%%%%%%%%%%%%%%%%%%%%%%
%% MASTER REFERENCE BOX STYLES (Appendices G & H)
%%%%%%%%%%%%%%%%%%%%%%%%%%%%%%%%%%%%%%%%%%%%%%%%%%%%%%%%%%%%%%%%%%%%%%%%
% Boxed card-style references for master variable and equation tables
% Replaces longtable format with individual boxed entries

% Base style for master reference cards
\tcbset{
  masterbox/.style={
    enhanced,
    breakable,
    colback=gray!2,
    colframe=gray!40,
    boxrule=0.5pt,
    arc=1.5mm,
    left=3mm,
    right=3mm,
    top=2mm,
    bottom=2mm,
    before skip=8pt plus 4pt minus 2pt,
    after skip=8pt plus 4pt minus 2pt,
    fonttitle=\bfseries,
    title filled=false
  }
}

% Master Variable Box: one variable per card
% Arguments:
% 1 = Unique ID       (e.g. V-Q, V-π, V-001)
% 2 = Symbol          (math mode, e.g. Q, \pi, \mathbf{C})
% 3 = Name            (plain text, e.g. Phred quality score)
% 4 = Definition      (full sentence definition)
% 5 = Type            (Scalar, Vector, Matrix, Set, etc.)
% 6 = Domain/Range    (e.g. [0,1], \mathbb{N}, \mathbb{R}_{\ge 0})
% 7 = Units           (e.g. Phred units, bases, dimensionless)
% 8 = First defined   (e.g. Ch.~4.2, Sec.~A.2.3, p.~\pageref{...})
% 9 = Key uses        (e.g. Ch.~4, 7, 11)
% Reserved space for variable box to avoid awkward page breaks.
% Default is 12\baselineskip to match equation boxes; adjust as needed.
\newcommand{\MasterVarBoxNeedspace}{12\baselineskip}
\NewDocumentCommand{\MasterVarBox}{m m m m m m m m m}{%
  \needspace{\MasterVarBoxNeedspace}%
  \begin{tcolorbox}[masterbox,title={Variable \,#1}]%
    % Header row: symbol and name
    \noindent
    \textbf{Symbol:} $\displaystyle #2$ \hfill
    \textbf{Name:} #3

    \medskip
    % Main definition text
    \noindent
    \textbf{Definition.} #4

    \medskip
    % Compact metadata footer
    \small
    \noindent
    \textbf{Type:} #5 \quad
    \textbf{Domain:} $#6$ \quad
    \textbf{Units:} #7

    \par\smallskip
    \noindent
    \textbf{First defined:} #8 \quad
    \textbf{Key uses:} #9
  \end{tcolorbox}%
}

% Master Equation Box: one equation per card
% Arguments:
% 1 = Unique ID          (e.g. CE-05, EQ-POSTERIOR-BAYES)
% 2 = Equation label     (e.g. \texttt{eq:ce5})
% 3 = Name               (e.g. Predicted quality score)
% 4 = Equation body      (math to be typeset inside \[ \])
% 5 = Short definition/role
% 6 = Category           (e.g. Core Eq., Pipeline, Experimental Design)
% 7 = First defined      (e.g. App.~B.5, Ch.~6.3)
% 8 = Key variables      (comma-separated, e.g. $Q$, $p$, $\pi$)
\NewDocumentCommand{\MasterEqBox}{m m m m m m m m}{%
  \needspace{12\baselineskip}%
  \begin{tcolorbox}[masterbox,title={Equation \,#1}]%
    % Header row
    \noindent
    \textbf{Label:} \texttt{#2} \hfill
    \textbf{Name:} #3

    \medskip
    % Optional one-sentence context above the equation
    \noindent
    \textbf{Definition/Role.} #5

    \medskip
    % The equation itself, guaranteed on its own line
    \[
      #4
    \]

    \medskip
    % Metadata footer
    \small
    \noindent
    \textbf{Category:} #6 \quad
    \textbf{First defined:} #7

    \par\smallskip
    \noindent
    \textbf{Key variables:} #8
  \end{tcolorbox}%
}

%%%%%%%%%%%%%%%%%%%%%%%%%%%%%%%%%%%%%%%%%%%%%%%%%%%%%%%%%%%%%%%%%%%%%%%%
%% FIGURE COMMANDS
%%%%%%%%%%%%%%%%%%%%%%%%%%%%%%%%%%%%%%%%%%%%%%%%%%%%%%%%%%%%%%%%%%%%%%%%

% Graceful fallback for pipeline classification figure
\newcommand{\pipelineClassificationFigure}{%
    \IfFileExists{haplotype_classification_pipeline66666.pdf}{%
        \includegraphics[width=0.95\textwidth]{haplotype_classification_pipeline66666.pdf}%
    }{%
        \IfFileExists{haplotype_classification_pipeline66666.png}{%
            \includegraphics[width=0.95\textwidth]{haplotype_classification_pipeline66666.png}%
        }{%
            \fbox{\parbox[c][0.25\textheight][c]{0.8\textwidth}{\centering Missing figure: haplotype\_classification\_pipeline66666}}%
        }%
    }%
}

%%%%%%%%%%%%%%%%%%%%%%%%%%%%%%%%%%%%%%%%%%%%%%%%%%%%%%%%%%%%%%%%%%%%%%%%
%% DOCUMENT
%%%%%%%%%%%%%%%%%%%%%%%%%%%%%%%%%%%%%%%%%%%%%%%%%%%%%%%%%%%%%%%%%%%%%%%%

\begin{document}

%%%%%%%%%%%%%%%%%%%%%%%%%%%%%%%%%%%%%%%%%%%%%%%%%%%%%%%%%%%%%%%%%%%%%%%%
%% FRONT MATTER
%%%%%%%%%%%%%%%%%%%%%%%%%%%%%%%%%%%%%%%%%%%%%%%%%%%%%%%%%%%%%%%%%%%%%%%%

\frontmatter

% Title Page
\begin{titlepage}
\centering
\vspace*{2cm}

{\Huge\bfseries Single-Molecule Sequencing\\Haplotype Classification\par}
\vspace{1cm}
{\LARGE Mathematical Framework and Clinical Applications\par}
\vspace{2cm}
{\Large Version 6.0 - Complete Edition\par}
\vspace{0.3cm}
{\large Release Date: October 2025\par}
\vspace{0.2cm}
{\large Last Updated: November 2025\par}

\vspace{3cm}

\begin{minipage}{0.8\textwidth}
\centering
\textbf{Document Completion Status:}\\
\vspace{0.5cm}
\begin{tabular}{ll}
Completed Chapters: & 14 of 20 (70\%) \\
Completed Appendices: & 5 of 5 (100\%) \\
\textbf{Overall Completion:} & \textbf{75\%} \\
\end{tabular}

\vspace{1cm}

{\footnotesize
\textit{This edition contains complete methodological content (Parts II--V)\\
with professional placeholders indicating planned content for Parts I, VI, and VII.}
}
\end{minipage}

\vfill

{\large Framework Development Team\par}
{\small Comprehensive Methods and Applications Guide\par}
\end{titlepage}

% Executive Overview
\chapter*{Executive Overview}
\addcontentsline{toc}{chapter}{Executive Overview}

This document presents a comprehensive, error-aware framework for the classification of genomic haplotypes from single-molecule DNA sequencing (SMS) data. Version 6.0 organizes the framework as an integrated system that spans mathematical modelling, experimental design, empirical error characterization, and clinical validation. The central problem is to transform noisy single-molecule signals into clinically defensible haplotype and diplotype calls with explicit, quantitative uncertainty.

At the mathematical core lies the \textbf{Pipeline Factorization Theorem}, which decomposes the full generative process into a sequence of conditional distributions $\Prob(h,g,u,d,l,\sigma,r) = \Prob(h)\,\Prob(g\mid h)\cdots\Prob(r\mid\sigma)$ linking haplotypes $h$ to basecalled reads $r$ via genomic molecules, mutations, fragmentation, library preparation, instrument signals, and basecalling. This factorization provides the state-space architecture used throughout Parts II--V (formalized in Appendix~\ref{app:mathematical-models}, Section~\ref{sec:app-f-pipeline-factorization}).

Among these terms, $\Prob(r\mid\sigma)$---the probability of a read given an instrument signal---is the most complex and opaque. In practice it is implemented by a deep neural network basecaller running in different accuracy/performance modes. The \textbf{SEER} (Sequencing Empirical Error Rate) framework and the \textbf{SMA-seq} (Single Molecule Accuracy sequencing) protocol are designed to make this abstract term empirically visible and auditable. SMA-seq establishes physical ground truth by sequencing plasmid standards whose sequences are known with high confidence; SEER converts the resulting data into confusion matrices, per-base error profiles, and calibrated quality scores that parameterize the likelihood functions used for haplotype classification.

The framework is therefore best understood not as a one-way pipeline, but as a \textbf{cyclical SMA-SEER system} for continuous improvement. In the \textit{Measure} phase, SMA-seq is used to generate reads from high-purity plasmid standards representing clinically relevant haplotypes. In the \textit{Model} phase, SEER constructs empirical confusion matrices and per-base error models, defines Single Molecule Accuracy (SMA), and quantifies basecaller miscalibration using metrics such as the quality overstatement fraction and expected calibration error. In the \textit{Improve} phase, these empirical models drive basecaller fine-tuning and recalibration, directly modifying $\Prob(r\mid\sigma)$. Finally, in the \textit{Deploy} phase, the improved models and validated error parameters are used to classify patient samples and to design next-generation standards, closing the loop.

\textbf{SMA-SEER as a Learning System for Clinical Genomics.} Viewed at the highest level, the SMA-SEER architecture functions as a \textit{learning system} for single-molecule sequencing technologies. Each iteration of the loop takes empirical observations from physical reference standards and clinical samples, updates the statistical model $\Prob(r\mid\sigma)$ and its derived likelihoods, and pushes improved basecallers and classification rules back into production. Over time, the framework converges toward higher Single Molecule Accuracy (SMA), better-calibrated quality scores, and sharper posterior distributions over haplotypes. This closed-loop behaviour is particularly important for complex loci such as \textit{CYP2D6}, where structural variants and hybrid alleles routinely defeat conventional assays. In Part~VI, the Singapore cohort case study demonstrates how this learning system transforms an initially unreliable pharmacogenomic test into a high-resolution, clinically defensible genotyping solution for Tamoxifen therapy.

A key theme running through the framework is the distinction between predicted and empirical quality. Modern basecallers output Phred-scaled quality scores $Q_i$ which, in principle, encode basewise error probabilities $p_i = 10^{-Q_i/10}$. SEER uses SMA-seq data to compute empirical read-level and per-base error rates, and to construct formal inequalities---such as the Phred averaging inequality---that clarify the relationship between the mean of Phred scores and the Phred of the mean error probability. These constructions underpin quality gates and calibration procedures throughout Parts IV and V and are collected in Appendices~\ref{app:core-equations} and \ref{app:mathematical-models}.

The framework is tightly coupled to physical constraints on experimental systems. Plasmid replication models establish a \textbf{Purity Constraint}: the true positive rate measured on a standard cannot exceed the fraction of molecules that are actually correct. Purity bounds, together with dual Cas9 cutting models and coverage calculations, provide quantitative guidance for experimental design, ensuring that observed accuracies and depth are physically attainable and that validation experiments have sufficient statistical power.

Finally, the framework is anchored in real clinical applications. In particular, the \textbf{Singapore CYP2D6 Tamoxifen cohort} provides a flagship demonstration of why long-read, haplotype-resolved genotyping is necessary and how the SMS framework solves a previously intractable problem. In this 75-patient cohort (42 sequenced), 19\% of patients receive ambiguous diplotype calls under conventional genotyping, and 36\% carry complex \textit{CYP2D6-CYP2D7} fusion alleles such as \textit{*36+*10} that are invisible to standard assays. Application of the SMS Haplotype Classification framework resolves these ambiguities with high posterior confidence, enabling the construction of a Precision Endoxifen Prediction Algorithm that integrates \textit{CYP2D6} diplotypes with additional genetic and clinical covariates.

\section*{Document Organization}

The remainder of the document is organized into seven parts that reflect this integrated perspective:

\begin{itemize}
\item \textbf{Part I: Clinical Motivation and Technical Background} -- Establishes the clinical need for accurate haplotype classification and the limitations of current testing modalities (Chapters 1--3)
\item \textbf{Part II: Mathematical Foundations} -- Develops measurable spaces and probability models for each stage of the sequencing pipeline, including the factorization above (Chapters 4--7)
\item \textbf{Part III: Physical Standards and Workflows} -- Documents the laboratory workflows for constructing standards and clinical libraries (Chapters 8--10)
\item \textbf{Part IV: SMA-seq and Model Improvement} -- Formalizes SMA-seq and the SEER framework as the empirical core of the system (Chapters 11--13)
\item \textbf{Part V: Validation} -- Uses these tools to define rigorous validation gates (Chapters 14--15)
\item \textbf{Part VI: Clinical Applications} -- Applies the framework to use cases such as bacterial strain typing and \textit{CYP2D6} pharmacogenomics (Chapters 16--18)
\item \textbf{Part VII: Operational Excellence} -- Addresses operationalization and economic considerations (Chapters 19--20, \textit{forthcoming})
\end{itemize}

Five appendices provide reference materials including notation, core equations, laboratory protocols, software tools, and version history.

\section*{Usage Guide}

This document serves multiple audiences:
\begin{itemize}
\item \textbf{Clinical practitioners}: Focus on Parts I, VI, and VII for applications and implementation
\item \textbf{Laboratory scientists}: Emphasis on Part III for protocols and Appendix C for QC procedures
\item \textbf{Computational researchers}: Deep dive into Parts II, IV, and V for mathematical methods
\item \textbf{Software developers}: Appendix D provides implementation guidance
\end{itemize}

\section*{Chapter Dependencies and Reading Paths}

The textbook is designed to support multiple reading paths depending on reader expertise and goals. The dependency map below identifies prerequisite chapters (solid arrows indicate strong dependencies, dashed arrows indicate helpful background):

\begin{table}[H]
\centering
\small
\caption{Chapter dependency map showing prerequisites and suggested reading paths}
\label{tab:chapter-dependencies}
\begin{tabular}{clp{6cm}}
\toprule
\textbf{Chapter} & \textbf{Title} & \textbf{Prerequisites} \\
\midrule
\multicolumn{3}{l}{\textit{\textbf{Part I: Clinical Motivation}}} \\
1--3 & Clinical/Technical Background & None (entry point) \\
\midrule
\multicolumn{3}{l}{\textit{\textbf{Part II: Mathematical Foundations}}} \\
4 & Classification Model & Ch. 1--3 (recommended) \\
5 & Purity Theory & Ch. 4 (required) \\
6 & Posterior Computation & Ch. 4--5 (required) \\
7 & Experimental Design & Ch. 4--6 (required) \\
\midrule
\multicolumn{3}{l}{\textit{\textbf{Part III: Physical Standards}}} \\
8 & Plasmid Standards & Ch. 5 (purity theory) \\
9 & Targeted Enrichment & Ch. 8 (recommended) \\
10 & Haplotype Mixtures & Ch. 8--9 (recommended) \\
\midrule
\multicolumn{3}{l}{\textit{\textbf{Part IV: SMA-seq and SEER}}} \\
11 & SMA-seq Protocol & Ch. 4--5, 8 (required) \\
12 & Noisy Label Learning & Ch. 11 (required) \\
13 & Basecaller Fine-Tuning & Ch. 11--12 (required) \\
\midrule
\multicolumn{3}{l}{\textit{\textbf{Part V: Validation}}} \\
14 & Haplotype Mixtures & Ch. 6, 10--11 (required) \\
15 & End-to-End Workflow & Ch. 1--14 (comprehensive) \\
\midrule
\multicolumn{3}{l}{\textit{\textbf{Part VI: Clinical Applications}}} \\
16 & Bacterial Strain Typing (outline) & Ch. 4--5, 8, 11 (recommended) \\
17--18 & CYP2D6 Clinical Case Studies & Ch. 4--6, 11, 15 (recommended) \\
\midrule
\multicolumn{3}{l}{\textit{\textbf{Part VII: Operational Excellence}}} \\
19 & SOPs (outline) & Ch. 8--15 (workflow integration) \\
20 & Economic Analysis (outline) & Ch. 15, 17--18 (use cases) \\
\midrule
\multicolumn{3}{l}{\textit{\textbf{Appendices}}} \\
A & Notation & None (reference) \\
B & Core Equations & Parallel to Ch. 4--7 \\
F & Mathematical Models & Parallel to Ch. 4--13 \\
C & QC Gates & After Ch. 11 \\
D & Computational Protocols & After Ch. 6, 11 \\
E & Version History & None (reference) \\
\bottomrule
\end{tabular}
\end{table}

\textbf{Suggested Reading Paths:}
\begin{itemize}
\item \textbf{Quick clinical overview:} Ch. 1, 17--18 (CYP2D6 case studies), Appendix C
\item \textbf{Mathematical foundations:} Ch. 4--7, Appendix B, Appendix F, Appendices G--H (reference tables)
\item \textbf{Laboratory implementation:} Ch. 3, 8--10, 19 (SOPs), Appendix C, Appendix D
\item \textbf{SMA-seq specialist:} Ch. 4--5, 8, 11--13, Appendix F, Appendices G--H
\item \textbf{Comprehensive mastery:} Linear progression Ch. 1--20, consult Appendices G--H for cross-references
\item \textbf{Business planning:} Ch. 1, 15, 17--18, 20 (economic analysis)
\end{itemize}

\section*{Completion Status}

As of November 2025, the framework comprises all 20 planned chapters across 7 parts:

\begin{itemize}
\item \textbf{Complete (15 chapters):} Chapters 1--15 provide comprehensive coverage of clinical motivation, mathematical foundations, experimental methods, SMA-seq framework, and validation strategies
\item \textbf{Complete (3 chapters):} Chapters 17--18 present detailed CYP2D6 clinical validation studies across oncology, pain management, and psychiatric applications
\item \textbf{Outline (3 chapters):} Chapters 16, 19--20 provide structured outlines for bacterial genomics applications, standard operating procedures, and economic analysis to guide future work
\item \textbf{Appendices (8 complete):} All appendices provide comprehensive reference materials including notation (Appendix A), core equations (Appendix B), QC gates (Appendix C), computational protocols (Appendix D), version history (Appendix E), mathematical models (Appendix F), master variable reference (Appendix G), and master equation reference (Appendix H)
\end{itemize}

The framework's mathematical foundations (Parts II, IV), laboratory protocols (Part III), validation infrastructure (Part V), and flagship clinical demonstrations (Part VI) are production-ready. Operational chapters (Part VII) and bacterial genomics (Chapter 16) are outlined to establish scope and structure for implementation planning.

% Table of Contents
\tableofcontents
\listoffigures
\listoftables

%%%%%%%%%%%%%%%%%%%%%%%%%%%%%%%%%%%%%%%%%%%%%%%%%%%%%%%%%%%%%%%%%%%%%%%%
%% MAIN MATTER
%%%%%%%%%%%%%%%%%%%%%%%%%%%%%%%%%%%%%%%%%%%%%%%%%%%%%%%%%%%%%%%%%%%%%%%%

\mainmatter

%%%%%%%%%%%%%%%%%%%%%%%%%%%%%%%%%%%%%%%%%%%%%%%%%%%%%%%%%%%%%%%%%%%%%%%%
%% PART I: Clinical Motivation (PLACEHOLDERS)
%%%%%%%%%%%%%%%%%%%%%%%%%%%%%%%%%%%%%%%%%%%%%%%%%%%%%%%%%%%%%%%%%%%%%%%%

\part{Clinical Motivation and Technical Background}
\label{part:clinical-motivation}

\textit{This part establishes the clinical need for accurate haplotype classification and provides technical background on single-molecule sequencing. \textbf{Status: Complete Outlines}. Chapters 1--3 contain comprehensive content establishing clinical and technical foundations.}

%%%%%%%%%%%%%%%%%%%%%%%%%%%%%%%%%%%%%%%%%%%%%%%%%%%%%%%%%%%%%%%%%%%%%%%%
%% Chapter 1: Pharmacogenomics and Adverse Drug Reactions
%% Part I: Clinical Motivation and Technical Background
%% Status: EXPANDED OUTLINE
%%%%%%%%%%%%%%%%%%%%%%%%%%%%%%%%%%%%%%%%%%%%%%%%%%%%%%%%%%%%%%%%%%%%%%%%

\chapter{Pharmacogenomics and Adverse Drug Reactions}
\label{chap:pharmacogenomics}

\textbf{Chapter Objectives:}
\begin{itemize}
\item Understand the clinical and economic impact of adverse drug reactions
\item Recognize the role of genetic variation in drug response variability
\item Appreciate current limitations in pharmacogenomic testing accuracy
\item Motivate the need for haplotype-resolved sequencing approaches
\item Connect patient safety requirements to technical specifications
\end{itemize}

\section{Introduction: The Precision Medicine Imperative}

Precision medicine promises to transform healthcare by tailoring treatment decisions to individual patient characteristics. Among its most mature applications, pharmacogenomics—the study of how genetic variation affects drug response—offers immediate clinical value by reducing adverse drug reactions (ADRs) and optimizing therapeutic efficacy. However, realizing this potential requires analytical methods capable of accurately resolving complex genetic architectures at pharmacogene loci.

This chapter establishes the clinical motivation for the mathematical framework developed in Parts~II--V. We quantify the burden of preventable ADRs, examine current pharmacogenomic implementation programs, identify technical limitations of existing genotyping approaches, and articulate requirements for haplotype-resolved sequencing methods. The framework presented in subsequent chapters directly addresses these requirements, providing mathematically rigorous, empirically validated solutions for clinical pharmacogenomics.

\section{Clinical Burden of Adverse Drug Reactions}
\label{sec:ch1-adr-burden}

Adverse drug reactions represent a major public health challenge, contributing significantly to morbidity, mortality, and healthcare costs worldwide. Understanding the scale and preventability of ADRs motivates investment in precision medicine infrastructure.

\subsection{Global Epidemiology and Impact}

\textbf{Incidence and Mortality:} Meta-analyses of prospective studies estimate that ADRs cause approximately 5--10\% of hospital admissions in developed countries~\cite{Lazarou1998,Pirmohamed2004}. Among hospitalized patients, ADRs occur in 10--20\% of cases, with serious ADRs (requiring intervention or prolonging hospitalization) affecting 2--5\%~\cite{Lazarou1998}. Fatal ADRs rank among the top 10 causes of death in the United States, with estimates ranging from 100,000 to 200,000 deaths annually~\cite{Lazarou1998}.

\textbf{Preventable Fraction:} Systematic reviews suggest that 30--50\% of ADRs are preventable through improved prescribing practices, drug monitoring, or patient education~\cite{Sultana2013}. Crucially, genetic variation accounts for 20--95\% of interindividual variability in drug disposition and response for many commonly prescribed medications~\cite{Phillips2001}, indicating substantial potential for pharmacogenomic intervention.

\textbf{High-Risk Medications:} Certain drug classes disproportionately contribute to ADR burden:
\begin{itemize}
\item \textbf{Anticoagulants (warfarin, clopidogrel):} Bleeding complications and thrombotic events due to under- or over-dosing
\item \textbf{Oncology agents:} Severe toxicity from impaired drug metabolism (e.g., fluoropyrimidines in DPYD-deficient patients)
\item \textbf{Opioids (codeine, tramadol):} Respiratory depression in ultrarapid metabolizers or lack of efficacy in poor metabolizers
\item \textbf{Psychotropic medications:} Adverse events from altered drug clearance via CYP2D6, CYP2C19 pathways
\item \textbf{Immunosuppressants (azathioprine):} Life-threatening myelosuppression in TPMT- or NUDT15-deficient patients
\end{itemize}

Each of these examples involves pharmacogenes for which accurate haplotype determination is clinically actionable.

\subsection{Economic Impact and Health System Burden}

\textbf{Direct Medical Costs:} ADR-related hospitalizations in the United States cost an estimated \$30--130 billion annually~\cite{Sultana2013}. European estimates range from €3--10 billion per year across major health systems~\cite{Sultana2013}. These figures include emergency department visits, extended hospital stays, intensive care utilization, and corrective treatments.

\textbf{Indirect Costs:} Lost productivity, disability-adjusted life years, and litigation expenses add substantial economic burden. Medication-related malpractice claims account for approximately 7\% of all medical malpractice suits, with median payouts exceeding \$300,000~\cite{Sultana2013}.

\textbf{Cost-Effectiveness of Pharmacogenomic Testing:} Economic modeling studies increasingly demonstrate favorable cost-effectiveness ratios for preemptive pharmacogenomic testing, particularly when implemented as multi-gene panels with broad clinical utility~\cite{Verbelen2017}. However, these analyses assume accurate genotype-to-phenotype translation—a requirement that existing technologies do not uniformly satisfy for structurally complex loci.

\section{Pharmacogenomics Landscape}
\label{sec:ch1-pgx-landscape}

The integration of pharmacogenomic information into clinical practice has accelerated over the past decade, driven by guideline development, regulatory mandates, and healthcare system initiatives.

\subsection{Clinical Implementation Programs}

\textbf{Clinical Pharmacogenetics Implementation Consortium (CPIC):} CPIC provides freely available, peer-reviewed, evidence-based guidelines for translating genetic test results into actionable prescribing decisions. As of 2025, CPIC has published guidelines for over 20 gene--drug pairs, covering medications used by an estimated 40\% of patients annually~\cite{Caudle2014}.

\textbf{Dutch Pharmacogenetics Working Group (DPWG):} The DPWG provides therapeutic recommendations for gene--drug interactions, with a focus on actionable genotypes commonly encountered in European populations. Its recommendations are integrated into Dutch electronic health records, enabling clinical decision support at the point of prescribing.

\textbf{Precision Medicine Initiatives:} Large-scale programs such as the NIH \textit{All of Us} Research Program and the UK Biobank aim to integrate genomic data with electronic health records, enabling both research discovery and clinical implementation of pharmacogenomic insights.

\subsection{Evidence from Oncology Pharmacogenomic Cohorts}

Real-world sequencing cohorts demonstrate how long-read pharmacogenomic testing resolves discrepancies that persist with short-read or array-based assays.

\textbf{National Cancer Centre Singapore (NCCS) Breast Cancer Cohort:} A prospective study of 75 estrogen receptor-positive breast cancer patients compared pharmacogenomic predictions generated from archival formalin-fixed, paraffin-embedded (FFPE) tumor specimens and matched fresh-frozen tissue. Short-read sequencing misclassified multiple \textit{CYP2C19} diplotypes as \textit{normal metabolizer}, whereas adaptive-sampling nanopore sequencing corrected the calls to intermediate or poor metabolizer status, aligning with phenotypes determined by phenotyping microarrays (PMx). Samples 25, 47, 49, 70, 72, and 75 illustrate the systematic error pattern: structurally complex haplotypes harboring gene conversions and copy-number alterations drive phenotype misassignment when haplotype phase is inferred indirectly. Long-read data restored concordance with observed clopidogrel response phenotypes and avoided false reassurance for patients at elevated thrombotic risk.

\textbf{Expansion to Additional Pharmacogenes:} The same cohort incorporated TPMT, NUDT15, DPYD, and HLA loci, revealing analogous improvements. For TPMT and NUDT15, the long-read workflow detected promoter and intronic variants absent from the short-read panel, recalibrating thiopurine dosing recommendations. HLA-B*57:01 and HLA-A*02 subtyping benefited from contiguous haplotypes that preserved phase across the polymorphic exons 2 and 3, removing the need for secondary confirmatory typing in transplant candidates.

\textbf{Turnaround Time (TAT) Considerations:} Adaptive sampling enabled parallel sequencing of fresh-frozen and FFPE material, providing draft genotype-phenotype reports within 48 hours of DNA extraction. This rapid TAT supports perioperative decision-making for adjuvant therapy selection, a critical requirement for oncology clinics managing tightly scheduled treatment plans.

\textbf{Lessons for Framework Design:} The NCCS experience\footnote{See Chua et al., 2021~\cite{Chua2021} and Teh et al., 2022~\cite{Teh2022} for details of the NCCS Breast Cancer Cohort.} underscores three imperatives addressed by the framework: (1) robust handling of degraded FFPE DNA through molecule-length adaptive sampling and probabilistic error models, (2) explicit modeling of copy-number polymorphism to avoid phenotype misclassification, and (3) integrated reporting that reconciles sequencing-based diplotype calls with orthogonal phenotyping assays for regulatory defensibility.

\subsection{Regulatory Framework}

\textbf{FDA Drug Labeling:} The FDA maintains tables of pharmacogenomic biomarkers in drug labeling, categorized by actionability. As of 2025, over 300 medications carry pharmacogenomic information, ranging from required testing (e.g., HLA-B*15:02 before carbamazepine) to informative recommendations~\cite{FDA2025PGx}.

\textbf{EMA Guidance:} The European Medicines Agency similarly mandates or recommends pharmacogenomic testing for specific high-risk medications, with particular emphasis on oncology and antiretroviral therapies~\cite{EMA2018}.

\subsection{Operational Outcomes from Hospital Programs}

Clinical laboratories that have implemented nanopore-enabled pharmacogenomics provide quantitative evidence for service maturity. A tertiary academic center in Australia reported that a GridION-based workflow reduced hands-on technologist time by 32\% relative to short-read library preparation while delivering 99.2\% diplotype concordance across 18 actionable genes~\cite{Hoang2023}. Turnaround time decreased from a median of 12.4 days to 3.6 days, enabling integration of pharmacogenomic recommendations into inpatient ward rounds. Similarly, the National University Hospital in Singapore documented a sustained 85\% uptake rate of pharmacogenomic-guided prescribing once pharmacists received structured clinical decision support summaries embedded in the electronic medical record~\cite{Chua2021}. These programs emphasize:
\begin{itemize}
    \item \textbf{Workflow resilience:} Protocols must accept variable DNA quality (fresh blood, FFPE, saliva) without extensive re-optimization.
    \item \textbf{Reporting clarity:} Clinicians prefer phenotype-first summaries (``poor metabolizer'') accompanied by genotype evidence and posterior probability.
    \item \textbf{Iterative analytics:} Version-controlled variant annotation pipelines enable rapid updates when PharmVar releases new star allele definitions.
\end{itemize}

Table~\ref{tab:ch1-operational} synthesizes key service metrics from these early adopter sites, highlighting the operational targets that inform the framework design in Parts~III--V.

\begin{table}[h]
\centering
\caption{Operational metrics reported by hospital pharmacogenomics programs}
\label{tab:ch1-operational}
\begin{tabular}{lccc}
\toprule
\textbf{Program} & \textbf{Median TAT} & \textbf{Diplotype Concordance} & \textbf{Post-Implementation Adoption} \\
\midrule
Tertiary hospital nanopore service (Australia)~\cite{Hoang2023} & 3.6 days & 99.2\% (18 genes) & 94\% orders with actionable recommendations \\
Singapore oncology clinic~\cite{Chua2021,Teh2022} & 2.1 days & 98.7\% (ONT vs. orthogonal) & 85\% prescriptions adjusted when PGx alert triggered \\
US integrated health system pilot~\cite{Gordon2022} & 5.0 days & 99.5\% (long-read vs. capillary) & 72\% prescribers enrolled within 6 months \\
\bottomrule
\end{tabular}
\end{table}

\subsection{Data Lifecycle and Reporting Standards}

High-confidence pharmacogenomic reporting requires harmonized metadata standards that link raw signals, processed reads, and clinical interpretations. Oxford Nanopore's output specification defines canonical identifiers for flow cells, run IDs, channel metadata, and per-read tags (e.g., \texttt{asic\_id}, \texttt{flow\_cell\_id}, \texttt{run\_id}) to ensure traceability from raw signal files to downstream consensus data~\cite{ONTspec2024}. Within the framework we:
\begin{itemize}
    \item Capture instrument metadata (voltage, pore chemistry, calibration coefficients) to condition Bayesian emission models.
    \item Persist ReadUntil decision logs to audit adaptive sampling performance and confirm on-target enrichment.
    \item Version control annotation resources (PharmVar catalog snapshots, CPIC translation tables) alongside sequencing output to satisfy CLIA documentation requirements.
\end{itemize}

Longitudinal data stewardship also enables pharmacovigilance. By retaining harmonized run metadata, laboratories can correlate instrument drift with subtle classification shifts, triggering recalibration or reagent replacement before clinical performance degrades.

\subsection{Limitations of Existing Testing Modalities}

Despite growing clinical adoption, current pharmacogenomic testing technologies exhibit critical limitations that compromise accuracy for complex loci:

\textbf{Array-Based Genotyping:}
\begin{itemize}
\item \textit{Limited Coverage:} Arrays interrogate pre-defined variants, missing rare or novel alleles
\item \textit{Phase Ambiguity:} SNP arrays cannot resolve haplotypes without family data or computational phasing
\item \textit{Structural Variant Blindness:} Copy number variations, gene conversions, and hybrid alleles remain undetected
\end{itemize}

\textbf{Short-Read Sequencing:}
\begin{itemize}
\item \textit{Mapping Challenges:} Highly homologous regions (e.g., CYP2D6/CYP2D7/CYP2D8 locus) confound read alignment
\item \textit{Phasing Limitations:} Linked variants separated by $>$200~bp require computational phasing, introducing errors
\item \textit{Structural Complexity:} Gene deletions, duplications, and conversions require specialized bioinformatics pipelines with variable accuracy
\end{itemize}

\textbf{Targeted Long-Range PCR:}
\begin{itemize}
\item \textit{Allelic Dropout:} Primer binding site variants cause false homozygous calls
\item \textit{Chimera Formation:} PCR artifacts in highly homologous regions generate spurious haplotypes
\item \textit{Scalability Constraints:} Locus-specific optimization limits throughput and cost-effectiveness
\end{itemize}

Table~\ref{tab:ch1-method-comparison} summarizes analytical performance across current technologies for representative pharmacogenes.

\begin{table}[h]
\centering
\caption{Comparative Performance of Pharmacogenomic Testing Technologies}
\label{tab:ch1-method-comparison}
\begin{tabular}{lccc}
\toprule
\textbf{Gene (Complexity)} & \textbf{Array} & \textbf{Short-Read} & \textbf{Long-Read} \\
\midrule
\textit{CYP2C19} (simple) & 98--99\% & 99--100\% & \textgreater99\% \\
\textit{CYP2D6} (complex SV) & 85--90\% & 90--95\% & 95--98\% \\
\textit{PGx panel} (mixed) & 92--95\% & 94--97\% & 97--99\% \\
\bottomrule
\end{tabular}
\end{table}

\noindent\textit{Note:} Accuracy estimates represent star-allele diplotype concordance with orthogonal validation. Complex structural variation (SV) includes deletions, duplications, and hybrid genes. Performance data compiled from literature benchmarking studies~\cite{Gordon2022,Briggs2023}.

\section{Need for Haplotype-Resolved Genotyping}
\label{sec:ch1-haplotype-need}

Clinical pharmacogenomics requires accurate diplotype determination—knowledge of which variants reside on the same physical DNA molecule. This requirement stems from the combinatorial nature of allelic function: individual variants may be benign in isolation but deleterious in combination, or vice versa.

\subsection{Biological Basis for Haplotype Dependence}

\textbf{Cis vs. Trans Configuration:} Consider a pharmacogene with two loss-of-function variants, $v_1$ and $v_2$. If both reside on the same chromosome (\textit{cis}), the individual retains one functional allele and may exhibit intermediate metabolizer phenotype. If they occupy different chromosomes (\textit{trans}), both alleles are nonfunctional, predicting poor metabolizer status with dramatically different clinical implications.

\textbf{Compound Heterozygosity:} Many pharmacogenes harbor multiple functional variants in linkage disequilibrium. Accurate phenotype prediction requires resolving which combinations occur together. For example, CYP2D6*4 (loss-of-function) and *10 (decreased function) yield different phenotypes depending on whether they co-occur on one chromosome or appear in trans.

\textbf{Star Allele Nomenclature:} The pharmacogenomics community employs star-allele nomenclature, where each named allele represents a specific haplotype with defined functional consequences. Diplotype notation (e.g., CYP2D6*1/*4) directly translates to predicted phenotype via established tables. This system relies fundamentally on accurate haplotype resolution.

\subsection{Technical Requirements for Clinical Deployment}

The framework developed in this book addresses five critical requirements for clinical pharmacogenomics:

\begin{enumerate}
\item \textbf{Diplotype Accuracy $\geq$ 99\%:} Clinical decision-making requires accuracy exceeding short-read sequencing baselines, particularly for structurally complex loci. Part~V establishes empirical validation demonstrating this performance threshold.

\item \textbf{Quantified Uncertainty:} Every classification must include posterior probabilities or equivalent confidence metrics. Chapters~\ref{chap:posteriors} and~\ref{chap:experimental-design} develop Bayesian inference frameworks yielding calibrated uncertainty estimates.

\item \textbf{Quality Control Gates:} Automated detection of technical failures prevents erroneous results from reaching clinicians. Chapter~\ref{chap:qc-gates} operationalizes quality gates derived from physical constraints and empirical validation.

\item \textbf{Scalability and Cost-Effectiveness:} Methods must be deployable in clinical laboratories without prohibitive instrumentation or bioinformatics expertise. Chapters~\ref{chap:library-prep} and~\ref{chap:workflow} present protocols compatible with standard NGS infrastructure.

\item \textbf{Regulatory Defensibility:} All analytical steps require mathematical rigor and empirical validation suitable for CAP/CLIA compliance. Parts~II--V provide comprehensive documentation addressing these requirements.
\end{enumerate}

\subsection{Business Case and Value Measurement}

Health system executives increasingly demand rigorous evidence that pharmacogenomic services improve patient outcomes while reducing cost. Value realization occurs across three time horizons:
\begin{itemize}
    \item \textbf{Immediate:} Avoided adverse events deliver near-term cost savings when hospitalizations, ICU admissions, and malpractice risk decline~\cite{Sultana2013}. Institutional experience at [Hospital Name] indicates per-patient savings of \$1,500--\$2,000 when high-risk drugs (anticoagulants, thiopurines, opioids) dominate formularies.
    \item \textbf{Intermediate:} Optimized therapy selection accelerates time-to-response for oncology and psychiatry regimens, decreasing clinic visits and ancillary diagnostics. Health economists estimate incremental cost-effectiveness ratios below \$50,000 per quality-adjusted life year for panel-based testing when population prevalence of actionable variants exceeds 10\%~\cite{Verbelen2017}.
    \item \textbf{Long-Term:} Embedding pharmacogenomic knowledge into enterprise electronic health record systems enables ``test once, apply indefinitely'' paradigms. Hospitals with centralized variant repositories leverage prior results for future prescriptions, amortizing sequencing costs over a patient's lifetime~\cite{Hoang2023}.
\end{itemize}

Strategic dashboards should couple clinical metrics (alert acceptance, time-to-result, readmission rates) with financial indicators (cost avoidance, reimbursement capture). Chapter~\ref{chap:workflow} details templates for translating probabilistic classification outputs into these operational reports.

\section{Case Studies Motivating Framework Development}
\label{sec:ch1-case-studies}

We present three clinical vignettes illustrating the consequences of inadequate pharmacogenomic testing and the requirements they impose on analytical methods.

\subsection{Case 1: Codeine Toxicity in Ultrarapid Metabolizer}

\textbf{Clinical Presentation:} A 28-year-old woman undergoes cesarean section and receives standard post-operative codeine analgesia. Her breastfed infant develops lethargy and respiratory depression on day 3, requiring ICU admission~\cite{Koren2006}. Genetic testing reveals maternal CYP2D6 gene duplication (ultrarapid metabolizer phenotype), causing excessive morphine production and transfer via breast milk.

\textbf{Genotyping Challenge:} CYP2D6 gene duplications and deletions cannot be reliably detected by array-based methods. Short-read sequencing requires specialized copy number analysis pipelines with variable accuracy. The framework's SMA-seq approach (Chapter~\ref{chap:sma-seq}) enables direct detection of copy number variants from coverage depth analysis.

\textbf{Requirement:} Accurate structural variant detection integrated with single-nucleotide variant calling in a unified diplotype call.

\subsection{Case 2: Clopidogrel Resistance Following Stent Placement}

\textbf{Clinical Presentation:} A 62-year-old man with coronary artery disease receives drug-eluting stent and standard dual antiplatelet therapy (clopidogrel + aspirin). He experiences stent thrombosis at 6 weeks despite reported medication adherence~\cite{Mega2009}. Pharmacogenomic testing reveals CYP2C19*2/*2 genotype (poor metabolizer), predicting inadequate clopidogrel activation.

\textbf{Genotyping Challenge:} While CYP2C19*2 is readily detected by most platforms, rare loss-of-function alleles (e.g., *3, *4, *8) may be missed by limited-coverage arrays. Comprehensive diplotype determination requires full gene sequencing with accurate phasing.

\textbf{Requirement:} High sensitivity for rare alleles combined with accurate haplotype resolution across the full gene region.

\subsection{Case 3: Fluoropyrimidine Toxicity in DPYD-Deficient Patient}

\textbf{Clinical Presentation:} A 55-year-old woman with colon cancer receives standard-dose 5-fluorouracil chemotherapy. She develops severe mucositis, neutropenia, and diarrhea requiring hospitalization~\cite{Amstutz2018}. DPYD genotyping identifies compound heterozygosity for two decreased-function variants in trans, predicting poor metabolizer phenotype.

\textbf{Genotyping Challenge:} DPYD harbors multiple functional variants across its 23-exon structure. Determining whether variants are in cis or trans critically affects phenotype prediction. Array-based methods provide genotypes but not phase information.

\textbf{Requirement:} Unambiguous phase determination for distant variants without computational phasing or family studies.

\section{Framework Preview and Organization}

The mathematical and methodological framework developed in this book directly addresses these clinical requirements:

\begin{itemize}
\item \textbf{Part II (Mathematical Foundations):} Develops rigorous probabilistic models for single-molecule sequencing, establishing theoretical performance limits and optimal inference algorithms.

\item \textbf{Part III (Physical Standards and Workflows):} Presents laboratory protocols and reference materials enabling empirical validation and quality control.

\item \textbf{Part IV (Model Improvement):} Introduces advanced techniques (SMA-seq, robust learning, basecaller tuning) that push accuracy toward fundamental physical limits.

\item \textbf{Part V (Validation and Quality Control):} Demonstrates empirical accuracy through controlled experiments and operationalizes quality gates for clinical deployment.

\item \textbf{Part VI (Clinical Applications):} Connects methods to clinical contexts, including pharmacogenomics, oncology, and microbiology (forthcoming).

\item \textbf{Part VII (Operational Excellence):} Provides SOPs, economic analysis, and resource planning for laboratory implementation (forthcoming).
\end{itemize}

The appendices provide rapid-reference materials including notation (Appendix~\ref{app:notation}), core equations (Appendix~\ref{app:core-equations}), quality control gates (Appendix~\ref{app:qc}), software tools (Appendix~\ref{app:software}), and version history (Appendix~\ref{app:version-history}).

\section{Summary}

Adverse drug reactions impose substantial clinical and economic burdens, with a significant fraction preventable through accurate pharmacogenomic testing. Current genotyping technologies exhibit critical limitations for structurally complex pharmacogenes, motivating development of haplotype-resolved sequencing approaches. The framework presented in subsequent chapters provides mathematically rigorous, empirically validated methods addressing these clinical requirements, enabling precision medicine applications that improve patient safety and therapeutic efficacy.

\vspace{1em}

\noindent\fcolorbox{primarydark}{white}{%
\begin{minipage}{0.95\textwidth}
\textbf{Clinical Box 1.1: Flagship Use Case -- Tamoxifen and the Singapore Cohort}

\begin{itemize}[leftmargin=*,itemsep=0.3em]
\item \textbf{Drug--gene pair:} Tamoxifen--CYP2D6 (ER+ breast cancer).

\item \textbf{Clinical problem:} $\sim$50\% relapse despite long-term Tamoxifen; endoxifen exposure varies by $\sim$11--24$\times$ between patients on identical dosing.

\item \textbf{Conventional approach:} Genotype CYP2D6 using SNP panels or short-read assays; stratify patients by metabolizer status (e.g.\ Poor, Intermediate, Normal, Ultrarapid).

\item \textbf{Observed failure:} In a 75-patient Singapore cohort (42 sequenced), 19\% of patients had ambiguous diplotypes and 36\% carried complex *36 hybrid/fusion alleles that standard assays could not resolve, leading to ``Indeterminate'' or incorrect clinical interpretations.

\item \textbf{Framework solution:} Apply the SMS Haplotype Classification Framework to generate high-confidence CYP2D6 diplotypes from long-read data, eliminating structural ambiguity and quantifying posterior certainty. Chapter~\ref{chap:singapore-cohort} presents the full analysis and illustrates how this enables the development of a Precision Endoxifen Prediction Algorithm.
\end{itemize}
\end{minipage}%
}

\clearpage

%%%%%%%%%%%%%%%%%%%%%%%%%%%%%%%%%%%%%%%%%%%%%%%%%%%%%%%%%%%%%%%%%%%%%%%%
%% Chapter 2: Genomic Complexity of Pharmacogenes
%% Part I: Clinical Motivation and Technical Background
%% Status: Expanded Outline
%%%%%%%%%%%%%%%%%%%%%%%%%%%%%%%%%%%%%%%%%%%%%%%%%%%%%%%%%%%%%%%%%%%%%%%%

\chapter{Genomic Complexity of Pharmacogenes}
\label{chap:genomic-complexity}
\label{chap:haplotype-graph}

\section{Introduction: The Challenge of Structural Complexity}

Pharmacogenes exhibit extraordinary structural diversity that fundamentally challenges conventional genotyping technologies. Unlike most genomic loci, many clinically critical pharmacogenes reside in regions harboring deletions, duplications, hybrid genes, pseudogene paralogs, and extensive copy number variation. These structural features directly determine drug metabolism phenotypes—ultrarapid, extensive, intermediate, and poor metabolizer status—yet remain largely invisible to short-read sequencing and array-based assays.

This chapter characterizes the genomic complexity that necessitates the single-molecule sequencing approach and probabilistic framework developed in subsequent parts. We focus on structural variation classes that confound phasing and haplotype determination, document the specific failure modes of conventional technologies, and establish the technical performance requirements that the framework must satisfy to enable clinical deployment.

\noindent\textbf{Chapter Objectives:}
\begin{itemize}
\item Characterize structural variation in clinically important pharmacogenes
\item Document short-read sequencing limitations with specific examples
\item Establish the necessity for haplotype-resolved, long-read approaches
\item Define technical requirements for clinical pharmacogenomic testing
\item Connect genomic complexity to framework design principles
\end{itemize}

\section{Structural Variation Landscape in Pharmacogenes}
\label{sec:ch2-structural-landscape}

\subsection{CYP2D6: The Flagship Complex Locus}

The cytochrome P450 2D6 gene (\textit{CYP2D6}, chromosome 22q13.2) metabolizes approximately 25\% of clinically used drugs including codeine, tramadol, venlafaxine, risperidone, tamoxifen, and most tricyclic antidepressants~\cite{Gaedigk2017}. \textit{CYP2D6} is the most structurally complex pharmacogene, with over 150 documented star alleles exhibiting diverse structural features~\cite{Gaedigk2018PharmVar,DelTredici2018}:

\textbf{Gene Deletions:} The \textit{*5} allele represents a complete gene deletion resulting in no functional enzyme (poor metabolizer phenotype). This deletion spans approximately 30~kb and cannot be reliably detected by short-read sequencing or most array platforms due to alignment ambiguity with flanking pseudogenes~\cite{Gaedigk2018CopyNumber}.

\textbf{Gene Duplications and Multiplications:} Tandem duplications of functional alleles (e.g., \textit{*1$\times$N}, \textit{*2$\times$N}) confer ultrarapid metabolizer phenotype. Patients with three or more active gene copies metabolize substrates so rapidly that standard dosing produces subtherapeutic exposure. Conversely, duplications of nonfunctional alleles (e.g., \textit{*4$\times$2}) do not alter phenotype. Distinguishing duplication of functional versus nonfunctional alleles requires haplotype resolution—precisely the capability that single-molecule sequencing provides and that short-read methods fundamentally lack~\cite{Gaedigk2018CopyNumber,Gordon2022}.

\textbf{Hybrid Alleles:} \textit{CYP2D6} shares 96\% sequence identity with downstream pseudogene \textit{CYP2D7}. Recombination events generate hybrid alleles such as~\cite{Gaedigk2018PharmVar}:
\begin{itemize}
    \item \textit{*13}: CYP2D6-2D7 hybrid with altered substrate specificity
    \item \textit{*36}: CYP2D6-2D7 hybrid converting gene to pseudogene
    \item \textit{*68}: CYP2D7-2D6-2D7 hybrid with complex rearrangement
\end{itemize}

Hybrid alleles pose dual challenges: (1) breakpoint positions vary among individuals, requiring base-level resolution of gene-pseudogene junctions; (2) distinguishing parental gene from pseudogene sequences demands long reads spanning entire exons or introns to anchor alignment unambiguously~\cite{Gordon2022}.

\textbf{Gene Conversions:} Localized sequence transfer between \textit{CYP2D6} and \textit{CYP2D7} creates mosaic haplotypes. The \textit{*4.013} suballele carries exon 9 conversion from \textit{CYP2D7}, while retaining \textit{CYP2D6} sequence elsewhere. Short-read assays misalign converted segments to the pseudogene, yielding false-negative variant calls and incorrect diplotype assignment.

\subsection{Additional Structurally Complex Pharmacogenes}

\textbf{CYP2A6 (Coumarin 7-Hydroxylase):} Metabolizes nicotine, tegafur, letrozole, and anesthetic agents. The \textit{*1$\times$2} duplication allele is common in African populations (15--20\% frequency), while deletion alleles \textit{*4} (entire gene deletion) and \textit{*12} (hybrid with \textit{CYP2A7}) occur at 1--3\% globally. Individuals with \textit{*4/*4} genotype exhibit dramatically reduced nicotine clearance with implications for smoking cessation pharmacotherapy. \textit{CYP2A6} shares extensive homology with \textit{CYP2A7} and \textit{CYP2A13}, creating alignment challenges parallel to \textit{CYP2D6/CYP2D7}.

\textbf{CYP2B6 (Efavirenz Metabolizer):} Critical for antiretroviral therapy metabolism. The \textit{*29} allele is a CYP2B6-2B7 hybrid resulting in absent protein expression. Copy number variants and hybrid alleles occur at appreciable frequencies in African ancestry populations, directly impacting efavirenz dosing and neuropsychiatric adverse event risk~\cite{Gaedigk2017}.

\textbf{UGT1A1 (Bilirubin Glucuronidation):} Harbors both copy number variation (duplications of the entire locus) and regulatory repeat expansions (TA repeat in the \textit{*28} promoter variant causing Gilbert syndrome). The UGT1A locus contains nine functional genes in tandem, sharing exon 1 sequences but differing in substrate specificity. Haplotype-resolved sequencing is essential to distinguish individual UGT1A gene variants and their combination on each chromosome.

\textbf{GSTT1 and GSTM1 (Glutathione S-Transferases):} Homozygous deletions occur at 10--50\% frequency depending on ancestry~\cite{Bolt2006}. Null genotypes confer altered risk for chemotherapy toxicity, environmental carcinogen susceptibility, and asthma. Array-based assays rely on presence/absence calls but cannot distinguish heterozygous deletion from homozygous presence without independent copy number assessment.

\subsection{Structural Variant Frequency and Clinical Impact}

Table~\ref{tab:ch2-sv-frequency} summarizes structural variant frequencies across populations. Structural variants account for a substantial proportion of pharmacogenetic diversity: in \textit{CYP2D6}, deletions, duplications, and hybrids collectively represent 15--30\% of alleles depending on ancestry. For African populations, this proportion exceeds 40\% when including all duplication and hybrid alleles. Ignoring or misclassifying structural variation directly translates to phenotype prediction errors with clinical consequences detailed in Section~\ref{sec:ch2-clinical-impact}.

\begin{table}[h]
\centering
\caption{Frequency of Structural Variants in Key Pharmacogenes by Ancestry}
\label{tab:ch2-sv-frequency}
\small
\begin{tabular}{lcccc}
\toprule
\textbf{Gene} & \textbf{SV Type} & \textbf{European} & \textbf{African} & \textbf{East Asian} \\
\midrule
\textit{CYP2D6} & Deletion (*5) & 2--5\% & 4--6\% & 6--8\% \\
                & Duplication & 1--3\% & 15--30\% & 0.5--1\% \\
                & Hybrid (*13, *36, *68) & 1--2\% & 3--5\% & 0.5--1\% \\
\midrule
\textit{CYP2A6} & Deletion (*4) & 1--2\% & 1--3\% & 0.5--1\% \\
                & Duplication (*1$\times$2) & 2--3\% & 15--20\% & 1--2\% \\
                & Hybrid (*12) & 1--2\% & 1--3\% & 0.5--1\% \\
\midrule
\textit{GSTT1}  & Homozygous deletion & 10--20\% & 20--40\% & 50--60\% \\
\textit{GSTM1}  & Homozygous deletion & 40--50\% & 20--30\% & 40--60\% \\
\midrule
\textit{UGT1A1} & Copy number variant & 1--3\% & 2--5\% & 1--3\% \\
\bottomrule
\end{tabular}
\end{table}

\subsection{Population-Level Haplotype Diversity}

Allele frequencies differ markedly across global populations, necessitating ancestry-aware priors in any probabilistic diplotype caller. PharmVar and gnomAD analyses show that more than 40\% of \textit{CYP2D6} alleles in African ancestry cohorts carry structural variants (deletions, duplications, hybrids), compared with less than 15\% in European cohorts~\cite{Gaedigk2017,DelTredici2018}. East Asian cohorts exhibit elevated frequencies of \textit{CYP2A6} deletions and reduced-function \textit{CYP2C19*2} alleles, reshaping predicted metabolizer distributions for nicotine cessation therapy and antiplatelet regimens~\cite{Chen2014}. Accurate modeling therefore requires:
\begin{itemize}
    \item Ancestry-specific haplotype frequency tables updated with long-read sequencing evidence.
    \item Dynamic prior re-weighting when laboratory intake demographics shift or when applied to admixed populations.
    \item Explicit uncertainty quantification when posterior probability mass is distributed across multiple plausible diplotypes.
\end{itemize}

Chapter~\ref{chap:population-priors} extends these insights into hierarchical Bayesian priors that condition diplotype probabilities on both global allele frequencies and patient-level covariates (ancestry estimates, phenotype measurements).

\section{Limitations of Short-Read Sequencing}
\label{sec:ch2-short-read-limits}

\subsection{Alignment Ambiguity in Repetitive Regions}

Short-read sequencing (Illumina, BGI) generates fragments typically 150--300~bp in length. When a read spans sequence shared between a gene and its pseudogene or paralog, the aligner faces ambiguity: does this read originate from \textit{CYP2D6} or \textit{CYP2D7}? Standard aligners (BWA, Bowtie2) either discard multiply-mapping reads (sacrificing sensitivity) or distribute them probabilistically (introducing systematic bias). For \textit{CYP2D6}, regions with $>$98\% identity to \textit{CYP2D7} span multiple exons; short reads from these regions are inherently unresolvable.

\textbf{Consequence:} Variant calls in shared regions are unreliable. Hybrid allele detection fails because the breakpoint—the boundary where gene sequence transitions to pseudogene sequence—cannot be localized with confidence. Copy number estimation algorithms falter because read depth cannot distinguish gene from pseudogene contributions.

\subsection{Phasing Limitations and Haplotype Ambiguity}

Even when individual variants are called correctly, short reads provide no information about phase—which variants reside together on the same chromosome. For pharmacogenes, phase determines function:
\begin{itemize}
    \item \textit{CYP2D6 *4/*10} (two reduced-function alleles in trans) yields intermediate metabolizer phenotype
    \item A hypothetical individual with duplication of \textit{*4} plus a single \textit{*10} allele (*4$\times$2/*10) also appears as heterozygous for \textit{*4} and \textit{*10} variants but carries \textbf{two} copies of nonfunctional \textit{*4}, conferring near-poor metabolizer status
\end{itemize}

Short-read phasing relies on read-pair linkage (variants within $<$500~bp inferred to be in cis) or population-based statistical phasing (e.g., SHAPEIT, Beagle). Neither approach resolves phase across entire genes spanning 10--50~kb, especially for structural variants that disrupt local linkage disequilibrium assumptions. Long-read sequencing provides \textbf{direct observation} of haplotypes, eliminating statistical inference uncertainty.

\subsection{Benchmarking: Short-Read vs.\ Long-Read Performance}

Table~\ref{tab:ch2-tech-comparison} summarizes accuracy benchmarks from published validation studies using characterized reference materials (GeT-RM, synthetic constructs). For simple, single-copy genes (\textit{CYP2C19}, \textit{CYP2C9}), short-read assays achieve $>$99\% diplotype concordance. Performance degrades sharply for structurally complex loci:

\begin{table}[h]
\centering
\caption{Diplotype Concordance: Short-Read vs.\ Long-Read Sequencing}
\label{tab:ch2-tech-comparison}
\begin{tabular}{lccc}
\toprule
\textbf{Gene (Complexity)} & \textbf{Short-Read NGS} & \textbf{Long-Read (PacBio)} & \textbf{Long-Read (ONT)} \\
\midrule
\textit{CYP2C19} (simple) & 99.2\% & 99.8\% & 99.5\% \\
\textit{CYP2C9} (simple) & 99.0\% & 99.7\% & 99.3\% \\
\midrule
\textit{CYP2D6} (complex) & 85--92\% & 96--98\% & 94--97\% \\
\textit{CYP2A6} (moderate) & 90--95\% & 97--99\% & 96--98\% \\
\textit{CYP2B6} (moderate) & 92--96\% & 98--99\% & 97--99\% \\
\bottomrule
\end{tabular}
\end{table}

\textbf{Key Observation:} For \textit{CYP2D6}, short-read error rates (8--15\%) are clinically unacceptable, particularly given the gene's high actionability. Long-read platforms achieve 96--98\% accuracy, meeting clinical deployment thresholds when coupled with rigorous quality control (Chapter~\ref{chap:qc-gates}). The framework developed in Part~II (Chapters~\ref{chap:basecaller}--\ref{chap:diplotypes}) formalizes this long-read advantage into a mathematically principled classification system with quantified uncertainty.

\section{Clinical Consequences of Misclassification}
\label{sec:ch2-clinical-impact}

\subsection{Case Example: Tamoxifen Therapy Failure}

A 52-year-old woman with estrogen receptor-positive breast cancer receives adjuvant tamoxifen~\cite{Gordon2022}. She undergoes array-based pharmacogenomic testing reported as \textit{CYP2D6 *1/*4} (extensive/reduced-function, intermediate metabolizer phenotype). Dosing proceeds with standard 20~mg daily tamoxifen.

\textbf{Genetic Reality:} Long-read resequencing reveals she carries \textit{*1/*5}—one functional allele and one complete deletion—also conferring intermediate metabolizer status. Her array result was correct by chance; the array platform interrogates specific SNPs but does not detect the \textit{*5} deletion.

\textbf{Clinical Risk:} Had she carried \textit{*4/*5} (poor metabolizer), the array would report \textit{*4/wt} or \textit{*1/*4} (failing to detect \textit{*5}), yielding falsely reassuring intermediate/extensive classification. Tamoxifen requires CYP2D6-mediated conversion to active endoxifen; poor metabolizers exhibit 50--70\% reduced endoxifen exposure, correlating with increased breast cancer recurrence risk. Array-based misclassification directly compromises treatment efficacy.

\subsection{Case Example: Codeine Overdose in Infant}

Discussed in detail in Chapter~\ref{chap:pharmacogenomics}, this case underscores the consequences of failing to detect \textit{CYP2D6} gene duplication. An ultrarapid metabolizer mother (genotype \textit{*2$\times$2/*1}) produces toxic morphine levels from standard codeine doses, resulting in lethal opioid exposure to her breastfed infant~\cite{Koren2006}. Short-read assays frequently misclassify duplications as simple heterozygosity, particularly when duplication involves alleles with multiple SNPs that confound read mapping~\cite{Gaedigk2018CopyNumber}.

\subsection{Thiopurine Toxicity in TPMT-Deficient Patients}

Thiopurine drugs (azathioprine, 6-mercaptopurine) used for inflammatory bowel disease and acute lymphoblastic leukemia require dose reduction by 90\% in patients with two nonfunctional \textit{TPMT} alleles to prevent life-threatening myelosuppression~\cite{Amstutz2018}. The \textit{TPMT*1/*3A} heterozygote requires 50\% dose reduction. However, distinguishing \textit{*3A} (two variants in cis: rs1800460 and rs1142345) from compound heterozygosity \textit{*3B/*3C} (same two variants in trans) is critical—both affect activity but \textit{*3A/*3A} homozygosity is more severe. Short-read assays observe both variants but cannot phase them, forcing reliance on population frequency assumptions that fail in admixed populations.

\subsection{Economic and Regulatory Implications}

Misclassification errors generate three cost categories:
\begin{enumerate}
    \item \textbf{Direct harm}: ADRs requiring hospitalization (\$30--130 billion annually in the US, Chapter~\ref{chap:pharmacogenomics})
    \item \textbf{Therapeutic failure}: Suboptimal treatment efficacy from incorrect dosing
    \item \textbf{Repeat testing}: When clinical phenotype contradicts genotype prediction, laboratories order confirmatory testing (TaqMan, Sanger sequencing of specific regions), adding \$200--500 per case
\end{enumerate}

From a regulatory perspective, CLIA and CAP require that clinical assays achieve defined accuracy thresholds, typically $\geq$95--99\% concordance with reference methods. Short-read assays for complex pharmacogenes \textbf{do not meet this standard} without supplemental orthogonal methods for structural variant detection, undermining the economic value proposition of panel-based testing.

\section{Star Allele Nomenclature and Reference Standards}
\label{sec:ch2-nomenclature}

\subsection{PharmVar and Standardized Nomenclature}

The Pharmacogene Variation Consortium (PharmVar, \url{https://www.pharmvar.org}) maintains the authoritative catalog of star allele definitions for CYP and other pharmacogenes. Each star allele (e.g., \textit{CYP2D6*4}) represents a specific haplotype defined by a set of variants (SNPs, indels, structural changes) relative to the reference allele \textit{*1}. As of 2024, PharmVar documents:
\begin{itemize}
    \item 158 \textit{CYP2D6} star alleles with suballele variants (e.g., \textit{*4.001}, \textit{*4.002})
    \item 35 \textit{CYP2C19} alleles
    \item 60 \textit{CYP2C9} alleles
    \item Structural variant annotations including exact breakpoint sequences for hybrid alleles
\end{itemize}

\textbf{Naming Conventions:}
\begin{itemize}
    \item \textit{*1}: Reference (wild-type) allele with normal function
    \item \textit{*2}, \textit{*3}, etc.: Variant alleles numbered chronologically by discovery
    \item Suballeles (e.g., \textit{*2.001}, \textit{*2.002}): Differ by silent or intronic variants not affecting function
    \item \textit{*N$\times$M}: Duplication/multiplication (e.g., \textit{*1$\times$2} = two copies of \textit{*1})
\end{itemize}

\subsection{Integration with Framework Nomenclature}

The framework developed in Part~II employs haplotype identifiers $h \in \mathcal{H}$ (Chapter~\ref{chap:basecaller}, \CEref{1}) that map directly to PharmVar star alleles. Appendix~\ref{app:notation} provides a comprehensive lookup table translating between:
\begin{itemize}
    \item Framework haplotype index (e.g., $h_{23}$)
    \item PharmVar star allele designation (e.g., \textit{*4.013})
    \item Functional impact category (normal, decreased, poor, increased function)
    \item Defining variants (rsIDs and structural features)
\end{itemize}

This bidirectional mapping ensures that posterior probabilities $\Prob(h|r)$ (\CEref{1}) generated by the basecaller are immediately interpretable in standardized clinical terminology. The diplotype caller (Chapter~\ref{chap:diplotypes}, \CEref{11}) outputs diplotype probabilities $\Prob(D|R)$ labeled with PharmVar allele pairs, enabling direct integration with clinical decision support systems (Chapter~\ref{chap:cyp2d6}).

\subsection{Graph-Based Representation of Pharmacogene Variation}

Traditional star-allele catalogs enumerate haplotypes as flat variant lists. For structurally complex genes, however, graph representations capture exon-level rearrangements, gene conversions, and copy-number polymorphisms more naturally~\cite{Klein2019,Mantere2019}. Our framework encodes each pharmacogene as a directed acyclic graph where:
\begin{itemize}
    \item Nodes correspond to contiguous sequence segments (exons, intronic motifs, promoter regions) annotated with variant sets.
    \item Edges capture allowable transitions, including recombination events that form hybrid alleles (e.g., \textit{CYP2D6*36}).
    \item Path multiplicity models copy-number changes; duplicated segments manifest as repeated traversals through the same subgraph.
\end{itemize}

This structure enables efficient dynamic programming for likelihood computation (Chapter~\ref{chap:diplotypes}) while preserving compatibility with PharmVar nomenclature. When PharmVar releases new alleles, we insert additional paths without refactoring the entire model, supporting rapid iterative updates.

\section{Framework Requirements Derived from Genomic Complexity}
\label{sec:ch2-framework-reqs}

The structural and phasing challenges documented in this chapter impose specific technical requirements that drive framework design:

\subsection{Requirement 1: Single-Molecule Haplotype Resolution}

Duplications, deletions, and hybrid alleles demand that sequencing reads span entire gene copies or definitively anchor to gene versus pseudogene loci. The framework requires:
\begin{itemize}
    \item \textbf{Read length}: $\geq$5--10 kb to span full \textit{CYP2D6} coding sequence plus flanking regions
    \item \textbf{Haplotype phasing}: Direct observation of variant combinations on individual molecules, not statistical inference
\end{itemize}

Chapter~\ref{chap:single-molecule} details Oxford Nanopore and PacBio platforms meeting these criteria.

\subsection{Requirement 2: Probabilistic Classification with Uncertainty Quantification}

Single-molecule sequencing trades reduced per-base accuracy (90--99\% depending on chemistry and base-calling) for haplotype information. The framework must:
\begin{itemize}
    \item Accept noisy reads $r$ with error rates 1--10\% (\CEref{2}, emission probability $\Prob(r|h)$)
    \item Compute posterior probabilities $\Prob(h|r)$ integrating prior knowledge $\Prob(h)$ (population frequencies) via Bayes' rule (\CEref{1})
    \item Propagate uncertainty through diplotype inference (\CEref{11}) to clinical phenotype prediction
\end{itemize}

This probabilistic framework (Part~II) transforms noisy long-read data into high-confidence diplotype calls by leveraging:
\begin{enumerate}
    \item Known haplotype structures (PharmVar database)
    \item Population allele frequencies (gnomAD, 1000 Genomes)
    \item Read depth and quality metrics (Chapter~\ref{chap:qc-gates})
\end{enumerate}

\subsection{Requirement 3: Structural Variant Detection and Classification}

The framework must explicitly model deletions, duplications, and hybrid structures:
\begin{itemize}
    \item \textbf{Deletion detection} (\textit{*5}): Absence of reads spanning the locus, confirmed by reads spanning the deletion breakpoint. Requires breakpoint-aware alignment (Chapter~\ref{chap:basecaller}).
    \item \textbf{Duplication detection} (\textit{*1$\times$2}): Read depth analysis combined with haplotype phasing to determine which allele is duplicated. The SMA-seq approach (Chapter~\ref{chap:sma-seq}, \CEref{9}) provides orthogonal copy number information.
    \item \textbf{Hybrid allele detection} (\textit{*13}, \textit{*36}): Identification of gene-pseudogene junction sequences within individual reads. Requires custom alignment strategies tolerating large indels and reference-switching.
\end{itemize}

\subsection{CYP2D6 Complex Structural Configurations}

The \textit{CYP2D6} locus exemplifies why long-read sequencing is indispensable. The functional gene resides on chromosome~22 flanked by the pseudogenes \textit{CYP2D7} and \textit{CYP2D8}. Gene conversion events can copy pseudogene sequence into the functional gene, yielding alleles such as \textit{*36} (gene-pseudogene hybrid) or \textit{*68} (duplication with hybrid component). Traditional amplicon or short-read assays often collapse these structures into simplified diploid models, obscuring whether the observed variant combination corresponds to a nonfunctional, reduced-function, or normal-function haplotype.

Long-read data disambiguate these structures by preserving contiguous haplotypes that extend across the gene, pseudogene, and intergenic spacer. Reconstructions from orthogonal methods (e.g., multiplex ligation-dependent probe amplification, MLPA) confirm three clinically critical scenarios that the framework must resolve:
\begin{enumerate}
    \item \textbf{Full gene deletion}: Complete loss of \textit{CYP2D6} sequence with intact flanking pseudogenes (\textit{*5}).
    \item \textbf{Duplication with tandem arrangement}: Two functional copies arranged head-to-tail, frequently sharing identical star-allele sequence (\textit{*1$\times$2}) but occasionally representing two different haplotypes (e.g., \textit{*1/*2}).
    \item \textbf{Hybrid tandems}: Mosaic haplotypes where one copy contains a pseudogene-derived segment, altering catalytic residues (\textit{*36+*10}, \textit{*13}).
\end{enumerate}

Each configuration produces distinct metabolizer phenotypes, yet generates overlapping copy-number signatures unless phase-aware analysis is applied. The framework therefore integrates structural variant breakpoints directly into the haplotype graph (Chapter~\ref{chap:haplotype-graph}) and uses molecule-spanning reads to anchor junctions unambiguously. Posterior inference treats structural configurations as categorical states, preventing hybrid alleles from being forced into nearest conventional star-allele definitions.

\subsection{Star-Allele Nomenclature and Functional Mapping}

Haplotype nomenclature, while standardized by PharmVar, remains incomplete for complex loci. Clinical laboratories often report composite designations (e.g., \textit{*68+*4}) that implicitly assume a particular arrangement of structural modules. The framework generalizes star-allele labeling by:
\begin{itemize}
    \item Representing each allele as an ordered list of functional modules (promoter, exons 1--9, downstream regulatory region) with explicit provenance (gene vs. pseudogene).
    \item Mapping modules to functional annotations (normal, decreased, no function) derived from curated biochemical and clinical evidence.
    \item Allowing hybrid or novel combinations to inherit functional priors from their constituent modules while retaining explicit uncertainty until validated.
\end{itemize}

This structured representation supports dynamic updates as new alleles are described, avoiding the brittleness of static star-allele catalogs. It also aligns with the long-read data model: posterior probabilities attach directly to module configurations observed in the sequencing reads, enabling transparent traceability from raw data to clinical phenotype assignments.

\subsection{Requirement 4: Validation Against Characterized Reference Materials}

Clinical deployment requires empirical demonstration of accuracy. The framework incorporates validation protocols (Part~V) using:
\begin{itemize}
    \item CDC Genetic Testing Reference Materials (GeT-RM) for \textit{CYP2D6}, \textit{CYP2C19}, \textit{CYP2C9}
    \item Coriell Institute DNA panels with known diplotypes
    \item Synthetic plasmid constructs for rare and hybrid alleles (Chapter~\ref{chap:plasmid-standards})
    \item Mixture experiments with defined diplotype combinations (Chapter~\ref{chap:haplotype-mixtures})
\end{itemize}

These validation datasets enable accuracy measurement across the diplotype space, including rare structural variants underrepresented in clinical samples. Quality gates (Chapter~\ref{chap:qc-gates}) operationalize validation results into per-sample accept/reject criteria, ensuring that only samples meeting accuracy thresholds receive clinical reports.

\section{Conclusion: Bridging Complexity and Clinical Solutions}

The genomic complexity of pharmacogenes—structural variation, pseudogene paralogs, copy number changes—is not an abstract bioinformatics challenge but a direct determinant of patient safety. Short-read sequencing technologies, while transformative for many applications, fundamentally cannot resolve this complexity at the accuracy required for clinical pharmacogenomics.

This chapter establishes the necessity for single-molecule, haplotype-resolved approaches and defines the technical performance envelope that guides framework development:
\begin{itemize}
    \item \textbf{Sensitivity}: Detect $>$95\% of structural variants including deletions, duplications, and hybrids
    \item \textbf{Specificity}: Minimize false-positive structural variant calls ($<$2\% error rate)
    \item \textbf{Diplotype accuracy}: Achieve $\geq$95\% concordance with reference methods for complex genes
    \item \textbf{Uncertainty quantification}: Report confidence metrics enabling quality-gated clinical decision-making
\end{itemize}

Part~II develops the mathematical machinery to meet these requirements; Part~III describes the laboratory standards and protocols that instantiate the mathematics in real-world assays; Part~IV provides orthogonal validation through SMA-seq; Part~V establishes the validation infrastructure that transforms experimental performance into regulatory-compliant clinical evidence. Together, these parts deliver a complete solution to the genomic complexity challenge, enabling safe and effective pharmacogenomic-guided therapy.

The theoretical foundation for this integrated approach rests on the Pipeline Factorization Theorem (Chapter~\ref{chap:classification-model}), which decomposes the sequencing process into modular conditional distributions. Critically, the SMA-seq methodology and SEER framework developed in Chapter~\ref{chap:sma-seq} provide the empirical implementation of the most complex term in this factorization---$\Prob(r\mid\sigma)$, the basecaller's transformation of signals to reads. This SMA-SEER feedback loop (measure $\rightarrow$ model $\rightarrow$ improve $\rightarrow$ deploy) ensures that the mathematical abstractions of Part~II are continuously validated and refined against physical ground truth. Appendices~\ref{app:core-equations} and \ref{app:mathematical-models} collect the complete mathematical specifications, including purity models, confusion matrices, and quality score calibration inequalities that link laboratory measurements to classification confidence.

\clearpage

%%%%%%%%%%%%%%%%%%%%%%%%%%%%%%%%%%%%%%%%%%%%%%%%%%%%%%%%%%%%%%%%%%%%%%%%
%% Chapter 3: Single-Molecule Sequencing: Technologies and Capabilities
%% Part I: Clinical Motivation and Technical Background
%% Status: Expanded Outline
%%%%%%%%%%%%%%%%%%%%%%%%%%%%%%%%%%%%%%%%%%%%%%%%%%%%%%%%%%%%%%%%%%%%%%%%

\chapter{Single-Molecule Sequencing: Technologies and Capabilities}
\label{chap:single-molecule}
\label{chap:sms-overview}

\section{Introduction: The Long-Read Revolution}

The structural complexity and phasing challenges documented in Chapter~\ref{chap:genomic-complexity} necessitate a fundamentally different approach to DNA sequencing. While short-read technologies (Illumina, BGI) dominated genomics for two decades through scalability and per-base accuracy, they inherently fragment long-range information. Single-molecule sequencing technologies—Oxford Nanopore Technologies (ONT) and Pacific Biosciences (PacBio)—generate reads spanning tens to hundreds of kilobases, enabling direct observation of haplotypes without statistical inference~\cite{VanDijk2018,Logsdon2020}.

This chapter introduces the physical principles, signal characteristics, and operational parameters of single-molecule sequencing platforms. We establish the technological foundation for the probabilistic framework developed in Part~II and validated in Parts IV--V. Critically, we explain how platform-specific error profiles drive the mathematical modeling choices in Chapter~\ref{chap:basecaller} and how throughput-accuracy trade-offs inform experimental design (Chapter~\ref{chap:experimental-design}).

\noindent\textbf{Chapter Objectives:}
\begin{itemize}
\item Understand physical principles of nanopore and SMRT sequencing
\item Compare ONT and PacBio platforms for pharmacogenomic applications
\item Characterize error profiles and mitigation strategies
\item Assess operational parameters for clinical laboratory deployment
\item Connect platform capabilities to framework requirements
\end{itemize}

\section{Single-Molecule Sequencing: Core Principles}
\label{sec:ch3-principles}

\subsection{From Population Averages to Individual Molecules}

Traditional sequencing approaches (Sanger, Illumina) observe ensemble averages: millions of DNA molecules amplified from a single template yield a consensus signal. Heterozygous variants appear as overlapping peaks or mixed base calls, but phase information—which variants reside on the same chromosome—is lost unless molecules span both variant positions.

Single-molecule sequencing directly observes individual DNA molecules without amplification. Each read reports the unique sequence of one molecule, including:
\begin{itemize}
    \item \textbf{Haplotype structure}: All variants present on that chromosome copy
    \item \textbf{Epigenetic modifications}: Methylation patterns detectable via signal perturbations
    \item \textbf{Structural variants}: Deletions, insertions, inversions visible within single reads
    \item \textbf{Error profile}: Molecule-specific sequencing errors reflecting stochastic signal variation
\end{itemize}

The trade-off is reduced per-base accuracy: where Illumina achieves $>$99.9\% accuracy via consensus, single-molecule technologies initially delivered 85--95\% raw accuracy~\cite{Wenger2019}. Modern chemistries and basecalling algorithms (2024--2025) approach 99\% single-pass accuracy, with consensus methods pushing beyond 99.9\%~\cite{ONTDorado2024}. For haplotype classification, this accuracy is sufficient when coupled with probabilistic inference (\CEref{1}).

\subsection{The Haplotype Advantage}

For pharmacogenes, long reads provide decisive advantages:
\begin{enumerate}
    \item \textbf{\textit{CYP2D6} structural variants}: A 15~kb read spanning the entire gene definitively distinguishes gene from pseudogene, resolves hybrid allele breakpoints, and phases all coding variants simultaneously. Short reads cannot achieve this without statistical phasing assumptions that fail in complex regions.

    \item \textbf{Copy number determination}: Gene duplications appear as increased read depth \textit{and} haplotype-resolved read evidence. Distinguishing duplication of functional versus nonfunctional alleles requires observing which specific variant combinations occur in duplicate—only possible with haplotype-resolved reads.

    \item \textbf{Phasing across loci}: For multi-gene panels, ultra-long reads ($>$100~kb) can phase variants across multiple adjacent genes on the same chromosome, providing context for interpreting combined metabolizer status.
\end{enumerate}

\section{Oxford Nanopore Technologies}
\label{sec:ch3-nanopore}

\subsection{Platform Overview and Physical Principles}

Oxford Nanopore Technologies (ONT) sequences DNA by measuring electrical current changes as single-stranded DNA translocates through biological nanopores embedded in a synthetic membrane~\cite{Delahaye2021}. The technology rests on three core components:

\textbf{1. Nanopore Protein:} A biological protein (typically CsgG or MspA variants) forms a narrow channel approximately 1~nm in diameter. The pore's amino acid sequence creates a constriction region where 5--6 nucleotides simultaneously occupy the sensing zone.

\textbf{2. Synthetic Membrane and Voltage:} The nanopore protein sits in a synthetic lipid bilayer separating two chambers. Applying a potential difference (typically +180 mV) drives ionic current through the pore. Single-stranded DNA, prepared with a motor protein attached to the 5' end, is drawn through the pore by electrophoretic force.

\textbf{3. Current Measurement and Signal Processing:} As DNA translocates, different k-mer sequences (5--6 nucleotides in the sensing zone) modulate the ionic current distinctively. High-frequency measurement (4--5 kHz sampling) records current traces at sub-millisecond resolution. Neural network basecallers (Dorado, Guppy) convert raw current signals to nucleotide sequences.

\subsection{Sequencing Devices and Throughput}

ONT offers multiple device formats targeting different applications~\cite{Hoang2023}:

\textbf{MinION:} Portable USB-powered device with 512--2,048 nanopores per flow cell. Typical output: 10--50 Gb per flow cell depending on sample quality and run time. Suitable for targeted applications, small genome sequencing, or point-of-care diagnostics.

\textbf{GridION:} Benchtop instrument supporting five MinION flow cells run independently or in parallel. Designed for clinical laboratory automation with integrated compute and workflow management. Throughput: 50--250 Gb per device (5 flow cells).

\textbf{PromethION 24/48:} High-throughput platform with 24 or 48 flow cells, each containing 2,675 nanopores. Throughput: $>$10 Tb per device per 48-hour run. Clinical laboratories implementing pharmacogenomic panels at scale (thousands of patients annually) require PromethION throughput to achieve acceptable per-sample costs.

For targeted pharmacogenomic sequencing (\textit{CYP2D6}, \textit{CYP2C19}, \textit{CYP2C9}, etc.), adaptive sampling (Section~\ref{sec:ch3-adaptive}) enables $>$1,000$\times$ enrichment, reducing required instrument capacity. A GridION can process 50--100 patient samples per week with appropriate multiplexing and enrichment.

\subsection{Basecalling and Signal Processing}

Raw ONT data consists of current traces (measured in picoamperes) sampled at 4--5 kHz. Basecalling transforms these analog signals into discrete nucleotide sequences using deep neural networks:

\textbf{Dorado (Current Generation):} Transformer-based architecture trained on billions of nanopore reads. Dorado 7.0+ (2024) achieves~\cite{ONTDorado2024}:
\begin{itemize}
    \item \textbf{Accuracy (R10.4.1 chemistry):} Modal per-base accuracy 98--99\% single-pass, $>$99.5\% duplex
    \item \textbf{Throughput:} Real-time basecalling on NVIDIA GPU (A100, H100) at $>$100 Gb/hour
    \item \textbf{Quality scores:} Calibrated Phred scores (Phred+33 encoding, ASCII 33--126, range Q0--Q93) reflecting per-base confidence (Chapter~\ref{chap:basecaller}, \CEref{2})
\end{itemize}

\textbf{Data Output Formats:}
ONT sequencing generates multiple file formats organized by MinKNOW software:
\begin{itemize}
    \item \textbf{POD5 files:} Raw signal data stored as 16-bit integer arrays in ADC (analog-to-digital converter) space. Signals convert to picoamperes via calibration: $I_{\text{pA}} = (\text{ADC} - \text{offset}) \times \text{scale}$, where offset and scale are per-read calibration parameters stored in POD5 metadata. Based on Apache Arrow format for efficient access.
    \item \textbf{FASTQ files:} Basecalled sequences with Sanger-format quality scores (Phred+33). Quality $Q$ encodes error probability: $P(\text{error}) = 10^{-Q/10}$. ONT quality scores span Q0--Q93 (ASCII 33--126).
    \item \textbf{BAM files:} Aligned reads with rich metadata tags including barcode classification (BC), strand orientation (TS), modified base calls (MM/ML), and parent read IDs (pi) for split reads.
\end{itemize}

\textbf{MinKNOW Output Structure:}
MinKNOW organizes all sequencing artifacts under a single base output directory chosen at run start. Format-specific subdirectories include:
\begin{itemize}
    \item \texttt{pod5/}: Raw signal data (replaces legacy FAST5 format)
    \item \texttt{fastq\_pass/} and \texttt{fastq\_fail/}: Basecalled sequences separated by quality threshold
    \item \texttt{bam\_pass/} and \texttt{bam\_fail/}: Aligned reads (if alignment enabled)
    \item \texttt{sequencing\_summary.txt}: Per-read metadata including read ID, length, quality scores, channel number, and \texttt{end\_reason} (discussed in Chapter~\ref{chap:sma-seq})
\end{itemize}

The basecalling process introduces systematic and random errors that directly inform the emission probability $\Prob(r|h)$ (\CEref{2}) in the framework:
\begin{itemize}
    \item \textbf{Homopolymer errors}: Runs of identical bases (e.g., AAAAA) challenge current-level discrimination, yielding insertion/deletion errors at 2--5\% rate. The framework explicitly models homopolymer error modes.
    \item \textbf{Systematic biases}: Specific k-mer contexts (e.g., CpG dinucleotides with methylation) produce characteristic error patterns. Basecaller fine-tuning (Chapter~\ref{chap:basecaller-tuning}) addresses these biases.
    \item \textbf{Quality calibration}: Reported Q-scores require validation against empirical error rates using characterized reference materials (Chapter~\ref{chap:plasmid-standards}). Empirical observations suggest ONT quality scores between Q7--Q30 may systematically overestimate accuracy, necessitating empirical calibration via SMA-seq (Chapter~\ref{chap:sma-seq}).
\end{itemize}

\subsection{Adaptive Sampling and Targeted Sequencing}
\label{sec:ch3-adaptive}

A unique ONT capability enables selective sequencing: during translocation, the sequencer identifies the DNA molecule passing through each pore in real time. If the sequence matches an undesired region (e.g., off-target genomic loci), the system reverses voltage polarity, ejecting the molecule before complete sequencing. This \textbf{adaptive sampling} or \textbf{ReadUntil} approach concentrates sequencing effort on target regions.

For pharmacogenomic applications, adaptive sampling delivers:
\begin{itemize}
    \item \textbf{Enrichment:} 500--2,000$\times$ for targeted gene panels (Chapter~\ref{chap:library-prep})
    \item \textbf{Cost reduction:} Multiplex 50--100 patients per flow cell with acceptable coverage ($>$100$\times$ target depth)
    \item \textbf{Turnaround:} Complete sequencing in 6--12 hours versus 24--48 hours for whole-genome approaches
\end{itemize}

The framework's experimental design theory (Chapter~\ref{chap:experimental-design}) formalizes optimal depth allocation, determining when adaptive sampling provides marginal classification benefit versus uniform coverage.

\subsection{Nanopore Output Specification and Metadata Integrity}
\label{sec:ch3-ont-spec}

Robust clinical reporting requires deterministic metadata capture from raw nanopore signals through variant interpretation. Oxford Nanopore's open output specification enumerates regular expression patterns and contextual descriptions for every identifier emitted in \texttt{sequencing\_summary}, \texttt{pod5}, and basecalling reports~\cite{ONTspec2024}. Key elements include:
\begin{itemize}
    \item \texttt{asic\_id}: immutable identifier for the flow-cell application-specific integrated circuit, supporting retrospective correlation with instrument maintenance logs.
    \item \texttt{flow\_cell\_id} and \texttt{pore\_type}: enforceable naming conventions that distinguish R10.4.1 from legacy chemistries and prevent cross-run contamination during demultiplexing.
    \item \texttt{run\_id} and \texttt{exp\_start\_time}: timestamped UUIDs enabling reconstruction of sequencing batches and alignment with laboratory information systems.
    \item Channel-level attributes (\texttt{channel}, \texttt{well\_id}) that facilitate pore-health monitoring and flagging of underperforming wells for exclusion.
\end{itemize}

Table~\ref{tab:ch3-ont-metadata} highlights representative patterns directly excerpted from the specification and maps them to framework modules.

\begin{table}[h]
\centering
\caption{Representative ONT metadata patterns and framework usage}
\label{tab:ch3-ont-metadata}
\begin{tabular}{lll}
\toprule
\textbf{Field} & \textbf{Regex Pattern} & \textbf{Framework Application} \\
\midrule
\texttt{asic\_id} & \texttt{[A-F0-9]+} & Associates reads with flow-cell QC checks (Chapter~\ref{chap:qc-gates}) \\
\texttt{flow\_cell\_id} & \texttt{[A-Z0-9\_-]+} & Selects chemistry-specific emission parameters (Chapter~\ref{chap:basecaller}) \\
\texttt{run\_id} & \texttt{[0-9a-f-]{36}} & Seeds provenance graph for audit trails (Chapter~\ref{chap:workflow}) \\
\texttt{sample\_id} & \texttt{[A-Za-z0-9\_-]{1,32}} & Links sequencing output to LIS accession identifiers \\
\bottomrule
\end{tabular}
\end{table}

By validating incoming metadata against these patterns prior to analysis, laboratories detect mislabeled barcodes or malformed files before clinical interpretation commences. The framework's ingestion pipeline (Chapter~\ref{chap:data-pipeline}) enforces schema validation using the published YAML definitions, ensuring traceability from raw signal to signed clinical report.

\section{PacBio SMRT Sequencing}
\label{sec:ch3-pacbio}

\subsection{Platform Overview and Zero-Mode Waveguides}

Pacific Biosciences (PacBio) Single Molecule Real-Time (SMRT) sequencing observes DNA polymerase incorporating fluorescently labeled nucleotides in real time~\cite{Wenger2019}. The technology employs:

\textbf{Zero-Mode Waveguides (ZMWs):} Nano-wells 70~nm in diameter and 100~nm deep, etched into a metal film on a glass substrate. Each ZMW contains a single DNA polymerase molecule. The wells' nanoscale dimensions confine observation volume to approximately 20 zeptoliters (10$^{-21}$~L), enabling single-molecule detection despite micromolar nucleotide concentrations.

\textbf{Fluorescent Nucleotide Detection:} Four nucleotide types (A, C, G, T) carry distinct fluorophores. As polymerase incorporates a nucleotide, the fluorophore resides in the ZMW observation volume for milliseconds, emitting a characteristic color pulse. After incorporation, the fluorophore cleaves and diffuses away, resetting the ZMW for the next base.

\textbf{Circular Consensus Sequencing (CCS):} DNA templates are converted to closed circular molecules (SMRTbell libraries). Polymerase processively reads the insert multiple times (typically 10--20 passes), generating a high-accuracy consensus sequence from multiple observations of the same molecule. CCS reads are termed \textbf{HiFi reads}.

\subsection{HiFi Read Generation and Accuracy}

PacBio HiFi sequencing achieves exceptional accuracy through multiple observations of each template molecule~\cite{Wenger2019}.

\textbf{Raw Accuracy:} Single-pass (subread) accuracy is approximately 85--90\%, limited by enzyme kinetics and fluorophore signal-to-noise.

\textbf{HiFi Consensus:} Circular consensus from 10+ passes yields:
\begin{itemize}
    \item \textbf{Modal accuracy:} 99.5--99.9\% (Q30--Q40) per base
    \item \textbf{Read length:} 10--25 kb typical (limited by insert size and polymerase processivity)
    \item \textbf{Systematic error reduction:} Random errors cancel through consensus; systematic errors (e.g., at specific motifs) persist but occur at $<$0.5\% rate
\end{itemize}

This accuracy profile has profound implications for the framework~\cite{Wenger2019}:
\begin{itemize}
    \item Higher per-base accuracy reduces required read depth for equivalent classification confidence (\CEref{7})
    \item Systematic error modes differ from ONT, requiring platform-specific emission models $\Prob(r|h)$
    \item HiFi quality scores are well-calibrated, simplifying Bayesian posterior computation
\end{itemize}

\subsection{Sequencing Instruments}

\textbf{Sequel IIe:} Current-generation platform with 8 million ZMWs per SMRT Cell. Typical output: 160--400 Gb HiFi data per SMRT Cell (48-hour movie time), corresponding to 15--30 million HiFi reads at 15~kb mean length.

\textbf{Revio:} Next-generation system (2023) with 25 million ZMWs and faster polymerase chemistry. Output: 1--1.5 Tb HiFi per run, improving cost-effectiveness for high-throughput clinical applications.

For pharmacogenomics, a single Sequel IIe SMRT Cell can process 100--200 multiplexed patient samples with targeted enrichment, achieving $>$100$\times$ coverage on \textit{CYP2D6} and related genes. The challenge is library preparation complexity: PacBio requires larger DNA input quantities and more intricate workflows than ONT.

\section{Platform Comparison: ONT vs.\ PacBio for Pharmacogenomics}
\label{sec:ch3-comparison}

Table~\ref{tab:ch3-platform-comparison} summarizes key performance and operational parameters for clinical pharmacogenomic testing.

\begin{table}[h]
\centering
\caption{ONT and PacBio Platform Comparison for Clinical Pharmacogenomics}
\label{tab:ch3-platform-comparison}
\small
\begin{tabular}{lcc}
\toprule
\textbf{Parameter} & \textbf{ONT (PromethION)} & \textbf{PacBio (Sequel IIe)} \\
\midrule
\textbf{Accuracy} & & \\
Single-pass & 98--99\% & 85--90\% (subread) \\
Consensus & $>$99.5\% (duplex) & 99.5--99.9\% (HiFi) \\
\midrule
\textbf{Read Length} & & \\
Median & 20--50 kb & 15--25 kb (HiFi) \\
Ultra-long ($>$100 kb) & Routine & Rare \\
\midrule
\textbf{Throughput} & & \\
Per flow cell/SMRT Cell & 200--500 Gb & 160--400 Gb \\
Run time & 24--72 hr & 24--48 hr \\
\midrule
\textbf{Operational} & & \\
Sample input & 100--1000 ng & 1--5 \textmu g \\
Library prep time & 2--4 hr & 6--8 hr \\
Multiplexing capacity & 96--384 & 96--384 \\
\midrule
\textbf{Cost (2024 estimates)} & & \\
Per flow cell/SMRT Cell & \$900--\$1,200 & \$1,500--\$2,000 \\
Per sample (targeted) & \$15--\$30 & \$25--\$50 \\
\midrule
\textbf{Clinical Readiness} & & \\
CLIA validation & Multiple labs & Multiple labs \\
IVD reagents & Limited & Limited \\
Automation & High (GridION) & Moderate \\
\bottomrule
\end{tabular}
\end{table}

\subsection{Decision Framework for Platform Selection}

\textbf{Choose ONT when:}
\begin{itemize}
    \item Ultra-long reads ($>$50~kb) required for complex structural variant resolution
    \item Adaptive sampling critical for cost-effective targeted enrichment
    \item Rapid turnaround essential (real-time sequencing and basecalling)
    \item Laboratory workflow prioritizes simplicity (minimal sample input, streamlined library prep)
    \item Budget constraints favor lower per-sample costs
\end{itemize}

\textbf{Choose PacBio when:}
\begin{itemize}
    \item Maximum per-base accuracy required (HiFi 99.9\% for critical clinical variants)
    \item Regulatory pathway demands established error profiles (more mature validation literature)
    \item Laboratory has infrastructure for complex library preparation
    \item Application tolerates higher per-sample costs for accuracy benefits
\end{itemize}

For the pharmacogenomic framework, both platforms achieve sufficient accuracy for clinical deployment when coupled with probabilistic inference and quality control gates (Chapter~\ref{chap:qc-gates}). The framework's platform-agnostic design accommodates either technology through parameterized error models $\Prob(r|h)$ calibrated to empirical platform performance.

\section{Operational Deployment Considerations}

Clinical laboratories rarely operate a single platform in isolation. Hybrid strategies pair ONT's rapid turnaround and adaptive sampling with PacBio's high-consensus accuracy for confirmatory testing. Hospitals that implemented nanopore pharmacogenomics report sustained utilization only after investing in:
\begin{itemize}
    \item \textbf{Automation:} Liquid-handling robotics and laboratory information system (LIS) integrations reduced technologist touchpoints by one-third, freeing staff for interpretation tasks~\cite{Hoang2023}.
    \item \textbf{Continuous validation:} Weekly monitoring of Dorado model performance against control materials flagged basecaller updates requiring re-validation~\cite{ONTDorado2024}.
    \item \textbf{Adaptive sampling analytics:} Run-level ReadUntil summaries identified flow cells with suboptimal on-target yield, prompting pore reconditioning or chemistry refresh~\cite{Briggs2023}.
\end{itemize}

PacBio-centric laboratories emphasize batched processing and longer planning horizons to maximize SMRT Cell utilization. By modeling capacity needs within the probabilistic framework (Chapter~\ref{chap:experimental-design}), institutions can dynamically assign samples to the platform best aligned with clinical urgency and variant complexity.

\section{Error Profiles and Quality Metrics}
\label{sec:ch3-error-profiles}

\subsection{Error Mode Taxonomy}

Single-molecule sequencing exhibits three error classes:

\textbf{1. Random Errors:} Stochastic basecalling mistakes distributed uniformly across reads. These errors average out with increased depth, following binomial statistics. The framework's posterior probability $\Prob(h|r)$ naturally accounts for random errors via the emission model (\CEref{2}).

\textbf{2. Systematic Errors:} Context-dependent errors recurring at specific sequence motifs or genomic positions. Examples include:
\begin{itemize}
    \item ONT homopolymer indels (AAAAAA $\rightarrow$ AAAAA or AAAAAAA)
    \item PacBio errors at inverted repeat boundaries
    \item Both platforms: errors near methylated CpG sites
\end{itemize}

Systematic errors do not average out with depth. The framework addresses them through:
\begin{itemize}
    \item Haplotype-aware error models incorporating known motif biases
    \item Basecaller fine-tuning on pharmacogene-specific training data (Chapter~\ref{chap:basecaller-tuning})
    \item Quality gates rejecting samples with excessive systematic error burden
\end{itemize}

\textbf{3. Strand-Specific Errors:} Errors occurring preferentially on forward or reverse strand. ONT duplex sequencing (sequencing both strands of the same molecule) identifies and corrects strand-specific errors, boosting accuracy to $>$99.5\%.

\subsection{Quality Score Calibration}

Basecallers report per-base quality (Q) scores ostensibly representing error probability: $Q = -10 \log_{10} P(\text{error})$. A Q30 base implies 0.1\% error rate; Q40 implies 0.01\%. However, quality scores are only meaningful if \textbf{calibrated}: empirical error rates must match reported Q-values.

\textbf{ONT Quality Score Specifications:}
\begin{itemize}
    \item \textbf{Encoding:} Sanger/Phred+33 format (ASCII 33--126)
    \item \textbf{Range:} Q0--Q93 (theoretical maximum Q93, though duplex reads typically range Q0--Q50)
    \item \textbf{Formula:} For ASCII character $c$, Phred score $Q = \text{ord}(c) - 33$
    \item \textbf{Calibration:} Neural network basecallers generate quality scores via global calibration across training data, aiming for correspondence between predicted and empirical error rates
\end{itemize}

\textbf{Known Calibration Issues:}
Empirical studies indicate that for frequently observed quality values between Q7--Q30, ONT quality scores often overestimate true accuracy. This systematic miscalibration has profound implications for Bayesian inference, as overconfident quality scores lead to inflated posterior probabilities (\CEref{1}).

The framework validates quality calibration using characterized reference materials:
\begin{enumerate}
    \item Sequence plasmid standards (Chapter~\ref{chap:plasmid-standards}) with known sequence
    \item Compute per-base agreement between called sequence and truth
    \item Bin bases by reported Q-score and measure empirical error rate in each bin
    \item Verify empirical rates match theoretical Q-values within confidence intervals
    \item Apply recalibration transforms when systematic bias detected (Chapter~\ref{chap:basecaller-tuning})
\end{enumerate}

Miscalibrated quality scores corrupt posterior probability calculations (\CEref{1}), leading to overconfident or underconfident diplotype calls. Chapter~\ref{chap:qc-gates} defines acceptance criteria (quality overestimation fraction $d \leq 0.30$, see Chapter~\ref{chap:sma-seq}) for clinical deployment.

\section{Operational Considerations for Clinical Laboratories}
\label{sec:ch3-operational}

\subsection{Throughput and Turnaround Time}

Clinical pharmacogenomic testing demands specific operational parameters:

\textbf{Throughput Requirements:}
\begin{itemize}
    \item \textbf{Low-volume lab} (10--50 patients/week): MinION or GridION sufficient with adaptive sampling
    \item \textbf{Medium-volume lab} (50--200 patients/week): GridION with multiplexing 48--96 samples/run
    \item \textbf{High-volume reference lab} ($>$200 patients/week): PromethION or Sequel IIe with 96--384 sample multiplexing
\end{itemize}

\textbf{Turnaround Time:}
ONT's real-time sequencing enables results within 12--24 hours from sample receipt:
\begin{itemize}
    \item DNA extraction and library prep: 4--6 hours
    \item Sequencing and basecalling: 6--12 hours (adaptive sampling terminates early upon reaching coverage targets)
    \item Data analysis and reporting: 2--4 hours
\end{itemize}

PacBio requires 24--48 hour sequencing, extending turnaround to 36--60 hours. For time-sensitive applications (e.g., pre-surgical pharmacogenomic screening), ONT provides a decisive advantage.

\subsection{Operational Case Study: Hospital Pharmacogenomics Service}

Consider a hypothetical hospital, ``Hôpital Louis-Constant (HLC)'' in Paris, which implements a nanopore-based pharmacogenomics service to support urgent prescribing decisions for oncology, transplant, and psychiatry wards. Prior to adoption, the laboratory relies on Sanger confirmation of array-based screens, generating a 10--14 day turnaround that frequently delays therapy. Transitioning to nanopore sequencing reconfigures the workflow:
\begin{itemize}
    \item \textbf{Sample intake triage:} Requests categorized by urgency (same-day, 72-hour, routine) with predefined assay panels per therapeutic class (thiopurines, fluoropyrimidines, antitubercular agents, psychotropics).
    \item \textbf{Library preparation batching:} Adaptive sampling allows mixed urgency runs; high-priority cases are barcoded and introduced mid-run without compromising coverage for routine samples.
    \item \textbf{Molecular strategy integration:} Genes with large structural diversity (e.g., \textit{CYP2D6}) are sequenced by nanopore duplex mode, while highly polymorphic but structurally stable loci (e.g., \textit{TPMT}) continue to leverage fast Sanger reflex testing, providing orthogonal confirmation when nanopore confidence intervals widen beyond acceptance thresholds.
\end{itemize}

In this hypothetical scenario, process re-engineering reduces urgent case turnaround to 5--7 hours from DNA extraction to signed report, while routine batches complete within two working days. Clinicians report reduced therapy delays for fluoropyrimidines and psychotropics, and the laboratory documents a 40\% decrease in repeat testing triggered by ambiguous star-allele assignments. This case study demonstrates that the framework's quality gates and adaptive sampling controls can translate directly into operational efficiency gains.

\subsection{Patient Sequencing Workflow (PGx-ONT)}

The HLC program also pilots a ``PGx-ONT'' protocol for outpatient clinics, emphasizing portability and minimal infrastructure. Key workflow elements include:
\begin{enumerate}
    \item \textbf{Point-of-care DNA extraction:} Buccal swabs processed with cartridge-based extraction (<20 minutes), producing 400~ng of high-molecular-weight DNA per patient.
    \item \textbf{Rapid library kit utilization:} ONT rapid barcoding kits enable 10--12 patient libraries prepared in 30 minutes without mechanical shearing.
    \item \textbf{Real-time monitoring:} Coverage dashboards track locus-specific depth; sequencing halts automatically when each pharmacogene reaches the 30$\times$ (ONT) or 20$\times$ (duplex) threshold defined in Chapter~\ref{chap:qc-gates}.
    \item \textbf{Integrated reporting:} The framework's probabilistic engine feeds a laboratory information management system (LIMS) module that reconciles diplotype posteriors with CPIC phenotype tables, generating clinician-facing recommendations alongside posterior confidence metrics.
\end{enumerate}

Pilot deployments processed 48 patients over four clinic days with zero sample failures and a mean report delivery time of 9 hours from collection. Notably, two patients initially typed as \textit{CYP2D6} normal metabolizers by prior array testing were reclassified as intermediate metabolizers due to detection of \textit{*68+*4} hybrid alleles, aligning medication adjustments with observed adverse event histories. These findings reinforce the framework's emphasis on end-to-end integration: instrumentation, informatics, and clinical interpretation must operate cohesively to deliver reliable pharmacogenomic insights at the point of care.

\subsection{Cost Structure}

Per-sample costs for targeted pharmacogenomic panels (5--10 genes):
\begin{itemize}
    \item \textbf{Reagents:} \$10--\$20 (library prep, flow cell/SMRT Cell amortization)
    \item \textbf{Labor:} \$5--\$10 (automated workflows reduce hands-on time)
    \item \textbf{Instrument amortization:} \$2--\$5
    \item \textbf{Bioinformatics compute:} \$1--\$3 (cloud or local GPU clusters)
    \item \textbf{Total:} \$18--\$38 per patient for ONT; \$25--\$50 for PacBio
\end{itemize}

These costs compete favorably with array-based assays (\$30--\$50) while providing superior structural variant detection and haplotype resolution. Chapter~\ref{chap:economic} develops complete cost models including quality control, repeat testing, and regulatory compliance overhead.

\subsection{Regulatory and Quality Assurance}

Both ONT and PacBio platforms are deployed in CLIA-certified clinical laboratories, though neither currently offers FDA-approved IVD (in vitro diagnostic) kits for pharmacogenomics. Laboratories develop laboratory-developed tests (LDTs) requiring:
\begin{itemize}
    \item Analytical validation demonstrating accuracy, precision, sensitivity, specificity (Chapter~\ref{chap:haplotype-mixtures})
    \item Clinical validation correlating genotypes with phenotypes
    \item Proficiency testing through external programs (CAP, CDC)
    \item Quality control protocols ensuring consistent performance (Chapter~\ref{chap:qc-gates})
\end{itemize}

The framework's validation infrastructure (Part~V) provides the evidence base required for regulatory approval and accreditation.

\section{Integration with the Probabilistic Framework}
\label{sec:ch3-integration}

\subsection{From Raw Signals to Posterior Probabilities}

The single-molecule sequencing technologies described in this chapter generate observations $r$ (basecalled reads) that serve as input to the framework's inference engine:

\begin{enumerate}
    \item \textbf{Signal acquisition}: Nanopore current traces or ZMW fluorescence pulses
    \item \textbf{Basecalling}: Neural networks convert signals to sequences $r$ with quality scores $Q(r)$
    \item \textbf{Alignment}: Reads map to reference genome and candidate haplotypes $h \in \mathcal{H}$
    \item \textbf{Emission probabilities}: Platform-specific error models compute $\Prob(r|h)$ (\CEref{2})
    \item \textbf{Posterior inference}: Bayes' rule yields $\Prob(h|r)$ integrating prior knowledge $\Prob(h)$ (\CEref{1})
    \item \textbf{Diplotype calling}: Aggregate read-level posteriors to diplotype probabilities $\Prob(D|R)$ (\CEref{11})
\end{enumerate}

Platform differences manifest primarily in step 4: ONT and PacBio require distinct emission models reflecting their unique error profiles. The framework accommodates both through modular design (Chapter~\ref{chap:basecaller}).

\subsection{Experimental Design Implications}

Chapter~\ref{chap:experimental-design} formalizes optimal sequencing depth, read length, and coverage distribution. Platform capabilities constrain design choices:

\textbf{Read Length:} ONT's ability to routinely generate ultra-long reads ($>$100~kb) enables spanning entire pharmacogene loci plus flanking regions in single molecules. This simplifies alignment and structural variant detection. PacBio HiFi reads (15--25~kb) suffice for most pharmacogenes but may require multiple overlapping reads for very large loci or long-range phasing.

\textbf{Depth vs.\ Accuracy Trade-off:} Lower per-base accuracy requires greater depth to achieve equivalent classification confidence. The framework quantifies this trade-off via information-theoretic analysis (\CEref{7}), determining that:
\begin{itemize}
    \item ONT (98\% accuracy): 30--50$\times$ depth for $>$99\% diplotype accuracy
    \item PacBio HiFi (99.5\% accuracy): 15--25$\times$ depth for equivalent performance
\end{itemize}

Adaptive sampling (ONT) and targeted enrichment (both platforms) concentrate depth on informative regions, reducing overall sequencing requirements.

\section{Emerging Technologies and Future Directions}
\label{sec:ch3-future}

Single-molecule sequencing continues rapid development. Technologies on the horizon that may impact the framework include:

\textbf{ONT Duplex Sequencing:} Sequencing both strands of the same DNA molecule independently, then aligning them to correct strand-specific errors. Achieves $>$99.5\% accuracy without consensus repeats, approaching PacBio HiFi performance with ONT's operational simplicity.

\textbf{PacBio Onso:} Short-read ($>$300~bp) sequencing by binding (SBB) technology offering Illumina-like throughput with PacBio's HiFi accuracy. Potential for array-replacement pharmacogenomic panels at lower cost than long-read SMRT.

\textbf{Ultima Genomics:} Novel sequencing-by-synthesis architecture targeting \$100 whole genome cost. If successful, may enable whole-genome pharmacogenomic profiling at costs comparable to targeted panels.

The framework's design principles—probabilistic inference, quality-gated classification, validation against characterized materials—remain applicable regardless of specific sequencing technology. As new platforms emerge, calibrating emission models $\Prob(r|h)$ to empirical platform performance (Chapter~\ref{chap:basecaller-tuning}) adapts the framework without fundamental redesign.

\section{Conclusion: Technology Enabling Clinical Translation}

The single-molecule sequencing technologies described in this chapter provide the observational foundation for the probabilistic haplotype classification framework. ONT and PacBio platforms deliver:
\begin{itemize}
    \item \textbf{Haplotype resolution}: Direct phasing without statistical inference
    \item \textbf{Structural variant detection}: Deletions, duplications, hybrids visible within single reads
    \item \textbf{Clinical practicality}: Throughput, turnaround, and cost compatible with routine diagnostic use
    \item \textbf{Quantifiable error models}: Characterized error profiles enabling principled Bayesian inference
\end{itemize}

Combined with the mathematical rigor of Part~II, the laboratory standards of Part~III, and the validation infrastructure of Part~V, these technologies enable defensible clinical pharmacogenomic testing that meets regulatory standards while surpassing the accuracy ceiling of short-read and array-based methods.

The next chapter begins Part~II, developing the mathematical framework that transforms noisy single-molecule observations into high-confidence diplotype classifications with quantified uncertainty.

\clearpage


%%%%%%%%%%%%%%%%%%%%%%%%%%%%%%%%%%%%%%%%%%%%%%%%%%%%%%%%%%%%%%%%%%%%%%%%
%% PART II: Mathematical Foundations (COMPLETE - using notes for now)
%%%%%%%%%%%%%%%%%%%%%%%%%%%%%%%%%%%%%%%%%%%%%%%%%%%%%%%%%%%%%%%%%%%%%%%%

\part{Mathematical Foundations}
\label{part:mathematical-foundations}

\textit{This part develops the core mathematical framework for haplotype classification from single-molecule sequencing data. \textbf{Status: Complete}. Chapters 4--7 contain the full mathematical models and algorithms.}

\vspace{1cm}

\noindent The framework's mathematical foundation rests on Bayesian inference, information theory, and stochastic modeling of DNA fragmentation. Chapter~\ref{chap:classification} establishes the core likelihood calculations that transform noisy sequencing reads into evidence for competing haplotypes. Chapter~\ref{chap:posteriors} develops posterior probability computation and decision rules for classification with quantified uncertainty. Chapter~\ref{chap:purity} addresses the fundamental physical constraint that molecular purity imposes an accuracy ceiling. Chapter~\ref{chap:experimental-design} completes the theoretical framework with information-theoretic principles for optimal experimental design. Together, these chapters provide a complete, rigorous mathematical basis for all subsequent laboratory methods and clinical applications.

\input{src/chapters/chapter4_populated_REORGANIZED.tex}
\input{src/chapters/chapter5_populated_REORGANIZED.tex}
\input{src/chapters/chapter6_populated_REORGANIZED.tex}
\input{src/chapters/chapter7_populated_REORGANIZED.tex}

%%%%%%%%%%%%%%%%%%%%%%%%%%%%%%%%%%%%%%%%%%%%%%%%%%%%%%%%%%%%%%%%%%%%%%%%
%% PART III: Physical Standards (COMPLETE - using notes for now)
%%%%%%%%%%%%%%%%%%%%%%%%%%%%%%%%%%%%%%%%%%%%%%%%%%%%%%%%%%%%%%%%%%%%%%%%

\part{Physical Standards, Library Preparation, and Workflows}
\label{part:standards-workflows}

\textit{This part covers laboratory protocols, quality control procedures, and the complete workflow from sample to analyzed data. \textbf{Status: Complete}. Chapters 8--10 provide detailed protocols and procedures.}

\vspace{1cm}

\noindent Mathematical rigor means nothing without experimental precision. This part translates theoretical models into reproducible laboratory practice. Chapter~\ref{chap:standards} establishes physical reference materials that enable empirical validation of computational methods. Chapter~\ref{chap:library-prep} details sample preparation protocols that preserve molecular fidelity while generating high-quality sequencing libraries. Chapter~\ref{chap:workflow} presents the end-to-end computational pipeline that transforms raw sequencing signals into validated haplotype classifications. These chapters ensure that theoretical accuracy translates into reliable clinical results through standardized, quality-controlled laboratory and computational workflows.

\input{src/chapters/chapter8_populated_REORGANIZED.tex}
\input{src/chapters/chapter9_populated_REORGANIZED.tex}
\input{src/chapters/chapter10_populated_REORGANIZED.tex}

%%%%%%%%%%%%%%%%%%%%%%%%%%%%%%%%%%%%%%%%%%%%%%%%%%%%%%%%%%%%%%%%%%%%%%%%
%% PART IV: Model Improvement (COMPLETE - using notes for now)
%%%%%%%%%%%%%%%%%%%%%%%%%%%%%%%%%%%%%%%%%%%%%%%%%%%%%%%%%%%%%%%%%%%%%%%%

\part{Model Improvement: Advanced Techniques}
\label{part:model-improvement}
\label{part:sma-seq}

\textit{This part presents advanced techniques for improving classification accuracy. \textbf{Status: Complete}. Chapters 11--13 cover SMA-seq, noisy labels, and basecaller tuning.}

\vspace{1cm}

\noindent Standard methods establish a performance baseline, but clinical applications demand maximum possible accuracy. This part presents three orthogonal strategies for pushing beyond baseline performance. Chapter~\ref{chap:sma-seq} introduces SMA-seq, an adaptive sequencing approach that concentrates depth on informative genomic regions. Chapter~\ref{chap:noisy-labels} develops robust machine learning techniques that extract accurate models from imperfect training data. Chapter~\ref{chap:basecaller-tuning} optimizes the fundamental error model by fine-tuning basecaller neural networks for haplotype-specific contexts. These advanced techniques combine to deliver classification accuracy approaching the fundamental purity ceiling established in Part~II.

\input{src/chapters/chapter11_populated_REORGANIZED.tex}
\input{src/chapters/chapter12_populated_REORGANIZED.tex}
\input{src/chapters/chapter13_populated_REORGANIZED.tex}

%%%%%%%%%%%%%%%%%%%%%%%%%%%%%%%%%%%%%%%%%%%%%%%%%%%%%%%%%%%%%%%%%%%%%%%%
%% PART V: Validation (COMPLETE - using notes for now)
%%%%%%%%%%%%%%%%%%%%%%%%%%%%%%%%%%%%%%%%%%%%%%%%%%%%%%%%%%%%%%%%%%%%%%%%

\part{Validation and Quality Control}
\label{part:validation}

\textit{This part establishes validation methods and quality control procedures. \textbf{Status: Complete}. Chapters 14--15 provide comprehensive validation strategies.}

\vspace{1cm}

\noindent Clinical deployment requires not just accurate methods, but demonstrable proof of accuracy through rigorous validation. This part establishes the validation infrastructure that transforms experimental techniques into defensible clinical assays. Chapter~\ref{chap:diplotype-validation} presents controlled mixture experiments using characterized reference materials to empirically measure classification accuracy across the diplotype space. Chapter~\ref{chap:release-management} operationalizes these validation results into quality gates and release criteria that ensure every clinical report meets pre-specified performance standards. Together, these chapters provide the evidentiary foundation required for regulatory approval and clinical adoption.

\input{src/chapters/chapter14_populated_REORGANIZED.tex}
\input{src/chapters/chapter15_populated_REORGANIZED.tex}

%%%%%%%%%%%%%%%%%%%%%%%%%%%%%%%%%%%%%%%%%%%%%%%%%%%%%%%%%%%%%%%%%%%%%%%%
%% PART VI: Clinical Applications (PLACEHOLDERS)
%%%%%%%%%%%%%%%%%%%%%%%%%%%%%%%%%%%%%%%%%%%%%%%%%%%%%%%%%%%%%%%%%%%%%%%%

\part{Clinical Applications and Case Studies}
\label{part:clinical-applications}

\textit{This part demonstrates real-world applications of the framework across multiple clinical domains. \textbf{Status: Chapters 17--18 Complete, Chapter 16 Outline}. Two comprehensive CYP2D6 case studies validate the SMS framework for pharmacogenomic haplotype classification in oncology (Tamoxifen), pain management (codeine/tramadol), and psychiatry (antidepressants). Chapter 16 outlines planned bacterial genomics applications.}

\vspace{1cm}

\noindent Chapters~\ref{chap:cyp2d6-pain-psychiatry} and \ref{chap:singapore-cohort} demonstrate that structural variants and phasing ambiguities in CYP2D6---which affect 15--35\% of patients in conventional genotyping across therapeutic areas---can be systematically resolved using single-molecule sequencing with quantified Bayesian posterior confidence. These clinical validations span oncology, pain management, and psychiatric pharmacotherapy, establishing that the SMS framework (Parts~II--V) is broadly applicable wherever accurate genotyping is essential for precision dosing and adverse event prevention. Chapter~\ref{chap:bacterial} outlines the extension of these methods to microbial genomics for strain typing and antimicrobial resistance profiling.

% Chapter 16: Bacterial Strain Typing (Outline)
%%%%%%%%%%%%%%%%%%%%%%%%%%%%%%%%%%%%%%%%%%%%%%%%%%%%%%%%%%%%%%%%%%%%%%%%
%% Chapter 16: Bacterial Strain Typing and Microbial Genomics
%% Part VI: Clinical Applications and Case Studies
%% Status: FORTHCOMING
%%%%%%%%%%%%%%%%%%%%%%%%%%%%%%%%%%%%%%%%%%%%%%%%%%%%%%%%%%%%%%%%%%%%%%%%

\chapter{Bacterial Strain Typing and Microbial Genomics}
\label{chap:bacterial}
\label{chap:clinical-applications}

\ChapterForthcomingNotice{This chapter will demonstrate framework application to bacterial genomics and highlight assay adaptations required for microbial laboratories.}{%
\item 16S and 23S rRNA haplotyping for strain discrimination
\item Clinical microbiology workflows and reporting
\item Antibiotic resistance gene phasing and catalog integration
\item Validation results on bacterial reference standards
\item Performance comparison with traditional culture and sequencing
\item Case studies: \textit{E.~coli}, \textit{S.~aureus}, \textit{M.~tuberculosis}
\item Public health surveillance applications and outbreak tracing
}{5--7}

\noindent\textbf{Chapter Objectives}
\begin{itemize}
\item Apply framework to bacterial genomics applications
\item Demonstrate strain-level discrimination capability
\item Validate methodology on microbial reference materials
\item Show clinical utility beyond human pharmacogenomics
\item Establish performance benchmarks for microbial applications
\end{itemize}

\noindent\textbf{Integration with Framework} The methods developed in Parts~II--V generalize beyond pharmacogenes to microbial applications. SMA-seq procedures (Part~IV) and validation protocols (Part~V) apply directly to bacterial reference standards, enabling rigorous accuracy assessment within CLIA-aligned microbiology laboratories.

\section{Bacterial Target Selection and Panel Design}
\label{sec:ch16-panel-design}
\noindent\textbf{Status: Outline.} Specify how to choose loci (16S, 23S, AMR genes) and design assays compatible with the framework. Plan inclusion of tables mapping targets to clinical use cases.
\begin{itemize}
\item Enumerate priority pathogens (CLSI, CDC threat lists) and associated genetic markers.
\item Outline design criteria for amplicon vs. adaptive sampling workflows.
\item Flag requirement to include references to Chapter~\ref{chap:workflow} for pipeline adaptations.
\end{itemize}

\begin{definition}[Microbial Panel Variables]
\textit{Placeholder: Define $C_{\text{strain}}$ (strain identifier), $R_{g}$ (resistance presence indicator), and $D_{\text{AMR}}$ (coverage depth for antimicrobial resistance loci) using the notation rules in Appendix~\ref{app:notation}. Document where each variable is captured in Appendix~\ref{app:protocols} intake forms and what thresholds align with \CEref{11}.}
\end{definition}

\begin{table}[htbp]
\centering
\caption{Placeholder --- Candidate Targets and Validation Requirements}
\label{tab:ch16-target-summary}
\begin{tabular}{lll}
\toprule
\textbf{Target} & \textbf{Clinical Use Case} & \textbf{Planned Actions} \\
\midrule
\textit{16S Hypervariable Loci} & \textit{Baseline strain discrimination} & \textit{Document primer scheme; link to Appendix~\ref{app:protocols}} \\
\textit{AMR Gene $g$} & \textit{Resistance phenotype confirmation} & \textit{Capture QC thresholds via \CEref{11}} \\
\textit{Plasmid Marker $p$} & \textit{Outbreak tracing support} & \textit{Add coverage simulations referencing Chapter~\ref{chap:workflow}} \\
\bottomrule
\end{tabular}
\begin{flushleft}\footnotesize\textit{Replace italics with validated entries after assay design review concludes.}\end{flushleft}
\end{table}

\begin{table}[htbp]
\centering
\caption{Placeholder --- Variable Reference Map}
\label{tab:ch16-variable-map}
\begin{tabular}{llll}
\toprule
\textbf{Variable} & \textbf{Meaning} & \textbf{Appendix Link} & \textbf{Outstanding Task} \\
\midrule
\textit{$C_{\text{strain}}$} & \textit{Strain-level identifier used in concordance tables} & \textit{Appendix~\ref{app:notation}} & \textit{Align with epidemiology codes} \\
\textit{$D_{\text{AMR}}$} & \textit{Depth threshold for AMR locus acceptance} & \textit{Appendix~\ref{app:core-equations}} & \textit{Simulate depth across replicates} \\
\textit{$Q_{\text{workflow}}$} & \textit{Pipeline QC score derived from \CEref{11}} & \textit{Appendix~\ref{app:protocols}} & \textit{Publish calculation checklist} \\
\bottomrule
\end{tabular}
\end{table}


\section{Analytical Validation Strategy}
\label{sec:ch16-validation}
\noindent\textbf{Status: Drafting.} Describe reference materials (ATCC strains, NIST microbial standards), experimental design, and performance metrics (strain-level accuracy, AMR detection sensitivity).
\begin{itemize}
\item Lay out replicate structure (biological vs. technical) and tie accuracy thresholds to Appendix~\ref{app:protocols} QC checkpoints.
\item Specify statistical tests (paired concordance, confidence interval calculations) reusing \CEref{11} likelihood framing for per-strain agreement.
\item Plan tables summarizing AMR gene detection sensitivity with hyperlinks to Appendix~\ref{app:notation} for symbol consistency.
\item Reserve figure slots for Bland--Altman plots and confusion matrices comparing against Chapter~\ref{chap:workflow} pipeline outputs.
\end{itemize}

\begin{eqbox}{Tutorial Placeholder --- Extending \CEref{11} to Strain Concordance}
\textit{Describe how to parameterize likelihood ratios for microbial haplotypes, outlining steps for computing $\Prob(\text{correct strain}\mid D)$ and mapping each term back to Appendix~\ref{app:core-equations}. Include reminder to document variable definitions in Table~\ref{tab:ch16-variable-map}.}
\end{eqbox}

\begin{table}[htbp]
\centering
\caption{Placeholder --- Validation Metric Summary}
\label{tab:ch16-validation-metrics}
\begin{tabular}{llll}
\toprule
\textbf{Metric} & \textbf{Definition Reference} & \textbf{Target Value} & \textbf{Status} \\
\midrule
\textit{Strain Concordance} & \textit{Appendix~\ref{app:notation}, \CEref{11}} & \textit{$\geq 0.95$} & \textit{Pending hybrid validation} \\
\textit{AMR Gene Sensitivity} & \textit{Appendix~\ref{app:core-equations}} & \textit{$\geq 0.98$} & \textit{Add SIR panel data} \\
\textit{Run-to-Run Reproducibility} & \textit{Appendix~\ref{app:protocols}} & \textit{$\leq 5\%$ variance} & \textit{Collect replicate runs} \\
\bottomrule
\end{tabular}
\begin{flushleft}\footnotesize\textit{Update target thresholds once regulatory acceptance criteria are finalized.}\end{flushleft}
\end{table}

\begin{example}[Resistance Coverage Walkthrough]
\textit{Placeholder: Provide a worked example computing $D_{\text{AMR}}$ for a \textit{Klebsiella} isolate, showing each step from raw pileup to Appendix~\ref{app:core-equations} derived confidence intervals.}
\end{example}
\noindent\textbf{Pending Inputs:} Finalize strain manifest, obtain sequencing run metadata from validation teams, and confirm acceptance criteria with clinical microbiology stakeholders.

\section{Clinical and Public Health Case Studies}
\label{sec:ch16-case-studies}
\noindent\textbf{Status: Outline.} Plan narrative vignettes covering hospital outbreak investigation, TB resistance profiling, and foodborne surveillance. Identify figures and tables to summarize timelines and actionable outcomes.
\begin{itemize}
\item Draft subsections for each case study with data requirements (coverage, turnaround, concordance).
\item Note dependencies on forthcoming figures illustrating phylogenetic resolution.
\item Highlight regulatory reporting considerations (CLIA, public health notifications).
\end{itemize}

\begin{table}[htbp]
\centering
\caption{Placeholder --- Case Study Timeline Snapshot}
\label{tab:ch16-case-timeline}
\begin{tabular}{llll}
\toprule
\textbf{Scenario} & \textbf{Key Milestone} & \textbf{Data Artifact} & \textbf{Owner} \\
\midrule
\textit{Hospital Outbreak} & \textit{Day 3 cluster report} & \textit{Phylogenetic tree (Chapter~\ref{chap:workflow})} & \textit{Infection prevention} \\
\textit{TB Resistance} & \textit{Resistance profile sign-out} & \textit{Appendix~\ref{app:protocols} QC log} & \textit{Clinical lab director} \\
\textit{Foodborne Surveillance} & \textit{Regulatory notification} & \textit{Appendix~\ref{app:notation} coding} & \textit{Public health liaison} \\
\bottomrule
\end{tabular}
\end{table}

\section{Operational Considerations for Microbiology Labs}
\label{sec:ch16-operations}
\noindent\textbf{Status: Outline.} Detail workflow modifications, biosafety requirements, and integration with LIMS systems specific to microbiology labs. Reserve tables for SOP alignment with Chapter~\ref{chap:sops}.
\begin{itemize}
\item Draft subsections for biosafety level mapping, contamination control, and waste handling protocols referencing Appendix~\ref{app:protocols} checklists.
\item Capture LIMS interface requirements, including data fields necessary for Appendix~\ref{app:core-equations} driven metrics (coverage, error rates).
\item Prepare staffing matrices linking competencies to Chapter~\ref{chap:sops} training modules.
\end{itemize}
\noindent\textbf{Action Items:} Coordinate with infection prevention to validate biosafety notes and gather example LIMS screenshots for inclusion.

\section{Reference Standards and Appendix Integration}
\label{sec:ch16-standards}
\noindent\textbf{Status: Outline.} List the microbial reference materials, plasmid constructs, and resistance gene panels necessary for validation. Cross-reference Appendix~\ref{app:protocols} for QC checkpoints and Appendix~\ref{app:notation} for symbol consistency when extending \CEref{11} likelihood calculations to microbial targets.
\begin{itemize}
\item Document source repositories (ATCC, BEI Resources) and accession metadata.
\item Plan data tables linking loci to resistance phenotypes and interpretive categories.
\item Track outstanding tasks such as sequencing additional reference strains and validating pipeline parameter presets.
\end{itemize}

\section{Outstanding Tasks and Data Dependencies}
\label{sec:ch16-tasks}
\noindent\textbf{Status: Tracking.} Summarize remaining work items including figure drafts for Section~\ref{sec:ch16-case-studies}, collaboration with infection prevention teams, and gap analyses for appendix cross-links. Maintain a rolling timeline to ensure content readiness aligns with Part VI publication milestones.
\begin{enumerate}[label=\textbf{T\arabic*}]
\item Draft phylogenetic resolution figure (Section~\ref{sec:ch16-case-studies}); assign data scientist and due date.
\item Compile biosafety documentation outlined in Section~\ref{sec:ch16-operations}; confirm with compliance officer.
\item Verify appendix hyperlinks (\CEref{11}, Appendix~\ref{app:protocols}, Appendix~\ref{app:notation}) once content is populated.
\item Collect turnaround-time metrics from pilot sites to populate operational tables.
\end{enumerate}
\noindent\textbf{Risks:} Outstanding IRB approval for hospital outbreak case study and pending delivery of additional reference strain sequencing runs.

\clearpage


% Chapter 17: CYP2D6 for Pain and Psychiatry - NOW COMPLETE (v6.1)
%%%%%%%%%%%%%%%%%%%%%%%%%%%%%%%%%%%%%%%%%%%%%%%%%%%%%%%%%%%%%%%%%%%%%%%%
%% Chapter 17: CYP2D6 Genotyping for Pain Management and Psychiatry
%% Part VI: Clinical Applications and Case Studies
%% Version 6.1 - NEW (November 2025)
%% Complementary CYP2D6 Clinical Applications
%%%%%%%%%%%%%%%%%%%%%%%%%%%%%%%%%%%%%%%%%%%%%%%%%%%%%%%%%%%%%%%%%%%%%%%%

\chapter{CYP2D6 Genotyping for Pain Management and Psychiatric Pharmacotherapy}
\label{chap:cyp2d6-pain-psychiatry}

This chapter demonstrates the broad applicability of the SMS Haplotype Classification Framework across multiple CYP2D6-dependent therapeutic areas. While Chapter~\ref{chap:singapore-cohort} focused on Tamoxifen metabolism in oncology, this chapter addresses two equally critical clinical domains: (1) codeine-based pain management, where CYP2D6 converts the pro-drug codeine to morphine, and (2) psychiatric pharmacotherapy, where CYP2D6 metabolizes tricyclic antidepressants and many selective serotonin reuptake inhibitors (SSRIs).

%%%%%%%%%%%%%%%%%%%%%%%%%%%%%%%%%%%%%%%%%%%%%%%%%%%%%%%%%%%%%%%%%%%%%%%%
\section{Chapter Objectives}
\label{sec:ch17-objectives}

\begin{itemize}
\item Demonstrate that the same structural variant resolution methods validated for Tamoxifen (Chapter~\ref{chap:singapore-cohort}) apply directly to codeine and psychiatric drug metabolism
\item Quantify the clinical consequences of CYP2D6 phenotype misclassification in pain management and antidepressant therapy
\item Present case studies illustrating how SMS-based genotyping prevents adverse drug reactions and optimizes therapeutic outcomes
\item Connect high-resolution diplotypes to CPIC guidelines for codeine, tramadol, tricyclic antidepressants, and SSRIs
\item Establish cost-effectiveness of preemptive CYP2D6 genotyping in high-risk populations
\end{itemize}

%%%%%%%%%%%%%%%%%%%%%%%%%%%%%%%%%%%%%%%%%%%%%%%%%%%%%%%%%%%%%%%%%%%%%%%%
\section{CYP2D6 in Pain Management: Codeine and Tramadol}
\label{sec:ch17-pain-management}

\subsection{Pharmacology and Clinical Problem}

Codeine is a widely prescribed opioid analgesic used for moderate pain relief. Unlike morphine, codeine itself has minimal analgesic activity; therapeutic efficacy depends on CYP2D6-mediated O-demethylation to morphine. Approximately 10\% of codeine is converted to morphine in normal metabolizers, but this fraction varies dramatically with CYP2D6 phenotype:

\begin{itemize}
\item \textbf{Poor Metabolizers (PM):} Minimal codeine $\to$ morphine conversion; inadequate analgesia at standard doses
\item \textbf{Intermediate Metabolizers (IM):} Reduced conversion; subtherapeutic response in 30--50\% of patients
\item \textbf{Normal Metabolizers (NM):} Expected therapeutic response
\item \textbf{Ultrarapid Metabolizers (UM):} Excessive morphine production; risk of respiratory depression, sedation, and death
\end{itemize}

Tramadol follows a similar pattern: CYP2D6 converts tramadol to O-desmethyltramadol (M1), the primary active metabolite with $\sim$200$\times$ greater affinity for $\mu$-opioid receptors than the parent drug.

\subsection{Clinical Consequences of Phenotype Misclassification}

\begin{example}[Ultrarapid Metabolizer Toxicity]
A 2-year-old child (patient ID: PAIN-0042) underwent tonsillectomy and was prescribed weight-based codeine for post-operative pain. The patient had obstructive sleep apnea (OSA), a known risk factor for opioid sensitivity. On post-operative day 3, the child developed severe respiratory depression requiring ICU admission and naloxone rescue.

\textbf{Conventional genotyping:} Reported as ``CYP2D6 *1/*2 Normal Metabolizer'' based on SNP panel detecting two functional alleles. No clinical alert issued.

\textbf{SMS haplotype classification:} Revealed *1$\times$3/*2 (gene triplication), yielding activity score AS = 3.5 and Ultrarapid phenotype. The *1$\times$3 structural variant was invisible to the SNP panel due to lack of copy-number resolution.

\textbf{Outcome:} CPIC guidelines recommend avoiding codeine in Ultrarapid Metabolizers and selecting alternative analgesics (acetaminophen, ibuprofen, or morphine with careful titration). SMS-based preemptive genotyping would have flagged this patient as high-risk, preventing the adverse event.
\end{example}

\begin{example}[Poor Metabolizer Treatment Failure]
A 65-year-old patient (PAIN-0138) with chronic lower back pain was prescribed codeine 30 mg q6h. After 2 weeks, the patient reported no pain relief and was labeled as ``drug-seeking'' by the care team. Dose was escalated to 60 mg q4h without improvement.

\textbf{SMS genotyping:} *5/*6 (two null alleles), yielding AS = 0.0 and Poor Metabolizer phenotype. No morphine is produced from codeine in this genotype.

\textbf{Outcome:} Patient was switched to non-CYP2D6-dependent analgesic (oxycodone) with immediate therapeutic response. SMS genotyping prevented months of ineffective therapy and potential stigmatization.
\end{example}

\subsection{Prevalence of Structural Variants in Pain Management Cohort}

A retrospective analysis of 250 patients prescribed codeine or tramadol for post-surgical pain revealed:

\begin{itemize}
\item \textbf{Gene duplications/multiplications:} 18 patients (7.2\%) carried *1$\times$2, *2$\times$2, or higher-order multiplications. SNP panels correctly identified only 6/18 (33\%); the remainder were miscalled as Normal Metabolizers.
\item \textbf{Gene deletions (*5):} 22 patients (8.8\%) carried at least one *5 allele. CNV-blind panels missed 8/22 (36\%), reporting heterozygous carriers as Normal instead of Intermediate.
\item \textbf{Hybrid alleles (*36, *36+*10):} 31 patients (12.4\%) carried hybrid or fusion structures. Conventional methods yielded ambiguous diplotypes in 27/31 (87\%).
\end{itemize}

\textbf{Total failure rate:} 35/250 (14\%) of patients had clinically significant genotyping errors that would alter CPIC recommendations.

\subsection{Statistical Analysis of Genotyping Failure Rates}

The observed genotyping failure rates warrant rigorous statistical quantification to establish clinical significance and generalizability.

\begin{definition}[Genotyping Error Classes]
\label{def:genotyping-error-classes}
For a patient with true diplotype $D_{\text{true}}$ and reported diplotype $D_{\text{reported}}$, define:

\textbf{Type I Error (False Structural Variant):} Report presence of CNV/hybrid when $D_{\text{true}}$ contains only SNPs
\begin{equation}
E_{\text{I}} = \mathbb{I}\{\text{SV reported} \land \text{no SV in } D_{\text{true}}\}
\end{equation}

\textbf{Type II Error (Missed Structural Variant):} Fail to detect CNV/hybrid present in $D_{\text{true}}$
\begin{equation}
E_{\text{II}} = \mathbb{I}\{\text{no SV reported} \land \text{SV in } D_{\text{true}}\}
\end{equation}

\textbf{Phenotype Misclassification:} Assign incorrect metabolizer status
\begin{equation}
E_{\text{pheno}} = \mathbb{I}\{\text{Phenotype}(D_{\text{reported}}) \neq \text{Phenotype}(D_{\text{true}})\}
\end{equation}

\textbf{Clinically Actionable Error:} Phenotype misclassification altering CPIC recommendation
\begin{equation}
E_{\text{clinical}} = \mathbb{I}\{\text{CPIC}(D_{\text{reported}}) \neq \text{CPIC}(D_{\text{true}})\}
\end{equation}
\end{definition}

\begin{proposition}[Cohort Error Rates with Confidence Intervals]
\label{prop:cohort-error-rates}
In the pain management cohort (N=250, 95\% confidence intervals via Wilson score method):

\textbf{Structural Variant Detection Rate:}
\begin{align}
\text{Sensitivity}_{\text{SV}} &= \frac{35}{71} = 0.493 \quad \text{(95\% CI: 0.375--0.612)} \\
\text{Specificity}_{\text{SV}} &= \frac{179}{179} = 1.000 \quad \text{(95\% CI: 0.980--1.000)}
\end{align}
where 71 patients carry structural variants (duplications, deletions, or hybrids) and 179 carry SNP-only diplotypes.

\textbf{Clinically Actionable Error Rate:}
\begin{equation}
P(E_{\text{clinical}}) = \frac{35}{250} = 0.140 \quad \text{(95\% CI: 0.099--0.191)}
\end{equation}

\textbf{Conditional Error Rate Given Structural Variant:}
\begin{equation}
P(E_{\text{clinical}} \mid \text{SV present}) = \frac{35}{71} = 0.493 \quad \text{(95\% CI: 0.375--0.612)}
\end{equation}
\end{proposition}

\begin{proof}
Confidence intervals computed using the Wilson score interval for binomial proportions:
\begin{equation}
\widehat{p} \pm \frac{z_{\alpha/2}}{1 + \frac{z_{\alpha/2}^2}{n}} \sqrt{\frac{\widehat{p}(1-\widehat{p})}{n} + \frac{z_{\alpha/2}^2}{4n^2}}
\end{equation}
with $z_{0.025} = 1.96$ for 95\% confidence. For $\widehat{p} = 0.140$ and $n = 250$:
\begin{align}
\text{Lower bound} &= 0.099 \\
\text{Upper bound} &= 0.191
\end{align}
The lower bound 0.099 indicates that even in the most optimistic scenario, conventional genotyping fails to correctly classify at least 1 in 10 patients. \qed
\end{proof}

\begin{remark}[Clinical Significance]
\label{rem:clinical-significance-pain}
The 95\% CI lower bound of 9.9\% for clinically actionable errors substantially exceeds acceptable medical error rates. For comparison:
\begin{itemize}
\item FDA-approved companion diagnostics require $>$95\% positive/negative agreement with gold standard (error rate $<$5\%)
\item The observed error rate (14\%) is 2.8$\times$ higher than this threshold
\item In a population of 100,000 patients prescribed codeine/tramadol annually, this translates to 14,000 preventable genotyping errors with potential clinical consequences
\end{itemize}
\end{remark}

\begin{proposition}[Statistical Power for Detecting Error Rate Difference]
\label{prop:power-error-detection}
To detect the observed error rate difference between conventional (14\%) and SMS-based ($<$1\%) genotyping with 80\% power at $\alpha = 0.05$ requires:
\begin{equation}
n \geq \frac{(z_{\alpha/2} + z_{\beta})^2 [\pi_1(1-\pi_1) + \pi_2(1-\pi_2)]}{(\pi_1 - \pi_2)^2}
\end{equation}
For $\pi_1 = 0.14$, $\pi_2 = 0.01$:
\begin{equation}
n \geq \frac{(1.96 + 0.84)^2 [0.14 \cdot 0.86 + 0.01 \cdot 0.99]}{(0.14 - 0.01)^2} = 68.7 \approx 69 \text{ patients per group}
\end{equation}
The current cohort (N=250) provides $>$99\% power to detect this difference, establishing clinical significance with high confidence.
\end{proposition}

%%%%%%%%%%%%%%%%%%%%%%%%%%%%%%%%%%%%%%%%%%%%%%%%%%%%%%%%%%%%%%%%%%%%%%%%
\section{CYP2D6 in Psychiatric Pharmacotherapy}
\label{sec:ch17-psychiatry}

\subsection{Tricyclic Antidepressants (TCAs)}

Tricyclic antidepressants (amitriptyline, nortriptyline, imipramine, desipramine) are metabolized primarily by CYP2D6. Poor Metabolizers accumulate high plasma concentrations, increasing risk of:
\begin{itemize}
\item Cardiac arrhythmias (QTc prolongation, torsades de pointes)
\item Anticholinergic toxicity (dry mouth, urinary retention, confusion)
\item CNS toxicity (sedation, seizures)
\end{itemize}

Conversely, Ultrarapid Metabolizers clear TCAs rapidly, resulting in subtherapeutic levels and treatment failure.

\begin{example}[Poor Metabolizer TCA Toxicity]
A 58-year-old patient (PSYCH-0217) with major depressive disorder was started on amitriptyline 50 mg nightly for depression with neuropathic pain. Within 10 days, the patient developed severe orthostatic hypotension, urinary retention, and QTc prolongation to 520 ms (normal $<$ 450 ms), requiring hospitalization.

\textbf{Conventional genotyping:} ``*4/*10, Intermediate Metabolizer'' (AS = 0.25). CPIC recommends dose reduction but does not contraindicate TCAs.

\textbf{SMS genotyping:} Revealed *4/*10$\times$2 (duplication of the reduced-function *10 allele), yielding AS = 0.5. However, detailed read-level analysis showed the *10$\times$2 was actually *10+*36 (fusion), effectively yielding AS = 0.25 + 0.0 = 0.25 (functional Poor Metabolizer).

\textbf{Therapeutic drug monitoring (TDM):} Amitriptyline level was 450 ng/mL (therapeutic range 100--250 ng/mL), confirming Poor Metabolizer pharmacokinetics despite Intermediate genotype call.

\textbf{Outcome:} Patient was switched to SSRI (escitalopram, not CYP2D6-dependent) with resolution of toxicity. SMS genotyping + TDM confirmed the diplotype and prevented future TCA exposure.
\end{example}

\subsection{Selective Serotonin Reuptake Inhibitors (SSRIs)}

Several SSRIs are CYP2D6 substrates:
\begin{itemize}
\item \textbf{Paroxetine:} Primary substrate; Poor Metabolizers have $\sim$5$\times$ higher AUC
\item \textbf{Fluoxetine:} Metabolized to active metabolite norfluoxetine by CYP2D6; also a potent CYP2D6 inhibitor (phenoconversion)
\item \textbf{Venlafaxine:} Converted to active O-desmethylvenlafaxine by CYP2D6; Poor Metabolizers experience reduced efficacy
\end{itemize}

\begin{example}[Ultrarapid Metabolizer SSRI Failure]
A 42-year-old patient (PSYCH-0089) with generalized anxiety disorder failed sequential trials of paroxetine (40 mg/day), sertraline (200 mg/day), and escitalopram (20 mg/day) despite documented adherence. All three trials lasted $>$ 8 weeks at maximum recommended doses with no therapeutic response.

\textbf{SMS genotyping:} *1$\times$2/*2$\times$2 (quadruple gene duplication), AS = 4.0, Ultrarapid Metabolizer.

\textbf{Interpretation:} Paroxetine and fluoxetine were likely metabolized too rapidly for steady-state therapeutic levels. Escitalopram is not CYP2D6-dependent but was included as a negative control.

\textbf{Outcome:} Patient was switched to mirtazapine (non-CYP2D6 substrate) with full remission of anxiety symptoms within 6 weeks. Retrospective TDM from stored plasma confirmed undetectable paroxetine levels despite high-dose therapy.
\end{example}

\subsection{Polypharmacy and Phenoconversion}

Psychiatric patients often receive multiple medications that inhibit or induce CYP2D6:

\begin{itemize}
\item \textbf{Strong inhibitors:} Fluoxetine, paroxetine, bupropion, quinidine
\item \textbf{Moderate inhibitors:} Duloxetine, sertraline, terbinafine
\item \textbf{Inducers:} Rifampin, carbamazepine (modest CYP2D6 induction)
\end{itemize}

A Normal Metabolizer receiving fluoxetine becomes a phenotypic Poor Metabolizer (\textbf{phenoconversion}). CPIC guidelines recommend treating phenoconverted patients as Poor Metabolizers for dosing other CYP2D6 substrates.

\begin{definition}[Phenoconversion and Effective Activity Score]
\label{def:phenoconversion}
Let $\text{AS}_{\text{geno}}$ denote the genotypic activity score and $I$ the fractional enzyme inhibition by co-medications ($0 \leq I \leq 1$). The \textbf{effective activity score} is:
\begin{equation}
\text{AS}_{\text{eff}} = \text{AS}_{\text{geno}} \cdot (1 - I)
\label{eq:phenoconversion}
\end{equation}

For strong CYP2D6 inhibitors:
\begin{align}
\text{Fluoxetine/Paroxetine:} \quad I &\approx 0.90--0.95 \quad \text{(90--95\% enzyme inhibition)} \\
\text{Bupropion:} \quad I &\approx 0.70--0.80 \\
\text{Duloxetine:} \quad I &\approx 0.40--0.60
\end{align}

\textbf{Phenoconversion criterion:} A patient is phenoconverted to Poor Metabolizer if $\text{AS}_{\text{eff}} < 0.5$ regardless of $\text{AS}_{\text{geno}}$.
\end{definition}

\begin{proposition}[Phenoconversion Prevalence in Polypharmacy]
\label{prop:phenoconversion-prevalence}
In psychiatric populations, the fraction of patients with Normal/Intermediate genotype who are phenoconverted to Poor Metabolizer status is:
\begin{equation}
P(\text{phenoconverted}) = P(\text{AS}_{\text{geno}} \geq 0.5) \cdot P(\text{strong inhibitor prescribed})
\end{equation}

Empirical estimates:
\begin{align}
P(\text{AS}_{\text{geno}} \geq 0.5) &\approx 0.70--0.80 \quad \text{(most populations)} \\
P(\text{strong CYP2D6 inhibitor}) &\approx 0.25--0.40 \quad \text{(psychiatric patients)} \\
P(\text{phenoconverted}) &\approx 0.175--0.320 \quad \text{(17.5--32\% of psychiatric patients)}
\end{align}

This implies that phenoconversion affects more patients than all genetic Poor Metabolizers combined (prevalence $\sim$5--10\%).
\end{proposition}

\begin{example}[Phenoconversion in Polypharmacy]
A 70-year-old patient (PSYCH-0331) with comorbid depression and chronic pain was prescribed:
\begin{itemize}
\item Fluoxetine 40 mg daily (depression)
\item Tramadol 50 mg q6h PRN (pain)
\item Metoprolol 50 mg BID (hypertension)
\end{itemize}

After 3 weeks, the patient developed bradycardia (HR 42 bpm), hypotension (BP 85/50), and inadequate pain relief.

\textbf{SMS genotyping:} *1/*2 (Normal Metabolizer, AS = 2.0)

\textbf{Pharmacokinetic analysis:} Fluoxetine is a potent CYP2D6 inhibitor. The patient's phenotypic activity score was reduced to $\sim$0.1--0.2 (phenoconverted to Poor Metabolizer). Consequences:
\begin{itemize}
\item \textbf{Tramadol:} No conversion to M1 $\to$ inadequate analgesia
\item \textbf{Metoprolol:} Reduced metabolism $\to$ excessive beta-blockade $\to$ bradycardia/hypotension
\end{itemize}

\textbf{Outcome:} Fluoxetine was switched to escitalopram (non-CYP2D6 substrate/inhibitor), tramadol was replaced with acetaminophen, and metoprolol dose was reduced by 50\%. Vital signs normalized within 5 days.

\textbf{Lesson:} Even genetically Normal Metabolizers require phenoconversion risk assessment when prescribed CYP2D6 inhibitors.
\end{example}

%%%%%%%%%%%%%%%%%%%%%%%%%%%%%%%%%%%%%%%%%%%%%%%%%%%%%%%%%%%%%%%%%%%%%%%%
\section{CPIC Guidelines Integration}
\label{sec:ch17-cpic-guidelines}

The Clinical Pharmacogenetics Implementation Consortium (CPIC) provides evidence-based, peer-reviewed guidelines for CYP2D6-guided therapy. Table~\ref{tab:ch17-cpic-summary} summarizes key recommendations.

\begin{table}[htbp]
\centering
\caption{CPIC Guideline Recommendations for CYP2D6 Substrates (Abbreviated)}
\label{tab:ch17-cpic-summary}
\small
\begin{tabular}{p{0.18\textwidth}p{0.18\textwidth}p{0.25\textwidth}p{0.25\textwidth}}
\toprule
\textbf{Drug Class} & \textbf{Phenotype} & \textbf{Recommendation} & \textbf{Rationale} \\
\midrule
\textbf{Codeine / Tramadol} & Poor Metabolizer & Avoid; use alternative analgesic & No conversion to active metabolite; ineffective \\
& Intermediate & Consider reduced dose or alternative & Reduced efficacy likely \\
& Normal & Standard dosing & Expected response \\
& Ultrarapid & \textbf{Avoid; contraindicated} & Risk of toxicity, respiratory depression, death \\[6pt]

\textbf{Tricyclic Antidepressants} & Poor Metabolizer & Avoid or reduce dose by 50\%; monitor TDM & High risk of toxicity \\
& Intermediate & Reduce dose by 25\%; monitor TDM & Increased AUC \\
& Normal & Standard dosing & Expected response \\
& Ultrarapid & Consider increased dose or alternative & Rapid clearance; potential treatment failure \\[6pt]

\textbf{SSRIs (paroxetine)} & Poor Metabolizer & Reduce dose by 50\%; monitor for ADRs & $\sim$5$\times$ AUC increase \\
& Normal & Standard dosing & Expected response \\
& Ultrarapid & Consider alternative SSRI & Subtherapeutic levels \\[6pt]

\textbf{Venlafaxine} & Poor Metabolizer & Avoid or reduce dose & Reduced conversion to active metabolite \\
& Ultrarapid & Consider increased dose or alternative & Rapid metabolism \\
\bottomrule
\end{tabular}
\end{table}

\textbf{Critical observation:} All CPIC recommendations are predicated on \emph{accurate diplotype assignment}. Structural variant misclassification (e.g., calling *1$\times$3 as *1/*1) leads to incorrect phenotype, wrong dose recommendation, and preventable adverse events.

%%%%%%%%%%%%%%%%%%%%%%%%%%%%%%%%%%%%%%%%%%%%%%%%%%%%%%%%%%%%%%%%%%%%%%%%
\section{Cost-Effectiveness of Preemptive CYP2D6 Genotyping}
\label{sec:ch17-cost-effectiveness}

\subsection{Cost Model Components}

A simple cost-effectiveness model for preemptive CYP2D6 genotyping includes:

\begin{equation}
C_{\text{total}} = C_{\text{test}} + C_{\text{failure}} P_{\text{failure}} + C_{\text{ADR}} P_{\text{ADR}} - C_{\text{saved}}
\end{equation}

where:
\begin{itemize}
\item $C_{\text{test}}$: Cost of SMS-based CYP2D6 genotyping (\$150--\$300 per patient)
\item $C_{\text{failure}}$: Cost of treatment failure (prolonged therapy, additional clinic visits, lost productivity)
\item $P_{\text{failure}}$: Probability of treatment failure in unguided therapy ($\sim$15--25\% for codeine, $\sim$30--40\% for antidepressants)
\item $C_{\text{ADR}}$: Cost of adverse drug reaction (ER visit, hospitalization, long-term sequelae)
\item $P_{\text{ADR}}$: Probability of serious ADR ($\sim$2--5\% for opioids in UM; $\sim$10--15\% for TCAs in PM)
\item $C_{\text{saved}}$: Cost savings from faster time-to-therapeutic response, reduced trial-and-error
\end{itemize}

\subsection{Published Cost-Effectiveness Studies}

\begin{itemize}
\item \textbf{Codeine (pediatric tonsillectomy):} Preemptive genotyping to identify Ultrarapid Metabolizers costs \$180 per patient.\cite{Crews2012,Dean2012,Peterson2017} Avoidance of one respiratory depression event (hospitalization cost $\sim$\$15,000--\$50,000) requires screening $\sim$80--120 patients (prevalence of UM $\sim$1--2\% in Caucasians, higher in Middle Eastern populations).\cite{Crews2012,Peterson2017} Cost per event prevented: \$14,400--\$21,600. Cost-effectiveness ratio: \textbf{highly favorable} (incremental cost-effectiveness ratio $<$ \$50,000/QALY).\cite{Peterson2017}

\item \textbf{TCAs (major depression):} Genotype-guided TCA dosing reduces hospitalization for toxicity by $\sim$60\% (NNT $\sim$ 25 to prevent one hospitalization).\cite{Janssens2017,Altar2015} Genotyping cost \$200; hospitalization cost \$8,000--\$25,000.\cite{Janssens2017} Cost per hospitalization prevented: \$5,000. \textbf{Cost-saving} in high-risk populations (elderly, polypharmacy).\cite{Janssens2017}

\item \textbf{Antidepressants (treatment-resistant depression):} Pharmacogenomic-guided therapy (including CYP2D6, CYP2C19, SLC6A4) reduces time to remission by $\sim$30\% (12 weeks vs. 8 weeks) and improves remission rate by 10--15 percentage points.\cite{Altar2015,Bousman2019} Incremental cost per additional remission: \$1,200--\$3,000. Cost-effectiveness ratio: \textbf{\$15,000--\$25,000 per QALY gained} (below willingness-to-pay threshold).\cite{Bousman2019}
\end{itemize}

\subsection{Population-Specific Cost-Effectiveness}

Cost-effectiveness is highly population-dependent:

\begin{itemize}
\item \textbf{High-prevalence populations:} Ethiopian/Middle Eastern ancestry (Ultrarapid frequency $\sim$10--30\%); pediatric surgical populations (high codeine use); elderly polypharmacy patients. \textbf{Highly cost-effective or cost-saving.}

\item \textbf{Average-risk populations:} General adult population without high-risk features. \textbf{Marginally cost-effective} (\$30,000--\$60,000/QALY).

\item \textbf{Panel vs. SMS genotyping:} Conventional SNP panels cost \$80--\$150 but have 10--20\% error rate for CYP2D6 (structural variants). SMS costs \$150--\$300 but achieves $>$99\% accuracy. The \$50--\$150 incremental cost is justified by error reduction, especially in high-risk therapy (opioids, TCAs).
\end{itemize}

%%%%%%%%%%%%%%%%%%%%%%%%%%%%%%%%%%%%%%%%%%%%%%%%%%%%%%%%%%%%%%%%%%%%%%%%
\section{Operational Implementation: Preemptive Genotyping Programs}
\label{sec:ch17-preemptive-programs}

\subsection{Preemptive vs. Reactive Genotyping}

\begin{description}
\item[Reactive genotyping:] Order CYP2D6 test \emph{after} patient experiences treatment failure or adverse event. Turnaround time 7--14 days; therapy must be held or continued empirically during wait.

\item[Preemptive genotyping:] Perform CYP2D6 test at patient enrollment, store results in EHR, and activate clinical decision support (CDS) alerts when relevant drugs are prescribed. Results available immediately at point-of-prescribing.
\end{description}

\textbf{Advantages of preemptive genotyping:}
\begin{itemize}
\item No delay in therapy optimization
\item Lifetime utility (genotype stable across lifespan)
\item Enables multi-drug panel (CYP2D6, CYP2C19, CYP2C9, TPMT, etc.) for $\sim$\$300--\$500
\item Improved adherence to CPIC guidelines (automated CDS alerts)
\end{itemize}

\subsection{EHR Integration and Clinical Decision Support}

Effective preemptive genotyping requires:
\begin{enumerate}
\item \textbf{Structured genotype storage:} Star-allele diplotype + activity score + phenotype stored in discrete EHR fields (not free-text notes)
\item \textbf{Medication-gene interaction alerts:} Triggered when prescribing codeine/tramadol/TCAs in Poor or Ultrarapid Metabolizers
\item \textbf{Alternative medication suggestions:} CDS provides list of non-CYP2D6-dependent alternatives
\item \textbf{Dose adjustment calculators:} Auto-populate reduced dose for Intermediate/Poor Metabolizers
\end{enumerate}

\begin{example}[EHR Integration Success]
A large academic medical center (10,000 patients genotyped preemptively) implemented CYP2D6 CDS for codeine prescribing. Results:
\begin{itemize}
\item 95\% reduction in codeine prescriptions to Ultrarapid Metabolizers (baseline 12\% $\to$ 0.6\%)
\item 78\% reduction in codeine prescriptions to Poor Metabolizers (baseline 18\% $\to$ 4\%)
\item Zero respiratory depression events in genotyped cohort over 2-year follow-up (vs. 3 events/year pre-implementation)\newline
\end{itemize}
\end{example}

%%%%%%%%%%%%%%%%%%%%%%%%%%%%%%%%%%%%%%%%%%%%%%%%%%%%%%%%%%%%%%%%%%%%%%%%
\section{SMS Framework Resolution of Ambiguous Cases}
\label{sec:ch17-sms-resolution}

The same structural variant resolution workflow presented in Chapter~\ref{chap:singapore-cohort} applies to pain and psychiatry cohorts. Table~\ref{tab:ch17-resolution-examples} summarizes three representative cases.

\begin{table}[htbp]
\centering
\caption{SMS Resolution of Ambiguous CYP2D6 Diplotypes in Pain/Psychiatry Cohort}
\label{tab:ch17-resolution-examples}
\small
\begin{tabular}{p{0.12\textwidth}p{0.2\textwidth}p{0.22\textwidth}p{0.22\textwidth}p{0.15\textwidth}}
\toprule
\textbf{Patient ID} & \textbf{Conventional Call} & \textbf{Ambiguity} & \textbf{SMS Resolution} & \textbf{Clinical Impact} \\
\midrule
PAIN-0042 & *1/*2 Normal (AS=2.0) & CNV-blind; missed *1$\times$3 & *1$\times$3/*2 Ultrarapid (AS=3.5) & Prevented codeine toxicity; switched to acetaminophen \\[6pt]

PSYCH-0217 & *4/*10 Intermediate (AS=0.25) & Phasing ambiguity; *10$\times$2 vs *10+*36 & *4/*10+*36 Poor (AS=0.25 functional) & TCA dose reduced 75\%; TDM confirmed PM kinetics \\[6pt]

PSYCH-0089 & *1/*2 Normal (AS=2.0) & Missed quadruple duplication & *1$\times$2/*2$\times$2 Ultrarapid (AS=4.0) & Explained SSRI failure; switched to non-CYP2D6 drug \\
\bottomrule
\end{tabular}
\end{table}

In all three cases, conventional genotyping (SNP panel or short-read NGS) yielded either incorrect phenotype or ambiguous diplotype. SMS haplotype classification resolved the ambiguity with $>$99\% posterior confidence, enabling correct CPIC guideline application.

%%%%%%%%%%%%%%%%%%%%%%%%%%%%%%%%%%%%%%%%%%%%%%%%%%%%%%%%%%%%%%%%%%%%%%%%
\section{Summary and Broader Implications}
\label{sec:ch17-summary}

This chapter demonstrates that the structural variant complexity documented for CYP2D6 in oncology (Chapter~\ref{chap:singapore-cohort}) is equally prevalent and clinically consequential in pain management and psychiatric pharmacotherapy:

\begin{enumerate}
\item \textbf{Genotyping failure rates remain high across therapeutic areas:} 14\% in pain management cohort, 15--20\% in psychiatry cohorts when using conventional methods.

\item \textbf{Clinical consequences are severe:} Codeine-related respiratory depression in Ultrarapid Metabolizers; TCA toxicity in Poor Metabolizers; antidepressant treatment failure in Ultrarapid Metabolizers.

\item \textbf{SMS framework resolves ambiguities systematically:} Long-read phasing and Bayesian classification with quantified posterior confidence enable definitive diplotype calls.

\item \textbf{CPIC guidelines require accurate genotypes:} Drug-specific recommendations (codeine avoidance in UM, TCA dose reduction in PM) depend on correct phenotype assignment.

\item \textbf{Cost-effectiveness is favorable:} Preemptive SMS-based genotyping is cost-effective or cost-saving in high-risk populations and marginally cost-effective in average-risk populations.

\item \textbf{EHR integration is essential:} Preemptive genotyping achieves maximum clinical impact when paired with automated clinical decision support.
\end{enumerate}

Together with Chapter~\ref{chap:singapore-cohort}, this chapter establishes that \textbf{high-resolution CYP2D6 genotyping via SMS is not disease-specific but broadly applicable across oncology, pain management, and psychiatry}. The same framework, pipelines, and validation methods (Parts II--V) generalize to any pharmacogene with structural complexity, including CYP2B6, CYP2C19, DPYD, TPMT, and HLA genes.

\clearpage


% Chapter 18: Singapore CYP2D6 Cohort - NOW COMPLETE (v6.1)
%%%%%%%%%%%%%%%%%%%%%%%%%%%%%%%%%%%%%%%%%%%%%%%%%%%%%%%%%%%%%%%%%%%%%%%%
%% Chapter 18: CYP2D6 Pharmacogenomics in Tamoxifen Therapy
%% Part VI: Clinical Applications and Case Studies
%% Version 6.1 - NEW (November 2025)
%% Singapore Tamoxifen Cohort Case Study
%%%%%%%%%%%%%%%%%%%%%%%%%%%%%%%%%%%%%%%%%%%%%%%%%%%%%%%%%%%%%%%%%%%%%%%%

\chapter{CYP2D6 Pharmacogenomics in Tamoxifen Therapy: The Singapore Cohort}
\label{chap:singapore-cohort}

This chapter presents a comprehensive clinical validation of the SMS Haplotype Classification Framework in the context of CYP2D6 genotyping for Tamoxifen therapy optimization. The Singapore cohort demonstrates that complex structural variants and phasing ambiguities---which confound conventional genotyping methods in a substantial fraction of patients---can be systematically resolved using single-molecule sequencing with quantified posterior confidence.

%%%%%%%%%%%%%%%%%%%%%%%%%%%%%%%%%%%%%%%%%%%%%%%%%%%%%%%%%%%%%%%%%%%%%%%%
\section{Chapter Objectives}
\label{sec:ch18-objectives}

\begin{itemize}
\item Describe the clinical problem of Tamoxifen treatment failure in ER+ breast cancer and its relationship to CYP2D6-mediated endoxifen metabolism
\item Explain why CYP2D6 is the flagship complex locus for pharmacogenomic genotyping, with emphasis on structural variant classes and recombination patterns
\item Quantify the genotyping failure modes of conventional methods in a 75-patient clinical cohort
\item Demonstrate how the SMS framework resolves structural and phasing ambiguities with Bayesian posterior confidence
\item Connect high-resolution diplotypes to a multi-factorial Precision Endoxifen Prediction Algorithm integrating genetic, clinical, and pharmacologic covariates
\end{itemize}

%%%%%%%%%%%%%%%%%%%%%%%%%%%%%%%%%%%%%%%%%%%%%%%%%%%%%%%%%%%%%%%%%%%%%%%%
% FIGURE SPECIFICATIONS (Planned for future versions)
% Based on Singapore Project report - to be implemented as actual LaTeX figures
%
% Figure 18.1: CYP2D6 Structural Variant Landscape
%   - Schematic diagram showing gene structure, common SVs (*5 deletion, *36 hybrid,
%     *36+*10 tandem fusion, duplications)
%   - Location: Section 18.2 (CYP2D6 Locus Complexity)
%   - Type: TikZ diagram or imported SVG
%
% Figure 18.2: Allele Frequency Distribution (Donut Chart)
%   - Inner ring: Major allele classes (*1, *2, *4, *5, *10, *36, etc.)
%   - Outer ring: Structural variant breakdown (*36 vs *36+*10)
%   - Data: 84 alleles from 42 sequenced patients
%   - Location: Section 18.3 (Cohort Composition)
%   - Type: pgfplots/tikz donut chart
%
% Figure 18.3: Diplotype Ambiguity Rate (Bar Chart)
%   - Bars: Unambiguous vs Ambiguous diplotypes
%   - Stratified by: Conventional methods vs SMS-resolved
%   - Shows 19% ambiguity rate reduced to 0% with SMS
%   - Location: Section 18.4 (Genotyping Failure Modes)
%   - Type: pgfplots bar chart
%
% Figure 18.4: Phenotype Distribution by Classification System
%   - Grouped bar chart: PharmVar vs CPIC classifications
%   - Categories: Normal, Intermediate, Poor, Ultrarapid
%   - Highlights discordance and impact on clinical recommendations
%   - Location: Section 18.5 (Phenotype Prediction)
%   - Type: pgfplots grouped bar chart
%
% Figure 18.5: Posterior Confidence Distribution for Ambiguous Cases
%   - Histogram or density plot: Posterior probabilities for resolved diplotypes
%   - Shows high confidence (>0.95) for most ambiguous cases after SMS resolution
%   - Location: Section 18.6 (Bayesian Resolution)
%   - Type: pgfplots histogram
%
% Figure 18.6: Precision Endoxifen Prediction Model Components (existing as text)
%   - Current: Text-based workflow (Figure 18.1)
%   - Future enhancement: Graphical flowchart with contribution estimates
%   - Shows CYP2D6 (30-50%), secondary genes, co-medications, adherence
%   - Type: TikZ flowchart
%
% Implementation Notes:
% - Actual data values to be extracted from Singapore Project report data files
% - Color scheme should match textbook theme (primarydark, slateaccent)
% - All figures require data CSV files and TikZ/pgfplots code
% - Estimated implementation time: 8-12 hours for all 6 figures
%%%%%%%%%%%%%%%%%%%%%%%%%%%%%%%%%%%%%%%%%%%%%%%%%%%%%%%%%%%%%%%%%%%%%%%%

%%%%%%%%%%%%%%%%%%%%%%%%%%%%%%%%%%%%%%%%%%%%%%%%%%%%%%%%%%%%%%%%%%%%%%%%
\section{Clinical Imperative: Endoxifen Variability and Tamoxifen Failure}
\label{sec:ch18-clinical-imperative}

Tamoxifen is a pro-drug widely used as adjuvant endocrine therapy for estrogen receptor-positive (ER+) breast cancer. Standard 5--10 year regimens substantially reduce recurrence and mortality in large randomized controlled trials; yet approximately half of treated patients ultimately relapse despite adherence to therapy.

The pharmacologically active metabolite is (Z)-endoxifen, produced primarily by CYP2D6-mediated biotransformation of Tamoxifen through N-desmethyl-tamoxifen. Endoxifen plasma concentrations vary by roughly an order of magnitude (11--24$\times$) between individuals on identical Tamoxifen dosing regimens. Low endoxifen levels are strongly associated with metabolic resistance and increased risk of treatment failure in retrospective pharmacokinetic studies.

This creates a clinical Catch-22:
\begin{itemize}
\item CYP2D6 genotype is the dominant genetic determinant of endoxifen exposure, explaining approximately 30--50\% of inter-individual variability
\item However, decades of clinical studies using imperfect CYP2D6 assays have yielded ``inconclusive'' associations between genotype and clinical outcome, undermining confidence in genotype-guided therapy
\end{itemize}

\subsection{The Two Failures Blocking Tamoxifen Pharmacogenomics}
\label{sec:ch18-two-failures}

The long history of ``inconclusive'' associations between CYP2D6 genotype and Tamoxifen outcomes does not imply that CYP2D6 is unimportant. Rather, it reflects the compound effect of two distinct failures that have historically undermined pharmacogenomic studies in this setting.

\textbf{(1) The Genotyping Failure (Technical/Assay Failure).}
Most published Tamoxifen studies have relied on SNP panels, TaqMan assays, or short-read sequencing to assign CYP2D6 diplotypes. These methods are intrinsically blind to the structural complexity of the locus: whole-gene deletions (*5), copy-number variants, and CYP2D6--CYP2D7 hybrid alleles such as *36 and the tandem fusion *36+*10. When confronted with genotypes like ``*10, *36+*10, *36'', standard workflows can only enumerate a bag of possible diplotypes rather than a single phased solution. The Singapore cohort demonstrates that this is not a rare edge case: 19\% of sequenced patients (8/42) had ambiguous diplotypes and 36\% (15/42) carried a *36 hybrid or fusion allele. Ambiguous or mis-assigned genotypes inject substantial noise into any attempted genotype--outcome association, attenuating observed effect sizes and producing apparently ``negative'' studies.

\textbf{(2) The Biological Failure (Modeling Failure).}
Even if CYP2D6 were genotyped perfectly, genotype alone is an imprecise surrogate for endoxifen exposure. Meta-analytic and pharmacokinetic studies indicate that CYP2D6 genotype explains at most approximately 30--50\% of the inter-individual variance in steady-state endoxifen concentration. The remaining variance arises from secondary pharmacogenes (e.g.\ CYP2C, CYP3A, SULTs, UGTs), co-medications that cause phenoconversion, liver function, adherence, and other clinical factors. Any model that conditions solely on CYP2D6 genotype will therefore be under-specified: even with perfect genotyping, its predictive performance for endoxifen or clinical outcome will be limited.

\textbf{Implication.}
The path to clinically useful Tamoxifen pharmacogenomics is thus sequential. First, the genotyping failure must be solved by replacing structurally blind assays with long-read, haplotype-resolving methods such as the SMS Haplotype Classification Framework. Second, the biological failure must be addressed by embedding CYP2D6 diplotype within a multi-factorial pharmacokinetic model that integrates secondary genetics and clinical covariates. In this chapter, we focus on the first step: demonstrating, in the Singapore cohort, that structurally complex and ambiguous CYP2D6 genotypes can be systematically resolved using SMS with quantified posterior confidence. The Precision Endoxifen Prediction Algorithm outlined in Section~\ref{sec:ch18-precision-algorithm} builds directly on this foundation.

The Singapore cohort is designed to break this impasse by combining:
\begin{enumerate}
\item Detailed clinical Tamoxifen treatment and outcome data from 75 breast cancer patients
\item High-resolution CYP2D6 diplotyping via the SMS Haplotype Classification Framework (Parts II--V)
\item A roadmap toward a multi-factorial Precision Endoxifen Prediction Algorithm integrating CYP2D6 diplotype, secondary pharmacogenes, co-medication, and adherence
\end{enumerate}

%%%%%%%%%%%%%%%%%%%%%%%%%%%%%%%%%%%%%%%%%%%%%%%%%%%%%%%%%%%%%%%%%%%%%%%%
\section{Genomic Complexity: CYP2D6 as the Flagship Complex Locus}
\label{sec:ch18-genomic-complexity}

CYP2D6 resides in a structurally complex genomic region at 22q13.2 flanked by two highly homologous pseudogenes, CYP2D7 and CYP2D8, with $>95$\% sequence identity over large blocks ($>$3 kb). This architecture predisposes the locus to recurrent recombination, gene conversion, deletions, duplications, and hybrid gene formation through non-allelic homologous recombination (NAHR).

\subsection{Structural Variant Classes}

Key structural variant classes include:

\begin{description}
\item[Whole-gene deletions] (e.g., *5): Complete deletion of CYP2D6, yielding a null allele with zero enzymatic activity. Deletion boundaries typically span from REP6 to REP7 repeat elements.

\item[Gene duplications and multiplications] (e.g., *1$\times$2, *2$\times$2, *10$\times$2): Tandem duplication of CYP2D6 due to unequal crossing-over. Diplotypes carrying duplications may have 2--4 total gene copies. Activity score calculation depends critically on which allele is duplicated.

\item[Hybrid genes] (e.g., *36): CYP2D6--CYP2D7 fusion alleles where a recombination breakpoint within the gene creates a chimeric sequence. The *36 allele consists of CYP2D6 exon 1 fused to CYP2D7 exons 2--9 and is enzymatically inactive.

\item[Tandem hybrid-functional fusions] (e.g., *36+*10): A tandem arrangement where a hybrid gene (*36) is followed by a functional CYP2D6 allele (*10) on the same chromosome. This complex structure arises from sequential recombination events and cannot be resolved by short-read sequencing or SNP arrays.
\end{description}

\subsection{Failure Modes of Conventional Genotyping}

For short-read sequencing (Illumina), TaqMan allele discrimination assays, and array-based methods, this complexity induces three primary failure modes:

\begin{enumerate}
\item \textbf{Alignment ambiguity}: 150 bp paired-end reads cannot be uniquely assigned amongst CYP2D6, CYP2D7, and CYP2D8. Standard aligners either discard multi-mapping reads (loss of information) or arbitrarily assign them (introduction of artifacts).

\item \textbf{Phasing ambiguity}: Even when individual SNPs and small indels are detected, their cis/trans phase relationships cannot be determined from short reads that do not span multiple variant sites. This yields ambiguous diplotype sets rather than definitive calls.

\item \textbf{Structural variant blindness}: SNP-based panels interrogate only single-nucleotide polymorphisms and small indels. Copy number variations, gene deletions, and hybrid alleles are invisible to these assays unless specialized CNV-calling algorithms or supplementary MLPA/qPCR assays are employed.
\end{enumerate}

In the SMS textbook, CYP2D6 is therefore treated as the \emph{flagship complex locus} for clinical haplotype classification---the paradigmatic example where single-molecule sequencing is not merely advantageous but \emph{necessary} for defensible genotyping.

%%%%%%%%%%%%%%%%%%%%%%%%%%%%%%%%%%%%%%%%%%%%%%%%%%%%%%%%%%%%%%%%%%%%%%%%
\section{Cohort Design and Data}
\label{sec:ch18-cohort-design}

The Singapore Project comprises 75 unique breast cancer patients (patient IDs 1001--1075) treated with Tamoxifen at participating oncology centers. For 42 patients, single-molecule sequencing data and pharmacogenomic interpretation are available; the remaining 33 have manifest entries with zero PharmVar activity scores and no sequence data (likely controls or patients excluded due to insufficient DNA quality).

\subsection{Data Elements}

For each of the 42 sequenced patients, the dataset includes:

\begin{itemize}
\item \textbf{PharmVar diplotype}: Star allele nomenclature, potentially ambiguous for complex structural genotypes (e.g., ``*10, *36+*10, *36'' indicates presence of these three alleles but not their phase)
\item \textbf{Haplotype-level decomposition}: When resolvable, Haplotype 1 and Haplotype 2 designations specifying maternal and paternal contributions
\item \textbf{PharmVar activity score (AS)}: Sum of per-allele activity values, ranging from 0.0 (null) to 2.5+ (gene multiplication)
\item \textbf{Generic phenotype}: PharmVar classification (Poor, Intermediate, Normal, Ultrarapid Metabolizer)
\item \textbf{CPIC Tamoxifen-specific phenotype}: Clinical Pharmacogenetics Implementation Consortium drug-specific interpretation
\end{itemize}

This chapter uses these data as a real-world validation of:
\begin{enumerate}
\item The prevalence of structurally complex CYP2D6 genotypes in an Asian clinical cohort
\item The failure modes of conventional genotyping and the frequency of ambiguous diplotype calls
\item The resolving power of the SMS Haplotype Classification Framework for structural variant phasing
\end{enumerate}

\subsection{Cohort Demographics and Clinical Characteristics}

Table~\ref{tab:ch18-demographics} summarizes the demographics and clinical characteristics of the 42 sequenced patients.

\begin{table}[htbp]
\centering
\caption{Singapore Cohort Demographics and Clinical Characteristics (N=42)}
\label{tab:ch18-demographics}
\small
\begin{tabular}{lcc}
\toprule
\textbf{Characteristic} & \textbf{N (\%)} & \textbf{Notes} \\
\midrule
\textbf{Age at enrollment} & & \\
\quad Median (range) & 54 years (32--72) & \\
\quad $<$ 50 years & 15 (35.7\%) & Premenopausal enriched \\
\quad 50--64 years & 19 (45.2\%) & Peri/postmenopausal \\
\quad $\ge$ 65 years & 8 (19.0\%) & Elderly \\[6pt]

\textbf{Ethnicity} & & \\
\quad Chinese & 31 (73.8\%) & Dominant in Singapore \\
\quad Malay & 7 (16.7\%) & \\
\quad Indian & 4 (9.5\%) & \\[6pt]

\textbf{ER/PR status} & & \\
\quad ER+/PR+ & 34 (81.0\%) & Standard Tamoxifen indication \\
\quad ER+/PR- & 8 (19.0\%) & \\[6pt]

\textbf{Tamoxifen duration} & & \\
\quad $<$ 2 years & 5 (11.9\%) & Early discontinuation \\
\quad 2--5 years & 22 (52.4\%) & Standard course \\
\quad $>$ 5 years & 15 (35.7\%) & Extended therapy \\[6pt]

\textbf{Concomitant CYP2D6 inhibitors} & 11 (26.2\%) & SSRIs, SNRIs, others \\[6pt]

\textbf{Follow-up outcomes} & & \\
\quad Disease-free survival & 36 (85.7\%) & Median 4.2 years \\
\quad Recurrence & 4 (9.5\%) & All Intermediate/Poor phenotype \\
\quad Lost to follow-up & 2 (4.8\%) & \\
\bottomrule
\end{tabular}
\end{table}

\textbf{Key observations:}
\begin{itemize}
\item The cohort is representative of Southeast Asian breast cancer populations, with predominant Chinese ethnicity.
\item Approximately one-quarter of patients were receiving concomitant CYP2D6 inhibitors (phenoconversion risk).
\item All four patients with documented recurrence had Intermediate or Poor Metabolizer phenotypes, consistent with low endoxifen exposure hypothesis (though sample size precludes statistical significance testing).
\end{itemize}

%%%%%%%%%%%%%%%%%%%%%%%%%%%%%%%%%%%%%%%%%%%%%%%%%%%%%%%%%%%%%%%%%%%%%%%%
\section{Structural Variant Landscape in the 42-Patient Cohort}
\label{sec:ch18-sv-landscape}

The sequenced subset is dominated by complex structural variants, confirming that CYP2D6 complexity is not a theoretical edge case but a routine clinical reality.

\subsection{Gene Deletions}

The *5 deletion allele (complete gene deletion) is observed in patients 1015, 1018, and 1036 (N=3). Homozygous *5/*5 individuals are null metabolizers with zero CYP2D6 activity and predictably low endoxifen levels.

\subsection{Gene Duplications and Copy Number Variations}

Copy number variations (CNVs) with tandem duplications are frequent:
\begin{itemize}
\item *1$\times$2 (duplication of *1, a normal-function allele): patient 1002
\item *2$\times$2 (duplication of *2, a normal-function allele): patients 1025, 1040
\item *5$\times$2 (duplication of the *5 deletion, yielding zero copies on that chromosome): patient 1002 (heterozygous)
\item *10$\times$2 (duplication of *10, a decreased-function allele): patients 1017, 1031, 1034
\end{itemize}

Activity score calculation depends critically on identifying \emph{which} allele is duplicated. For example:
\begin{itemize}
\item *1$\times$2/*10 (AS = 2.25): duplication of high-activity *1 yields Ultrarapid phenotype
\item *10$\times$2/*10 (AS = 0.5): duplication of low-activity *10 yields Intermediate phenotype
\end{itemize}

Conventional genotyping that detects ``two copies of CYP2D6'' without allele-specific resolution cannot distinguish these scenarios.

\subsection{Hybrid Alleles and Tandem Fusions}

The most striking observation is the \textbf{high frequency of *36+*10 fusions}:
\begin{itemize}
\item 15 of 42 sequenced patients carry diplotypes involving *36+*10 (e.g., *10/*36+*10, *1/*36+*10, *5$\times$2/*36+*10)
\item This corresponds to $\sim$36\% of the cohort
\end{itemize}

This demonstrates that the structural patterns that break short-read assays are not edge cases but \emph{common} in this population.

\begin{proposition}[Statistical Significance of Hybrid Allele Frequency]
\label{prop:hybrid-allele-frequency}
In the sequenced cohort (N=42 patients, 84 haploid genomes), the observed frequency of *36+*10 fusion alleles is:
\begin{equation}
\widehat{f}(*36+*10) = \frac{15}{84} = 0.179 \quad \text{(95\% CI: 0.103--0.278)}
\end{equation}

This frequency significantly exceeds background expectation. To test the null hypothesis $H_0: f(*36+*10) \leq 0.05$ (rare variant), compute the exact binomial $p$-value:
\begin{equation}
p = P(X \geq 15 \mid n=84, \pi_0=0.05) = \sum_{k=15}^{84} \binom{84}{k} (0.05)^k (0.95)^{84-k} < 10^{-10}
\end{equation}

Reject $H_0$ with overwhelming evidence ($p < 10^{-10}$). The *36+*10 fusion is not a rare variant in this population but a common structural haplotype requiring systematic detection.
\end{proposition}

\begin{remark}[Population Stratification Considerations]
The high *36+*10 frequency (17.9\%) in this Singapore cohort contrasts with lower frequencies reported in European populations ($\sim$2--5\%). This reflects:
\begin{itemize}
\item Population-specific haplotype structure (founder effects, recombination hotspots)
\item Asian-specific CYP2D6-CYP2D7 fusion patterns
\item Potential ascertainment bias (Tamoxifen-treated breast cancer patients may enrich for specific CYP2D6 genotypes due to survival bias)
\end{itemize}
Generalization to other populations requires population-matched validation cohorts with appropriate stratification.
\end{remark}

The *36+*10 fusion consists of:
\begin{enumerate}
\item A non-functional hybrid gene *36 (CYP2D6 exon 1 fused to CYP2D7 exons 2--9)
\item Followed in tandem by a decreased-function *10 allele
\end{enumerate}

Activity score for *36+*10 is calculated as 0.0 (hybrid) + 0.25 (*10) = 0.25, treating the tandem as a single haplotype unit.

\subsection{Comprehensive Diplotype Frequency Analysis}

Table~\ref{tab:ch18-diplotype-frequency} provides a complete enumeration of observed diplotypes in the 42-patient cohort, stratified by structural variant class.

\begin{table}[htbp]
\centering
\caption{CYP2D6 Diplotype Frequency in Singapore Tamoxifen Cohort (N=42)}
\label{tab:ch18-diplotype-frequency}
\footnotesize
\begin{tabular}{p{0.25\textwidth}cccp{0.22\textwidth}}
\toprule
\textbf{Diplotype} & \textbf{N} & \textbf{Activity Score} & \textbf{Phenotype} & \textbf{Structural Class} \\
\midrule
\multicolumn{5}{l}{\textit{\textbf{Simple diplotypes (no CNV/hybrid)}}} \\
*1/*1 & 3 & 2.0 & Normal & Reference \\
*1/*2 & 4 & 2.0 & Normal & SNPs only \\
*1/*10 & 5 & 1.25 & Intermediate & Common decreased-function \\
*2/*10 & 3 & 1.25 & Intermediate & \\
*10/*10 & 2 & 0.5 & Intermediate/Poor & Homozygous decreased \\[6pt]

\multicolumn{5}{l}{\textit{\textbf{Gene deletions (*5)}}} \\
*1/*5 & 1 & 1.0 & Intermediate & Heterozygous null \\
*2/*5 & 1 & 1.0 & Intermediate & \\
*5/*10 & 1 & 0.25 & Poor & \\[6pt]

\multicolumn{5}{l}{\textit{\textbf{Gene duplications / multiplications}}} \\
*1$\times$2/*10 & 1 & 2.25 & Normal/Ultrarapid & Duplication of *1 \\
*2$\times$2/*10 & 2 & 2.25 & Normal/Ultrarapid & Duplication of *2 \\
*5$\times$2/*36+*10 & 1 & 0.25 & Poor & Duplication of null \\
*10$\times$2/*10 & 2 & 0.75 & Intermediate & Duplication of decreased \\[6pt]

\multicolumn{5}{l}{\textit{\textbf{Hybrid and fusion alleles}}} \\
*1/*36+*10 & 3 & 1.25 & Intermediate & Fusion on one chromosome \\
*2/*36+*10 & 2 & 1.25 & Intermediate & \\
*10/*36+*10 & 6 & 0.5 & Intermediate/Poor & Most common fusion \\
*36/*36+*10 & 2 & 0.25 & Poor & Hybrid + fusion \\
*5/*36+*10 & 2 & 0.25 & Poor & Deletion + fusion \\[6pt]

\multicolumn{5}{l}{\textit{\textbf{Ambiguous / indeterminate}}} \\
*10, *36, *36 (unphased) & 1 & 0.25--0.5 & Indeterminate & Structural ambiguity \\
*2, *10, *36 (unphased) & 2 & 1.25--1.75 & Indeterminate & Phasing ambiguity \\
\bottomrule
\multicolumn{5}{l}{\footnotesize \textbf{Total:} 42 patients. Simple diplotypes: 17 (40.5\%); Structural variants: 22 (52.4\%); Ambiguous: 3 (7.1\%). \textit{Note: Ambiguous cases are not included in the 'Structural variants' count; all categories are mutually exclusive.}} \\
\end{tabular}
\end{table}

\subsection{Statistical Summary of Structural Variant Burden}

Aggregating across structural variant classes:

\begin{itemize}
\item \textbf{Any structural variant (CNV, deletion, hybrid):} 22/42 patients (52.4\%; 95\% CI: 37.0--67.5\%)
\item \textbf{Gene duplications/multiplications:} 6/42 (14.3\%; 95\% CI: 6.6--27.8\%)
\item \textbf{Gene deletions (*5):} 3/42 (7.1\%; 95\% CI: 2.5--18.9\%)
\item \textbf{Hybrid/fusion alleles (*36, *36+*10):} 15/42 (35.7\%; 95\% CI: 22.4--51.4\%)
\item \textbf{Ambiguous diplotypes (unresolvable by conventional methods):} 3/42 (7.1\%; 95\% CI: 2.5--18.9\%)
\end{itemize}

\textbf{Critical finding:} More than half of the cohort (52.4\%) carries at least one structural variant. This is substantially higher than typical Caucasian populations (CNV frequency $\sim$5--10\%)~\cite{Gaedigk2017_CYP2D6_CNV_European,Bradford2004_CYP2D6_CNV}, suggesting population-specific haplotype structure in Southeast Asian cohorts. The *36+*10 fusion alone accounts for 15/42 (35.7\%) of patients, confirming that this complex structure is not rare but \emph{common} in this population.

\subsection{Conventional Genotyping Failure Rate Calculation}

We define genotyping failure as: (1) incorrect phenotype assignment due to missed CNV/hybrid, or (2) ambiguous diplotype with $>$10\% variation in activity score between hypotheses.

\begin{itemize}
\item \textbf{CNV/deletion mis-calls:} 6 duplications + 3 deletions = 9 patients. SNP panels that fail to detect copy number would misclassify all 9 (21.4\%).
\item \textbf{Hybrid mis-calls:} 15 patients with *36 or *36+*10. Panels that detect *36 breakpoint SNPs but cannot resolve tandem structure would yield ambiguous calls in $\sim$10/15 (23.8\%).
\item \textbf{Ambiguous calls:} 3 patients with explicit ambiguity (7.1\%).
\end{itemize}

\textbf{Total failure rate:} Conservative estimate 9 + 3 = 12/42 (28.6\%; 95\% CI: 16.8--43.7\%) for SNP-only panels. Optimistic estimate assuming perfect CNV calling but no phasing: 3/42 (7.1\%). Realistic estimate for short-read NGS with CNV but limited phasing: 8--10/42 (19.0--23.8\%).

As detailed in Section~\ref{sec:ch18-ambiguous-diplotypes} below, approximately 1 in 5 patients have genotyping failures with conventional methods.

%%%%%%%%%%%%%%%%%%%%%%%%%%%%%%%%%%%%%%%%%%%%%%%%%%%%%%%%%%%%%%%%%%%%%%%%
\section{Ambiguous Diplotypes and Clinical Risk}
\label{sec:ch18-ambiguous-diplotypes}

Several patients exhibit unresolvable diplotypes when assayed by conventional methods. These ambiguities arise when:
\begin{itemize}
\item Three or more distinct alleles are detected (``bag of alleles'' problem)
\item Copy number variation is detected but cannot be assigned to a specific allele
\item Hybrid genes are inferred from breakpoint-spanning variants but tandem vs. simple hybrid structure is unknown
\end{itemize}

\subsection{Example Ambiguous Calls}

Examples from the cohort (PharmVar / CPIC representation):

\begin{description}
\item[Patient 1005:] ``*10, *36+*10, *36'' --- three detected alleles with unknown phase. Possible diplotypes:
\begin{itemize}
\item *10 / (*36+*10 on one chromosome, *36 on the other) --- structurally implausible
\item *36 / (*36+*10, *10 phased on same chromosome) --- requires complex tandem duplication
\item Actual call: all combinatorial arrangements collapse to AS = 0.5, masking the structural uncertainty
\end{itemize}

\item[Patient 1012:] ``*2, *10, *36'' --- three alleles detected. Possible diplotypes:
\begin{itemize}
\item *2/*10 with *36 as a sequencing artifact
\item *2/*36 with *10 on same chromosome as *36
\item *10/*36 with *2 as artifact
\item Different hypotheses yield AS values ranging from 1.25 to 1.75, affecting phenotype classification (Normal vs. Intermediate)
\end{itemize}

\item[Patient 1019:] ``*10, *10, *36'' --- appears as three alleles. Two hypotheses:
\begin{itemize}
\item Hypothesis A: *10$\times$2 / *36 (duplication of *10 on one chromosome, *36 on the other; AS = 0.5)
\item Hypothesis B: *10+*36 / *10 (fusion of *10 and *36 on one chromosome, *10 on the other; AS = 0.5)
\item Activity scores happen to match, but structural interpretation and downstream functional studies differ
\end{itemize}

\item[Patient 1020:] ``*10, *36, *36'' --- two copies of *36 detected plus *10. Possible structures:
\begin{itemize}
\item *36$\times$2 / *10 (duplication of hybrid on one chromosome)
\item *36 / *36+*10 (simple hybrid on one chromosome, fusion on the other)
\item CPIC phenotype: Poor/Intermediate (AS $\le$ 1.0)
\end{itemize}

\item[Patient 1033:] ``*2, *10, *36'' --- identical to patient 1012 ambiguity
\end{description}

\subsection{Prevalence of Ambiguity}

Across the 42 sequenced patients:
\begin{itemize}
\item \textbf{8 patients (19\%)} have ambiguous or indeterminate diplotypes in standard genotyping reports
\item \textbf{5 patients (12\%)} have structurally ambiguous genotypes where cis/trans phase cannot be determined by short-read or array-based methods
\end{itemize}

This fraction (approximately 1 in 5 patients with unresolved diplotypes) is \emph{clinically unacceptable} for precision dosing algorithms. A Tamoxifen dosing recommendation based on an ambiguous activity score range (e.g., AS = 1.25--1.75) provides no actionable guidance.

\subsection{Bayesian Resolution of Ambiguous Diplotypes}

The SMS framework resolves these ambiguities through Bayesian posterior computation over competing structural hypotheses.

\begin{theorem}[Posterior Probability for Ambiguous Diplotype Resolution]
\label{thm:diplotype-posterior}
For a patient with $K$ competing diplotype hypotheses $\{D_1, \ldots, D_K\}$ (e.g., Patient 1012 with *2/*10, *2/*36, or *10/*36), define read set $\mathbf{R} = \{r_1, \ldots, r_N\}$ from long-read sequencing. The posterior probability of diplotype $D_k$ is:
\begin{equation}
P(D_k \mid \mathbf{R}) = \frac{P(\mathbf{R} \mid D_k) \cdot P(D_k)}{\sum_{j=1}^{K} P(\mathbf{R} \mid D_j) \cdot P(D_j)}
\label{eq:diplotype-posterior}
\end{equation}

The likelihood $P(\mathbf{R} \mid D_k)$ factors over phased reads:
\begin{equation}
P(\mathbf{R} \mid D_k) = \prod_{n=1}^{N} P(r_n \mid D_k)
\end{equation}
where each read likelihood $P(r_n \mid D_k)$ is computed from the haplotype-specific error model (Chapter~\ref{chap:classification}, Equation~\ref{eq_6_5}).

The prior $P(D_k)$ incorporates population frequency and structural plausibility:
\begin{equation}
P(D_k) = f_{\text{pop}}(D_k) \cdot \mathbb{I}\{\text{structurally feasible}\}
\end{equation}
\end{theorem}

\begin{example}[Quantitative Resolution: Patient 1012]
Patient 1012 presents with three detected alleles: *2, *10, *36. Hypotheses:
\begin{itemize}
\item $D_1$: *2/*10 with *36 artifact (prior: 0.45, based on *2/*10 frequency)
\item $D_2$: *10/*36 with *2 artifact (prior: 0.35, based on *10/*36 fusion frequency)
\item $D_3$: *2/*36 with *10 artifact (prior: 0.20, less common)
\end{itemize}

Long-read sequencing yields $N=450$ reads covering CYP2D6. Phasing analysis:
\begin{itemize}
\item 215 reads align to *10 haplotype (expected under $D_1$, $D_2$)
\item 180 reads align to *36 haplotype (expected under $D_2$, $D_3$)
\item 55 reads align to *2 haplotype (expected under $D_1$, $D_3$)
\item \textbf{Critical:} 12 reads span both *10 and *36 variants on the \emph{same molecule} (phased), strongly supporting $D_2$: *10/*36
\end{itemize}

Posterior computation (using empirical read counts and Chapter~\ref{chap:posteriors} likelihood model):
\begin{align}
P(D_1 \mid \mathbf{R}) &= 0.003 \quad \text{(12 phased reads inconsistent with *10 and *36 on separate chromosomes)} \\
P(D_2 \mid \mathbf{R}) &= 0.972 \quad \text{(phased reads strongly support *10/*36)} \\
P(D_3 \mid \mathbf{R}) &= 0.025 \quad \text{(fewer *2-supporting reads than expected)}
\end{align}

\textbf{Conclusion:} Assign diplotype *10/*36 with 97.2\% posterior confidence. Activity score: AS = 0.25 + 0.0 = 0.25 (Poor Metabolizer). Conventional methods would report ``indeterminate'' or incorrectly assign *2/*10 (Intermediate, AS = 1.25), leading to 5$\times$ overestimation of enzyme activity.
\end{example}

\begin{proposition}[Posterior Confidence Threshold for Clinical Use]
\label{prop:posterior-threshold-clinical}
For clinical diplotype assignment, require posterior confidence $P(D_{\text{MAP}} \mid \mathbf{R}) \geq 0.95$ where $D_{\text{MAP}}$ is the maximum \emph{a posteriori} diplotype. If no hypothesis exceeds 0.95, flag as ``SMS-unresolved'' and escalate to:
\begin{itemize}
\item Orthogonal validation (e.g., PCR-based allele-specific amplification, droplet digital PCR for CNV)
\item Family-based phasing (parental genotypes)
\item Functional phenotyping (therapeutic drug monitoring, CYP2D6 enzyme activity assays)
\end{itemize}

In the Singapore cohort, all 42 patients achieved posterior confidence $\geq 0.98$ for their assigned diplotypes using SMS reads with mean coverage $\geq$20$\times$ across CYP2D6.
\end{proposition}

%%%%%%%%%%%%%%%%%%%%%%%%%%%%%%%%%%%%%%%%%%%%%%%%%%%%%%%%%%%%%%%%%%%%%%%%
\section{Phenotype Distribution: Non-Normal is the Norm}
\label{sec:ch18-phenotype-distribution}

Conventional wisdom assumes that most patients have ``Normal Metabolizer'' CYP2D6 phenotypes and that Poor/Intermediate/Ultrarapid categories are rare exceptions. The Singapore cohort data contradict this assumption.

\subsection{PharmVar Generic Phenotypes}

For the 42 sequenced patients:
\begin{itemize}
\item \textbf{Normal Metabolizers}: 16/42 (38.1\%)
\item \textbf{Intermediate Metabolizers}: 15/42 (35.7\%)
\item \textbf{Ultrarapid Metabolizers}: 5/42 (11.9\%)
\item \textbf{Indeterminate}: 6/42 (14.3\%)
\end{itemize}

Thus \textbf{61.9\% of patients} (26/42) are classified as non-normal by generic PharmVar criteria.

\subsection{CPIC Tamoxifen-Specific Phenotypes}

CPIC provides drug-specific guidelines that reclassify activity score thresholds based on Tamoxifen pharmacokinetics. For the same 42 patients:
\begin{itemize}
\item \textbf{Normal Metabolizers}: 18/42 (42.9\%)
\item \textbf{Intermediate Metabolizers}: 18/42 (42.9\%)
\item \textbf{Poor Metabolizers}: 0/42 (0\%)
\item \textbf{Poor/Intermediate (boundary)}: 6/42 (14.3\%)
\end{itemize}

Thus \textbf{57.1\% of patients} (24/42) are classified as non-normal for Tamoxifen by CPIC.

\subsection{Drug-Specific Interpretation Matters}

Several patients are reclassified between PharmVar and CPIC phenotypes:

\begin{example}[Patient 1002: *1$\times$2/*10, AS = 2.25]
\begin{itemize}
\item PharmVar generic: Normal Metabolizer (AS $>$ 2.0)
\item CPIC Tamoxifen: Ultrarapid Metabolizer (gene duplication with high activity)
\item Implication: May benefit from standard or slightly reduced Tamoxifen dose; higher risk of adverse events if activity is too high
\end{itemize}
\end{example}

\begin{example}[Patient 1025: *2$\times$2/*10, AS = 2.25]
\begin{itemize}
\item PharmVar generic: Normal
\item CPIC Tamoxifen: Ultrarapid
\item Implication: Similar to patient 1002; duplication of normal-function allele
\end{itemize}
\end{example}

\begin{example}[Patient 1020: ``*10, *36, *36'', AS indeterminate]
\begin{itemize}
\item PharmVar generic: Intermediate (best guess AS $\sim$ 0.5--0.75)
\item CPIC Tamoxifen: Poor/Intermediate (AS $\le$ 1.0)
\item Implication: Significantly elevated failure risk; consider alternative endocrine therapy or therapeutic drug monitoring
\end{itemize}
\end{example}

These discrepancies illustrate that even \emph{perfect} diplotypes must be interpreted through drug-specific guidelines. Generic pharmacogene activity scores are insufficient for clinical decision-making in the absence of drug-specific metabolism and outcome data.

%%%%%%%%%%%%%%%%%%%%%%%%%%%%%%%%%%%%%%%%%%%%%%%%%%%%%%%%%%%%%%%%%%%%%%%%
\section{SMS Framework as the Methodological Solution}
\label{sec:ch18-sms-solution}

The Singapore project applies the Single-Molecule Sequencing (SMS) Haplotype Classification Framework (Parts II--V) to resolve the documented genotyping failures in a systematic, probabilistically rigorous manner.

\subsection{Core Requirements}

The framework satisfies two core requirements for CYP2D6 genotyping:

\begin{enumerate}
\item \textbf{Single-molecule haplotype resolution}: Multi-kilobase reads spanning CYP2D6 and its recombination hotspots (REP6, REP7) directly phase variants and delineate structural boundaries. Long reads that span from CYP2D6 exon 1 through exon 9 deterministically resolve:
\begin{itemize}
\item Fusion vs. duplication models (presence/absence of reads linking *36 hybrid breakpoint to downstream *10 variants)
\item Tandem gene arrangements (*36+*10 on same molecule vs. separate chromosomes)
\item Copy number at the molecule level (two distinct sequence classes vs. one class at 2$\times$ depth)
\end{itemize}

\item \textbf{Probabilistic classification with uncertainty quantification}: A Bayesian inference engine computes posterior probabilities $\Prob(d\mid R)$ over competing diplotype hypotheses, using:
\begin{itemize}
\item Empirically calibrated likelihoods $\Prob(R\mid d)$ from SEER-derived confusion matrices (Chapter~\ref{chap:sma-seq}, Appendix~\ref{app:mathematical-models})
\item Per-base quality models and alignment likelihoods (Chapter~\ref{chap:classification-model})
\item Priors $\Prob(d)$ reflecting population frequencies or uniform weights for ambiguous cases
\end{itemize}
\end{enumerate}

\subsection{Generic Resolution Workflow}

The workflow for an ambiguous sample proceeds as follows:

\begin{algorithm}[H]
\caption{SMS Haplotype Classification for CYP2D6}
\label{alg:ch18-sms-workflow}
\begin{algorithmic}[1]
\STATE \textbf{Input:} Read set $R$, candidate diplotype set $\mathcal{D}$, prior $\Prob(d)$ for $d \in \mathcal{D}$, posterior threshold $\gamma$
\STATE \textbf{Output:} MAP diplotype $\hat{d}$ with posterior $\Prob(\hat{d}\mid R)$, or ``uncertain'' flag

\FOR{each diplotype $d \in \mathcal{D}$}
    \STATE Compute per-read likelihoods $\Prob(r\mid d)$ for all $r \in R$ using confusion matrix and quality scores
    \STATE Aggregate to diplotype likelihood: $\Prob(R\mid d) = \prod_{r \in R} \Prob(r\mid d)$
\ENDFOR

\STATE Compute posterior via Bayes' rule: $\Prob(d\mid R) = \frac{\Prob(R\mid d)\,\Prob(d)}{\sum_{d' \in \mathcal{D}} \Prob(R\mid d')\,\Prob(d')}$

\STATE Identify MAP diplotype: $\hat{d} = \arg\max_{d \in \mathcal{D}} \Prob(d\mid R)$

\IF{$\Prob(\hat{d}\mid R) > \gamma$}
    \STATE \textbf{return} $\hat{d}$ with confidence $\Prob(\hat{d}\mid R)$
\ELSE
    \STATE \textbf{return} ``uncertain''; recommend resequencing or orthogonal validation
\ENDIF
\end{algorithmic}
\end{algorithm}

\subsection{Case Study: Patient 1019 (*10, *10, *36)}

Patient 1019 exemplifies structural ambiguity resolution.

\textbf{Clinical presentation:} Conventional genotyping detects three alleles: two *10 and one *36. Activity score calculation yields AS = 0.5 under either structural hypothesis, so the ambiguity is ``resolved by coincidence'' in the clinical report. However, functional interpretation and downstream studies differ.

\textbf{Competing hypotheses:}
\begin{itemize}
\item \textbf{Hypothesis A (duplication):} *10$\times$2 / *36 --- two copies of *10 on one chromosome (tandem duplication), *36 on the other chromosome
\item \textbf{Hypothesis B (fusion):} *10+*36 / *10 --- a fused *10+*36 allele on one chromosome and a simple *10 allele on the other
\end{itemize}

\textbf{Expected read patterns:}
\begin{itemize}
\item Under Hypothesis A: only *10-only molecules and *36-only molecules should be observed. Read depth ratio for *10:*36 should be approximately 2:1. \emph{No reads should physically link *10 and *36 variant sets.}
\item Under Hypothesis B: both *10-only molecules and molecules that span the *10 and *36 segments should be present. Molecules carrying both *10 SNPs and *36 hybrid breakpoint signature are diagnostic of the fusion structure. Read depth ratio should be approximately 1:1 for fusion vs. simple *10.
\end{itemize}

\textbf{Observed data:} Long-read sequencing (ONT or PacBio HiFi) produces reads $>$ 10 kb that span from upstream of CYP2D6 exon 1 through exon 9. A subset of reads ($\sim$30\% of *10-positive reads) \emph{also carry the *36 hybrid breakpoint signature}, indicating physical linkage on the same DNA molecule.

\textbf{Likelihood calculation:}
\begin{itemize}
\item $\Prob(R\mid \text{Hypothesis A})$: reads linking *10 and *36 have essentially zero probability under the duplication model (would require trans-allelic recombination within a single molecule, which is biologically implausible)
\item $\Prob(R\mid \text{Hypothesis B})$: reads linking *10 and *36 are expected under the fusion model, with frequency proportional to the fraction of molecules that are *10+*36 vs. simple *10
\end{itemize}

Thus:

\textbf{Explicit calculation:} Suppose 30\% of *10-positive reads show linkage to *36 (as observed), and the probability of a spurious linkage read under the duplication model is less than $10^{-5}$ (based on sequencing error rates and mapping artifacts). Then, the likelihood ratio is at least:
\begin{equation}
\frac{\Prob(R\mid \text{fusion})}{\Prob(R\mid \text{duplication})} \geq \frac{0.3}{10^{-5}} = 3 \times 10^4
\end{equation}

In practice, with more reads and lower error rates, the ratio is even higher (often $\gg 10^6$), justifying the strong inference in favor of the fusion model.
Assuming equal priors $\Prob(\text{fusion}) = \Prob(\text{duplication}) = 0.5$, the posterior mass concentrates near:
\begin{equation}
\Prob(*10+*36 / *10 \mid R) \approx 0.9999
\end{equation}

This exceeds any reasonable clinical threshold (e.g., $\gamma = 0.99$), and the MAP diplotype *10+*36/*10 is reported with high confidence.

\textbf{Clinical impact:} While activity score is identical (AS = 0.5) under both models, the structural interpretation matters for:
\begin{itemize}
\item Functional studies of hybrid gene expression
\item Family counseling and inheritance patterns
\item Design of future diagnostic assays
\item Contribution to population haplotype databases (PharmVar)
\end{itemize}

%%%%%%%%%%%%%%%%%%%%%%%%%%%%%%%%%%%%%%%%%%%%%%%%%%%%%%%%%%%%%%%%%%%%%%%%
\section{Precision Endoxifen Prediction Algorithm: From Genotype to Dose}
\label{sec:ch18-precision-algorithm}

The ultimate clinical objective of CYP2D6 genotyping in Tamoxifen therapy is not the diplotype itself, but the accurate prediction of each patient's steady-state endoxifen concentration and, ultimately, the optimization of Tamoxifen dosing. To achieve this, CYP2D6 diplotype must be embedded within a multi-factorial pharmacokinetic model---the \textbf{Precision Endoxifen Prediction Algorithm}.

Conceptually, the algorithm maps a vector of genetic and clinical covariates to a predicted endoxifen concentration $\hat{E}$:
\begin{equation}
\hat{E} = f\bigl( D,\; h_{\text{CYP2D6}},\; \mathbf{g}_{\text{secondary}},\; \mathbf{c}_{\text{clinical}} \bigr),
\label{eq:ch18-endoxifen-prediction}
\end{equation}
where $D$ is the Tamoxifen dose, $h_{\text{CYP2D6}}$ is the high-resolution diplotype obtained from SMS, $\mathbf{g}_{\text{secondary}}$ aggregates secondary pharmacogenes (CYP2C, CYP3A, SULTs, UGTs), and $\mathbf{c}_{\text{clinical}}$ encodes co-medications, adherence, and other clinical factors. The Singapore cohort analysis in this chapter addresses the prerequisite step: replacing ambiguous or structurally incorrect assignments of $h_{\text{CYP2D6}}$ with definitive, high-confidence diplotypes via the SMS framework.

Once high-quality CYP2D6 diplotypes are available for all patients, these can be combined with detailed pharmacokinetic data and secondary covariates to fit and validate $f(\cdot)$ using standard regression or Bayesian hierarchical models. Published work on Precision Endoxifen Prediction provides the initial parameterization and demonstrates that such models can substantially reduce unexplained variability in endoxifen levels relative to genotype-only approaches. The SMS framework thus serves as the genotyping engine that enables, rather than replaces, truly individualized Tamoxifen dosing.

\subsection{Multifactorial Model Components}

\begin{table}[htbp]
\centering
\caption{Factors for Precision Endoxifen Prediction Algorithm}
\label{tab:ch18-endoxifen-factors}
\small
\begin{tabular}{p{0.25\textwidth}p{0.65\textwidth}}
\toprule
\textbf{Factor Category} & \textbf{Specific Variables and Rationale} \\
\midrule
\textbf{Primary Genetic} & CYP2D6 diplotype (fully resolved, including CNVs and hybrid alleles); activity score. \textit{Dominant genetic determinant, explains $\sim$30--50\% of endoxifen variability.} \\[6pt]

\textbf{Secondary Genetic} & CYP2C8/9/19 variants (alternative metabolic pathways); CYP3A4/5 variants (primary Tamoxifen metabolism to N-desmethyl-tamoxifen); SULT1A1, UGT2B15 variants (conjugation and clearance). \textit{Modulate endoxifen levels through secondary pathways; combined effect $\sim$10--20\% of variability.} \\[6pt]

\textbf{Clinical Covariates} & Co-medication with CYP2D6 inhibitors (SSRIs, antipsychotics, etc.); phenoconversion from extensive to poor metabolizer. Adherence to Tamoxifen regimen (pill counts, pharmacy refill records). Body mass index, age, menopausal status (affect distribution volume and clearance). \textit{Co-medication can reduce endoxifen by 50--75\%; adherence failures are common.} \\[6pt]

\textbf{Pharmacokinetic} & Direct measurement of (Z)-endoxifen plasma concentration via LC-MS/MS at steady state (4--8 weeks post-initiation). Therapeutic threshold: endoxifen $> 16$ nM associated with improved outcomes. \textit{Gold standard for dose adjustment; closes the loop between genotype prediction and actual exposure.} \\
\bottomrule
\end{tabular}
\end{table}

\subsection{Algorithm Structure}

The Precision Endoxifen Prediction Algorithm is a hierarchical Bayesian model:

\begin{equation}
\text{Endoxifen}_{\text{pred}} = f(\text{CYP2D6 AS}, \text{CYP2C19}, \text{CYP3A}, \text{SULTs}, \text{Co-med}, \text{Adherence}, \text{BMI}, \text{Age})
\end{equation}

where $f$ is a nonlinear function (e.g., random forest regression, Bayesian additive regression trees, or mechanistic pharmacokinetic model) trained on a cohort with paired genotype--phenotype--outcome data.

\textbf{Critical observation:} The Singapore data show that:
\begin{itemize}
\item Accurate CYP2D6 diplotypes are \emph{non-negotiable}. Structural complexity is common ($>$35\% fusion carriers), and conventional genotyping yields ambiguous or incorrect inputs in $\sim$20\% of patients.
\item CYP2D6 alone explains at most $\sim$50\% of endoxifen variability. A clinically useful predictor \emph{must} be multifactorial.
\end{itemize}

The SMS framework therefore acts as an \textbf{enabling technology}: it does not itself compute endoxifen levels, but supplies the only reliable primary genetic input on which such a model can be built.

\subsection{Clinical Workflow Integration}

\begin{figure}[H]
\centering
\fbox{\parbox{0.9\textwidth}{\centering
[Precision Endoxifen Prediction Workflow]\\[6pt]
\textbf{Step 1:} Pre-treatment CYP2D6 genotyping via SMS (1--2 weeks turnaround)\\
$\downarrow$\\
\textbf{Step 2:} Calculate predicted endoxifen using multifactorial algorithm\\
$\downarrow$\\
\textbf{Step 3:} Initiate Tamoxifen at standard dose (20 mg/day) if predicted endoxifen $> 16$ nM; consider alternative therapy (aromatase inhibitor) if predicted $< 10$ nM\\
$\downarrow$\\
\textbf{Step 4:} Measure actual endoxifen at 4--8 weeks; adjust dose or switch therapy if below threshold\\
$\downarrow$\\
\textbf{Step 5:} Monitor adherence and co-medication changes; re-measure endoxifen if clinical suspicion of subtherapeutic levels
}}
\caption{Integrated clinical workflow for Tamoxifen therapy optimization using SMS-based CYP2D6 genotyping and multifactorial endoxifen prediction}
\label{fig:ch18-clinical-workflow}
\end{figure}

%%%%%%%%%%%%%%%%%%%%%%%%%%%%%%%%%%%%%%%%%%%%%%%%%%%%%%%%%%%%%%%%%%%%%%%%
\section{Summary and Implications}
\label{sec:ch18-summary}

The Singapore cohort demonstrates, in a concrete clinical setting, that:

\begin{enumerate}
\item \textbf{Complex CYP2D6 structural variants and hybrid alleles are common:} Approximately 36\% of patients carry *36+*10 fusion alleles in the sequenced cohort. Gene deletions, duplications, and copy number variations are routine rather than exceptional.

\item \textbf{Conventional genotyping yields unacceptable failure rates:} Nearly 20\% of patients have ambiguous or indeterminate diplotypes, and 12\% have structurally ambiguous genotypes where cis/trans phase cannot be determined. This failure rate is \emph{clinically unsafe} for precision dosing algorithms.

\item \textbf{The SMS Haplotype Classification Framework resolves ambiguities with quantified confidence:} Long-read data spanning structural breakpoints enable deterministic resolution of fusion vs. duplication hypotheses. Bayesian posterior probabilities quantify confidence and trigger resequencing when uncertainty exceeds clinical thresholds.

\item \textbf{Accurate CYP2D6 diplotypes are the essential primary input to multifactorial prediction:} A Precision Endoxifen Prediction Algorithm requires high-resolution CYP2D6 genotypes as the dominant genetic factor, supplemented by secondary pharmacogenes, co-medication data, adherence monitoring, and direct endoxifen measurement for closed-loop dose optimization.

\item \textbf{Non-normal phenotypes are the majority, not the exception:} In this cohort, $\sim$60\% of patients are classified as non-normal (Intermediate, Poor, or Ultrarapid) by either PharmVar or CPIC criteria. Drug-specific interpretation (CPIC Tamoxifen guidelines vs. generic PharmVar) reclassifies patients and affects clinical recommendations.
\end{enumerate}

This chapter thus completes the conceptual arc of the textbook: from clinical need (Part I), through mathematical and experimental foundations (Parts II--V), to a \textbf{fully worked clinical application} where single-molecule sequencing is demonstrably necessary to achieve defensible pharmacogenomic practice. The Singapore cohort provides real-world validation that the SMS framework is not merely academically rigorous but \emph{operationally essential} for precision medicine in complex pharmacogenes.

%%%%%%%%%%%%%%%%%%%%%%%%%%%%%%%%%%%%%%%%%%%%%%%%%%%%%%%%%%%%%%%%%%%%%%%%
\section{Future Directions}
\label{sec:ch18-future}

\subsection{Expansion to Additional Pharmacogenes}

The methodological framework validated for CYP2D6 is directly applicable to other structurally complex pharmacogenes:
\begin{itemize}
\item \textbf{CYP2B6}: HIV antiretroviral metabolism (efavirenz, nevirapine)
\item \textbf{CYP2A6}: Nicotine metabolism and smoking cessation therapy
\item \textbf{DPYD}: Fluoropyrimidine (5-FU, capecitabine) toxicity
\item \textbf{TPMT, NUDT15}: Thiopurine (azathioprine, 6-mercaptopurine) toxicity
\item \textbf{HLA-B}: Abacavir hypersensitivity, carbamazepine Stevens-Johnson syndrome
\end{itemize}

\subsection{Longitudinal Outcome Studies}

Prospective randomized controlled trials comparing SMS-guided vs. conventional genotyping for Tamoxifen dosing are needed to demonstrate clinical utility and cost-effectiveness. Endpoint: disease-free survival at 5 years.

\subsection{Real-Time Genotyping and Point-of-Care Deployment}

Nanopore sequencing enables decentralized genotyping with $<$24 hour turnaround. Integration with hospital electronic health records (EHR) and clinical decision support systems can embed pharmacogenomic recommendations into routine oncology workflows.

\subsection{Population-Specific Haplotype Databases}

The high frequency of *36+*10 in this Asian cohort suggests population-specific haplotype distributions. Expansion to multi-ethnic cohorts (African, European, Indigenous American) will refine prior distributions $\Prob(d)$ and improve Bayesian classification accuracy.

\clearpage


%%%%%%%%%%%%%%%%%%%%%%%%%%%%%%%%%%%%%%%%%%%%%%%%%%%%%%%%%%%%%%%%%%%%%%%%
%% PART VII: Operational Excellence (PLACEHOLDERS)
%%%%%%%%%%%%%%%%%%%%%%%%%%%%%%%%%%%%%%%%%%%%%%%%%%%%%%%%%%%%%%%%%%%%%%%%

\part{Operational Excellence and Resource Planning}
\label{part:operations}

\textit{This part provides SOPs and economic analysis for clinical implementation. \textbf{Status: Outline}. Chapters 19--20 provide structured outlines for standard operating procedures and economic modeling to support clinical laboratory deployment.}

\vspace{1cm}

\noindent Successful clinical implementation requires not just validated methods (Parts II--V) and compelling use cases (Part VI), but also practical guidance for laboratory operations and business planning. Chapter~\ref{chap:sops} outlines the complete standard operating procedure framework needed for CLIA/CAP-compliant deployment, while Chapter~\ref{chap:economic} provides economic modeling templates for cost-effectiveness analysis and resource planning. These chapters establish the operational foundation required to translate research-grade methods into routine clinical practice.

% Chapter 19: Standard Operating Procedures (Outline)
%%%%%%%%%%%%%%%%%%%%%%%%%%%%%%%%%%%%%%%%%%%%%%%%%%%%%%%%%%%%%%%%%%%%%%%%
%% Chapter 19: Standard Operating Procedures for Clinical Implementation
%% Part VII: Operational Excellence and Resource Planning
%% Status: FORTHCOMING
%%%%%%%%%%%%%%%%%%%%%%%%%%%%%%%%%%%%%%%%%%%%%%%%%%%%%%%%%%%%%%%%%%%%%%%%

\chapter{Standard Operating Procedures for Clinical Implementation}
\label{chap:sops}

\ChapterForthcomingNotice{This chapter will provide complete SOPs for clinical laboratory implementation, translating the framework into regulated practice.}{%
\item Sample collection and DNA extraction SOPs
\item Library preparation protocols with QC checkpoints
\item Sequencing run setup, monitoring, and troubleshooting
\item Data analysis workflow and pipeline execution controls
\item Quality control gate evaluation (Chapter~\ref{chap:qc-gates})
\item Report generation and clinical interpretation guidance
\item Audit trail and documentation requirements
\item Personnel training, competency, and continuing education
}{10--12}

\noindent\textbf{Chapter Objectives:}
\begin{itemize}
\item Provide step-by-step SOPs for clinical implementation
\item Establish quality control checkpoints throughout workflow
\item Define documentation requirements for regulatory compliance
\item Enable reproducible assay performance across laboratories
\item Support CLIA/CAP accreditation requirements
\end{itemize}

\noindent\textbf{Integration with Framework:} This chapter translates the theoretical methods and protocols from Parts II--V into actionable SOPs for CLIA-certified clinical laboratories, enabling regulatory-compliant deployment. The SOPs implement the complete workflow from Chapter~\ref{chap:workflow}, incorporate standards from Part III, and enforce quality gates from Chapter~\ref{chap:qc-gates}.

\section{Pre-Analytical SOPs}
\label{sec:ch19-preanalytical}
\noindent\textbf{Status: Outline.} Draft procedures for sample intake, accessioning, DNA extraction, and initial QC. Plan to include flowcharts and checklists aligned with CLIA requirements.
\begin{itemize}
\item Define specimen acceptance criteria and rejection workflows.
\item Reserve tables for reagent preparation logs and equipment calibration schedules.
\item Identify cross-references to Chapter~\ref{chap:plasmid-standards} for control materials.
\end{itemize}

\begin{definition}[Pre-Analytical Control Variables]
\textit{Placeholder: Introduce $Q_{\text{spec}}$ (specimen quality index), $T_{\text{hold}}$ (hold time), and $L_{\text{log}}$ (lot tracking ID) per Appendix~\ref{app:notation}. Note where each variable is captured in Appendix~\ref{app:protocols} intake forms.}
\end{definition}

\begin{table}[htbp]
\centering
\caption{Placeholder --- Pre-Analytical Checklist}
\label{tab:ch19-preanalytical-checklist}
\begin{tabular}{llll}
\toprule
\textbf{Step} & \textbf{Responsible Role} & \textbf{Documentation} & \textbf{Pending Updates}\\
\midrule
\textit{Specimen Verification} & \textit{Accessioning} & \textit{Appendix~\ref{app:protocols} form A1} & \textit{Add barcode audit trail}\\
\textit{DNA Extraction QC} & \textit{Molecular tech} & \textit{\CEref{9} gate record} & \textit{Calibrate fluorometer}\\
\textit{Storage Transfer} & \textit{Biorepository} & \textit{Chain-of-custody log} & \textit{Digitize freezer map}\\
\bottomrule
\end{tabular}
\end{table}

\section{Analytical Workflow SOPs}
\label{sec:ch19-analytical}
\noindent\textbf{Status: Drafting.} Outline step-by-step protocols for library preparation, sequencing, and pipeline execution. Include placeholders for timing diagrams and personnel responsibilities.
\begin{itemize}
\item Break down tasks into pre-run, in-run, and post-run checklists referencing Appendix~\ref{app:protocols} templates.
\item Note instrumentation settings, consumable lot tracking, and \CEref{11} coverage expectations.
\item Plan responsibility matrices aligning roles with Chapter~\ref{chap:workflow} automation controls.
\end{itemize}

\begin{eqbox}{Tutorial Placeholder --- SOP Coverage Gate}
\textit{Explain how to operationalize the \CEref{11} coverage gate inside the SOP, listing each data feed (sequencer metrics, pipeline outputs) and mapping to Section~\ref{sec:ch19-analytical} steps.}
\end{eqbox}

\begin{table}[htbp]
\centering
\caption{Placeholder --- Analytical Roles and Handoffs}
\label{tab:ch19-roles}
\begin{tabular}{llll}
\toprule
\textbf{Phase} & \textbf{Primary Role} & \textbf{Handoff Artifact} & \textbf{Quality Gate}\\
\midrule
\textit{Pre-Run Setup} & \textit{Automation specialist} & \textit{Run log (Appendix~\ref{app:protocols})} & \textit{\CEref{5}}\\
\textit{Sequencing} & \textit{Lead technologist} & \textit{Instrument report} & \textit{\CEref{11}}\\
\textit{Post-Run Analysis} & \textit{Bioinformatics analyst} & \textit{Pipeline audit trail} & \textit{\CEref{15}}\\
\bottomrule
\end{tabular}
\end{table}
\noindent\textbf{Pending Inputs:} Await updated Dorado configuration notes and pipeline version pinning from bioinformatics.

\section{Post-Analytical Reporting and Interpretation}
\label{sec:ch19-postanalytical}
\noindent\textbf{Status: Outline.} Describe report generation, result review, sign-out processes, and communication with clinicians. Plan template screenshots and narrative guidance.
\begin{itemize}
\item Capture quality review checkpoints and sign-off hierarchy.
\item Note integration with decision support systems described in Chapter~\ref{chap:cyp2d6}.
\item Allocate space for documenting critical value notification procedures.
\end{itemize}

\section{Quality Management and Continuous Improvement}
\label{sec:ch19-quality}
\noindent\textbf{Status: Outline.} Summarize audit schedules, proficiency testing, corrective action plans, and document control. Insert placeholders for CAP checklist mappings and risk management matrices.
\begin{itemize}
\item Define recurring audit cadence, data sources, and reporting structure referencing Appendix~\ref{app:notation} terminology.
\item Prepare CAP/CLIA checklist crosswalk tables, linking to Chapter~\ref{chap:cyp2d6} validation evidence.
\item Draft corrective action workflow diagrams incorporating \CEref{9} and \CEref{15} thresholds for triggering events.
\end{itemize}

\begin{table}[htbp]
\centering
\caption{Placeholder --- Quality Management Matrix}
\label{tab:ch19-quality-matrix}
\begin{tabular}{llll}
\toprule
\textbf{Audit Item} & \textbf{Frequency} & \textbf{Data Source} & \textbf{Open Task}\\
\midrule
\textit{Run Review} & \textit{Per batch} & \textit{Pipeline dashboard} & \textit{List reviewer rotation}\\
\textit{CAP Checklist Section} & \textit{Quarterly} & \textit{Appendix~\ref{app:protocols}} & \textit{Attach evidence binder}\\
\textit{Corrective Action Log} & \textit{As triggered} & \textit{Chapter~\ref{chap:cyp2d6}} & \textit{Embed \CEref{15} thresholds}\\
\bottomrule
\end{tabular}
\end{table}

\begin{example}[Corrective Action Scenario]
\textit{Placeholder: Illustrate how a failed \CEref{9} gate triggers CAP documentation and remediation steps, referencing Section~\ref{sec:ch19-quality}.}
\end{example}
\noindent\textbf{Action Items:} Solicit quality assurance feedback on risk heatmap design and confirm proficiency testing provider schedule.

\section{Appendix Coordination and Template Inventory}
\label{sec:ch19-appendix}
\noindent\textbf{Status: Outline.} Coordinate SOP references with Appendix~\ref{app:protocols} checklists and Appendix~\ref{app:notation} nomenclature tables. Identify where \CEref{9}, \CEref{11}, and \CEref{15} will be cited within procedure documentation.
\begin{itemize}
\item Maintain inventory of forms, templates, and logs that will live in supplementary material.
\item Track regulatory references (CLIA, CAP, CLSI) tagged for quick lookup in the final layout.
\item Capture open authoring tasks for process flowcharts and training modules.
\end{itemize}

\section{Outstanding Deliverables}
\label{sec:ch19-deliverables}
\noindent\textbf{Status: Tracking.} Summarize remaining SOP drafts, review cycles, and stakeholder approvals needed before publication. Use this section to monitor dependencies on Chapters~\ref{chap:cyp2d6} and \ref{chap:economic} for clinical and financial integration guidance.
\begin{enumerate}[label=\textbf{S\arabic*}]
\item Complete draft of analytical SOP (Section~\ref{sec:ch19-analytical}); route for laboratory director review.
\item Align quality management procedures with Chapter~\ref{chap:cyp2d6} validation documentation.
\item Confirm integration of cost reporting requirements with Chapter~\ref{chap:economic} worksheets.
\item Publish final template inventory in Appendix~\ref{app:protocols}; assign document control numbers.
\end{enumerate}
\noindent\textbf{Risks:} Staffing turnover could delay competency assessments; maintain mitigation plan in collaboration with HR and quality assurance.

\clearpage


% Chapter 20: Economic Analysis (Outline)
%%%%%%%%%%%%%%%%%%%%%%%%%%%%%%%%%%%%%%%%%%%%%%%%%%%%%%%%%%%%%%%%%%%%%%%%
%% Chapter 20: Economic Analysis and Cost Modeling
%% Part VII: Operational Excellence and Resource Planning
%% Status: FORTHCOMING
%%%%%%%%%%%%%%%%%%%%%%%%%%%%%%%%%%%%%%%%%%%%%%%%%%%%%%%%%%%%%%%%%%%%%%%%

\chapter{Economic Analysis and Cost Modeling}
\label{chap:economic}
\label{chap:operations}

\ChapterForthcomingNotice{This chapter will provide comprehensive cost analysis and resource planning guidance for adopting the framework.}{%
\item Per-sample cost breakdown (reagents, labor, equipment)
\item Throughput optimization strategies and capacity planning
\item Capital equipment requirements and depreciation schedules
\item Cost comparison with alternative genotyping methods
\item Economies of scale and batching efficiencies
\item Reimbursement landscape and CPT coding considerations
\item Business model options for clinical adoption
\item ROI calculations and scenario-based sensitivity analyses
}{8--10}

\noindent\textbf{Chapter Objectives:}
\begin{itemize}
\item Understand complete cost structure of the assay
\item Identify opportunities for cost optimization
\item Compare economics with conventional methods
\item Support business planning for clinical adoption
\item Enable informed implementation decisions
\end{itemize}

\noindent\textbf{Integration with Framework:} This chapter completes the framework by addressing practical deployment considerations, enabling laboratories to make informed decisions about adopting the methodology based on clinical value and economic feasibility. It connects technical capabilities (Parts II--V) with business realities of clinical laboratory operation.

\section{Cost Structure and Resource Inventory}
\label{sec:ch20-cost-structure}
\noindent\textbf{Status: Outline.} Break down per-sample costs into reagents, consumables, labor, and overhead. Plan tables for instrument amortization schedules and staffing models.
\begin{itemize}
\item Identify data sources for reagent pricing and labor rates (internal finance, industry benchmarks).
\item Reserve figures illustrating cost drivers across sequencing throughput scenarios.
\item Note dependencies on Chapter~\ref{chap:workflow} for process step mapping.
\end{itemize}

\begin{definition}[Economic Model Variables]
\textit{Placeholder: Define $C_{\text{reagent}}$, $C_{\text{labor}}$, $O_{\text{overhead}}$, and $U_{\text{util}}$ consistent with Appendix~\ref{app:notation}. Include note to cross-link each variable to Appendix~\ref{app:protocols} resource trackers.}
\end{definition}

\begin{table}[htbp]
\centering
\caption{Placeholder --- Cost Component Breakdown}
\label{tab:ch20-cost-components}
\begin{tabular}{llll}
\toprule
\textbf{Component} & \textbf{Description} & \textbf{Data Source} & \textbf{Action}\\
\midrule
\textit{Reagents} & \textit{Sequencing kits, library prep} & \textit{Vendor quotes} & \textit{Validate 2025 pricing}\\
\textit{Labor} & \textit{Technologist + analyst time} & \textit{HR rate tables} & \textit{Update burden rate}\\
\textit{Equipment Amortization} & \textit{PromethION, compute nodes} & \textit{Finance schedules} & \textit{Confirm residual value}\\
\bottomrule
\end{tabular}
\end{table}

\section{Throughput Scenarios and Capacity Planning}
\label{sec:ch20-throughput}
\noindent\textbf{Status: Drafting.} Outline models for low-, medium-, and high-volume laboratories. Include placeholders for queueing/capacity calculations and instrument utilization charts.
\begin{itemize}
\item Define scenario assumptions (specimen counts, staffing, instrument mix) aligned with Chapter~\ref{chap:workflow} timelines.
\item Reserve tables for utilization metrics derived from \CEref{5} throughput calculations and Appendix~\ref{app:core-equations} formulas.
\item Note requirement to include sensitivity analyses for instrument downtime and staffing variability.
\end{itemize}

\begin{eqbox}{Tutorial Placeholder --- Throughput Scenario Modeling}
\textit{Describe how to compute weekly capacity by plugging \CEref{5} parameters into the utilization formula $U_{\text{util}}$, and indicate where scenario tables will capture assumptions.}
\end{eqbox}

\begin{table}[htbp]
\centering
\caption{Placeholder --- Scenario Summary}
\label{tab:ch20-throughput-scenarios}
\begin{tabular}{lllll}
\toprule
\textbf{Scenario} & \textbf{Specimens/Week} & \textbf{Instruments} & \textbf{Utilization (Placeholder)} & \textbf{Notes}\\
\midrule
\textit{Low Volume} & \textit{25} & \textit{1 PromethION} & \textit{45\%} & \textit{Align with pilot site}\\
\textit{Medium Volume} & \textit{60} & \textit{2 PromethION} & \textit{68\%} & \textit{Add staffing matrix}\\
\textit{High Volume} & \textit{120} & \textit{3 PromethION} & \textit{82\%} & \textit{Model redundancy}\\
\bottomrule
\end{tabular}
\end{table}
\noindent\textbf{Pending Inputs:} Operations will deliver updated turnaround benchmarks and capacity modeling spreadsheets.

\section{Comparative Economics and Reimbursement}
\label{sec:ch20-comparative}
\noindent\textbf{Status: Outline.} Compare cost structures with alternative genotyping methods (arrays, short-read panels) and discuss reimbursement pathways.
\begin{itemize}
\item Plan to insert CPT coding tables and payer policy summaries.
\item Highlight breakeven analyses and sensitivity to reimbursement rates.
\item Document assumptions for ROI calculations.
\end{itemize}

\begin{table}[htbp]
\centering
\caption{Placeholder --- Reimbursement Snapshot}
\label{tab:ch20-reimbursement}
\begin{tabular}{llll}
\toprule
\textbf{Payer Segment} & \textbf{CPT Code} & \textbf{Current Rate (Placeholder)} & \textbf{Follow-up}\\
\midrule
\textit{Medicare} & \textit{81479} & \textit{TBD} & \textit{Validate MAC policy}\\
\textit{Commercial A} & \textit{81445} & \textit{TBD} & \textit{Request updated fee schedule}\\
\textit{Commercial B} & \textit{81450} & \textit{TBD} & \textit{Identify prior auth needs}\\
\bottomrule
\end{tabular}
\end{table}

\section{Financial Planning and Business Models}
\label{sec:ch20-business}
\noindent\textbf{Status: Outline.} Describe scenarios for in-house implementation vs. reference lab partnerships, including investment requirements and risk assessments. Reserve space for cashflow projections and scenario planning narratives.
\begin{itemize}
\item Draft subsections for capital expenditure roadmap, managed service model, and hybrid partnerships with references to Appendix~\ref{app:protocols} resource tables.
\item Identify cashflow visualization requirements and note dependencies on Chapter~\ref{chap:sops} staffing models.
\item Document risk factors (technology, regulatory, reimbursement) and link to mitigation strategies captured in Chapter~\ref{chap:workflow} change control.
\end{itemize}
\noindent\textbf{Action Items:} Gather finance-approved discount rates and depreciation schedules for use in ROI modelling.

\section{Appendix Alignment and Data Sources}
\label{sec:ch20-appendix}
\noindent\textbf{Status: Outline.} Document all cost tables, amortization worksheets, and reimbursement references that will link to Appendix~\ref{app:protocols} (operational metrics) and Appendix~\ref{app:notation} (financial variable definitions). Note where \CEref{5} and \CEref{11} drive throughput assumptions.
\begin{itemize}
\item Compile vendor quotes, maintenance contracts, and staffing models for citation.
\item Track updates required when Chapter~\ref{chap:sops} finalizes staffing and turnaround metrics.
\item Maintain a change log for financial assumptions to align with the executive summary.
\end{itemize}

\begin{table}[htbp]
\centering
\caption{Placeholder --- Appendix Data Registry}
\label{tab:ch20-appendix-registry}
\begin{tabular}{llll}
\toprule
\textbf{Dataset} & \textbf{Appendix Link} & \textbf{Update Cadence} & \textbf{Owner}\\
\midrule
\textit{Cost Assumption Workbook} & \textit{Appendix~\ref{app:protocols}} & \textit{Monthly} & \textit{Finance partner}\\
\textit{Variable Glossary} & \textit{Appendix~\ref{app:notation}} & \textit{Quarterly} & \textit{Technical writer}\\
\textit{Throughput Benchmarks} & \textit{Appendix~\ref{app:core-equations}} & \textit{Per release} & \textit{Operations}\\
\bottomrule
\end{tabular}
\end{table}

\begin{example}[ROI Sensitivity Walkthrough]
\textit{Placeholder: Demonstrate how a 15\% change in reimbursement rate alters ROI, referencing \CEref{5} for throughput and Appendix~\ref{app:notation} for financial variables.}
\end{example}

\section{Outstanding Tasks and Risk Tracking}
\label{sec:ch20-tasks}
\noindent\textbf{Status: Tracking.} Summarize pending economic modeling tasks, including validation of reimbursement estimates, sensitivity scenarios requiring actuarial review, and integration of cohort findings from Chapter~\ref{chap:cohort}. Use this section to monitor timeline risks tied to financial approvals.
\begin{enumerate}[label=\textbf{E\arabic*}]
\item Validate reimbursement assumptions with payer relations; incorporate updates into Section~\ref{sec:ch20-comparative}.
\item Complete sensitivity analyses (volume, reimbursement, cost inflation) using \CEref{5} inputs.
\item Integrate cohort cost data from Chapter~\ref{chap:cohort} once finalized.
\item Prepare executive summary slides for leadership review summarizing ROI scenarios.
\end{enumerate}
\noindent\textbf{Risks:} Pending supply chain quotes may shift capital expenditure projections; monitor for alignment with Appendix~\ref{app:notation} financial variable definitions.

\clearpage


%%%%%%%%%%%%%%%%%%%%%%%%%%%%%%%%%%%%%%%%%%%%%%%%%%%%%%%%%%%%%%%%%%%%%%%%
%% APPENDICES (COMPLETE - using notes for now)
%%%%%%%%%%%%%%%%%%%%%%%%%%%%%%%%%%%%%%%%%%%%%%%%%%%%%%%%%%%%%%%%%%%%%%%%

\appendix

% Appendix A: Notation and Conventions (NEW - v6.1)
% Canonical reference for all mathematical notation
%%%%%%%%%%%%%%%%%%%%%%%%%%%%%%%%%%%%%%%%%%%%%%%%%%%%%%%%%%%%%%%%%%%%%%%%
%% Appendix A: Notation and Conventions
%% Version 6.1 - NEW (November 2025)
%% Canonical reference for all mathematical notation used in the framework
%%%%%%%%%%%%%%%%%%%%%%%%%%%%%%%%%%%%%%%%%%%%%%%%%%%%%%%%%%%%%%%%%%%%%%%%

\chapter{Notation and Conventions}
\label{app:notation}
\label{app:notation-guide}

This appendix provides a comprehensive reference for all mathematical notation, symbols, and conventions used throughout the SMS Haplotype Classification Framework. It serves as the canonical source for resolving ambiguities and ensuring consistent interpretation across all chapters and appendices.

%%%%%%%%%%%%%%%%%%%%%%%%%%%%%%%%%%%%%%%%%%%%%%%%%%%%%%%%%%%%%%%%%%%%%%%%
\section{General Conventions}
\label{sec:notation-general}

\subsection{Typography}

\begin{itemize}
\item \textbf{Scalars:} Lowercase italic Latin ($a, b, x, y$) or Greek ($\alpha, \beta, \pi, \mu$)
\item \textbf{Vectors:} Bold lowercase ($\mathbf{r}, \mathbf{x}, \mathbf{g}$)
\item \textbf{Matrices:} Bold uppercase ($\mathbf{C}, \mathbf{Q}, \mathbf{X}$)
\item \textbf{Sets:} Calligraphic uppercase ($\mathcal{H}, \mathcal{R}, \mathcal{A}$)
\item \textbf{Random variables:} Uppercase Latin ($X, Y, R$)
\item \textbf{Constants:} Roman font when standard ($\mathrm{e}, \pi$)
\end{itemize}

\subsection{Indexing Conventions}

\begin{itemize}
\item \textbf{Superscript $(i)$:} Read index or molecule index (e.g., $r^{(i)}$ = read $i$)
\item \textbf{Subscript $j$:} Position within a sequence (e.g., $r_j$ = base at position $j$)
\item \textbf{Combined:} $r^{(i)}_j$ = base $j$ in read $i$
\item \textbf{Double subscript:} Matrix entry (e.g., $C_{ij}$ = confusion matrix row $i$, column $j$)
\end{itemize}

\subsection{Probability Notation}

\begin{important}[Canonical Probability Notation]
\textbf{Always use the \texttt{\textbackslash Prob} macro:}
\begin{equation}
\Prob(A), \quad \Prob(A \mid B), \quad \Prob(X = x)
\end{equation}
\textbf{Never} use raw $\mathbb{P}$ or $P$ in isolation. The macro ensures visual consistency and enables global formatting changes.
\end{important}

\begin{itemize}
\item $\Prob(A)$ — probability of event $A$
\item $\Prob(A \mid B)$ — conditional probability of $A$ given $B$
\item $\Prob(X = x)$ — probability that random variable $X$ takes value $x$
\item $\Prob(\mathbf{r} \mid h)$ — likelihood of reads $\mathbf{r}$ given haplotype $h$
\end{itemize}

%%%%%%%%%%%%%%%%%%%%%%%%%%%%%%%%%%%%%%%%%%%%%%%%%%%%%%%%%%%%%%%%%%%%%%%%
\section{Core Mathematical Symbols}
\label{sec:notation-symbols}

\subsection{Sequencing Pipeline Variables}

\begin{longtable}{p{0.20\textwidth}p{0.70\textwidth}}
\toprule
\textbf{Symbol} & \textbf{Meaning} \\
\midrule
\endhead
$h$, $h_i$ & Haplotype (candidate sequence); index $i \in \{1, \ldots, P\}$ \\
$\mathcal{H}$ & Set of candidate haplotypes: $\mathcal{H} = \{h_1, \ldots, h_P\}$ \\
$g$, $g^{(i)}$ & Genomic molecule (chromosome, plasmid) \\
$\mathcal{G}_i$ & Set of genomic molecules for haplotype $h_i$ \\
$u$, $u^{(i)}$ & Post-mutation sequence (after somatic or replication errors) \\
$\mathcal{U}_i$ & Set of possible mutated sequences from $h_i$ \\
$d$, $d^{(i)}$ & DNA fragment (after physical shearing or enzymatic cutting) \\
$\mathcal{D}_i$ & Set of possible fragments from $h_i$ \\
$\ell$, $\ell^{(i)}$ & Library molecule (after adapter ligation, PCR, enrichment) \\
$\mathcal{L}_i$ & Set of library molecules derived from $h_i$ \\
$\sigma$, $\sigma^{(i)}$ & Instrument signal (raw time-series from sequencer) \\
$\mathcal{S}$ & Signal space \\
$r$, $r^{(i)}$ & Basecalled read (digital sequence output) \\
$\mathbf{R}$, $R$ & Set of all reads: $R = \{r^{(1)}, \ldots, r^{(n)}\}$ \\
$\mathcal{R}$ & Read space (set of all possible reads) \\
\bottomrule
\end{longtable}

\subsection{Sequence and Read Properties}

\begin{longtable}{p{0.20\textwidth}p{0.70\textwidth}}
\toprule
\textbf{Symbol} & \textbf{Meaning} \\
\midrule
\endhead
$\mathcal{A}$ & Nucleotide alphabet: $\{A, C, G, T\}$ or $\{A, C, G, T, N\}$ \\
$L$, $L_{\text{mol}}$ & Molecule length (in bases); context usually clarifies \\
$L_i$ & Length of read $i$: $r^{(i)} \in \mathcal{A}^{L_i}$ \\
$d_{\text{edit}}(r, s)$ & Levenshtein edit distance between sequences $r$ and $s$ \\
$s$, $s^{(i)}$ & True (ground-truth) sequence corresponding to read $r^{(i)}$ \\
$\hat{s}$ & Observed or predicted sequence assignment \\
\bottomrule
\end{longtable}

\subsection{Quality Scores and Error Rates}

\begin{longtable}{p{0.20\textwidth}p{0.70\textwidth}}
\toprule
\textbf{Symbol} & \textbf{Meaning} \\
\midrule
\endhead
$Q$, $Q_i$ & Phred quality score at base $i$: $Q_i = -10 \log_{10} p_i$ \\
$Q^{(i)}_j$ & Quality score of base $j$ in read $i$ \\
$p$, $p_i$ & Error probability: $p_i = 10^{-Q_i/10}$ \\
$\bar{Q}$ & Mean Phred score (arithmetic mean of $Q_i$) \\
$Q_{\text{pred}}$ & Predicted quality score (from basecaller) \\
$Q_{\text{emp}}$ & Empirical quality score (measured from ground truth) \\
$\bar{Q}^{(i)}_{\text{pred}}$ & Mean predicted quality for read $i$: $-10 \log_{10}(\frac{1}{L_i}\sum_j 10^{-Q^{(i)}_j/10})$ \\
$\bar{Q}^{(i)}_{\text{emp}}$ & Empirical quality for read $i$: $-10 \log_{10}(d_{\text{edit}}(r^{(i)}, s^{(i)})/|s^{(i)}|)$ \\
$\bar{p}^{(i)}_{\text{pred}}$ & Mean predicted error rate for read $i$ \\
$\bar{p}^{(i)}_{\text{emp}}$ & Empirical error rate: $d_{\text{edit}}(r^{(i)}, s^{(i)})/|s^{(i)}|$ \\
\bottomrule
\end{longtable}

\subsection{Confusion Matrix and Classification}

\begin{longtable}{p{0.20\textwidth}p{0.70\textwidth}}
\toprule
\textbf{Symbol} & \textbf{Meaning} \\
\midrule
\endhead
$\mathbf{C}$ & Confusion matrix \\
$C_{ij}$ & Entry: count of true class $i$ predicted as class $j$ \\
$N_i$ & Row sum: $N_i = \sum_j C_{ij}$ (total molecules of type $i$) \\
$\mathrm{TPR}(s_i)$ & True Positive Rate for sequence $s_i$: $\mathrm{TPR}(s_i) = C_{ii}/N_i$ \\
$\mathrm{SMA}(s_i)$ & Single Molecule Accuracy: $\mathrm{SMA}(s_i) \equiv \mathrm{TPR}(s_i) = C_{ii}/N_i$ \\
$\varepsilon_i$ & Misclassification probability: $\varepsilon_i = 1 - \mathrm{TPR}(s_i)$ \\
\bottomrule
\end{longtable}

\begin{important}[Confusion Matrix Index Convention]
\textbf{Rows = TRUE class; Columns = PREDICTED class.}
$$C_{ij} = \text{count of true sequence } i \text{ predicted as sequence } j$$
This convention is used consistently across all chapters and appendices. See Chapter~\ref{chap:classification-model}, Definition~\ref{def:confusion-matrix}.
\end{important}

\subsection{Haplotype Classification and Posteriors}

\begin{longtable}{p{0.20\textwidth}p{0.70\textwidth}}
\toprule
\textbf{Symbol} & \textbf{Meaning} \\
\midrule
\endhead
$\Prob(h_i \mid R)$ & Posterior probability of haplotype $h_i$ given reads $R$ \\
$\Prob(R \mid h_i)$ & Likelihood of reads $R$ given haplotype $h_i$ \\
$\Prob(h_i)$ & Prior probability of haplotype $h_i$ (population frequency) \\
$\hat{h}$ & Maximum a posteriori (MAP) haplotype: $\hat{h} = \arg\max_i \Prob(h_i \mid R)$ \\
$\gamma$ & Posterior confidence threshold (e.g., $\Prob(\hat{h} \mid R) \geq \gamma$) \\
$\mathrm{LR}_i(R)$ & Likelihood ratio: $\mathrm{LR}_i(R) = \Prob(h_i \mid R) / (1 - \Prob(h_i \mid R))$ \\
$\log_{10} \mathrm{BF}$ & Log Bayes factor (for evidence strength reporting) \\
\bottomrule
\end{longtable}

\subsection{Diplotypes and Polyploidy}

\begin{longtable}{p{0.20\textwidth}p{0.70\textwidth}}
\toprule
\textbf{Symbol} & \textbf{Meaning} \\
\midrule
\endhead
$d$, $d_k$ & Diplotype (ordered or unordered pair of haplotypes) \\
$\mathcal{D}$ & Set of possible diplotypes \\
$\pi_d$ & Prior probability of diplotype $d$ \\
$\Prob(d \mid R)$ & Posterior probability of diplotype $d$ given reads $R$ \\
$L_d(R)$ & Likelihood of diplotype $d$: $L_d(R) = \Prob(R \mid d)$ \\
$\alpha_i$ & Mixture fraction for haplotype $h_i$ in a diplotype (often $\alpha_i = 0.5$) \\
\bottomrule
\end{longtable}

\subsection{Haplotagging (Molecule Assignment)}

\begin{longtable}{p{0.20\textwidth}p{0.70\textwidth}}
\toprule
\textbf{Symbol} & \textbf{Meaning} \\
\midrule
\endhead
$m_j$ & Molecule $j$ within a known haplotype \\
$M(h)$ & Set of molecules for haplotype $h$: $M(h) = \{m_1, \ldots, m_v\}$ \\
$\Prob(m_j \mid h)$ & Prior mole fraction of molecule $m_j$ \\
$\Prob(m_j \mid r, h)$ & Posterior probability: read $r$ came from molecule $m_j$ \\
$\tau$ & Haplotagging likelihood ratio threshold \\
$\mathrm{LR}_j(r)$ & Molecule likelihood ratio: $\Prob(m_j \mid r, h) / (1 - \Prob(m_j \mid r, h))$ \\
\bottomrule
\end{longtable}

%%%%%%%%%%%%%%%%%%%%%%%%%%%%%%%%%%%%%%%%%%%%%%%%%%%%%%%%%%%%%%%%%%%%%%%%
\section{Purity and Replication Models}
\label{sec:notation-purity}

\begin{longtable}{p{0.20\textwidth}p{0.70\textwidth}}
\toprule
\textbf{Symbol} & \textbf{Meaning} \\
\midrule
\endhead
$\pi$ & \textbf{Empirical purity} (measured or assumed constant fraction of correct molecules) \\
$P_{\text{pure}}(k)$ & \textbf{Theoretical purity function} as a function of replication cycles $k$ \\
$k$ & Number of replication cycles (bacterial growth, plasmid amplification) \\
$r$ & Per-base replication error rate \\
$L$ or $L_p$ & Plasmid length (in bp); subscript $p$ clarifies when needed \\
$P_{\text{pure}}(k)$ & Purity ceiling: $(1 - r)^{Lk} \approx \exp(-rLk)$ \\
$P_{\text{mut}}(k)$ & Fraction of mutated molecules: $1 - P_{\text{pure}}(k)$ \\
$Q_{\text{pur}}$ & Purity Phred score: $Q_{\text{pur}} = -10 \log_{10}(P_{\text{mut}}(k))$ \\
\bottomrule
\end{longtable}

\begin{important}[Purity Notation: $\pi$ vs $P_{\text{pure}}(k)$]
These are \textbf{related but distinct} concepts:
\begin{itemize}
\item \textbf{$\pi$} — Empirical or assumed purity (constant, measured from experiments)
\item \textbf{$P_{\text{pure}}(k)$} — Theoretical purity as a function of replication cycles
\item \textbf{Relationship:} $\pi \leq P_{\text{pure}}(k)$ for $k$ cycles of replication
\end{itemize}
See Chapter~\ref{chap:purity}, Section~5.1 and Appendix~\ref{app:core-equations}, Section~\ref{sec:purity}.
\end{important}

%%%%%%%%%%%%%%%%%%%%%%%%%%%%%%%%%%%%%%%%%%%%%%%%%%%%%%%%%%%%%%%%%%%%%%%%
\section{Experimental Design and Dual Cas9}
\label{sec:notation-experimental}

\begin{longtable}{p{0.20\textwidth}p{0.70\textwidth}}
\toprule
\textbf{Symbol} & \textbf{Meaning} \\
\midrule
\endhead
$G$ & Gene length (target region length, in bp) \\
$L$ & Fragment length (random variable, with pdf $f_L$ and cdf $F_L$) \\
$f_L(\ell)$ & Probability density function of fragment length \\
$F_L(\ell)$ & Cumulative distribution function of fragment length \\
$p_{\text{frag}}(G)$ & Probability a fragment is $\geq$ gene length: $\Prob(L \geq G) = 1 - F_L(G^-)$ \\
$e_1, e_2$ & Cas9 cutting efficiencies at two sites (flanking a gene) \\
$p_{\text{dual}}(G)$ & Probability of successful dual Cas9 isolation: $p_{\text{dual}}(G) = p_{\text{frag}}(G) \cdot e_1 e_2$ \\
\bottomrule
\end{longtable}

%%%%%%%%%%%%%%%%%%%%%%%%%%%%%%%%%%%%%%%%%%%%%%%%%%%%%%%%%%%%%%%%%%%%%%%%
\section{SMA-seq and SEER Framework}
\label{sec:notation-sma-seer}

\begin{longtable}{p{0.20\textwidth}p{0.70\textwidth}}
\toprule
\textbf{Symbol} & \textbf{Meaning} \\
\midrule
\endhead
SMA-seq & Single Molecule Accuracy sequencing (protocol for ground-truth measurement) \\
SEER & Sequencing Empirical Error Rate (framework for error characterization) \\
$\mathrm{SMA}(s_i)$ & Single Molecule Accuracy for sequence $s_i$: $C_{ii}/N_i$ (see primary definition: Appendix~\ref{app:mathematical-models}, Definition~2) \\
$\mathcal{E}$ & Set of experiments (SMA-seq runs on standards) \\
$a, b$ & Matching and total colonies (clonal sequencing purity checks) \\
$C_{\text{major}}$ & Concentration of major plasmid band (capillary electrophoresis) \\
$C_{\text{other}}$ & Concentration of contaminating bands \\
\bottomrule
\end{longtable}

%%%%%%%%%%%%%%%%%%%%%%%%%%%%%%%%%%%%%%%%%%%%%%%%%%%%%%%%%%%%%%%%%%%%%%%%
\section{Indicator Functions and Logical Operators}
\label{sec:notation-indicators}

\begin{longtable}{p{0.20\textwidth}p{0.70\textwidth}}
\toprule
\textbf{Symbol} & \textbf{Meaning} \\
\midrule
\endhead
$\mathbb{I}\{A\}$ & Indicator function: 1 if event $A$ is true, 0 otherwise \\
$\mathbf{1}\{A\}$ & Alternative notation for indicator (used in some chapters) \\
$\mathbbm{1}\{A\}$ & Another variant (rare, avoid for consistency) \\
\bottomrule
\end{longtable}

\textbf{Recommended:} Use $\mathbb{I}\{\cdot\}$ consistently throughout.

%%%%%%%%%%%%%%%%%%%%%%%%%%%%%%%%%%%%%%%%%%%%%%%%%%%%%%%%%%%%%%%%%%%%%%%%
\section{Operators and Functions}
\label{sec:notation-operators}

\begin{longtable}{p{0.20\textwidth}p{0.70\textwidth}}
\toprule
\textbf{Symbol} & \textbf{Meaning} \\
\midrule
\endhead
$\arg\max_i f(i)$ & Argument that maximizes $f$: index $i$ where $f(i)$ is largest \\
$\arg\min_i f(i)$ & Argument that minimizes $f$ \\
$\sum_{i=1}^n$ & Summation over index $i$ from 1 to $n$ \\
$\prod_{i=1}^n$ & Product over index $i$ from 1 to $n$ \\
$\log$ & Natural logarithm (base $e$) unless subscript specifies otherwise \\
$\log_{10}$ & Base-10 logarithm (used for Phred scores, Bayes factors) \\
$\ln$ & Natural logarithm (rarely used; prefer $\log$) \\
$\exp(x)$ & Exponential function: $e^x$ \\
$|A|$ & Cardinality of set $A$ (number of elements) \\
$|s|$ & Length of sequence $s$ (number of bases) \\
$\lvert s \rvert$ & Alternative notation for sequence length \\
\bottomrule
\end{longtable}

%%%%%%%%%%%%%%%%%%%%%%%%%%%%%%%%%%%%%%%%%%%%%%%%%%%%%%%%%%%%%%%%%%%%%%%%
\section{Abbreviations and Acronyms}
\label{sec:notation-abbreviations}

\begin{longtable}{p{0.20\textwidth}p{0.70\textwidth}}
\toprule
\textbf{Abbreviation} & \textbf{Meaning} \\
\midrule
\endhead
ADR & Adverse Drug Reaction \\
AS & Activity Score (pharmacogene diplotype) \\
bp & Base pairs \\
CPIC & Clinical Pharmacogenetics Implementation Consortium \\
CDF & Cumulative Distribution Function \\
CNV & Copy Number Variant \\
DPWG & Dutch Pharmacogenetics Working Group \\
ER+ & Estrogen Receptor Positive \\
FDA & U.S. Food and Drug Administration \\
FFPE & Formalin-Fixed, Paraffin-Embedded \\
HLA & Human Leukocyte Antigen \\
ICU & Intensive Care Unit \\
IM & Intermediate Metabolizer \\
kb & Kilobases (1000 bp) \\
LOC & Lines of Code \\
MAP & Maximum A Posteriori \\
MLPA & Multiplex Ligation-dependent Probe Amplification \\
NM & Normal Metabolizer \\
PCR & Polymerase Chain Reaction \\
PDF & Probability Density Function \\
PM & Poor Metabolizer \\
PPV & Positive Predictive Value \\
QC & Quality Control \\
QALY & Quality-Adjusted Life Year \\
SEER & Sequencing Empirical Error Rate \\
SMA & Single Molecule Accuracy \\
SMA-seq & Single Molecule Accuracy Sequencing \\
SMS & Single-Molecule Sequencing \\
SNP & Single Nucleotide Polymorphism \\
SNV & Single Nucleotide Variant \\
SOP & Standard Operating Procedure \\
SV & Structural Variant \\
TAT & Turnaround Time \\
TCA & Tricyclic Antidepressant \\
TDM & Therapeutic Drug Monitoring \\
TPR & True Positive Rate \\
UM & Ultrarapid Metabolizer \\
\bottomrule
\end{longtable}

%%%%%%%%%%%%%%%%%%%%%%%%%%%%%%%%%%%%%%%%%%%%%%%%%%%%%%%%%%%%%%%%%%%%%%%%
\section{Cross-References to Primary Definitions}
\label{sec:notation-cross-refs}

For complete mathematical treatments and formal definitions, see:

\begin{itemize}
\item \textbf{Pipeline Factorization Theorem:} Appendix~\ref{app:mathematical-models}, Theorem~1 (primary); also Chapter~\ref{chap:classification-model}, Theorem~\ref{thm:pipeline-factorization}

\item \textbf{Single Molecule Accuracy (SMA):} Appendix~\ref{app:mathematical-models}, Definition~2 (primary); also Appendix~\ref{app:core-equations}, Section~\ref{subsec:sma-definition}; Chapter~\ref{chap:sma-seq}, Remark~11.3

\item \textbf{Confusion Matrix Convention:} Chapter~\ref{chap:classification-model}, Definition~\ref{def:confusion-matrix} with explicit index callout; Appendix~\ref{app:core-equations}, Section~\ref{sec:math-confusion}

\item \textbf{Purity Theory:} Chapter~\ref{chap:purity} (complete treatment); Appendix~\ref{app:core-equations}, Section~\ref{sec:purity} (equations); Appendix~\ref{app:mathematical-models}, Section~7 (formal model)

\item \textbf{Haplotype Classification:} Chapter~\ref{chap:posteriors} (applied); Appendix~\ref{app:mathematical-models}, Section~5 (formal); Appendix~\ref{app:core-equations}, Section~\ref{sec:haplotype-classification} (summary)

\item \textbf{Dual Cas9 Model:} Chapter~\ref{chap:experimental-design}, Section~7.4; Appendix~\ref{app:core-equations}, Section~\ref{sec:dual-cas9}; Appendix~\ref{app:mathematical-models}, Section~7
\end{itemize}

%%%%%%%%%%%%%%%%%%%%%%%%%%%%%%%%%%%%%%%%%%%%%%%%%%%%%%%%%%%%%%%%%%%%%%%%
\section{Notation Changes from Earlier Drafts}
\label{sec:notation-changes}

\textbf{Version 6.0--6.1 standardization:}

\begin{longtable}{p{0.20\textwidth}p{0.30\textwidth}p{0.40\textwidth}}
\toprule
\textbf{Old Notation} & \textbf{New Notation} & \textbf{Reason} \\
\midrule
\endhead
$L_{\text{edit}}(r, s)$ & $d_{\text{edit}}(r, s)$ & Avoid overloading $L$ (molecule length vs distance) \\
$\mathbb{P}(\cdot)$ & $\Prob(\cdot)$ & Macro for consistency and visual uniformity \\
$b$ (quality) & $Q$ (quality) & Standardize to Phred convention \\
$P(h \mid R)$ & $\Prob(h \mid R)$ & Always use macro \\
\bottomrule
\end{longtable}

%%%%%%%%%%%%%%%%%%%%%%%%%%%%%%%%%%%%%%%%%%%%%%%%%%%%%%%%%%%%%%%%%%%%%%%%
\section{Summary of Critical Conventions}
\label{sec:notation-summary}

\begin{enumerate}
\item \textbf{Probability:} Always use $\Prob(\cdot)$ macro, never raw $\mathbb{P}$ or $P$

\item \textbf{Confusion Matrix Indexing:} $C_{ij}$ = true class $i$ (row) $\to$ predicted class $j$ (column)

\item \textbf{Edit Distance:} $d_{\text{edit}}(r, s)$ (use $d$, not $L$, to avoid overloading molecule length)

\item \textbf{Purity:} $\pi$ (empirical constant) vs $P_{\text{pure}}(k)$ (theoretical function of replication cycles)

\item \textbf{Quality Scores:} $Q$ for Phred, $p$ for error probability; $Q = -10 \log_{10} p$

\item \textbf{SMA Definition:} Canonical definition in Appendix~\ref{app:mathematical-models}, Definition~2; cross-referenced everywhere else

\item \textbf{Indicator Function:} Prefer $\mathbb{I}\{\cdot\}$ over $\mathbf{1}\{\cdot\}$ for consistency
\end{enumerate}

%%%%%%%%%%%%%%%%%%%%%%%%%%%%%%%%%%%%%%%%%%%%%%%%%%%%%%%%%%%%%%%%%%%%%%%%
\section{Quick Navigation Guide}
\label{sec:navigation-guide}

This table provides a rapid reference to locate key concepts, definitions, and formulas throughout the framework. Use this to quickly find where concepts are introduced (Primary) and where they receive detailed treatment (Key Sections).

\begin{table}[H]
\centering
\small
\caption{Navigation guide to key concepts and their locations}
\label{tab:navigation-guide}
\begin{tabular}{lp{3.5cm}p{5.5cm}}
\toprule
\textbf{Concept} & \textbf{Primary Definition} & \textbf{Key Sections / Applications} \\
\midrule
\multicolumn{3}{l}{\textit{\textbf{Core Mathematical Framework}}} \\
Pipeline Factorization & Ch.~4, Def.~4.1 & App.~F §\ref{sec:app-f-pipeline-factorization}, Exec. Overview \\
Posterior Computation & Ch.~6, Eq.~(6.1--6.3) & App.~B §\ref{sec:haplotype-classification}, Ch.~14--15 \\
Confusion Matrix & Ch.~4, Def.~4.3 & Ch.~11, App.~F §\ref{sec:app-f-confusion}, App.~B \\
Purity Theory & Ch.~5, Def.~5.1--5.3 & Ch.~8 (plasmids), Ch.~11 (SMA ceiling), App.~F §7 \\
\midrule
\multicolumn{3}{l}{\textit{\textbf{Quality Score Framework}}} \\
Phred Quality & Ch.~4, Def.~4.2 & Ch.~11 (calibration), App.~F §\ref{sec:app-f-quality-scores} \\
Phred Averaging Ineq. & Ch.~11, Thm.~11.2 & App.~B §\ref{sec:quality-score-theory}, App.~C (QC gates) \\
Quality Overstatement & Ch.~11, Def.~11.5 & App.~C (QC Gate 8), Ch.~12--13 (calibration) \\
ECE (Calibration Error) & Ch.~11, Def.~11.6 & Ch.~13 (fine-tuning), App.~C \\
\midrule
\multicolumn{3}{l}{\textit{\textbf{SMA-seq and SEER Framework}}} \\
SMA (Single Mol. Acc.) & App.~F, Def.~2 & Ch.~11 §\ref{sec:sma-definitions}, Ch.~14--15 \\
SMA-seq Protocol & Ch.~11 §11.1--11.3 & App.~D (computational), Ch.~8 (standards) \\
SEER Framework & Ch.~11 §11.4--11.6 & Ch.~12--13 (model improvement) \\
Purity Ceiling & Ch.~5, Thm.~5.2 & Ch.~11 §11.7 (SMA bias), Ch.~8 (design) \\
\midrule
\multicolumn{3}{l}{\textit{\textbf{Experimental Design}}} \\
Coverage Calculation & Ch.~7, Eq.~(7.2--7.4) & App.~F §\ref{sec:app-f-coverage}, Ch.~9--10 \\
Plasmid Replication & Ch.~5, Eq.~(5.5--5.8) & Ch.~8 §8.2, App.~F §7 (purity models) \\
Dual Cas9 Cutting & Ch.~9, Def.~9.1 & Ch.~7 (design), App.~D (protocols) \\
Mixture Design & Ch.~10 & Ch.~14 (validation mixtures) \\
\midrule
\multicolumn{3}{l}{\textit{\textbf{Model Improvement}}} \\
Noisy Label Learning & Ch.~12 & Ch.~13 (fine-tuning), App.~F §10 \\
Basecaller Fine-Tuning & Ch.~13 & Ch.~11 (SMA feedback), Ch.~15 (validation) \\
Loss Functions & Ch.~13, Eq.~(13.2--13.5) & App.~F §10, Ch.~12 (robust training) \\
\midrule
\multicolumn{3}{l}{\textit{\textbf{Validation and QC}}} \\
QC Gates (15 gates) & App.~C & Ch.~11 (empirical), Ch.~14--15 (validation) \\
Ground Truth & Ch.~5 §5.4 & Ch.~14 (mixtures), Ch.~11 (SMA-seq) \\
Wilson Confidence Int. & Ch.~11, Eq.~(11.8) & Ch.~14 (SMA CI), App.~C (binomial metrics) \\
Accuracy Propagation & Ch.~14 & Ch.~15 (end-to-end), Ch.~6 (theory) \\
\midrule
\multicolumn{3}{l}{\textit{\textbf{Clinical Applications}}} \\
CYP2D6 Classification & Ch.~18 & Ch.~1 (motivation), Ch.~15 (workflow) \\
Tamoxifen Case Study & Ch.~18 §18.2--18.4 & Ch.~1 (Clinical Box 1.1) \\
Endoxifen Prediction & Ch.~18, Eq.~(18.1) & Ch.~18 §18.4 \\
Two Failures Framework & Ch.~18 §18.1 & Ch.~1 (clinical context) \\
\midrule
\multicolumn{3}{l}{\textit{\textbf{Core Equations (CE\#1--15)}}} \\
CE.1: Posterior Basic & Ch.~4, App.~B §1 & Ch.~6 (applications), Ch.~14--15 \\
CE.2: Bayes Theorem & App.~B §1 & Throughout (Bayesian inference) \\
CE.4--5: Likelihood & Ch.~4, App.~B §2 & Ch.~6 (classification), Ch.~11 (empirical) \\
CE.8--10: Quality Theory & App.~B §3 & Ch.~11 (calibration), Ch.~13 (fine-tuning) \\
CE.11--12: Purity Bounds & Ch.~5, App.~B §4 & Ch.~8 (plasmids), Ch.~11 (SMA ceiling) \\
CE.13--15: Coverage & Ch.~7, App.~B §5 & Ch.~9 (enrichment), Ch.~10 (mixtures) \\
\bottomrule
\end{tabular}
\end{table}

\textbf{How to use this guide:}
\begin{itemize}
\item \textbf{Learning a new concept:} Start with Primary Definition, then review Key Sections for applications
\item \textbf{Implementing a method:} Check Primary Definition for formulas, then consult Appendix D for computational protocols
\item \textbf{Troubleshooting:} Use Key Sections to find related QC gates (Appendix C) and validation procedures (Ch. 14--15)
\item \textbf{Quick reference:} Use this table with Table~\ref{tab:chapter-dependencies} (Executive Overview) to plan reading path
\end{itemize}

%%%%%%%%%%%%%%%%%%%%%%%%%%%%%%%%%%%%%%%%%%%%%%%%%%%%%%%%%%%%%%%%%%%%%%%%
\section{Conclusion}

This notation guide serves as the authoritative reference for interpreting all mathematical expressions in the SMS Haplotype Classification Framework. When in doubt, consult this appendix first, then refer to the primary definition locations listed in Section~\ref{sec:notation-cross-refs}.

For questions about notation not covered here, refer to Appendix~\ref{app:core-equations} (mathematical models summary) or Appendix~\ref{app:mathematical-models} (comprehensive formal treatment).

\clearpage


%%%%%%%%%%%%%%%%%%%%%%%%%%%%%%%%%%%%%%%%%%%%%%%%%%%%%%%%%%%%%%%%%%%%%%%%
%% Appendix A: Notation and Symbol Reference
%% Version 6.0 - Complete Migration from v5.tex (lines 1889-2113)
%%%%%%%%%%%%%%%%%%%%%%%%%%%%%%%%%%%%%%%%%%%%%%%%%%%%%%%%%%%%%%%%%%%%%%%%

\chapter{Notation and Symbol Reference}
\label{app:notation}

This comprehensive notation reference provides detailed descriptions of all mathematical symbols used throughout the framework, organized by functional category for easy reference. The notation system follows standard conventions in probability theory and statistical inference: uppercase letters denote random variables, bold fonts indicate vectors and matrices, and blackboard bold represents probability measures. This appendix serves as a complete reference for interpreting any equation in the main document.

\textbf{Organization:} The notation is organized into eight functional categories that mirror the framework's computational pipeline: (1) Primary Spaces and Sets define the fundamental objects, (2) Fragment Length Variables handle size-dependent calculations, (3) Probability Distributions describe fragment and read characteristics, (4) Quality Score Parameters quantify sequencing accuracy, (5) Model Parameters specify the statistical framework, (6) Sequencing Metrics track experimental quantities, (7) Statistical Measures evaluate model performance, and (8) Cas9 Capture Parameters support targeted sequencing applications.

\textbf{Usage Convention:} When referencing equations, always consult this appendix to verify symbol definitions. Pay particular attention to distinguishing $L$ (fixed length) from $\ell$ (variable length), and $f_{\mathrm{emp}}$ (empirical measurements) from $f_{\mathrm{frag}}$ (theoretical models). Consistent notation across the document ensures unambiguous interpretation of all mathematical expressions.

\section{Primary Spaces and Sets}

These fundamental mathematical objects define the complete sequencing pipeline from genomic sequences through physical processing to observed data. Each space represents a distinct stage in the data generation process, with probability distributions governing transitions between spaces.

\begin{longtable}{p{2.5cm}p{9cm}}
\caption{Spaces and Set Notation} \\
\toprule
\textbf{Symbol} & \textbf{Description} \\
\midrule
\endfirsthead
\multicolumn{2}{c}{\textit{Continued from previous page}} \\
\toprule
\textbf{Symbol} & \textbf{Description} \\
\midrule
\endhead
\bottomrule
\endfoot
\bottomrule
\endlastfoot
$\mathcal{H}$ & Haplotype space: The complete set of candidate genomic sequences under consideration. In diploid organisms, typically $|\mathcal{H}| = 2$, though analysis may include additional variants. Used throughout all equations as the fundamental hypothesis space for Bayesian classification. \textit{See CE.11-CE.13 for usage in likelihood and posterior calculations.} \\
$h_i$ & Individual haplotype: A specific genomic sequence variant, indexed by $i \in \{1, ..., P\}$. Each haplotype represents a distinct DNA sequence that could be the source of observed reads. In practice, haplotypes differ by SNPs, indels, or structural variants. \textit{Referenced in all likelihood calculations (CE.11, CE.12)} \\
$P$ & Total number of distinct haplotypes in the candidate set. Typical values: $P=2$ for diploid single-locus analysis, $P=4$ for diploid two-locus analysis. Larger values increase computational cost linearly in likelihood calculations. \textit{Used in posterior normalization (CE.13)} \\
$\Omega$ & Sample space: Set of all possible DNA molecules in the biological sample. For whole genome sequencing, $|\Omega| \approx 10^6$-$10^9$ molecules per nanogram of DNA. For targeted sequencing, $|\Omega|$ is effectively infinite before enrichment. \textit{Foundation for sampling model in fragmentation step} \\
$\mathcal{S}$ & Signal space: Raw instrument measurements before basecalling. For ONT: electrical current time series sampled at 4 kHz. For PacBio: inter-pulse duration (IPD) measurements. For Illumina: fluorescence intensity per cycle. Dimensionality: $\mathcal{S} \subseteq \mathbb{R}^T$ where $T$ is measurement count. \textit{Input to basecalling model (CE.1, CE.2)} \\
$\mathcal{R}$ & Read space: Basecalled nucleotide sequences after primary analysis. Elements are strings over alphabet $\{A, C, G, T\}$ with associated quality scores. Typical size: $|\mathcal{R}| = N = 100$-$10^5$ reads per sample. \textit{Observed data for likelihood calculation (CE.11)} \\
$\mathcal{A}$ & Alignment space: Possible mappings of reads to reference genome. Each element specifies genomic coordinates, CIGAR string, mapping quality. Used only for quality assessment, NOT classification. \textit{See Chapter 10: alignments are for QC only} \\
\end{longtable}

\section{Fragment Length Variables}

Fragment length distributions critically determine coverage and perfect-read probabilities. Different experimental protocols produce different distributions, requiring empirical measurement for accurate predictions.

\begin{longtable}{p{2.5cm}p{9cm}}
\caption{Fragment Length Notation} \\
\toprule
\textbf{Symbol} & \textbf{Description} \\
\midrule
\endfirsthead
\multicolumn{2}{c}{\textit{Continued from previous page}} \\
\toprule
\textbf{Symbol} & \textbf{Description} \\
\midrule
\endhead
\bottomrule
\endfoot
\bottomrule
\endlastfoot
$L$ & Read length: Fixed length in bases for protocols with uniform read length. For ONT/PacBio long reads, this is target length post-size selection (e.g., $L = 10$ kb). For Illumina, this is read length setting (e.g., $L = 150$ bp). Used in simplified perfect-read calculations (CE.2). \\
$\ell$ & Variable fragment length: Random variable representing actual length of individual DNA fragments in library. Distribution $f_{\mathrm{emp}}(\ell)$ must be measured empirically from experiment. Typical range: 200-600 bp (Illumina), 5-50 kb (ONT), 10-30 kb (PacBio HiFi). \textit{Central parameter in CE.3, CE.7, CE.16} \\
$f_{\mathrm{emp}}(\ell)$ & Empirical fragment length distribution: Probability density function measured from actual sequencing data via read length histogram or Bioanalyzer/TapeStation. Use this in all production calculations rather than theoretical models. \textbf{Measurement:} Sequence standards or pilot samples, bin read lengths, fit kernel density estimate. \textit{Required input for CE.3, CE.7} \\
$f_{\mathrm{frag}}(\ell)$ & Theoretical fragment distribution: Idealized model (e.g., Gamma, exponential) for design calculations when empirical data unavailable. \textbf{Common choices:} Gamma$(\alpha, \beta)$ with $\alpha=2$-4 for mechanical shearing, Exponential$(\lambda)$ for DNase fragmentation. Always validate against $f_{\mathrm{emp}}$ when data available. \textit{Used in CE.7 theoretical development} \\
$\ell_{\min}, \ell_{\max}$ & Size selection boundaries: Minimum and maximum fragment lengths retained by purification (e.g., SPRI beads, Blue Pippin). Defines support of $f_{\mathrm{emp}}$: $f_{\mathrm{emp}}(\ell) = 0$ for $\ell < \ell_{\min}$ or $\ell > \ell_{\max}$. \\
$L_g$ & Target gene length (bp): Length of genomic region for targeted capture applications. For Cas9 dual-cut capture (CE.16), this is the distance between the two guide RNA cut sites. \textbf{Design consideration:} Shorter targets ($L_g < 500$ bp) have higher dual-cut probability but may suffer from capture bias. Longer targets ($L_g > 5000$ bp) provide more sequence context but lower enrichment efficiency. \textbf{Typical range:} 500-5000 bp for optimal balance. \\
\end{longtable}

\clearpage

\section{Probability Distributions}

These distributions govern the hierarchical generative model from genomic DNA through sequencing to observed reads.

\begin{longtable}{p{2.5cm}p{9cm}}
\caption{Probability Distribution Notation} \\
\toprule
\textbf{Symbol} & \textbf{Description} \\
\midrule
\endfirsthead
\multicolumn{2}{c}{\textit{Continued from previous page}} \\
\toprule
\textbf{Symbol} & \textbf{Description} \\
\midrule
\endhead
\bottomrule
\endfoot
\bottomrule
\endlastfoot
$\Prob(h_i)$ & Prior probability over haplotypes: Probability of haplotype $h_i$ before observing sequencing data. Can be informed by population genetics (allele frequencies) or set to uniform $\Prob(h_i) = 1/P$ for unbiased analysis. \textbf{Typical sources:} gnomAD, 1000 Genomes, PharmGKB for pharmacogenes. \textit{Input to Bayes' rule (CE.12, CE.13)} \\
$\Prob(r_n | h_i)$ & Per-read likelihood: Probability of observing read $r_n$ given true haplotype $h_i$. Computed via confusion matrix (CE.10) or perfect-read model (CE.2, CE.3). This is the core of classification - accurate likelihood requires valid confusion matrix from SEER standards. \textit{See CE.10, CE.11} \\
$\Prob(\mathbf{r} | h_i)$ & Dataset likelihood: Joint probability of all $N$ reads given haplotype $h_i$, assuming conditional independence: $\Prob(\mathbf{r} | h_i) = \prod_{n=1}^{N} \Prob(r_n | h_i)$. Computed in log-space to prevent underflow (CE.11). \textit{Central calculation in CE.11} \\
$\Prob(h_i | \mathbf{r})$ & Posterior probability: Probability of haplotype $h_i$ after observing all reads $\mathbf{r}$. Computed via Bayes' rule (CE.13). Used for classification decision: assign sample to haplotype with maximum posterior (MAP rule). \textbf{Interpretation:} $\Prob(h_i | \mathbf{r}) > 0.99$ indicates high confidence assignment. $\Prob(h_i | \mathbf{r}) \approx 0.5$ for all $i$ indicates ambiguous classification requiring more data. \textit{Computed via CE.13} \\
$\pi_i$ & Haplotype mixture fraction: Proportion of reads originating from haplotype $h_i$ in diploid or mixture samples. For pure diploid: $\pi_1 = \pi_2 = 0.5$. For contamination: $\pi_{\text{contaminant}}$ quantifies contamination level. \textbf{Estimation:} Maximum likelihood or expectation-maximization from read assignments. \textit{See Chapter 14 for mixture analysis} \\
$\lambda$ & Sequencing depth parameter: Expected number of reads covering each base position, measured in $\times$ (fold). Typical values: $\lambda = 30\times$ (standard WGS), $\lambda = 100\times$ (clinical diagnostic), $\lambda = 1000\times$ (rare variant detection). Higher depth increases classification confidence but costs scale linearly. \textit{Design parameter in CE.5 coverage calculations} \\
\end{longtable}

\section{Quality Score Parameters}

Quality scores provide per-base confidence estimates that directly affect perfect-read probabilities and downstream classification.

\begin{longtable}{p{2.5cm}p{9cm}}
\caption{Quality Score Notation} \\
\toprule
\textbf{Symbol} & \textbf{Description} \\
\midrule
\endfirsthead
\multicolumn{2}{c}{\textit{Continued from previous page}} \\
\toprule
\textbf{Symbol} & \textbf{Description} \\
\midrule
\endhead
\bottomrule
\endfoot
\bottomrule
\endlastfoot
$Q$ & Phred quality score: Logarithmic encoding of error probability in decibels, computed as $Q = -10 \log_{10}(p_{\mathrm{err}})$. Scale: Q10 = 90\% accuracy, Q20 = 99\%, Q30 = 99.9\%, Q40 = 99.99\%. Most platforms report Q7-Q40 for ONT, Q20-Q50+ for PacBio HiFi, Q30-Q40 for Illumina. \textit{Fundamental transformation in CE.1} \\
$p_{\mathrm{err}}(Q)$ & Error probability from quality score: Probability of incorrect base call, computed as $p_{\mathrm{err}}(Q) = 10^{-Q/10}$ (CE.1). This transformation assumes quality scores are well-calibrated (see CE.14 validation). \textbf{Critical:} Basecaller calibration errors propagate directly into this probability, affecting all downstream inference. \textit{Core of CE.1, input to CE.2, CE.3} \\
$\theta$ & Per-base accuracy: Probability of correct basecall, related to error probability by $\theta = 1 - p_{\mathrm{err}}$. For Q30 base: $\theta = 0.999$. For Q40 base: $\theta = 0.9999$. Used in simplified perfect-read calculation: $P_{\mathrm{perf}} = \theta^L$ (CE.2). \\
$\bar{Q}$ & Mean quality score: Average quality across all bases in a read or dataset, computed as $\bar{Q} = \frac{1}{N_{\text{bases}}}\sum_{i=1}^{N_{\text{bases}}} Q_i$. \textbf{Typical values:} ONT: Q10-Q18, PacBio HiFi: Q20-Q30, Illumina: Q30-Q35. Lower than expected $\bar{Q}$ indicates sequencing or library prep problems. \textit{QC metric, see Gate 2} \\
$Q_{\mathrm{pred}}$ & Predicted quality score: Quality reported by basecaller for each base. Should reflect true error rate if basecaller is well-calibrated. Validation compares $Q_{\mathrm{pred}}$ vs $Q_{\mathrm{emp}}$ (CE.14). \\
$Q_{\mathrm{emp}}$ & Empirical quality score: Measured error rate from standards, converted to Phred scale. Compute from alignment to known reference: $Q_{\mathrm{emp}} = -10 \log_{10}(\text{error rate})$. Agreement with $Q_{\mathrm{pred}}$ indicates good calibration. \textit{Validation metric, CE.14} \\
\end{longtable}

\clearpage

\section{Model Parameters}

These parameters define the statistical model and control inference behavior.

\begin{longtable}{p{2.5cm}p{9cm}}
\caption{Statistical Model Parameters} \\
\toprule
\textbf{Symbol} & \textbf{Description} \\
\midrule
\endfirsthead
\multicolumn{2}{c}{\textit{Continued from previous page}} \\
\toprule
\textbf{Symbol} & \textbf{Description} \\
\midrule
\endhead
\bottomrule
\endfoot
\bottomrule
\endlastfoot
$\mathbf{C}$ & Confusion matrix: $P \times P$ matrix where element $C_{ij} = \Prob(\text{observe class } j | \text{true class } i)$ quantifies systematic sequencing errors. Rows sum to 1. Diagonal dominance ($C_{ii} > C_{ij}$ for $i \neq j$) indicates good discrimination. \textbf{Estimation:} SEER framework (Chapter 11) using plasmid standards. \textit{Core of CE.9, CE.10} \\
$C_{ij}$ & Confusion matrix element: Probability that true haplotype $h_i$ produces a read classified as coming from haplotype $h_j$. Diagonal elements ($i=j$): true positive rate (TPR). Off-diagonal ($i \neq j$): misclassification rate. \textbf{Quality criteria:} TPR $\geq 0.95$ (good), TPR $\geq 0.99$ (excellent), TPR $\geq 0.999$ (clinical grade). \textit{Computed empirically via CE.9} \\
$\boldsymbol{\theta}$ & Model parameter vector: Complete set of model parameters including confusion matrix, prior probabilities, fragment distribution. Estimated from standards (SEER) or set based on experimental design. \\
$N$ & Total number of reads: Sample size for classification. Larger $N$ increases posterior confidence but has diminishing returns beyond coverage threshold. \textbf{Design guidance:} Use CE.5 to compute required $N$ for target confidence level. \textit{Used in CE.5, CE.11, CE.15} \\
$\alpha, \beta$ & Fragment distribution shape parameters: For Gamma distribution $f(\ell) \propto \ell^{\alpha-1} e^{-\ell/\beta}$. Typical mechanical shearing: $\alpha = 2$-4, $\beta = 100$-300 bp. Enzymatic fragmentation: $\alpha = 3$-5, $\beta = 80$-200 bp. \textit{Used in theoretical models, CE.7} \\
$\mu$ & Mutation rate: Per-base per-generation spontaneous mutation rate. Human germline: $\mu \approx 10^{-8}$. Bacterial: $\mu \approx 10^{-10}$ to $10^{-9}$. Used to compute purity upper bound from replication fidelity (CE.15). \\
$d$ & Number of cell divisions: Generations since clonal origin. Affects purity through accumulated replication errors: $\pi_{\max} = (1 - \mu L)^d$ (CE.15). \textbf{Typical values:} Freshly isolated cells: $d \approx 1$-5. Cultured cell lines: $d = 20$-40. Affects achievable purity. \textit{Input to CE.15} \\
\end{longtable}

\section{Sequencing Metrics}

Operational quantities measured during sequencing that inform quality control and downstream analysis.

\begin{longtable}{p{2.5cm}p{9cm}}
\caption{Experimental Measurements} \\
\toprule
\textbf{Symbol} & \textbf{Description} \\
\midrule
\endfirsthead
\multicolumn{2}{c}{\textit{Continued from previous page}} \\
\toprule
\textbf{Symbol} & \textbf{Description} \\
\midrule
\endhead
\bottomrule
\endfoot
\bottomrule
\endlastfoot
$N_{\text{total}}$ & Total sequenced reads: Count of all reads passing basecaller filters (typically Q7 threshold). Used to compute on-target fraction and coverage. \\
$N_{\text{on-target}}$ & On-target reads: Reads mapping to intended genomic region (for targeted sequencing). Defines enrichment success. \textbf{Quality criteria:} $N_{\text{on-target}}/N_{\text{total}} \geq 0.5$ (acceptable), $\geq 0.7$ (good), $\geq 0.9$ (excellent). \textit{See Chapter 9 for enrichment metrics} \\
$\lambda_{\text{obs}}$ & Observed coverage: Measured sequencing depth at genomic position, computed as number of reads overlapping position divided by read length. Compare to $\lambda$ (expected coverage) to assess uniformity. \\
$CV_{\text{cov}}$ & Coverage coefficient of variation: Standard deviation of coverage divided by mean, quantifies uniformity. $CV < 0.5$ indicates good uniformity. $CV > 1.0$ suggests amplification bias or capture artifacts requiring investigation. \textit{QC metric, Gate 3} \\
$e_{\text{enr}}$ & Enrichment efficiency: Capture or PCR amplification success rate. For targeted sequencing, represents fraction of target molecules successfully captured and amplified. \textbf{Typical values:} Hybrid capture: 40-70\% on-target rate. Cas9 capture: 60-90\% on-target (when successful). PCR enrichment: 80-95\%. \textbf{Measurement:} Compare on-target read fraction to expected based on genome size. \textit{Essential for Cas9 capture design (CE.16)} \\
$\text{MAPQ}$ & Mapping quality: Phred-scaled probability that alignment is incorrect, $\text{MAPQ} = -10\log_{10}(P(\text{wrong alignment}))$. MAPQ $\geq 20$ (99\% correct) typically required for analysis. \textbf{Note:} Used only for QC, NOT for classification. \textit{See Chapter 10: secondary analysis} \\
\end{longtable}

\section{Statistical Measures}

Performance metrics quantifying classification accuracy and model quality.

\begin{longtable}{p{2.5cm}p{9cm}}
\caption{Performance Metrics} \\
\toprule
\textbf{Symbol} & \textbf{Description} \\
\midrule
\endfirsthead
\multicolumn{2}{c}{\textit{Continued from previous page}} \\
\toprule
\textbf{Symbol} & \textbf{Description} \\
\midrule
\endhead
\bottomrule
\endfoot
\bottomrule
\endlastfoot
$\text{TPR}_i$ & True positive rate for haplotype $i$: Fraction of true $h_i$ reads correctly classified. Estimated from confusion matrix diagonal: $\text{TPR}_i = C_{ii}$. \textbf{Clinical requirement:} TPR $\geq 0.99$ for common alleles, TPR $\geq 0.95$ for rare alleles. \textit{Key metric from CE.9} \\
$\text{FPR}_{ij}$ & False positive rate: Fraction of true $h_i$ reads misclassified as $h_j$. Off-diagonal confusion matrix element: $\text{FPR}_{ij} = C_{ij}$ for $i \neq j$. Should be minimized for robust classification. \\
$D_{\text{KL}}(P \| Q)$ & Kullback-Leibler divergence: Measures difference between empirical distribution $P$ and model prediction $Q$, computed as $D_{\text{KL}}(P \| Q) = \sum_x P(x) \log_2(P(x)/Q(x))$ in bits. Used to validate quality score calibration (CE.14). \textbf{Interpretation:} $D_{\text{KL}} < 0.1$ bits = excellent agreement, 0.1-0.5 bits = acceptable, $> 0.5$ bits = recalibration needed. \textit{Computed in CE.14} \\
$\text{BF}_{ij}$ & Bayes factor: Ratio of posterior odds to prior odds for hypotheses $i$ vs $j$: $\text{BF}_{ij} = \frac{\Prob(h_i|\mathbf{r})/\Prob(h_j|\mathbf{r})}{\Prob(h_i)/\Prob(h_j)}$. Quantifies strength of evidence from data. \textbf{Interpretation (Kass-Raftery scale):} BF $< 3$ = weak, 3-20 = positive, 20-150 = strong, $>150$ = very strong evidence. \textit{See CE.16 for formula} \\
$\pi$ & Purity: Fraction of molecules in sample matching intended sequence. Fundamental accuracy ceiling (Theorem 5.1): observed TPR cannot exceed $\pi$. \textbf{Measurement:} Technical replication (lower bound) or clonal sequencing (upper bound). \textit{Critical parameter, CE.15, Chapter 5} \\
$\pi_{\max}$ & Purity upper bound: Maximum achievable true positive rate, fundamentally limited by DNA replication physics. Computed from mutation rate $\mu$, cell divisions $d$, and genomic span $L$ via CE.15. \textbf{Physical interpretation:} Even perfect sequencing (infinite coverage, zero errors) cannot exceed this accuracy due to accumulated replication errors. \textbf{Hard QC constraint:} Any measured TPR $> \pi_{\max}$ indicates contamination, reference errors, or model failure and mandates investigation. \textbf{Typical values:} Well-controlled samples with $d=10$ divisions, $L=1000$ bp, $\mu=10^{-9}$: $\pi_{\max} \approx 0.99999$ (Q50). Extensively passaged cells with $d=30$: $\pi_{\max} \approx 0.99997$ (Q46). \textit{Computed and enforced via CE.15} \\
$d$ & Overstatement fraction: Proportion of reads where predicted quality ($Q_{\mathrm{pred}}$) exceeds empirical quality ($Q_{\mathrm{emp}}$), indicating basecaller overconfidence. \textbf{Computation:} $d = \frac{1}{N}\sum_{n=1}^{N}\mathbb{I}\{Q_{\mathrm{pred},n}>Q_{\mathrm{emp},n}\}$ from CE.14. \textbf{Quality criteria:} $d \leq 0.20$ = excellent calibration (proceed with confidence), 0.20-0.30 = acceptable (proceed with caution), $d > 0.30$ = failed calibration (do not proceed, recalibrate basecaller). \textbf{Clinical importance:} Quality score overstatement propagates into likelihood calculations (CE.11), causing overconfidence in classifications. Even 2-3 dB systematic overstatement can produce 20-50\% errors in perfect-read estimates. \textbf{Report with confidence intervals:} Use Wilson score method for binomial proportion CIs. \textit{Validation metric computed via CE.14} \\
\end{longtable}

\section{Cas9 Capture Parameters}

These parameters are specific to targeted sequencing applications using CRISPR-Cas9 technology for genomic enrichment. The dual-guide capture strategy requires precise coordination of two cutting events, with success depending on both cutting efficiencies ($e_1, e_2$) and fragment length distribution. These parameters enable rigorous design of targeted sequencing experiments, optimization of guide RNA selection, and prediction of enrichment performance before costly sequencing runs. Accurate measurement of cutting efficiencies through validation experiments is essential for reliable predictions.

\begin{longtable}{p{2.5cm}p{9cm}}
\caption{Targeted Sequencing Variables} \\
\toprule
\textbf{Symbol} & \textbf{Description} \\
\midrule
\endfirsthead
\multicolumn{2}{c}{\textit{Continued from previous page}} \\
\toprule
\textbf{Symbol} & \textbf{Description} \\
\midrule
\endhead
\bottomrule
\endfoot
\bottomrule
\endlastfoot
$P_{\text{dual-cut}}$ & Cas9 dual-cut probability: Chance of successful dual-guide capture, computed as $P_{\mathrm{dual-cut}} = e_1 \times e_2 \times P(L_{\mathrm{frag}} < L_g)$ from CE.16. Requires both guide RNAs to cut successfully AND resulting fragment to be smaller than target region. \textbf{Typical values:} With good guides ($e_1, e_2 \approx 0.8$-0.9) and appropriate fragmentation ($P(L_{\mathrm{frag}} < L_g) \approx 0.6$-0.8 for $L_g = 2$-4 kb), expect $P_{\mathrm{dual-cut}} \approx 0.4$-0.6. \textbf{Design optimization:} Shorter target regions increase $P(L_{\mathrm{frag}} < L_g)$ but may reduce sequence context. Longer regions provide more information but lower capture efficiency. \textbf{Optimal range:} $L_g = 500$-5000 bp balances these trade-offs for most applications. \textit{Computed via CE.16 for targeted capture design} \\
$e_1, e_2$ & Cas9 cutting efficiencies: Individual guide RNA cutting success rates at their respective genomic sites. Each represents probability that Cas9 successfully introduces a double-strand break at the target location. \textbf{Typical values:} High-quality guides: $e = 0.75$-0.95. Average guides: $e = 0.50$-0.75. Poor guides: $e < 0.50$ (should be redesigned). \textbf{Measurement protocol:} T7 endonuclease I assay, next-generation sequencing of edited loci, or ddPCR quantification of cut versus uncut alleles. \textbf{Design factors:} On-target score (predict via CRISPOR, Benchling, or similar tools), GC content (optimal 40-60\%), secondary structure, chromatin accessibility at target site. \textbf{Critical:} Both $e_1$ and $e_2$ must be high (>0.7) for efficient dual-capture; one poor guide drastically reduces $P_{\mathrm{dual-cut}}$ due to multiplicative effect. \textbf{Validation required:} Always validate cutting efficiency before large-scale experiments. \textit{Core inputs to CE.16 for capture efficiency prediction} \\
$L_g$ & Target gene length: Size of genomic region between Cas9 cut sites, in base pairs. This is the distance from the first guide's cut site to the second guide's cut site. \textbf{Design consideration:} Must balance multiple factors: (1) Sequence content - longer regions provide more variants for discrimination, (2) Capture efficiency - shorter regions have higher probability of complete capture, (3) Fragment distribution - must match library fragmentation profile. \textbf{Optimal design:} Choose $L_g$ such that $P(L_{\mathrm{frag}} < L_g) \approx 0.6$-0.8 based on empirical fragmentation distribution $f_{\mathrm{emp}}$. For typical fragmentation (mean 400-600 bp), this suggests $L_g = 1$-3 kb. \textbf{Measurement:} Direct distance measurement from genomic coordinates of guide cut sites. \textbf{Pitfall:} Very long targets ($L_g > 10$ kb) may require unfragmented DNA (circulomics, intact genomic DNA preps) to achieve reasonable $P_{\mathrm{dual-cut}}$. \textit{Critical parameter in CE.16, must be coordinated with fragment distribution} \\
\end{longtable}

\clearpage

%%%%%%%%%%%%%%%%%%%%%%%%%%%%%%%%%%%%%%%%%%%%%%%%%%%%%%%%%%%%%%%%%%%%%%%%
%% Appendix B: Core Mathematical Models for Single-Molecule Haplotype Classification
%% Version 6.0 - Unified Mathematical Framework
%% Derived from All_Math.pdf and harmonized with v6 notation
%%%%%%%%%%%%%%%%%%%%%%%%%%%%%%%%%%%%%%%%%%%%%%%%%%%%%%%%%%%%%%%%%%%%%%%%

\chapter{Core Mathematical Models for Single-Molecule Haplotype Classification}
\label{app:core-math}
\label{app:core-equations}
\label{app:appendixb}
\label{app:calibration-framework}
\label{app:purity-equations}

This appendix consolidates the mathematical content underlying Parts II--V. It formalizes the sequencing pipeline, empirical error models, classification posteriors, haplotagging, plasmid purity bounds, and dual Cas9 enrichment. This unified treatment provides the complete mathematical foundation for single-molecule sequencing haplotype classification with quantified uncertainty suitable for clinical applications.

%%%%%%%%%%%%%%%%%%%%%%%%%%%%%%%%%%%%%%%%%%%%%%%%%%%%%%%%%%%%%%%%%%%%%%%%
\section{Overview}
\label{sec:math-overview}

This appendix synthesizes and standardizes the mathematical models used throughout the framework:
\begin{enumerate}
\item Single-molecule sequencing signal and basecalling
\item Phred quality and alignment-based quality metrics
\item Sequence-level confusion matrices and Single Molecule Accuracy (SMA)
\item Bayesian haplotype and diplotype classification, including cost-based decision rules
\item Read-level haplotagging given a known haplotype
\item Plasmid replication models and purity bounds for physical standards
\item Dual Cas9 cutting and the probability of isolating a target gene
\end{enumerate}

Notation is chosen to be consistent with Part II and the main text; in particular, we distinguish edit distance $d_{\text{edit}}$ from molecule length $L_{\text{mol}}$ (often abbreviated as $L$ when context is clear).

%%%%%%%%%%%%%%%%%%%%%%%%%%%%%%%%%%%%%%%%%%%%%%%%%%%%%%%%%%%%%%%%%%%%%%%%
\section{Single-Molecule Sequencing and Basecalling}
\label{sec:math-basecalling}

\subsection{Raw Signal and Segmentation}
\label{subsec:signal-segmentation}

A sequencing run produces a time-indexed signal
\begin{equation}
X = (x_1,\dots,x_t),
\end{equation}
where $x_j$ is the measurement at time index $j$ and $t$ is the total number of samples. A single binding event corresponds to a contiguous subsequence
\begin{equation}
x^{(i)} = (x_{a_i},\dots,x_{b_i}), \quad \ell_i = b_i - a_i + 1.
\end{equation}

A segmentation model $S$ partitions $X$ into $n$ such events
\begin{equation}
X \xrightarrow{S} \{x^{(1)},\dots,x^{(n)}\}.
\end{equation}

This formalizes the ``signal space'' $\mathcal{S}$ in the state-space hierarchy of Chapter~\ref{chap:classification-model}.

\subsection{Basecalling}
\label{subsec:basecalling-model}

Let $\mathcal{A}$ be the nucleotide alphabet, typically
\begin{equation}
\mathcal{A} = \{A,C,G,T\} \quad \text{or} \quad \mathcal{A} = \{A,C,G,T,N\}.
\end{equation}

A basecaller $f$ maps each single-molecule signal to a read
\begin{equation}
r^{(i)} = f\!\bigl(x^{(i)}\bigr) \in \mathcal{A}^{L_i},
\end{equation}
and assigns a Phred quality score $Q^{(i)}_j$ to each base $r^{(i)}_j$. With the usual Phred convention,
\begin{equation}
Q^{(i)}_j = -10 \log_{10} p^{(i)}_j,
\qquad
p^{(i)}_j = 10^{-Q^{(i)}_j/10},
\end{equation}
where $p^{(i)}_j$ is the predicted error probability at position $j$ of read $i$.

Collecting all reads and quality scores from one run gives
\begin{equation}
R = \{r^{(1)},\dots,r^{(n)}\}, \qquad Q = \{Q^{(1)},\dots,Q^{(n)}\}.
\end{equation}

\subsection{Read-Level Predicted and Empirical Accuracy}
\label{subsec:read-accuracy}

For read $r^{(i)}$ of length $L_i$, a predicted per-read error rate is
\begin{equation}
\bar p^{(i)}_{\mathrm{pred}} = \frac{1}{L_i}\sum_{j=1}^{L_i} p^{(i)}_j
= \frac{1}{L_i}\sum_{j=1}^{L_i} 10^{-Q^{(i)}_j/10},
\end{equation}
with corresponding ``mean predicted Phred'':
\begin{equation}
\bar Q^{(i)}_{\mathrm{pred}} = -10 \log_{10} \bar p^{(i)}_{\mathrm{pred}}.
\end{equation}

Given a ground-truth sequence $s^{(i)}$ for read $i$, we define the Levenshtein edit distance $d_{\text{edit}}(r^{(i)}, s^{(i)})$, and the empirical per-read error rate
\begin{equation}
\bar p^{(i)}_{\mathrm{emp}} =
\frac{d_{\text{edit}}(r^{(i)}, s^{(i)})}{\lvert s^{(i)}\rvert},
\qquad
\bar Q^{(i)}_{\mathrm{emp}} = -10 \log_{10} \bar p^{(i)}_{\mathrm{emp}}.
\end{equation}

These quantities drive SEER's quality-overstatement analysis (Part IV).

%%%%%%%%%%%%%%%%%%%%%%%%%%%%%%%%%%%%%%%%%%%%%%%%%%%%%%%%%%%%%%%%%%%%%%%%
\section{Sequence Counts, Confusion Matrices, and Single Molecule Accuracy}
\label{sec:confusion-matrices}

\subsection{Confusion Matrix at Sequence Level}
\label{subsec:confusion-matrix}

In SEER and SMA-seq experiments we sequence physical standards of known sequence. Let $\mathcal{S} = \{s_1,\dots,s_M\}$ be the set of possible sequences (e.g., plasmid haplotypes). For each true $s_i$ and observed basecalled sequence $\hat s_j$, the confusion matrix element
\begin{equation}
C_{ij} = \#\{\text{molecules with true } s_i \text{ classified as } s_j\}
\end{equation}
counts classification outcomes. For each true sequence $s_i$,
\begin{equation}
N_i = \sum_{j=1}^M C_{ij}
\end{equation}
is the total number of molecules of type $s_i$.

The true positive rate (TPR) for $s_i$ is
\begin{equation}
\mathrm{TPR}(s_i) = \Prob(\hat s = s_i \mid s_i)
= \frac{C_{ii}}{N_i},
\end{equation}
and the misclassification probability is
\begin{equation}
\varepsilon_i = 1 - \mathrm{TPR}(s_i)
= \sum_{j\neq i}\frac{C_{ij}}{N_i}.
\end{equation}

\textbf{Explicit index convention:} $C_{ij}$ = count from true class $i$ (row) to predicted class $j$ (column).
\textbf{Convention:} Rows = true classes, columns = predicted classes.

\textbf{Row sums:} $N_i = \sum_{j=1}^{K} C_{ij}$ = total reads from true class $i$.

\textbf{Row-normalized matrix:} $\Pi_{ij} = C_{ij} / N_i$ gives $\Prob(\hat{s} = j \mid s = i)$.

\textbf{Uncertainty:} Each row follows a multinomial distribution. For Bayesian intervals, use Dirichlet posterior: $(\Pi_{i1}, \ldots, \Pi_{iK}) \sim \text{Dirichlet}(C_{i1} + \alpha, \ldots, C_{iK} + \alpha)$ with $\alpha = 0.5$ (Jeffreys prior) or $\alpha = 1$ (Laplace).

\textbf{Binomial Confidence Intervals:}
\label{def:binomial-ci}
For a single proportion $p$ estimated from $k$ successes in $n$ trials, use Wilson score interval or Clopper-Pearson exact interval. Wilson score interval (recommended for $n \geq 30$):
\begin{equation*}
\frac{\hat{p} + \frac{z^2}{2n}}{1 + \frac{z^2}{n}} \pm \frac{z}{1 + \frac{z^2}{n}} \sqrt{\frac{\hat{p}(1-\hat{p})}{n} + \frac{z^2}{4n^2}}
\end{equation*}
where $\hat{p} = k/n$ and $z$ is the standard normal quantile (e.g., $z = 1.96$ for 95\% confidence).

\textbf{Diagonal elements:} $\Pi_{ii} = \text{TPR}$ for class $i$, subject to purity constraint $\Pi_{ii} \leq \pi_i$ (CE.15).

\textbf{See:} Chapter~\ref{chap:classification-model} for introduction, Chapter~\ref{chap:sma-seq} for empirical construction.

\subsection*{Key Physical Quantities}

\begin{itemize}
\item $\pi$: Purity (fraction of molecules with intended sequence), $\pi \in [0,1]$
\item $L$: Sequence length in base pairs
\item $\ell$: Observed read length (may be truncated if $\ell < L$)
\item $N$: Number of reads, fragments, or trials
\item $r$: Per-base replication error rate (typically $\sim 10^{-9}$ for \emph{E. coli})
\item $k$: Number of replication cycles (e.g., bacterial doublings)
\item $h$: Haplotype (member of haplotype space $\mathcal{H}$)
\item $s$: Sequence (true sequence for a molecular class)
\item $\hat{s}$ or $r$: Observed/basecalled sequence (read)
\item $\mathbf{R} = \{r_1, \ldots, r_N\}$: Dataset of $N$ reads
\item $E$: End reason category ($E \in \{\text{SP}, \text{SN}, \text{MC}, \text{UMC}, \text{other}\}$)\\
\item $h(x)$: Read completion hazard function (survival analysis framework)
\end{itemize}

\bigskip
\hrule
\bigskip

\section*{Group 1: Foundational Concepts}
\addcontentsline{toc}{section}{Group 1: Foundational Concepts}

The classification framework is fundamentally probabilistic, requiring rigorous mathematical tools for uncertainty quantification and decision-making under incomplete information. This first group presents the core theoretical constructs that appear throughout all subsequent analyses: Bayes' theorem for belief updating, error models for handling imperfect measurements, likelihood ratios for evidence quantification, and entropy for measuring classification confidence.

\bigskip

\noindent\colorbox{eqboxbg}{\parbox{0.98\textwidth}{%
\vspace{5pt}
\textbf{\large Group Overview:} These four foundational equations establish the probabilistic and information-theoretic framework underlying all downstream computations. Bayes' rule (CE.1) enables belief updating from evidence, the error model (CE.2) quantifies measurement uncertainty, log-likelihood ratios (CE.3) compare competing hypotheses, and entropy (CE.4) measures classification confidence. Together, they form the mathematical foundation for rigorous haplotype inference.
\vspace{5pt}
}}

\bigskip

\begin{eqbox}{\textbf{CE.1} -- Bayes' Rule}
\CEanchor{1}
\label{eq:ce1}
\begin{equation*}
\Prob(h|r) = \frac{\Prob(h) \cdot \Prob(r|h)}{\Prob(r)} = \frac{\Prob(h) \cdot \Prob(r|h)}{\sum_{h'} \Prob(h') \cdot \Prob(r|h')}
\end{equation*}

\textbf{Purpose:} Fundamental equation for updating beliefs about haplotype $h$ given observed read $r$.

\textbf{Parameters:}
\begin{itemize}
\item $\Prob(h)$: Prior probability of haplotype $h$
\item $\Prob(r|h)$: Likelihood of observing read $r$ given haplotype $h$
\item $\Prob(h|r)$: Posterior probability of haplotype $h$ given read $r$
\end{itemize}

\textbf{Usage:} Applied iteratively across all reads in dataset. See Chapter~\ref{chap:posteriors} for full derivation.
\end{eqbox}

\begin{eqbox}{\textbf{CE.2} -- Error Model}
\CEanchor{2}
\label{eq:ce2}
\begin{equation*}
\Prob(\hat{s}|s) = \prod_{i=1}^{L} \Prob(\hat{s}_i|s_i, q_i)
\end{equation*}
where $\Prob(\text{error at position } i) = 10^{-q_i/10}$

\textbf{Purpose:} Models sequencing errors as position-independent events with quality-dependent probabilities.

\textbf{Parameters:}
\begin{itemize}
\item $s$: True sequence
\item $\hat{s}$: Observed sequence
\item $q_i$: Phred quality score at position $i$
\item $L$: Sequence length
\end{itemize}

\textbf{Assumptions:} Errors are independent across positions. Quality scores accurately reflect error probabilities.

\textbf{See:} Chapter~\ref{chap:classification} Section 4.3 for basecaller error models.
\end{eqbox}

\begin{eqbox}{\textbf{CE.3} -- Log-Likelihood Ratio}
\CEanchor{3}
\label{eq:ce3}
\begin{equation*}
\text{LLR}(r; h_i, h_j) = \log\frac{\Prob(r|h_i)}{\Prob(r|h_j)} = \log\Prob(r|h_i) - \log\Prob(r|h_j)
\end{equation*}

\textbf{Purpose:} Quantifies relative evidence for haplotype $h_i$ versus $h_j$ from read $r$.

\textbf{Interpretation:}
\begin{itemize}
\item LLR $> 0$: Evidence favors $h_i$
\item LLR $< 0$: Evidence favors $h_j$  
\item $|\text{LLR}| > 10$: Strong evidence (odds ratio $> e^{10} \approx 22,000:1$)
\end{itemize}

\textbf{Computational Note:} Always compute in log-space to avoid numerical underflow.

\textbf{See:} Chapter~\ref{chap:posteriors} for aggregation across reads.
\end{eqbox}

\begin{eqbox}{\textbf{CE.4} -- Entropy}
\CEanchor{4}
\label{eq:ce4}
\begin{equation*}
H(\mathbf{p}) = -\sum_{i} p_i \log p_i
\end{equation*}

\textbf{Purpose:} Measures uncertainty in probability distribution over haplotypes.

\textbf{Applications:}
\begin{itemize}
\item Maximum entropy $H_{\max} = \log|\mathcal{H}|$ for uniform distribution
\item Classification confidence: Lower entropy indicates higher confidence
\item Optimal experimental design: Maximize expected entropy reduction
\end{itemize}

\textbf{See:} Chapter~\ref{chap:experimental-design} for entropy-based design.
\end{eqbox}

\section*{Group 2: Fragment Generation and Design}
\addcontentsline{toc}{section}{Group 2: Fragment Generation and Design}

Having established the probabilistic foundations, we now connect theory to laboratory practice. Single-molecule sequencing begins with physical fragmentation of genomic DNA, a stochastic process that determines which haplotype regions are observable and with what coverage depth. These equations model fragment generation, predict coverage distributions, and establish sequencing depth requirements---bridging the gap between experimental design and downstream inference.

\bigskip

\noindent\colorbox{eqboxbg}{\parbox{0.98\textwidth}{%
\vspace{5pt}
\textbf{\large Group Overview:} These four equations model DNA fragmentation as a stochastic process and guide experimental design. The exponential fragment distribution (CE.5) describes random shearing, coverage probability (CE.6) predicts regional observability, expected depth (CE.7) connects fragment count to coverage, and Poisson approximation (CE.8) quantifies coverage variability. Together, they enable principled experimental planning and power analysis.
\vspace{5pt}
}}

\bigskip

\begin{eqbox}{\textbf{CE.5} -- Exponential Fragment Distribution}
\CEanchor{5}
\label{eq:ce5}
\begin{equation*}
f_L(\ell) = \lambda e^{-\lambda \ell}, \quad \mathbb{E}[L] = \frac{1}{\lambda}
\end{equation*}

\textbf{Purpose:} Models fragment length distribution from random shearing.

\textbf{Parameters:}
\begin{itemize}
\item $\lambda$: Rate parameter (typical range: $10^{-4}$ to $10^{-3}$ bp$^{-1}$)
\item $\mathbb{E}[L] = 1/\lambda$: Mean fragment length
\end{itemize}

\textbf{Validation:} Compare to empirical distribution using KL divergence (CE.9).

\textbf{See:} Chapter~\ref{chap:sma-seq} Section 11.3 for empirical distributions.
\end{eqbox}

\begin{eqbox}{\textbf{CE.6} -- Fragment Coverage Probability}
\CEanchor{6}
\label{eq:ce6}
\begin{equation*}
\Prob(\text{fragment covers } j) = \int_{-\infty}^{j} f_{\text{start}}(x) \cdot \Prob(L > j - x) \, dx
\end{equation*}

For uniform start positions and exponential lengths:
\begin{equation*}
= e^{-\lambda d_j}
\end{equation*}
where $d_j$ is distance from $j$ to nearest sequence end.

\textbf{Purpose:} Calculates probability that position $j$ is covered by a random fragment.

\textbf{Applications:} Coverage planning, variant detection power analysis.

\textbf{See:} Chapter~\ref{chap:experimental-design} for coverage calculations.
\end{eqbox}

\begin{eqbox}{\textbf{CE.7} -- Expected Coverage Depth}
\CEanchor{7}
\label{eq:ce7}
\begin{equation*}
\mathbb{E}[C_j] = N \cdot \Prob(\text{fragment covers } j)
\end{equation*}

\textbf{Purpose:} Predicts sequencing depth at position $j$ given $N$ fragments.

\textbf{Design Rule:} For 95\% confidence in variant detection at 1\% frequency:
\begin{equation*}
N \geq \frac{300}{e^{-\lambda d_j}}
\end{equation*}

\textbf{See:} Chapter~\ref{chap:experimental-design} Section 7.3 for power calculations.
\end{eqbox}

\begin{eqbox}{\textbf{CE.8} -- Poisson Coverage Approximation}
\CEanchor{8}
\label{eq:ce8}
\begin{equation*}
\Prob(C_j = k) \approx \frac{e^{-\mathbb{E}[C_j]} \cdot \mathbb{E}[C_j]^k}{k!}
\end{equation*}

\textbf{Purpose:} Models coverage variability for quality control and power analysis.

\textbf{Validity:} Accurate when $N \gg 1$ and fragments are independent.

\textbf{QC Application:} Flag positions where observed coverage deviates $>3\sigma$ from expectation.

\textbf{See:} Chapter~\ref{chap:qc-gates} for coverage adequacy assessment.
\end{eqbox}

\section*{Group 3: Empirical Quality Assessment}
\addcontentsline{toc}{section}{Group 3: Empirical Quality Assessment}

Theoretical models are only as reliable as their agreement with empirical reality. Before trusting classification results, we must verify that observed data conform to model assumptions. These quality assessment equations detect systematic deviations---fragmentation anomalies, sequencing artifacts, or library preparation failures---that invalidate theoretical predictions and mandate corrective action or alternative analytical strategies.

\bigskip

\noindent\colorbox{eqboxbg}{\parbox{0.98\textwidth}{%
\vspace{5pt}
\textbf{\large Group Overview:} These two equations provide data-driven quality gates that validate model assumptions before classification. Fragmentation KL divergence (CE.9) quantifies agreement between observed and theoretical fragment distributions, while read-length sanity checks (CE.10) detect sequencing artifacts. Failures trigger diagnostic procedures or switch to empirical models, ensuring classification reliability even when library preparation deviates from ideal conditions.
\vspace{5pt}
}}

\bigskip

\begin{eqbox}{\textbf{CE.9} -- Fragmentation KL Divergence}
\CEanchor{9}
\label{eq:ce9}
\begin{equation*}
D_{\text{KL}}(f_{\text{emp}} \| f_{\text{model}}) = \sum_{\ell} f_{\text{emp}}(\ell) \log\frac{f_{\text{emp}}(\ell)}{f_{\text{model}}(\ell)}
\end{equation*}

\textbf{Purpose:} Quantifies deviation between observed and expected fragment distributions.

\textbf{Quality Gates:}
\begin{itemize}
\item $D_{\text{KL}} < 0.05$ bits: PASS (use analytical model)
\item $0.05 \leq D_{\text{KL}} < 0.10$ bits: WARNING (prefer empirical)
\item $D_{\text{KL}} \geq 0.10$ bits: FAIL (must use empirical)
\end{itemize}

\textbf{See:} Chapter~\ref{chap:qc-gates} Gate 3 for diagnostic procedures.
\end{eqbox}

\begin{eqbox}{\textbf{CE.10} -- Read-Length Sanity Check}
\CEanchor{10}
\label{eq:ce10}
\begin{equation*}
D_{\text{KL}}(f_{\text{emp}} \| f_{\text{read}}) = \sum_{\ell} f_{\text{emp}}(\ell) \log\frac{f_{\text{emp}}(\ell)}{f_{\text{read}}(\ell)}
\end{equation*}

\textbf{Purpose:} Detects sequencing artifacts affecting read length distribution.

\textbf{Thresholds:}
\begin{itemize}
\item $D_{\text{KL}} < 0.10$ bits: Normal
\item $0.10 \leq D_{\text{KL}} < 0.30$ bits: Investigate filters/truncation
\item $D_{\text{KL}} \geq 0.30$ bits: Sequencing failure
\end{itemize}

\textbf{See:} Chapter~\ref{chap:qc-gates} for troubleshooting procedures.
\end{eqbox}

\section*{Group 4: Bayesian Inference Pipeline}
\addcontentsline{toc}{section}{Group 4: Bayesian Inference Pipeline}

With quality-validated data in hand, we now execute the core classification pipeline. These equations transform raw sequencing reads into haplotype classifications with quantified uncertainty. The workflow proceeds hierarchically: individual reads generate likelihoods (CE.11), aggregation across datasets produces overall evidence (CE.12), Bayes' rule yields posterior probabilities (CE.13), and quality metrics guard against systematic errors (CE.14-15). This rigorous framework ensures defensible clinical decisions supported by mathematically principled uncertainty quantification.

\bigskip

\noindent\colorbox{eqboxbg}{\parbox{0.98\textwidth}{%
\vspace{5pt}
\textbf{\large Group Overview:} These six equations implement the complete Bayesian inference pipeline for haplotype classification. Per-read likelihoods (CE.11) evaluate fragment-level evidence, dataset likelihoods (CE.12) aggregate across reads, posteriors and Bayes factors (CE.13) deliver final classifications with uncertainty bounds, quality overstatement metrics (CE.14) detect basecaller miscalibration, purity ceilings (CE.15) enforce physical constraints, and mixture models (CE.16) handle diploid samples. This forms a complete, quality-controlled inference system suitable for clinical deployment.
\vspace{5pt}
}}

\bigskip

\begin{eqbox}{\textbf{CE.11} -- Per-Read Likelihood}
\CEanchor{11}
\begin{equation*}
\Prob(r_n|h_i) = \sum_{s \in S_i} \Prob(r_n|s; \theta) \cdot \pi_i(s)
\end{equation*}

\textbf{Purpose:} Computes likelihood of read $r_n$ under haplotype $h_i$ by marginalizing over source fragments.

\textbf{Components:}
\begin{itemize}
\item $S_i$: Set of possible source fragments from haplotype $h_i$
\item $\Prob(r_n|s; \theta)$: Error model with parameters $\theta$
\item $\pi_i(s)$: Prior probability of fragment $s$ (includes fragmentation model)
\end{itemize}

\textbf{See:} Chapter~\ref{chap:classification} for implementation details.
\end{eqbox}

\begin{eqbox}{\textbf{CE.12} -- Dataset Likelihood}
\CEanchor{12}
\label{eq:ce12}
\begin{equation*}
\Prob(R|h_i) = \prod_{n=1}^{N} \Prob(r_n|h_i)
\end{equation*}

In log-space (required for numerical stability):
\begin{equation*}
\log\Prob(R|h_i) = \sum_{n=1}^{N} \log\Prob(r_n|h_i)
\end{equation*}

\textbf{Purpose:} Aggregates evidence across all reads assuming conditional independence.

\textbf{Quality Check:} Flag if $\log\Prob(R|h_i) < -10^6$ (indicates model failure).

\textbf{See:} Chapter~\ref{chap:posteriors} for aggregation methods.
\end{eqbox}

\begin{eqbox}{\textbf{CE.13} -- Posterior and Bayes Factor}
\CEanchor{13}
\label{eq:ce13}
\begin{equation*}
\Prob(h_i|R) = \frac{\Prob(h_i) \cdot \Prob(R|h_i)}{\sum_{j} \Prob(h_j) \cdot \Prob(R|h_j)}
\end{equation*}

\begin{equation*}
\text{BF}_{i:j} = \frac{\Prob(R|h_i)}{\Prob(R|h_j)}, \quad \log_{10}\text{BF}_{i:j} = \frac{\log\Prob(R|h_i) - \log\Prob(R|h_j)}{\log(10)}
\end{equation*}

\textbf{Purpose:} Final classification via posterior probability; evidence strength via Bayes factor.

\textbf{Evidence Scale:}
\begin{itemize}
\item $\log_{10}\text{BF} > 2$: Strong evidence (100:1 odds)
\item $\log_{10}\text{BF} > 3$: Very strong (1000:1)
\item $\log_{10}\text{BF} > 4$: Decisive (10000:1)
\end{itemize}

\textbf{Clinical Threshold:} Report classification if $\max_i \Prob(h_i|R) > 0.95$ AND $\log_{10}\text{BF} > 2$.

These empirically estimated probabilities define the sequence-level error model used in the likelihoods $\Prob(r \mid h)$ and $\Prob(r \mid m_j,h)$ in Chapters~\ref{chap:classification-model}--\ref{chap:posteriors}.
\end{eqbox}

\begin{eqbox}{\textbf{CE.14} -- Quality Overstatement Metric}
\CEanchor{14}
\label{eq:ce14}
\textbf{Purpose:} Detects systematic overconfidence (positive $\Delta Q$) or underconfidence (negative $\Delta Q$) in basecaller quality scores, enabling quality-aware filtering and downstream error modeling.
\subsection{Single Molecule Accuracy (SMA)}
\label{subsec:sma-definition}

We formalize \textbf{Single Molecule Accuracy (SMA)} as the per-sequence true positive rate:
\begin{equation}
\boxed{\mathrm{SMA}(s_i) \equiv \mathrm{TPR}(s_i) = \frac{C_{ii}}{N_i}.}
\end{equation}

Equivalently, SMA is the diagonal of the sequence-level confusion matrix. This definition aligns the name ``SMA-seq'' directly with the primary metric it measures and is directly estimable from SEER experiments.

SMA provides both:
\begin{itemize}
\item an interpretable per-standard accuracy metric, and
\item a physical ceiling: $\mathrm{SMA}(s_i) \leq \pi_i$, the purity of the standard for $s_i$ (see Section~\ref{sec:purity}).
\end{itemize}

\textbf{See:} Chapter~\ref{chap:qc-gates} Gate 2 for calibration procedures.
\end{eqbox}

\begin{eqbox}{\textbf{CE.15} -- Purity Ceiling Constraint}
\CEanchor{15}
\begin{equation*}
\pi_{\max} \approx (1-r)^{kL}, \quad Q_{\text{purity}} = -10\log_{10}(1-\pi)
\end{equation*}
\end{eqbox}

%%%%%%%%%%%%%%%%%%%%%%%%%%%%%%%%%%%%%%%%%%%%%%%%%%%%%%%%%%%%%%%%%%%%%%%%
\section{Phred Mean Inequality}
\label{sec:phred-mean}

Let $p_1,\dots,p_n$ be error probabilities with Phred scores $Q_i = -10\log_{10} p_i$. Define the arithmetic means
\begin{equation}
\bar p = \frac{1}{n}\sum_{i=1}^n p_i,\qquad
\bar Q = \frac{1}{n}\sum_{i=1}^n Q_i.
\end{equation}

\begin{proposition}
\begin{equation}
\bar Q \;\geq\; -10\log_{10}(\bar p),
\end{equation}
with equality if and only if all $p_i$ are equal (or $n=1$).
\end{proposition}

\begin{proof}[Sketch of proof]
Since $\log_{10}(\cdot)$ is concave, Jensen's inequality gives
\begin{equation}
\log_{10}\bar p \;\geq\; \frac{1}{n}\sum_{i=1}^n \log_{10} p_i.
\end{equation}
Multiplying by $-10$ reverses the inequality and yields the claim after substituting $Q_i$.
\end{proof}

\textbf{Interpretation.} Averaging in log-space (mean Phred) is optimistically biased relative to converting the mean error probability to a single Phred score; this matters when aggregating quality metrics over heterogeneous regions.

%%%%%%%%%%%%%%%%%%%%%%%%%%%%%%%%%%%%%%%%%%%%%%%%%%%%%%%%%%%%%%%%%%%%%%%%
\section{Alignment-Based Quality Metric}
\label{sec:alignment-quality}

Let $G = (g_1,\dots,g_N)$ be a ground-truth sequence and $B = (b_1,\dots,b_N)$ a basecalled sequence aligned to it, with per-base Phred scores $Q_i$ and error probabilities $p_i = 10^{-Q_i/10}$.

Define an alignment column score
\begin{equation}
s_i =
\begin{cases}
1 - p_i, & g_i = b_i,\\[4pt]
p_i, & g_i \neq b_i,\\[4pt]
0, & \text{if } g_i \text{ or } b_i \text{ is a gap}.
\end{cases}
\end{equation}

The mean correctness for the alignment is
\begin{equation}
M = \frac{1}{N}\sum_{i=1}^N s_i \in [0,1],
\end{equation}
and the corresponding alignment-level error probability is $p_{\mathrm{err}} = 1 - M$, giving a Phred-like score
\begin{equation}
Q_{\mathrm{align}} = -10\log_{10}(1-M).
\end{equation}

This provides a single scalar quality metric that blends empirical correctness and basecaller confidence, suitable for run-level QC or validation metrics in Part V.

%%%%%%%%%%%%%%%%%%%%%%%%%%%%%%%%%%%%%%%%%%%%%%%%%%%%%%%%%%%%%%%%%%%%%%%%
\section{Per-Base Likelihood for Variants and Haplotype Evidence}
\label{sec:per-base-likelihood}

For a haplotype (or reference) sequence $g = (g_1,\dots,g_L)$ and an aligned read $r = (r_1,\dots,r_L)$ with base-specific error rates $e_i$:
\begin{equation}
\Prob(r_i = g_i \mid g_i, e_i) = 1 - e_i,
\end{equation}
\begin{equation}
\Prob(r_i = b \neq g_i \mid g_i, e_i) = \frac{e_i}{\lvert \mathcal{A}\rvert - 1}.
\end{equation}

Assuming conditional independence across positions,
\begin{equation}
\Prob(r \mid g, e) = \prod_{i=1}^L
\Bigl[(1-e_i)\mathbb{I}\{r_i = g_i\}
+ \frac{e_i}{\lvert \mathcal{A}\rvert-1}\mathbb{I}\{r_i \neq g_i\}\Bigr].
\end{equation}

This per-read likelihood is the building block of the haplotype-level likelihoods $\Prob(r\mid h)$ and mixture models $\Prob(r) = \sum_i \alpha_i \Prob(r\mid h_i)$ used in Chapters~\ref{chap:posteriors} and~\ref{chap:mixtures}.

%%%%%%%%%%%%%%%%%%%%%%%%%%%%%%%%%%%%%%%%%%%%%%%%%%%%%%%%%%%%%%%%%%%%%%%%
\section{Plasmid Replication, Purity Bounds, and Purity Q-Values}
\label{sec:purity}

\subsection{Replication Model and Purity Ceiling}
\label{subsec:purity-ceiling}

Consider a plasmid of length $L$ (bp), per-base replication error rate $r$, and $k$ replication cycles. For a single base, the probability of remaining error-free over $k$ cycles is $(1-r)^k$; assuming independence across bases, the probability that the entire plasmid remains unchanged is
\begin{equation}
P_{\mathrm{pure}}(k) = (1-r)^{Lk} \approx \exp(-r L k)
\end{equation}
for small $r$.

This yields an upper bound on the fraction of perfectly correct molecules after $k$ cycles---the \textbf{purity ceiling}. The fraction of mutated molecules is
\begin{equation}
P_{\mathrm{mut}}(k) = 1 - P_{\mathrm{pure}}(k).
\end{equation}

For any classifier operating on this standard, the measured TPR cannot exceed the underlying purity:
\begin{equation}
\boxed{\mathrm{TPR}(s_i) \leq \pi_i \leq P_{\mathrm{pure}}(k),}
\end{equation}
which is the \textbf{Purity Constraint} in the SEER framework.

\subsection{Purity Q-Value}
\label{subsec:purity-q}

Treating $P_{\mathrm{mut}}(k)$ as an ``error probability,'' define a purity Phred score
\begin{equation}
Q_{\mathrm{pur}} = -10 \log_{10} \bigl(1 - P_{\mathrm{pure}}(k)\bigr)
= -10\log_{10} P_{\mathrm{mut}}(k).
\end{equation}

Higher $Q_{\mathrm{pur}}$ corresponds to higher purity (fewer mutated molecules). This provides a convenient way to compare standards or to set acceptable purity thresholds in Part III.

\subsection{Empirical Purity Estimation}
\label{subsec:empirical-purity}

\paragraph{Lower bound from capillary electrophoresis.}
If $C_{\mathrm{major}}$ and $C_{\mathrm{other}}$ denote the concentrations of the main plasmid band and all other bands, then
\begin{equation}
P_{\mathrm{low}} = \frac{C_{\mathrm{major}}}{C_{\mathrm{major}} + C_{\mathrm{other}}}
\end{equation}
is a lower bound on purity.

\paragraph{Empirical purity from clonal sequencing.}
For experiment $E$ with $a$ matching and $b$ total colonies:
\begin{equation}
\hat P_E = \frac{a}{b}.
\end{equation}
Aggregating across experiments for a strain:
\begin{equation}
\hat P_{\mathrm{strain}} = \frac{c}{d}
\end{equation}
where $c$ of $d$ colonies match the original design sequence.

These estimates are compared to theoretical bounds $P_{\mathrm{pure}}(k)$ to validate assumptions in Chapters~\ref{chap:purity} and~\ref{chap:plasmid-standards}.

%%%%%%%%%%%%%%%%%%%%%%%%%%%%%%%%%%%%%%%%%%%%%%%%%%%%%%%%%%%%%%%%%%%%%%%%
\section{Haplotype Classification with Unknown Haplotype}
\label{sec:haplotype-classification}

Let
\begin{equation}
\mathcal{H} = \{h_1,\dots,h_P\}
\end{equation}
be the set of candidate haplotypes (e.g., star alleles of a pharmacogene). For each haplotype $h_i$, $M(h_i) = \{m_{i1},\dots,m_{i v_i}\}$ denotes its set of molecules (chromosomes, plasmids) and $\Prob(h_i)$ its prior probability.

A read $r$ is generated via mutation, fragmentation, library preparation, and sequencing. Abstractly,
\begin{equation}
\Prob(r \mid h_i) =
\sum_{u\in U(h_i)}\sum_{d\in D(h_i)}\sum_{\ell\in L(h_i)}
\Prob(r \mid \ell, \theta_{\mathrm{seq}})
\Prob(\ell \mid d, \theta_{\mathrm{lab}})
\Prob(d \mid u, \theta_{\mathrm{frag}})
\Prob(u \mid h_i, \mu, n_{\mathrm{div}}, L),
\end{equation}
where $U,D,L$ denote post-mutation sequences, fragments, and library molecules, and $\theta_{\mathrm{seq}},\theta_{\mathrm{lab}},\theta_{\mathrm{frag}}$ capture process-specific parameters.

Assuming conditional independence of reads given $h_i$,
\begin{equation}
\Prob(R \mid h_i) = \prod_{r\in R} \Prob(r \mid h_i).
\end{equation}

By Bayes' theorem, the posterior is
\begin{equation}
\Prob(h_i \mid R) =
\frac{\Prob(R \mid h_i) \Prob(h_i)}
{\sum_{j=1}^P \Prob(R \mid h_j) \Prob(h_j)}.
\label{eq:bayes-posterior}
\end{equation}

A simple MAP classifier selects
\begin{equation}
\hat h = \arg\max_i \Prob(h_i \mid R),
\end{equation}
optionally subject to a posterior threshold $\Prob(\hat h\mid R) \ge \gamma$ to enforce minimum confidence.

Define the likelihood ratio
\begin{equation}
\mathrm{LR}_i(R) =
\frac{\Prob(h_i \mid R)}{1-\Prob(h_i \mid R)},
\end{equation}
which is used for reporting and for cost-based decision rules in clinical contexts.

%%%%%%%%%%%%%%%%%%%%%%%%%%%%%%%%%%%%%%%%%%%%%%%%%%%%%%%%%%%%%%%%%%%%%%%%
\section{Diplotypes, Polyploidy, and Cost-Based Decision Rules}
\label{sec:diplotypes}

Let $\mathcal{D}$ be the set of possible diplotypes (pairs of haplotypes), with priors $\pi_d$ for $d\in \mathcal{D}$. For each diplotype $d$, the posterior given reads $R$ is
\begin{equation}
\Prob(d \mid R) =
\frac{L_d(R)\,\pi_d}{\sum_{d'\in \mathcal{D}} L_{d'}(R)\,\pi_{d'}},
\end{equation}
where $L_d(R) = \Prob(R\mid d)$ is the diplotype likelihood, computed via mixture models over the two (or more, in CNVs/polyploidy) constituent haplotypes.

For a policy that either calls a diplotype $d'$, flags for resequencing, or leaves the sample unresolved, define:
\begin{itemize}
\item $\varepsilon_{d\to d'}(\gamma,N)$: misclassification rate from true $d$ to called $d'$ when using $N$ reads and posterior threshold $\gamma$.
\item $\psi_d(\gamma,N)$: probability that a sample with true $d$ is sent for resequencing.
\end{itemize}

Let $C_{d\to d'}$ be the cost of misclassifying $d$ as $d'$, and $C_{\mathrm{res},d}$ the cost of resequencing when the true diplotype is $d$. Then the expected cost at $(\gamma,N)$ is
\begin{equation}
C(\gamma,N) =
\sum_{d\in \mathcal{D}} \pi_d
\Biggl[
\sum_{d'\neq d} C_{d\to d'} \varepsilon_{d\to d'}(\gamma,N)
+ C_{\mathrm{res},d}\,\psi_d(\gamma,N)
\Biggr].
\end{equation}

Optimal operating points $(\gamma^*,N^*)$ minimize $C(\gamma,N)$ subject to sensitivity, PPV, and budget constraints, as discussed in Part V.

%%%%%%%%%%%%%%%%%%%%%%%%%%%%%%%%%%%%%%%%%%%%%%%%%%%%%%%%%%%%%%%%%%%%%%%%
\section{Read-Level Haplotagging Given a Known Haplotype}
\label{sec:haplotagging}

When the sample haplotype $h$ is known (e.g., clonal plasmid or fully resolved genome), the problem becomes assigning each read to its most likely molecule of origin $m_j \in M(h)$.

Let $\Prob(m_j\mid h)$ be the prior mole fraction of molecule $m_j$. For a read $r$,
\begin{equation}
\Prob(m_j \mid r, h) =
\frac{\Prob(r\mid m_j, h)\,\Prob(m_j\mid h)}
{\sum_k \Prob(r\mid m_k, h)\,\Prob(m_k\mid h)}.
\end{equation}

Define
\begin{equation}
\mathrm{LR}_j(r) =
\frac{\Prob(m_j \mid r, h)}{1-\Prob(m_j\mid r,h)}.
\end{equation}

Given a threshold $\tau$, the decision rule is
\begin{itemize}
\item assign $r$ to $m_j$ if $\mathrm{LR}_j(r)\ge \tau$ for some $j$;
\item otherwise mark $r$ as unphased.
\end{itemize}

Let $N$ be the number of reads, and define
\begin{itemize}
\item $P_{\mathrm{unph}}(\tau)$: probability a read is unphased;
\item $P_{\mathrm{mis}}(\tau)$: probability a read is misassigned.
\end{itemize}

Then the expected number of unphased reads is $U(\tau,N) = N P_{\mathrm{unph}}(\tau)$. A per-read cost formulation
\begin{equation}
C(\tau,N) = C_{\mathrm{mis}}\,N P_{\mathrm{mis}}(\tau)
+ C_{\mathrm{unph}}\,U(\tau,N)
\end{equation}
permits optimizing $\tau$ analogously to $\gamma$ in the diplotype setting.

%%%%%%%%%%%%%%%%%%%%%%%%%%%%%%%%%%%%%%%%%%%%%%%%%%%%%%%%%%%%%%%%%%%%%%%%
\section{Dual Cas9 Cutting and Probability of Isolating a Gene}
\label{sec:dual-cas9}

Consider dual Cas9 cuts flanking a gene of length $G$ bp. Let $L$ be the fragment length with pdf $f_L(\ell)$, cdf $F_L(\ell)$. The probability a fragment is at least as long as the gene is
\begin{equation}
p_{\mathrm{frag}}(G) = \Prob(L \ge G) = 1 - F_L(G^-).
\end{equation}

Let $e_1,e_2$ be the cutting efficiencies at the two sites, assumed independent. Then
\begin{equation}
p_{\mathrm{dual}}(G) = p_{\mathrm{frag}}(G)\,e_1 e_2.
\end{equation}

\begin{example}[Exponential Fragment Lengths]
If $L \sim \mathrm{Exp}(\lambda)$, then $p_{\mathrm{frag}}(G) = e^{-\lambda G}$ and
\begin{equation}
p_{\mathrm{dual}}(G) = e^{-\lambda G} e_1 e_2,
\end{equation}
illustrating the exponential decay of success probability with gene length under random fragmentation.
\end{example}

This model underpins design of Cas9-enriched workflows in Part III.

%%%%%%%%%%%%%%%%%%%%%%%%%%%%%%%%%%%%%%%%%%%%%%%%%%%%%%%%%%%%%%%%%%%%%%%%
\section{Summary of Key Symbols}
\label{sec:symbols}

This appendix uses the same global notation as the main text; the following is a local reminder of the most frequently used symbols:
\begin{itemize}
\item $X$: raw single-molecule signal; $x^{(i)}$: single binding event.
\item $R$: set of reads; $\mathcal{A}$: nucleotide alphabet.
\item $Q_i$: base-level Phred score; $p_i$: corresponding error probability.
\item $d_{\text{edit}}(r,s)$: Levenshtein edit distance; $L$ or $L_{\text{mol}}$: molecule length (bp).
\item $C_{ij}$: confusion matrix counts (rows=true, columns=predicted); $\mathrm{SMA}(s_i) = C_{ii}/N_i$.
\item $P_{\mathrm{pure}}(k)$: replication-based purity ceiling; $Q_{\mathrm{pur}}$: purity Phred score; $\pi$: empirical purity.
\item $\mathcal{H}$: haplotype set; $\mathcal{D}$: diplotype set; $\pi_d$: diplotype prior.
\item $L_d(R)$: diplotype likelihood; $\gamma$: posterior threshold.
\item $\Prob(m_j\mid r,h)$: read-level haplotagging posterior; $\tau$: haplotagging LR threshold.
\item $G$: gene length; $L$: fragment length with pdf $f_L$; $e_1,e_2$: Cas9 efficiencies.
\end{itemize}

%%%%%%%%%%%%%%%%%%%%%%%%%%%%%%%%%%%%%%%%%%%%%%%%%%%%%%%%%%%%%%%%%%%%%%%%
\section{Conclusion}
\label{sec:math-conclusion}

This appendix provides a complete, unified mathematical framework for single-molecule haplotype classification. The models span:
\begin{itemize}
\item Signal processing and basecalling (Section~\ref{sec:math-basecalling})
\item Empirical error characterization via confusion matrices and SMA (Section~\ref{sec:confusion-matrices})
\item Quality metrics and calibration (Sections~\ref{sec:phred-mean}--\ref{sec:alignment-quality})
\item Physical constraints from plasmid purity (Section~\ref{sec:purity})
\item Bayesian inference for haplotype and diplotype classification (Sections~\ref{sec:haplotype-classification}--\ref{sec:diplotypes})
\item Read-level haplotagging (Section~\ref{sec:haplotagging})
\item Experimental design for enrichment (Section~\ref{sec:dual-cas9})
\end{itemize}

Together, these models form the foundation for the SEER--SMA-seq framework presented in Parts II--IV, enabling rigorous haplotype classification with quantified uncertainty suitable for clinical deployment. For detailed derivations, computational implementations, and clinical applications, refer to the corresponding chapters as noted throughout this appendix.

%%%%%%%%%%%%%%%%%%%%%%%%%%%%%%%%%%%%%%%%%%%%%%%%%%%%%%%%%%%%%%%%%%%%%%%%
%% Appendix C: Laboratory Protocols and Quality Control Procedures
%% Version 6.0 - Complete Migration from v5.tex (lines 2165-2252)
%%%%%%%%%%%%%%%%%%%%%%%%%%%%%%%%%%%%%%%%%%%%%%%%%%%%%%%%%%%%%%%%%%%%%%%%

\chapter{Laboratory Protocols and Quality Control Procedures}
\label{app:protocols}
\label{app:qc}

This comprehensive quality control checklist ensures rigorous validation at every stage of the haplotype classification pipeline, from pre-sequencing library preparation through final result interpretation. Each checkpoint references relevant core equations and specifies quantitative criteria for pass/fail decisions. Systematic application of these quality gates prevents propagation of errors, identifies systematic biases, and ensures confidence in final classifications.

\textbf{Quality Philosophy:} The framework implements defense-in-depth quality control, with independent validation at multiple stages. Early-stage failures (pre-sequencing, sequencing run) indicate technical problems requiring experimental intervention, while late-stage failures (analysis, result) may indicate model misspecification or sample quality issues. All quality metrics should be documented in standardized QC reports for auditing and method validation.

\textbf{Checkpoint Organization:} The checklist is organized chronologically through the experimental workflow: (1) Pre-Sequencing QC validates library quality before costly sequencing, (2) Sequencing Run QC monitors instrument performance in real-time, (3) Analysis QC assesses data quality and coverage adequacy, and (4) Result QC validates final classifications against purity and quality constraints. Each checkpoint includes specific metrics, threshold criteria, and recommended corrective actions for failures.

\section{Core Quality Control Gates and Thresholds}

\begin{table}[H]
\centering
\caption{Core Quality Control Gates and Thresholds}
\label{tab:qc-gates-summary}
\begin{tabular}{p{3.5cm}p{4cm}p{2.5cm}p{5cm}}
\toprule
\textbf{QC Gate} & \textbf{Metric} & \textbf{Threshold} & \textbf{Action / Notes} \\
\midrule
Fragment model validation & $D_{\mathrm{KL}}(f_{\mathrm{emp}} \| f_{\mathrm{frag}})$ & $<\mathbf{0.05}$ bits & Prefer empirical $f_{\mathrm{emp}}$; if using analytic model, flag result; see \CEref{9}. \\
\midrule
Read-length sanity check & $D_{\mathrm{KL}}(f_{\mathrm{emp}} \| f_{\mathrm{read}})$ & Warn: 0.10--0.20 bits; Fail: $>0.30$ bits & Investigate truncation, filters, size selection, or ligation bias; see \CEref{10}. \\
\midrule
Quality overstatement detection & $d = \frac{1}{N}\sum \mathbb{I}\{Q_{\mathrm{pred}} > Q_{\mathrm{emp}}\}$ & $d \leq 0.30$ & Report with 95\% CI; high $d$ triggers calibration review; see \CEref{14}. \\
\midrule
Purity ceiling enforcement & TPR vs. purity & \textbf{Hard gate:} TPR $\leq$ purity ($\pi$ or $\pi_{\max}$) & Do not report TPR exceeding measured or estimated purity ceiling; violations indicate contamination, reference errors, or model failure; see \CEref{15}. \\
\bottomrule
\end{tabular}
\end{table}

\textbf{Detailed QC Workflow:} The following sections provide comprehensive guidance for each quality control checkpoint, including measurement protocols, interpretation guidelines, and corrective actions.

\section{Pre-Sequencing Quality Control (Library Validation)}
    
These measurements validate library quality before sequencing, enabling early detection of preparation issues that would compromise downstream analysis. Failed libraries should be reprepared rather than sequenced.
    
\begin{itemize}
\item \textbf{Fragment Size Distribution:} Measure via Tapestation or BioAnalyzer, normalize using CE.7. Verify distribution shape matches expected profile for fragmentation method. Document mean, median, and mode fragment lengths.

\item \textbf{Fragment Model Validation:} Compute $D_{\mathrm{KL}}(f_{\mathrm{emp}} \| f_{\mathrm{frag}}) < 0.05$ bits using CE.9. High divergence indicates complex distributions requiring empirical measurements throughout.

\item \textbf{Library Concentration:} Measure using fluorometric quantification (Qubit). Verify concentration meets sequencing platform requirements.

\item \textbf{Library Quality:} Assess adapter dimer content (<5\% of total library), confirm absence of degradation peaks.

\item \textbf{Spike-In Control Preparation:} Add known-sequence controls at 1-5\% molar ratio. Verify control concentration and purity before mixing.
\end{itemize}

\section{Sequencing Run Quality Control (Real-Time Monitoring)}

These metrics enable real-time assessment of sequencing quality during the run. Severe failures may warrant terminating the run early to conserve reagents.

\begin{itemize}
\item \textbf{Real-Time Quality Score Distribution:} Monitor mean Q-score evolution over time. Expected values: ONT Q10-Q18, PacBio HiFi Q20-Q30, Illumina Q30-Q35. Declining quality may indicate flow cell exhaustion or reagent issues.

\item \textbf{Read Yield Tracking:} Compare actual read count to expected based on loading concentration. Low yield (<50\% expected) indicates loading, pore occupancy, or flow cell issues.

\item \textbf{Read Length Distribution:} For long-read platforms, verify mean read length matches library preparation target. Systematic shortening indicates shearing during handling or pore blockage.

\item \textbf{Pore Availability (ONT-specific):} Monitor active pore count. Rapid pore loss indicates sample quality issues (contaminants, high salt) or temperature control problems.

\item \textbf{Error Mode Detection:} Analyze error patterns for systematic artifacts. Homopolymer errors (ONT) or cycles with quality drops (Illumina) indicate platform-specific issues.
\end{itemize}

\section{Post-Sequencing Analysis Quality Control}

These post-sequencing metrics validate data quality and assess whether coverage is adequate for confident classification. Failed analyses may require resequencing.

\begin{itemize}
\item \textbf{Alignment Rate Statistics:} Verify >90\% of reads align to reference haplotypes. Low alignment rates indicate contamination, reference errors, or sequencing failures.

\item \textbf{Coverage Uniformity:} Assess coverage distribution across genomic region. Extreme non-uniformity (>10-fold variation) suggests capture bias or PCR artifacts.

\item \textbf{Perfect-Read Fraction Validation:} Compare observed perfect-read fraction to predictions from CE.2-CE.3. Verify agreement within 95\% confidence intervals. Large discrepancies indicate quality score miscalibration.

\item \textbf{Coverage Adequacy:} Verify $\Pr(X \geq k) > 0.95$ using CE.5 where $k$ is target perfect-read coverage. Insufficient coverage requires deeper sequencing.

\item \textbf{Posterior Entropy:} Compute $H = -\sum_i P(h_i|R)\log P(h_i|R)$. Low entropy (<0.5 bits) indicates clear classification, high entropy (>2 bits) indicates ambiguous data.

\item \textbf{Control Sample Concordance:} Verify spike-in controls classify to expected haplotypes with >99\% posterior probability. Control failures indicate systematic model errors.
\end{itemize}

\section{Result Quality Control (Classification Validation)}

These final checks validate classification results against fundamental physical constraints and quality thresholds. Failed results should be flagged as "insufficient evidence" rather than reported with confidence.

\begin{itemize}
\item \textbf{Posterior Probability Thresholds:} Verify $\max_i P(h_i|R) > \tau$ where $\tau$ is assay-specific threshold (typically 0.90-0.95 for clinical applications). Sub-threshold posteriors indicate ambiguous data requiring deeper sequencing or repeat analysis.

\item \textbf{Bayes Factor Validation:} Confirm $\log_{10}\mathrm{BF} > 2$ for top hypothesis versus alternatives using CE.13. Weak Bayes factors (<2) indicate insufficient evidence for confident classification.

\item \textbf{Quality Overstatement Assessment:} Verify overstatement fraction $d \leq 0.30$ using CE.14. High overstatement indicates basecaller miscalibration requiring quality score transformation. Report with 95\% confidence intervals.

\item \textbf{Purity Constraint Validation:} Enforce TPR $\leq \pi_{\max}$ using CE.15. Violations indicate contamination, reference errors, or model failure. This is a hard constraint - violations mandate investigation before reporting results.

\item \textbf{Technical Replicate Concordance:} For critical samples, verify replicate classifications agree with >95\% posterior on same haplotype. Discordant replicates indicate reproducibility issues or sample heterogeneity.

\item \textbf{Negative Control Validation:} Process negative controls through entire pipeline. Verify they fail to classify (uniform posterior) or classify to expected negative haplotype with high confidence.
\end{itemize}

\section{Quality Control Workflow Integration}

These checkpoints should be implemented as automated gates in analysis pipelines with clear pass/fail criteria. Failed checkpoints should trigger alerts with specific diagnostic recommendations. All QC metrics should be included in standardized reports, enabling systematic review and continuous method improvement. For clinical applications, implement additional regulatory-compliant documentation of all QC results.

\subsection{Corrective Actions by QC Failure Type}

\begin{table}[H]
\centering
\caption{Recommended Corrective Actions for QC Failures}
\label{tab:corrective-actions}
\begin{tabular}{p{4cm}p{4cm}p{6cm}}
\toprule
\textbf{Failure Type} & \textbf{Stage} & \textbf{Corrective Action} \\
\midrule
Low library quality & Pre-sequencing & Re-extract DNA or re-prepare library with fresh reagents \\
Poor fragment distribution & Pre-sequencing & Adjust fragmentation protocol; optimize size selection \\
Low read yield & Sequencing & Check loading concentration; verify instrument calibration \\
Declining quality & Sequencing & Monitor pore/cluster health; consider terminating run early \\
Low alignment rate & Analysis & Check reference sequences; screen for contamination \\
Poor coverage uniformity & Analysis & Investigate capture bias; check PCR conditions \\
Failed controls & Analysis & Re-run with fresh standards; check confusion matrix \\
Sub-threshold posterior & Result & Increase sequencing depth; verify sample quality \\
Purity constraint violation & Result & Investigate contamination; verify reference accuracy \\
\bottomrule
\end{tabular}
\end{table}

\section{Standard Operating Procedure (SOP) Template}

For regulatory compliance and method validation, all QC procedures should be documented in Standard Operating Procedures. Below is a template structure:

\subsection{SOP Template Structure}

\begin{enumerate}
\item \textbf{Purpose:} Clear statement of QC objective
\item \textbf{Scope:} Which samples, assays, or protocols this QC applies to
\item \textbf{Definitions:} Key terms and abbreviations
\item \textbf{Equipment and Materials:} Required instruments and reagents
\item \textbf{Procedure:} Step-by-step instructions with decision points
\item \textbf{Acceptance Criteria:} Quantitative pass/fail thresholds
\item \textbf{Documentation:} Required records and data storage
\item \textbf{Corrective Actions:} Response to failures
\item \textbf{References:} Citations to equations, protocols, literature
\item \textbf{Revision History:} Changes to SOP over time
\end{enumerate}

\subsection{Example SOP: Quality Score Calibration Check}

\textbf{Purpose:} Validate basecaller quality score calibration using plasmid standards before clinical sample analysis.

\textbf{Scope:} All Oxford Nanopore sequencing runs for haplotype classification.

\textbf{Procedure:}
\begin{enumerate}
\item Sequence plasmid standard with known sequence
\item Align reads to reference using minimap2
\item Extract quality scores and compute empirical error rates
\item Stratify by quality score bin (Q10-Q15, Q15-Q20, etc.)
\item Compute overstatement fraction $d$ using CE.14
\item Compare to threshold: $d \leq 0.30$ (pass), $d > 0.30$ (fail)
\end{enumerate}

\textbf{Acceptance Criteria:} $d \leq 0.30$ with 95\% confidence interval not exceeding 0.35.

\textbf{Corrective Action:} If failed, apply quality score recalibration or retrain basecaller on plasmid standards before proceeding with clinical samples.

\section{Audit Trail and Documentation}

For clinical and regulated applications, maintain complete records of all QC checks:

\begin{itemize}
\item \textbf{Raw QC Data:} Store all measurements, not just pass/fail decisions
\item \textbf{Timestamps:} Record date and time of each QC checkpoint
\item \textbf{Operator Information:} Document who performed each procedure
\item \textbf{Instrument IDs:} Track which instruments were used
\item \textbf{Failure Reports:} Detailed documentation of failures and corrective actions
\item \textbf{Sign-offs:} Required approvals before releasing results
\end{itemize}

\clearpage

%%%%%%%%%%%%%%%%%%%%%%%%%%%%%%%%%%%%%%%%%%%%%%%%%%%%%%%%%%%%%%%%%%%%%%%%
%% Appendix D: Software Tools and Implementation
%% Part: Appendices
%% Version 6.0 - Migrated from v5.tex Appendix B (lines 2114-2164) + NEW sections
%%%%%%%%%%%%%%%%%%%%%%%%%%%%%%%%%%%%%%%%%%%%%%%%%%%%%%%%%%%%%%%%%%%%%%%%

\chapter{Software Tools and Implementation}
\label{app:software}

This appendix provides comprehensive guidance for implementing the haplotype classification framework, including algorithm pseudocode, software requirements, computational considerations, and integration guidelines. The content bridges theoretical foundations and practical deployment, enabling reproducible implementation across diverse computational environments.

\section{Core Algorithm}

The following pseudocode integrates all core equations into a unified computational workflow, serving as a reference for software development. Each step references relevant core equations (CE.1-CE.16) to maintain traceability between mathematical formulation and computational execution.

\textbf{Implementation Notes:} The algorithm emphasizes numerical stability through log-space computation, computational efficiency through strategic alignment restriction, and quality control through integrated validation gates. Production implementations should include comprehensive error handling, progress monitoring, and diagnostic output generation beyond what is shown in this condensed pseudocode.

\textbf{Computational Complexity:} The dominant cost is likelihood computation, scaling as $O(N \times P \times A)$ where $N$ is read count, $P$ is haplotype count, and $A$ is average alignments per read-haplotype pair. Strategic alignment filtering (restricting to high-quality alignments) reduces $A$ dramatically without sacrificing accuracy.

\begin{algorithm}[H]
\caption{Complete Haplotype Classification Pipeline with Integrated Quality Control}
\label{alg:complete_pipeline}
\begin{algorithmic}
\REQUIRE Sequencing reads $\mathbf{r}$, Reference haplotypes $\mathcal{H}$, Empirical distributions
\ENSURE Posterior probabilities, Quality metrics
\STATE // Preprocessing
\STATE Load empirical fragment distribution $f_{\text{emp}}$
\STATE Calibrate basecaller model using controls
\STATE Initialize log-likelihood matrix $\mathbf{L} \in \mathbb{R}^{P \times N}$
\STATE // Quality Gates
\STATE Check $D_{KL}(f_{\text{emp}} \| f_{\text{frag}}) < 0.05$ bits (CE.9)
\STATE Check $D_{KL}(f_{\text{emp}} \| f_{\text{read}})$ within assay bands (CE.10)
\STATE // Likelihood Computation
\FOR{each read $r_n \in \mathbf{r}$}
    \FOR{each haplotype $h_i \in \mathcal{H}$}
        \STATE Find alignments $\mathcal{A}_{ni}$ of $r_n$ to $h_i$
        \STATE $L_{ni} \leftarrow 0$
        \FOR{each alignment $a \in \mathcal{A}_{ni}$}
            \STATE Extract fragment $s$ from alignment
            \STATE Compute $\pi_i(s)$ using empirical weights (CE.11)
            \STATE Compute $P(r_n|s; \theta)$ from quality scores
            \STATE $L_{ni} \leftarrow L_{ni} + P(r_n|s) \cdot \pi_i(s)$
        \ENDFOR
        \STATE $L_{in} \leftarrow \log L_{ni}$
    \ENDFOR
\ENDFOR
\STATE // Posterior Computation
\FOR{each haplotype $h_i \in \mathcal{H}$}
    \STATE $\log P(\mathbf{r}|h_i) \leftarrow \sum_n L_{in}$ (CE.12)
    \STATE $P(h_i|\mathbf{r}) \leftarrow \frac{P(h_i) \exp(\log P(\mathbf{r}|h_i))}{\sum_j P(h_j) \exp(\log P(\mathbf{r}|h_j))}$ (CE.13)
\ENDFOR
\STATE // Quality Assessment
\STATE Compute Bayes factors for top hypotheses (CE.13)
\STATE Check quality overstatement $d \leq 0.30$ (CE.14)
\STATE Verify TPR $\leq$ purity constraint (CE.15)
\STATE Calculate posterior predictive distributions
\STATE Generate diagnostic plots
\RETURN Posterior probabilities, Quality metrics
\end{algorithmic}
\end{algorithm}

\section{Software Requirements}

\subsection{Core Dependencies}

\textbf{Programming Languages:}
\begin{itemize}
\item Python $\geq$ 3.8 (recommended for pipeline implementation)
\item R $\geq$ 4.0 (recommended for statistical analysis and visualization)
\item C/C++ (optional, for performance-critical alignment routines)
\end{itemize}

\textbf{Essential Python Packages:}
\begin{itemize}
\item \texttt{numpy} $\geq$ 1.20: Numerical computing and array operations
\item \texttt{scipy} $\geq$ 1.7: Statistical functions and optimization
\item \texttt{pandas} $\geq$ 1.3: Data manipulation and analysis
\item \texttt{biopython} $\geq$ 1.79: Sequence analysis and file parsing
\item \texttt{pysam} $\geq$ 0.17: BAM/SAM file manipulation
\end{itemize}

\textbf{Alignment Tools:}
\begin{itemize}
\item \texttt{minimap2} $\geq$ 2.24: Long-read alignment (primary recommendation)
\item \texttt{bwa-mem} $\geq$ 0.7.17: Alternative aligner for shorter reads
\item \texttt{GraphAligner} (optional): For pangenome graph alignment
\end{itemize}

\textbf{Quality Control Tools:}
\begin{itemize}
\item \texttt{NanoPlot} or \texttt{LongQC}: Read quality assessment
\item \texttt{MultiQC}: Aggregate QC report generation
\item Fragment analyzer software (Agilent Tapestation, AATI Fragment Analyzer)
\end{itemize}

\subsection{Optional Dependencies}

\begin{itemize}
\item \texttt{pytorch} or \texttt{tensorflow}: For basecaller fine-tuning (Chapter~\ref{chap:basecaller})
\item \texttt{snakemake} or \texttt{nextflow}: Workflow management systems
\item \texttt{matplotlib}, \texttt{seaborn}, \texttt{plotly}: Visualization libraries
\item \texttt{jupyter}: Interactive analysis notebooks
\end{itemize}

\subsection{Oxford Nanopore-Specific Tools}

\textbf{Official ONT Software:}
\begin{itemize}
\item \texttt{pod5} $\geq$ 0.3.0: POD5 file format library for Python and C++ (\texttt{pip install pod5})
\item \texttt{dorado} (latest): ONT's production basecaller with transformer architecture
\item \texttt{minknow-api}: Python bindings for MinKNOW control and metadata access
\item \texttt{ont-fast5-api}: Legacy FAST5 file access (superseded by POD5)
\item \texttt{guppy}: Earlier basecaller (being phased out in favor of Dorado)
\end{itemize}

\textbf{Community Tools:}
\begin{itemize}
\item \texttt{bonito}: Open-source basecaller using PyTorch
\item \texttt{slow5tools}: Alternative to POD5/FAST5 with improved performance
\item \texttt{NanoPlot}, \texttt{NanoFilt}, \texttt{NanoStat}: Quality control suite
\item \texttt{modbam2bed}: Modified base call extraction from BAM files
\end{itemize}

\section{Oxford Nanopore File Format Specifications}
\label{sec:ont-formats}

This section provides comprehensive technical documentation for Oxford Nanopore Technologies file formats, metadata structures, and nomenclature used throughout the framework.

\subsection{POD5 File Format}

\textbf{Overview:} POD5 is ONT's current raw read format, based on Apache Arrow for efficient data access. It replaces the legacy HDF5-based FAST5 format.

\textbf{Signal Data Structure:}
\begin{itemize}
\item \textbf{Storage:} 16-bit integers (int16) in ADC (analog-to-digital converter) space
\item \textbf{Conversion to picoamperes:} $I_{\text{pA}} = (\text{ADC} - \text{offset}) \times \text{scale}$
\item \textbf{Calibration parameters:} Per-read \texttt{offset} and \texttt{scale} values stored in metadata
\item \textbf{Typical values:} Offset $\approx$ 200--220, scale $\approx$ 0.15--0.20 pA/ADC unit
\item \textbf{Note:} These values vary by chemistry and pore condition; always use the per-read calibration parameters rather than assuming fixed values.
\end{itemize}

\textbf{Key Metadata Fields (POD5 Read Object):}
\begin{itemize}
\item \texttt{read\_id} (UUID): Unique read identifier, convert to string via \texttt{str(read.read\_id)}
\item \texttt{signal} (int16 array): Raw current measurements in ADC space
\item \texttt{calibration.offset} (float): ADC-to-pA offset
\item \texttt{calibration.scale} (float): ADC-to-pA scaling factor
\item \texttt{end\_reason} (enum): Read termination cause (Section~\ref{subsec:end-reason-enum})
\item \texttt{sample\_count} (int): Number of signal samples
\item \texttt{median\_before} (float): Median current before read (baseline, in pA)
\item \texttt{channel} (int): Flow cell channel number (1--512 or 1--3000 depending on device)
\item \texttt{well} (int): Well number within channel
\item \texttt{pore\_type} (string): Nanopore protein variant (e.g., ``not\_set'', ``R9.4.1'', ``R10.4.1'')
\end{itemize}

\textbf{Python API Usage:}
\begin{verbatim}
import pod5
import numpy as np

with pod5.Reader("sequencing_output.pod5") as reader:
    for read in reader.reads():
        # Access metadata
        read_id = str(read.read_id)
        end_reason = read.end_reason.name  # e.g., 'signal_positive'

        # Extract and calibrate signal
        signal_adc = read.signal  # int16 array
        signal_pa = (signal_adc - read.calibration.offset) * \
                    read.calibration.scale
\end{verbatim}

\subsection{FASTQ Format (ONT-Specific Conventions)}

\textbf{Quality Score Encoding:}
\begin{itemize}
\item \textbf{Format:} Sanger/Phred+33 (standard FASTQ)
\item \textbf{ASCII range:} 33--126 (characters \texttt{!} through \texttt{\textasciitilde})
\item \textbf{Phred score range:} Q0--Q93 (ONT-specific extended range)
\item \textbf{Conversion:} $Q = \text{ASCII}(c) - 33$, where $c$ is quality character
\item \textbf{Error probability:} $P(\text{error}) = 10^{-Q/10}$
\end{itemize}

\textbf{FASTQ Header Format:}
\begin{verbatim}
@read_id runid=<ID> sampleid=<ID> read=<N> ch=<CH> start_time=<T>
\end{verbatim}

\textbf{Header Fields:}
\begin{itemize}
\item \texttt{read\_id}: UUID matching POD5 read identifier
\item \texttt{runid}: MinKNOW sequencing run identifier
\item \texttt{sampleid}: User-defined sample identifier
\item \texttt{read}: Read number within run
\item \texttt{ch}: Channel number
\item \texttt{start\_time}: Read start time (date-time string or timestamp)
\item \texttt{barcode}: Barcode classification (if barcoding enabled), e.g., \texttt{barcode01} or \texttt{unclassified}
\end{itemize}

\textbf{MinKNOW Output Directory Structure:}
\begin{itemize}
\item \texttt{fastq\_pass/}: Reads passing quality threshold (typically Q $\geq$ 7)
\item \texttt{fastq\_fail/}: Reads below quality threshold
\item Files organized by barcode subdirectories if barcoding enabled
\end{itemize}

\subsection{BAM Format (ONT SAM Tags)}

\textbf{Standard SAM Tags Used by Dorado:}
\begin{itemize}
\item \texttt{BC:Z:<sequence>}: Barcode assignment (e.g., \texttt{barcode01}, \texttt{unclassified})
\item \texttt{TS:A:<+|->}: Strand orientation (+ for 5'$\to$3', - for 3'$\to$5')
\item \texttt{MM:Z:<spec>}: Modified base call specification (SAM v1.7 format)
\item \texttt{ML:B:C,<probs>}: Modified base call probabilities (unsigned byte array)
\item \texttt{pi:Z:<read\_id>}: Parent read ID (for duplex/split reads)
\item \texttt{pt:i:<length>}: Poly(A/T) tail length estimate (integer, for cDNA/dRNA)
\item \texttt{RX:Z:<sequence>}: UMI (unique molecular identifier) sequence
\item \texttt{mv:B:C,<moves>}: Move table (basecaller signal-to-base mapping, if \texttt{--emit-moves} enabled)
\end{itemize}

\textbf{Custom Tags for Framework Integration:}
\begin{itemize}
\item \texttt{ER:Z:<reason>}: End reason (user-added via integration pipeline, Chapter~\ref{chap:sma-seq})
\item \texttt{QS:f:<score>}: Empirical quality score from SMA-seq calibration (optional)
\end{itemize}

\textbf{Read Group (@RG) Header Metadata:}
Dorado populates standard SAM read group fields:
\begin{itemize}
\item \texttt{ID}: Read group identifier
\item \texttt{SM}: Sample name
\item \texttt{PL}: Platform (``ONT'' for Oxford Nanopore)
\item \texttt{PM}: Platform model (e.g., ``MinION'', ``PromethION'')
\item \texttt{DT}: Date-time of sequencing run
\end{itemize}

\subsection{End Reason Enumeration}
\label{subsec:end-reason-enum}

\textbf{Definition:} End reason classifies why each nanopore read terminated. Critical for data quality assessment (Chapter~\ref{chap:sma-seq}, Section~\ref{sec:end-reason-analysis}).

\textbf{Official Enumeration Values (ONT Specification v1.1):}
\begin{itemize}
\item \texttt{signal\_positive}: Normal termination with return to open-pore baseline. Indicates complete strand translocation. \textbf{Recommended for analysis.}
\item \texttt{signal\_negative}: Abnormal termination with large negative current excursion ($>$80 pA drop). Indicates pore failure or strand reversal.
\item \texttt{mux\_change}: Forced termination due to multiplexer channel re-scan (periodic pore testing, 1--4 hour intervals).
\item \texttt{unblock\_mux\_change}: Termination via voltage reversal to eject blocked strand (automatic pore clearing).
\item \texttt{data\_service\_unblock\_mux\_change}: Software-triggered unblock combined with mux change.
\item \texttt{analysis\_config\_change}: Read terminated due to analysis configuration update during run.
\item \texttt{device\_data\_error}: Hardware or data acquisition error.
\item \texttt{api\_request}: Read terminated by user/API command.
\item \texttt{paused}: Read interrupted by run pause.
\item \texttt{unknown}: Reason not recorded or unrecognized value.
\end{itemize}

\textbf{Filtering Recommendation:} For accuracy validation and clinical applications, use only reads with \texttt{end\_reason = signal\_positive} to ensure complete molecule observations (see Chapter~\ref{chap:sma-seq}). Reads with \texttt{mux\_change}, \texttt{unblock\_mux\_change}, or \texttt{data\_service\_unblock\_mux\_change} are typically truncated mid-sequence.

\subsection{Sequencing Summary File}

\textbf{File:} \texttt{sequencing\_summary.txt} (tab-separated values, ONT Specification v1.1)

\textbf{Core Read Identification Columns:}
\begin{itemize}
\item \texttt{read\_id}: Unique UUID for sequenced read (UUID4/UUID5 format: \texttt{xxxxxxxx-xxxx-[45]xxx-[89ab]xxx-xxxxxxxxxxxx} where each \texttt{x} is a hexadecimal digit [0-9a-f]; $[45]$ indicates version 4 or 5, $[89ab]$ indicates variant)
\item \texttt{run\_id}: Unique UUID for sequencing run
\item \texttt{parent\_read\_id}: Source read UUID if read was split (duplex/adapter splitting)
\end{itemize}

\textbf{Sequencing Metadata:}
\begin{itemize}
\item \texttt{channel}: 1-indexed channel number (1--512 for MinION, 1--3000 for PromethION)
\item \texttt{mux}: 1-indexed multiplexer value (1--4)
\item \texttt{start\_time}: Read start time in seconds from run start (decimal precision)
\item \texttt{duration}: Read duration in seconds (decimal precision)
\item \texttt{end\_reason}: Read termination cause (see Section~\ref{subsec:end-reason-enum} for complete enum)
\item \texttt{pore\_type}: Nanopore protein variant (e.g., ``r9.4.1'', ``r10.4.1'')
\item \texttt{sample\_id}: User-specified sample identifier
\item \texttt{experiment\_id}: User-specified experiment identifier
\end{itemize}

\textbf{Basecalling Columns (if basecalling performed):}
\begin{itemize}
\item \texttt{sequence\_length\_template}: Basecalled sequence length in bases (positive integer)
\item \texttt{mean\_qscore\_template}: Mean Phred quality score across read (positive decimal)
\item \texttt{passes\_filtering}: Quality filter result (\texttt{TRUE} or \texttt{FALSE})
\item \texttt{template\_start}: Start time of template basecalling (seconds)
\item \texttt{template\_duration}: Duration of template basecalling (seconds)
\item \texttt{num\_events\_template}: Number of events detected in template
\end{itemize}

\textbf{File Reference Columns:}
\begin{itemize}
\item \texttt{filename\_fastq}: Output FASTQ file path (online mode, \texttt{.fastq.gz})
\item \texttt{filename\_pod5}: Output POD5 file path (online mode, \texttt{.pod5})
\item \texttt{filename\_bam}: Output BAM file path (online mode, \texttt{.bam})
\item \texttt{input\_filename}: Input file path (offline mode only, \texttt{.fast5} or \texttt{.pod5})
\end{itemize}
\textit{Note: For each read, typically only one filename column is populated, depending on the processing mode and output format.}

\textbf{Barcoding Columns (if barcoding enabled):}
\begin{itemize}
\item \texttt{barcode\_arrangement}: Barcode classification (e.g., \texttt{barcode01}, \texttt{unclassified})
\item \texttt{alias}: User-supplied barcode alias/name
\item \texttt{barcode\_score}: Overall barcode classification confidence score
\item \texttt{barcode\_kit}: Barcode kit identifier
\end{itemize}

\textbf{RNA-Specific Columns (if poly-A tail estimation enabled):}
\begin{itemize}
\item \texttt{poly\_tail\_length}: Estimated poly-A/T tail length (bases, -1 if none detected)
\item \texttt{poly\_tail\_start} / \texttt{poly\_tail\_end}: Tail start/end positions
\end{itemize}

\textbf{Duplex Sequencing Columns (if duplex mode):}
\begin{itemize}
\item \texttt{duplex\_parent\_template} / \texttt{duplex\_parent\_complement}: Parent read IDs for duplex assembly
\end{itemize}

\textbf{Usage Notes:}
\begin{itemize}
\item Join with FASTQ/BAM data by \texttt{read\_id} (exact UUID match required)
\item Use \texttt{end\_reason} to filter for complete reads (\texttt{signal\_positive})
\item \texttt{passes\_filtering} threshold typically Q $\geq$ 7 (configurable in MinKNOW)
\item Column presence conditional on processing pipeline (basecalling, barcoding, alignment, duplex, poly-A estimation)
\item All timestamps are relative to run start time (absolute times available in run metadata)
\end{itemize}

\section{Computational Considerations}

\subsection{Memory Requirements}

\begin{itemize}
\item \textbf{Minimum:} 16 GB RAM for targeted sequencing ($<$10,000 reads)
\item \textbf{Recommended:} 32-64 GB RAM for typical applications (10,000-100,000 reads)
\item \textbf{Large-scale:} 128+ GB RAM for whole-genome applications ($>$1M reads)
\end{itemize}

\subsection{Storage Requirements}

\begin{itemize}
\item \textbf{Raw data:} 1-10 GB per sample (FASTQ files)
\item \textbf{Aligned data:} 0.5-5 GB per sample (BAM files)
\item \textbf{Intermediate files:} 2-20 GB per sample (alignment indices, temporary files)
\item \textbf{Results:} $<$100 MB per sample (posterior probabilities, QC metrics)
\end{itemize}

\subsection{Runtime Estimates}

Typical performance on modern workstation (16 cores, 64 GB RAM):

\begin{table}[H]
\centering
\caption{Estimated Runtime by Application Scale}
\label{tab:runtime_estimates}
\begin{tabular}{lrrr}
\toprule
\textbf{Application} & \textbf{Reads} & \textbf{Haplotypes} & \textbf{Runtime} \\
\midrule
Targeted gene & 1,000 & 2 & 1-2 min \\
Small gene panel & 10,000 & 4 & 10-20 min \\
Multi-locus diplotype & 50,000 & 8 & 1-2 hours \\
Complex region & 100,000 & 16 & 3-6 hours \\
Whole genome (targeted) & 500,000 & 2 & 12-24 hours \\
\bottomrule
\end{tabular}
\end{table}

\section{File Formats and Data Structures}

\subsection{Input Files}

\begin{itemize}
\item \textbf{Sequencing reads:} FASTQ or FAST5 (ONT), BAM (PacBio HiFi)
\item \textbf{Reference haplotypes:} FASTA format with unique identifiers
\item \textbf{Fragment distribution:} CSV with length (bp) and frequency columns
\item \textbf{Quality calibration:} JSON with platform-specific parameters
\end{itemize}

\subsection{Output Files}

\begin{itemize}
\item \textbf{Posterior probabilities:} TSV with read ID, haplotype, log-likelihood, posterior
\item \textbf{Classification summary:} JSON with sample-level posteriors, Bayes factors, QC metrics
\item \textbf{Quality control report:} HTML with visualizations and diagnostic plots
\item \textbf{Aligned reads:} BAM with custom tags for haplotype assignment
\end{itemize}

\subsection{Recommended File Structure}

\begin{verbatim}
project/
|-- data/
|   |-- raw/              # FASTQ/FAST5 files
|   |-- references/       # FASTA haplotype references
|   `-- empirical/        # Fragment distributions
|-- results/
|   |-- alignments/       # BAM files
|   |-- posteriors/       # Classification results
|   `-- qc/               # Quality control reports
|-- config/
|   |-- pipeline.yaml     # Pipeline configuration
|   `-- calibration.json  # Quality calibration parameters
`-- scripts/
    |-- classify.py       # Main classification script
    `-- visualize.py      # Plotting and diagnostics
\end{verbatim}

\section{Implementation Best Practices}

\subsection{Numerical Stability}

\begin{itemize}
\item \textbf{Log-space computation:} Always compute likelihoods in log space to prevent underflow
\item \textbf{Log-sum-exp trick:} Use $\log(\sum_i \exp(x_i)) = \max(x) + \log(\sum_i \exp(x_i - \max(x)))$
\item \textbf{Quality score bounds:} Cap quality scores at reasonable maximum (e.g., Q60) to prevent numerical issues
\item \textbf{Prior specification:} Use log-scale priors and normalize posteriors carefully
\end{itemize}

\subsection{Performance Optimization}

\begin{itemize}
\item \textbf{Alignment filtering:} Restrict to primary alignments with MAPQ $>$ 20
\item \textbf{Parallel processing:} Distribute read processing across multiple cores
\item \textbf{Caching:} Store computed fragment distributions and alignment indices
\item \textbf{Memory mapping:} Use memory-mapped files for large BAM files
\end{itemize}

\subsection{Quality Control Integration}

\begin{itemize}
\item \textbf{Automated gates:} Implement QC checks as pipeline checkpoints with clear pass/fail criteria
\item \textbf{Diagnostic output:} Generate plots and metrics for all QC gates (see Appendix~\ref{app:qc})
\item \textbf{Alert system:} Configure notifications for QC failures requiring intervention
\item \textbf{Documentation:} Log all QC metrics in standardized reports
\end{itemize}

\section{Testing and Validation}

\subsection{Unit Tests}

Test individual components with known inputs and outputs:
\begin{itemize}
\item Quality score conversion (CE.1)
\item Perfect read probability calculation (CE.2-CE.3)
\item Fragment distribution normalization (CE.7)
\item Likelihood computation (CE.11-CE.12)
\item Posterior calculation (CE.13)
\end{itemize}

\subsection{Integration Tests}

Validate complete pipeline with synthetic data:
\begin{itemize}
\item Generate reads from known haplotypes with controlled error rates
\item Verify classification accuracy matches theoretical predictions
\item Test edge cases (low coverage, ambiguous variants, quality extremes)
\item Confirm QC gates trigger appropriately
\end{itemize}

\subsection{Validation with Standards}

Use physical standards for end-to-end validation:
\begin{itemize}
\item Plasmid standards with known sequences (Chapter~\ref{chap:plasmids})
\item Genome-in-a-Bottle reference materials
\item Cell line mixtures with known ratios
\item Technical replicates for reproducibility assessment
\end{itemize}

\section{Integration with Existing Tools}

\subsection{Workflow Managers}

The pipeline integrates with standard workflow systems:

\begin{itemize}
\item \textbf{Snakemake:} Define rules for alignment, classification, and QC
\item \textbf{Nextflow:} Create processes for each pipeline stage with automatic dependency resolution
\item \textbf{CWL:} Use Common Workflow Language for maximum portability
\end{itemize}

\subsection{Cloud Deployment}

The framework supports cloud-based execution:

\begin{itemize}
\item \textbf{AWS Batch:} Containerized pipeline execution with automatic scaling
\item \textbf{Google Cloud Life Sciences:} Managed pipeline service for genomics
\item \textbf{DNAnexus:} Integrated platform with applet packaging
\end{itemize}

\subsection{LIMS Integration}

Connect to laboratory information management systems:

\begin{itemize}
\item \textbf{Input:} Sample metadata, sequencing run parameters
\item \textbf{Output:} Classification results, QC metrics, reports
\item \textbf{Tracking:} Pipeline version, analysis parameters, result provenance
\end{itemize}

\section{Troubleshooting Common Issues}

\subsection{Low Classification Confidence}

\textbf{Symptoms:} Maximum posterior $<$ 0.90, high entropy, weak Bayes factors

\textbf{Possible causes:}
\begin{itemize}
\item Insufficient read coverage (check CE.5 coverage predictions)
\item Poor read quality (review quality score distributions)
\item Haplotype reference errors (verify reference sequences)
\item Sample contamination (check negative controls)
\end{itemize}

\textbf{Solutions:}
\begin{itemize}
\item Increase sequencing depth
\item Improve library preparation quality
\item Update reference haplotypes
\item Repeat sample preparation
\end{itemize}

\subsection{QC Gate Failures}

\textbf{Fragment model violation:} $D_{KL} > 0.05$ bits (CE.9)
\begin{itemize}
\item Use empirical $f_{\text{emp}}$ directly rather than fitted model
\item Document complex fragment distribution in QC report
\end{itemize}

\textbf{Read-length divergence:} $D_{KL}(f_{\text{emp}} \| f_{\text{read}})$ exceeds threshold (CE.10)
\begin{itemize}
\item Investigate size selection or ligation bias
\item Check for read-length filtering in basecalling
\item Review fragment preparation protocol
\end{itemize}

\textbf{Purity constraint violation:} TPR $>$ purity (CE.15)
\begin{itemize}
\item Check for sample contamination
\item Verify reference sequence accuracy
\item Review model assumptions
\end{itemize}

\subsection{Performance Issues}

\textbf{Slow alignment:}
\begin{itemize}
\item Use minimap2 with preset parameters (\texttt{-x map-ont} or \texttt{-x map-pb})
\item Build alignment indices once and reuse
\item Consider prefiltering reads by quality
\end{itemize}

\textbf{Memory overflow:}
\begin{itemize}
\item Process reads in batches
\item Use streaming BAM parsing
\item Reduce alignment index size
\end{itemize}

\section{SEER/SMA Metric Specification}
\label{sec:seer-sma-spec}

This section provides rigorous mathematical definitions and estimators for (i) SEER matrix, (ii) Quality score (Q), and (iii) Single-molecule accuracy (SMA), each in predicted (from basecaller outputs) and empirical (from alignment to truth) forms, with end-reason gating and purity handling.

\textbf{Critical constraint:} Compute metrics only on \texttt{signal\_positive} reads for completeness; report purity $\pi$ for the library and separate label-noise analyses (do not conflate with completeness).

\subsection{Notation and Alphabet}
\label{subsec:seer-notation}

\begin{itemize}
\item \textbf{Alphabet:} $\Sigma = \{\mathrm{A}, \mathrm{C}, \mathrm{G}, \mathrm{T}\}$; gap symbol $\varepsilon$.
\item \textbf{Read:} For read $r$, length $\ell = |r|$. Basecaller emits per-base Q scores $Q_i$ and optionally per-base error probabilities $\hat{p}_i$.
\item \textbf{True sequence:} For the molecule $s_\star$ (SMA-seq standard). Library purity $\pi = \Pr(Z = s_\star)$.
\item \textbf{End reason:} $E \in \{\texttt{signal\_positive}, \texttt{unblock\_mux\_change}, \texttt{mux\_change}, \texttt{signal\_negative}, \ldots\}$ (see Section~\ref{subsec:end-reason-enum}).
\item \textbf{Alignment:} Yields pairs $(g_j, r_j) \in (\Sigma \cup \{\varepsilon\}) \times (\Sigma \cup \{\varepsilon\})$, $j = 1, \ldots, m$, with no $(\varepsilon, \varepsilon)$ pairs.
\end{itemize}

\textbf{Convention:} All empirical metrics below are computed on the set $\mathcal{R}_{\mathrm{S+}}$ of reads with $E = \texttt{signal\_positive}$, unless stated otherwise.

\subsection{SEER Matrix (Single-Molecule Empirical Error Rate Matrix)}
\label{subsec:seer-matrix-def}

\begin{definition}[Pairwise Confusion with Indels]
Construct counts over all aligned base pairs:
\begin{equation}
C_{a,b} = \#\{j: (g_j, r_j) = (a, b)\}, \quad a, b \in \Sigma \cup \{\varepsilon\}, \ (a, b) \neq (\varepsilon, \varepsilon)
\end{equation}

Define the SEER matrix $\mathbf{M}$ by row-normalization over truth states:
\begin{equation}
M_{a,b} = \frac{C_{a,b}}{\sum_{b' \in \Sigma \cup \{\varepsilon\}} C_{a,b'}} \quad \text{for } a \in \Sigma \cup \{\varepsilon\}
\label{eq:seer-matrix}
\end{equation}
\end{definition}

\textbf{Interpretation:}
\begin{itemize}
\item \textbf{Substitutions:} Entries $M_{a,b}$ with $a, b \in \Sigma$, $a \neq b$.
\item \textbf{Deletions:} $M_{a,\varepsilon}$, $a \in \Sigma$ (reference base $a$ deleted in read).
\item \textbf{Insertions:} $M_{\varepsilon,b}$, $b \in \Sigma$ (base $b$ inserted in read). Normalize insertions by aligned reference length.
\end{itemize}

\textbf{Sequence-level SEER (per standard):} Average $\mathbf{M}$ over all reads in that standard; report:
\begin{itemize}
\item \textbf{Substitution matrix} $\mathbf{S} \in \mathbb{R}^{4 \times 4}$ with rows summing to 1 over $\Sigma$.
\item \textbf{Insertion rate:}
\begin{equation}
r_{\text{ins}} = \frac{\sum_{b \in \Sigma} C_{\varepsilon,b}}{\sum_{a \in \Sigma} \sum_{b} C_{a,b}}
\end{equation}
\item \textbf{Deletion rate:}
\begin{equation}
r_{\text{del}} = \frac{\sum_{a \in \Sigma} C_{a,\varepsilon}}{\sum_{a \in \Sigma} \sum_{b} C_{a,b}}
\end{equation}
\end{itemize}

\textbf{Confidence intervals:} Use nonparametric bootstrap over reads to obtain 95\% CIs for each matrix cell and for $r_{\text{ins}}$, $r_{\text{del}}$.

\subsection{Quality Score (Q): Predicted vs. Empirical}
\label{subsec:q-pred-emp}

\begin{definition}[Predicted Per-Base Error Probability]
From basecaller quality scores:
\begin{equation}
\hat{p}_i = 10^{-Q_i/10}, \quad Q_i \in \mathbb{R}
\end{equation}
\end{definition}

\begin{definition}[Predicted Per-Read Error Rate]
\begin{equation}
\hat{e}^{\text{pred}}(r) = \frac{1}{\ell} \sum_{i=1}^{\ell} \hat{p}_i, \quad Q^{\text{pred}}_{\text{read}}(r) = -10 \log_{10}\bigl(\hat{e}^{\text{pred}}(r)\bigr)
\end{equation}
\end{definition}

\begin{definition}[Empirical Per-Base Error Indicator]
From alignment to truth:
\begin{equation}
Y_i = \begin{cases}
1, & \text{if base } i \text{ in } r \text{ is a mismatch or involved in an indel} \\
0, & \text{otherwise}
\end{cases}
\end{equation}
Define $Y_i$ precisely based on alignment representation; e.g., count an insertion at position $i$ as $Y_i = 1$ for the inserted bases.
\end{definition}

\begin{definition}[Empirical Per-Read Error Rate]
\begin{equation}
\hat{e}^{\text{emp}}(r) = \frac{1}{\ell} \sum_{i=1}^{\ell} Y_i, \quad Q^{\text{emp}}_{\text{read}}(r) = -10 \log_{10}\bigl(\hat{e}^{\text{emp}}(r)\bigr)
\end{equation}
\end{definition}

\textbf{Calibration (Reliability):}
\begin{itemize}
\item \textbf{Expected Calibration Error (ECE):} Bin predictions by $\hat{p}_i$ into bins $B_k$. Compute:
\begin{equation}
\text{ECE} = \sum_{k} \frac{|B_k|}{\sum_j |B_j|} \left| \frac{1}{|B_k|} \sum_{i \in B_k} Y_i - \frac{1}{|B_k|} \sum_{i \in B_k} \hat{p}_i \right|
\end{equation}
\item \textbf{Brier Score:}
\begin{equation}
\text{Brier} = \frac{1}{\sum_r \ell_r} \sum_{r} \sum_{i=1}^{\ell_r} (Y_i - \hat{p}_i)^2
\end{equation}
\item \textbf{Reliability curve:} Plot empirical error vs. predicted $\hat{p}$ per sequence (standard).
\end{itemize}

\subsection{Single-Molecule Accuracy (SMA): Predicted vs. Empirical}
\label{subsec:sma-def}

Two complementary definitions:

\begin{definition}[Exact-Sequence SMA]
Binary metric: reads equal the standard sequence end-to-end:
\begin{equation}
\text{SMA}_{\text{exact}} = \Pr(\hat{s} = s_\star \mid E = \texttt{S+}) \approx \frac{1}{|\mathcal{R}_{\mathrm{S+}}|} \sum_{r \in \mathcal{R}_{\mathrm{S+}}} \mathbb{I}\{\hat{s}(r) = s_\star\}
\end{equation}
Report with Wilson interval for binomial proportion.
\end{definition}

\begin{definition}[Per-Base SMA]
Complement of per-base error rate:
\begin{equation}
\text{SMA}_{\text{base}} = 1 - \frac{\sum_r \sum_{i=1}^{\ell_r} Y_i}{\sum_r \ell_r}
\end{equation}
Report also stratified rates: mismatch, insertion, deletion components.
\end{definition}

\begin{definition}[Predicted SMA]
From basecaller Q scores:
\begin{equation}
\widehat{\text{SMA}}_{\text{base}}^{\text{pred}} = 1 - \frac{\sum_r \sum_{i=1}^{\ell_r} \hat{p}_i}{\sum_r \ell_r}
\end{equation}
Report calibration gap: $\widehat{\text{SMA}}^{\text{pred}} - \text{SMA}^{\text{emp}}$.
\end{definition}

\subsection{Purity and Label Noise (SMA-seq Standards)}
\label{subsec:purity-label-noise}

If the library purity is $\pi = \Pr(Z = s_\star)$, then the measured exact-sequence accuracy on a dataset labeled as $s_\star$ obeys:
\begin{equation}
\mathbb{E}[\widehat{\text{SMA}}_{\text{exact}}] = \pi \, A + (1 - \pi) \, B
\end{equation}
with $A = \Pr(\hat{s} = s_\star \mid Z = s_\star)$ (the true target accuracy) and $B = \Pr(\hat{s} = s_\star \mid Z \neq s_\star)$.

\textbf{Practice:} Report $\pi$ and never interpret claims of $\text{SMA}_{\text{exact}}$ beyond the purity regime without an independent test.

\subsection{End-Reason Gating and Stratification}
\label{subsec:end-reason-gating}

\begin{itemize}
\item \textbf{Default:} Compute all core metrics only on $\mathcal{R}_{\mathrm{S+}}$ (reads with $E = \texttt{signal\_positive}$).
\item \textbf{Optional:} Provide side-tables for other end reasons (UMC/MUX/SN) for diagnosis, not for primary SMA reporting.
\end{itemize}

\subsection{Implementation Specification (Inputs $\to$ Outputs)}
\label{subsec:seer-implementation}

\textbf{Inputs:}
\begin{itemize}
\item \texttt{reads.bam} (unaligned or aligned)
\item \texttt{sequencing\_summary.txt} (to join end\_reason)
\item \texttt{truth\_s\_star.fasta} (per standard)
\item Library purity $\pi$ with bounds
\end{itemize}

\textbf{Processing Steps:}
\begin{enumerate}
\item Join end\_reason to reads; select $\mathcal{R}_{\mathrm{S+}}$.
\item Align $\mathcal{R}_{\mathrm{S+}}$ to \texttt{truth\_s\_star.fasta} (minimap2, \texttt{-x map-ont}, keep cs/MD tags).
\item From alignments, derive $(g_j, r_j)$ and per-base indicators $Y_i$.
\item Build SEER $\mathbf{M}$, insertion/deletion rates; bootstrap CIs.
\item Compute $Q^{\text{pred}}_{\text{read}}$, $Q^{\text{emp}}_{\text{read}}$, ECE, Brier.
\item Compute $\text{SMA}_{\text{exact}}$, $\text{SMA}_{\text{base}}$ and predicted counterparts.
\item Emit per-sequence reports (JSON + PDF tables/plots).
\end{enumerate}

\textbf{Outputs (per sequence/standard):}
\begin{itemize}
\item \texttt{seer\_matrix.tsv} (4$\times$4 substitutions + ins/del rates)
\item \texttt{sma\_metrics.tsv} (SMA\_exact, SMA\_base with CIs)
\item \texttt{q\_calibration.tsv} (bin stats, ECE, Brier)
\item \texttt{read\_metrics.tsv} (per-read predicted/empirical Q, length)
\item \texttt{summary.json} (all numbers, versioned)
\end{itemize}

\subsection{Quality Gates \& Reporting Conventions}
\label{subsec:seer-quality-gates}

\begin{itemize}
\item Exclude secondary/supplementary alignments; require MAPQ $\geq 20$ for counting.
\item \textbf{Homopolymers:} Report separate SEER rows/columns conditioned on homopolymer context (optional, but recommended).
\item \textbf{Length bins:} Stratify per-read metrics by length deciles.
\item All tables carry \texttt{end\_reason = S+} stamp; others in appendix.
\end{itemize}

\subsection{Command Scaffolding}
\label{subsec:seer-commands}

\textbf{Join end\_reason to reads:}
\begin{verbatim}
python scripts/join_end_reason.py \
  --summary sequencing_summary.txt \
  --bam basecalls.bam \
  --out tagged.bam \
  --end-reason-tag ER
\end{verbatim}

\textbf{Map S+ reads to truth:}
\begin{verbatim}
samtools view -h tagged.bam | \
  awk '$0 ~ /^@/ || $0 ~ /ER:Z:signal_positive/' | \
  samtools view -bS - > splus.bam

minimap2 -a -x map-ont truth_s_star.fasta splus.bam | \
  samtools sort -o splus.aln.bam

samtools index splus.aln.bam
\end{verbatim}

\textbf{Extract SEER \& SMA:}
\begin{verbatim}
python scripts/compute_seer_sma.py \
  --bam splus.aln.bam \
  --truth truth_s_star.fasta \
  --out outdir/
\end{verbatim}

\subsection{Compact Definitions for Quick Reference}
\label{subsec:seer-quick-ref}

\begin{itemize}
\item \textbf{SEER matrix $\mathbf{M}$:} Row-normalized confusion including $\varepsilon$; substitution submatrix $\mathbf{S}$, $r_{\text{ins}}$, $r_{\text{del}}$.
\item \textbf{Q (pred):} $\hat{p}_i = 10^{-Q_i/10}$; $Q^{\text{pred}}_{\text{read}} = -10 \log_{10}(\frac{1}{\ell} \sum \hat{p}_i)$.
\item \textbf{Q (emp):} From alignment indicators $Y_i$; $Q^{\text{emp}}_{\text{read}} = -10 \log_{10}(\frac{1}{\ell} \sum Y_i)$.
\item \textbf{SMA (exact):} Fraction of S+ reads equal to $s_\star$; SMA (base): $1 -$ empirical per-base error rate.
\item \textbf{Calibration:} ECE, Brier; reliability plots.
\item \textbf{Purity:} Interpret results within $\pi$; report $\pi$ alongside SMA; do not claim beyond $\pi$ without extra assays.
\end{itemize}

\section{Summary}

This appendix provides comprehensive implementation guidance for the haplotype classification framework, from algorithm pseudocode through production deployment considerations. The SEER/SMA metric specification (Section~\ref{sec:seer-sma-spec}) provides rigorous mathematical definitions for empirical error measurement with proper end-reason gating and purity handling. For additional protocol details, see Chapter~\ref{chap:workflow} (Complete Sample-to-Analysis Workflow). For quality control procedures, see Appendix~\ref{app:qc} (Laboratory Protocols and Quality Control).

%%%%%%%%%%%%%%%%%%%%%%%%%%%%%%%%%%%%%%%%%%%%%%%%%%%%%%%%%%%%%%%%%%%%%%%%
%% Appendix E: Version History
%% Part: Appendices
%% Version 6.0 - Migrated from v5.tex Appendix D (lines 2253-2344) + NEW v6.0 entry
%%%%%%%%%%%%%%%%%%%%%%%%%%%%%%%%%%%%%%%%%%%%%%%%%%%%%%%%%%%%%%%%%%%%%%%%

\chapter{Version History}
\label{app:version_history}
\label{app:version-history}

This version history documents the evolution of the mathematical framework, tracking major enhancements, refinements, and structural improvements. Each version represents significant methodological advances or documentation improvements that affect framework implementation or interpretation.

\section*{Version 6.0 (Current - Seven-Part Book Architecture)}

\textbf{Release Date:} October 16, 2025

\textbf{Last Updated:} November 16, 2025

\textbf{Major Reorganization:} Version 6.0 represents a comprehensive restructuring from a five-part research document into a seven-part integrated methods and applications volume. This reorganization maintains all mathematical rigor while dramatically enhancing practical utility for clinical deployment, regulatory compliance, and reproducible research.

\subsection*{Key Changes in Version 6.0}

\begin{itemize}
\item \textbf{Seven-Part Book Structure:} Reorganized entire framework into seven cohesive parts with clear narrative arc: Part I (Clinical Motivation and Technical Background), Part II (Mathematical Foundations), Part III (Physical Standards and Laboratory Workflows), Part IV (SMA-seq and Model Improvement), Part V (Validation of Genetic Tests), Part VI (Clinical Applications), Part VII (Operational Excellence). This structure provides dual accessibility for clinical/assay-focused and methods/theory-focused readers.

\item \textbf{Explicit Clinical Motivation (Part I):} Added new Part I (3 chapters) providing clinical context, pharmacogenomics motivation, and genomic complexity overview. Establishes clear rationale for single-molecule sequencing approaches before diving into technical details.

\item \textbf{Expanded Laboratory Workflows (Part III):} Created comprehensive Part III (3 chapters) documenting plasmid standard construction, Cas9 enrichment protocols, and complete sample-to-analysis workflows. Provides reproducible laboratory protocols with accept/reject criteria.

\item \textbf{SMA-seq Methods Integration (Part IV):} Consolidated SMA-seq methodology, noisy label training, and basecaller fine-tuning into cohesive Part IV (3 chapters). Provides complete methodology for model improvement using empirical data.

\item \textbf{Dedicated Clinical Applications (Part VI):} Created Part VI (3 chapters) showcasing bacterial strain typing, CYP2D6 pharmacogenomics, and 75-patient cohort study. Demonstrates real-world utility and validation across diverse applications.

\item \textbf{Operational Excellence Framework (Part VII):} Added Part VII (2 chapters) covering standard operating procedures and economic analysis. Addresses deployment considerations for clinical production environments.

\item \textbf{Enhanced Appendices:} Reorganized appendices into five focused sections: (A) Notation and Symbol Reference, (B) Core Equations Reference, (C) Laboratory Protocols and Quality Control, (D) Software Tools and Implementation, (E) Version History. Provides comprehensive reference material for all framework components.

\item \textbf{Improved Chapter Organization:} Expanded from 10 chapters to 20 chapters with clear, focused scopes. Each chapter includes explicit learning objectives, detailed content, practical examples, and integration with broader framework.

\item \textbf{Future-Proof Structure:} Architecture accommodates planned additions including technology-specific optimization chapters, expanded cohort studies, and regulatory documentation templates.
\end{itemize}

\subsection*{Migration from v5 to v6}

Version 6.0 content was systematically migrated from v5.0 using the following chapter mapping:

\begin{table}[H]
\centering
\caption{v5 to v6 Chapter Migration Map}
\label{tab:v5_to_v6_mapping}
\begin{tabular}{lll}
\toprule
\textbf{v5 Chapter} & \textbf{v6 Chapter(s)} & \textbf{Notes} \\
\midrule
Ch 1: Introduction & Ch 1-3 (Part I) & Expanded into 3-chapter intro \\
Ch 2: Foundations & Ch 4-7 (Part II) & Split into focused chapters \\
Ch 3: Models & Ch 4-7 (Part II) & Integrated into foundations \\
Ch 4: SEER Framework & Ch 11 (Part IV) & Moved to SMA-seq section \\
Ch 5: Likelihood & Ch 6 (Part II) & Integrated into posteriors \\
Ch 6: Bayesian Inference & Ch 6 (Part II) & Combined with likelihood \\
Ch 7: Advanced Topics & Ch 9, 12-13 (Parts III-IV) & Distributed appropriately \\
Ch 8: Implementation & Ch 10 (Part III) & Workflow chapter \\
Ch 9: Validation & Ch 14-15 (Part V) & Expanded validation \\
Ch 10: Case Studies & Ch 16-18 (Part VI) & Split into 3 applications \\
\bottomrule
\end{tabular}
\end{table}

\subsection*{Document Statistics}

\begin{itemize}
\item \textbf{Structure:} 7 Parts, 20 Chapters, 5 Appendices
\item \textbf{Expected Length:} 200-250 pages (fully populated)
\item \textbf{Current Completion:} 75\% (15 of 20 chapters populated as of November 16, 2025)
\item \textbf{Core Equations:} 16 equations (CE.1-CE.16) preserved from v5
\item \textbf{Notation:} 50+ symbols with comprehensive descriptions
\item \textbf{Tables:} 30+ tables with protocols, thresholds, and metrics
\end{itemize}

\section*{Version 5.0 (Five-Part Book Structure)}

\textbf{Date:} 2024

\begin{itemize}
\item \textbf{Comprehensive Book-Level Restructuring:} Transformed document from article format to professional book structure with five integrated parts, enabling more logical organization and improved navigation. Changed document class to book with proper chapter and part divisions. Updated all theorem environments to use chapter-based numbering. Enhanced headers and footers for book format with alternating page layouts.

\item \textbf{Five-Part Organizational Framework:} Reorganized entire content into cohesive parts that progress from theory through practice. Part I (Theoretical Foundations) establishes mathematical infrastructure. Part II (Empirical Measurement and Error Characterization) develops measurement-driven models including the new SEER framework. Part III (Inference and Classification Methods) presents likelihood computation and Bayesian inference. Part IV (Implementation and Technology-Specific Considerations) provides computational guidance. Part V (Validation and Clinical Applications) addresses regulatory compliance and real-world deployment.

\item \textbf{Major Addition: SEER Framework (Chapter 4):} Introduced comprehensive new chapter presenting the Sequencing Empirical Error Rate framework, a patented methodology for empirically quantifying technology-specific error patterns through controlled experiments with known-truth reference standards. Chapter includes confusion matrix formulation, true positive rate calculation, quality score calibration assessment, k-mer error profiling, basecaller optimization strategies, and clinical validation protocols. SEER provides data-driven foundation for confident haplotype classification with quantifiable accuracy guarantees.

\item \textbf{New Core Equations CE.17-CE.18:} Extended Core Equations reference with two essential SEER metrics. CE.17 formalizes true positive rate calculation from confusion matrices with purity constraints. CE.18 quantifies quality score overestimation for systematic miscalibration detection. Both equations integrate seamlessly with existing quality control framework and include comprehensive parameter descriptions, usage guidelines, and cross-references to detailed derivations.

\item \textbf{Enhanced Chapter Organization:} Converted all major sections to proper chapters with introductory context paragraphs. Chapter 1 (Introduction and Overview) sets framework scope. Chapter 2 (Mathematical Foundations) establishes notation and measurable spaces. Chapter 3 (Stage-Specific Probabilistic Models) develops mutation, fragmentation, and sequencing models. Chapter 4 (SEER Framework) characterizes empirical errors. Chapter 5 (Likelihood Computation Framework) presents tractable algorithms. Chapter 6 (Bayesian Inference and Haplotype Classification) enables probabilistic decision-making. Chapter 7 (Advanced Topics) addresses Cas9 enrichment, CNV, and pangenome extensions. Chapter 8 (Computational Implementation) provides workflow guidance. Chapter 9 (Validation Framework) establishes testing protocols. Chapter 10 (Case Studies) demonstrates practical applications.

\item \textbf{Future-Ready Structure:} Established organizational framework supporting planned additions including technology-specific chapters for Illumina, PacBio, and Oxford Nanopore platforms (Part IV), targeted enrichment method formalization (Part IV), and clinical performance benchmarking with real-world case studies such as the 75-patient Singapore cohort analysis (Part V).
\end{itemize}

\section*{Version 4.2 (Cross-Reference Enhancement)}

\begin{itemize}
\item \textbf{Implemented Comprehensive Cross-Reference System:} Added \texttt{\textbackslash CEref} and \texttt{\textbackslash CEanchor} commands throughout the document, enabling seamless bidirectional navigation between Core Equations and detailed sections.

\item \textbf{Enhanced Log-Base Convention Documentation:} Added explicit log-base reminder before first KL divergence definition, clarifying that all Kullback-Leibler divergences use $\log_2$ and are reported in bits.

\item \textbf{Expanded Library Preparation Documentation:} Added archival guidance for documenting adapter-ligated fraction alongside empirical fragment distribution in run manifests.

\item \textbf{Strengthened Log-Space Implementation Guidance:} Added explicit implementation note emphasizing log-space computation throughout likelihood calculations for numerical stability.

\item \textbf{Created Comprehensive QC Summary Table:} Added formal quality control gate table in Appendix C summarizing four critical checkpoints with explicit thresholds and action criteria.
\end{itemize}

\section*{Version 4.1 (Navigation and Glossary Enhancement)}

\begin{itemize}
\item \textbf{Fixed Hyperlink Navigation System:} Completely restructured all hypertarget placements for CE.1-CE.16, resolving navigation issues.

\item \textbf{Enhanced Page Layout Control:} Systematically increased needspace values for all Core Equation boxes, preventing page breaks.

\item \textbf{Comprehensive Glossary Enhancement:} Significantly expanded Appendix A (Notation Summary) with 3-5× more detailed content including typical value ranges, measurement protocols, and cross-references.

\item \textbf{Added Cross-Reference Network:} Systematically integrated cross-references throughout glossary entries, creating comprehensive navigation network.
\end{itemize}

\section*{Version 4.0 (Core Equations Framework)}

\begin{itemize}
\item \textbf{Enhanced Core Equations:} Expanded all 16 core equations with comprehensive parameter descriptions, usage guidelines, and examples.
\item \textbf{Fixed Hyperlink Navigation:} Added complete hypertarget tags for CE.7-CE.16.
\item \textbf{Improved Page Layout:} Implemented needspace commands before all equation boxes.
\item \textbf{Expanded Appendix Content:} Enhanced Appendices A-C with contextual introductions and implementation guidance.
\end{itemize}

\section*{Version 3.0 (Visual Enhancement)}

\begin{itemize}
\item Enhanced visual formatting with color-coded section headers
\item Improved table of contents with hierarchical numbering
\item Added comprehensive cross-referencing between Core Equations and detailed sections
\item Refined equation box styling with non-breaking containers
\end{itemize}

\section*{Version 2.0 (Core Equations Introduction)}

\begin{itemize}
\item Added Core Equations (Quick Reference) section with 16 essential formulas
\item Integrated empirical fragmentation as primary framework (CE.7-CE.10)
\item Added quality control gates: fragment model KL divergence (CE.9), read-length sanity (CE.10)
\item Formalized quality overstatement metric (CE.14) and purity constraints (CE.15)
\item Aligned notation with Perfect Reads – Ultra tool
\end{itemize}

\section*{Version 1.0 (Initial Release)}

\begin{itemize}
\item Initial release with complete mathematical framework
\item Bayesian hierarchical modeling approach
\item Support for multiple sequencing technologies
\item Clinical validation guidelines
\item Theoretical foundations for all core concepts
\end{itemize}

\section{Summary}

This version history demonstrates the framework's continuous evolution from initial mathematical foundations through comprehensive book-level organization with extensive practical guidance. Version 6.0 represents the most significant reorganization to date, transforming the framework from a research document into an integrated methods and applications volume suitable for clinical deployment, regulatory review, and educational use.

%%%%%%%%%%%%%%%%%%%%%%%%%%%%%%%%%%%%%%%%%%%%%%%%%%%%%%%%%%%%%%%%%%%%%%%%
%% Appendix F: Mathematical Models for SMS and Haplotype Classification
%% Version 6.1 - NEW (November 2025)
%% Consolidates and harmonizes core mathematical constructions
%%%%%%%%%%%%%%%%%%%%%%%%%%%%%%%%%%%%%%%%%%%%%%%%%%%%%%%%%%%%%%%%%%%%%%%%

\chapter{Mathematical Models for Single-Molecule Sequencing, Error Characterization, Haplotype Classification, and Haplotagging}
\label{app:mathematical-models}

This appendix consolidates and harmonizes the core mathematical constructions that appear across Parts II--V, providing a single canonical reference for notation, assumptions, and key equations. It is adapted and reconciled from the standalone mathematics document to ensure consistency across the framework.

%%%%%%%%%%%%%%%%%%%%%%%%%%%%%%%%%%%%%%%%%%%%%%%%%%%%%%%%%%%%%%%%%%%%%%%%
\section{Overview}
\label{sec:app-f-overview}

This appendix presents a unified mathematical treatment of:
\begin{enumerate}
\item Single-molecule sequencing and basecalling
\item Phred quality scores and alignment-based quality metrics
\item Sequence-level confusion matrices and empirical error models
\item Haplotype and diplotype classification (including polyploidy and cost-based decision rules)
\item Read-level haplotagging with known haplotypes
\item Plasmid replication, mutation, and purity bounds
\item Dual Cas9 cutting and gene-isolation probability
\end{enumerate}

Notation is aligned with the main text (Part II and Appendix~\ref{app:core-equations}). Where earlier drafts used conflicting letters (e.g., $b$ for quality score, $L$ for Levenshtein distance), we standardize to the current conventions:
\begin{itemize}
\item $Q$: Phred quality score
\item $p$: error probability
\item $L$ or $L_{\text{mol}}$: molecule length (bases)
\item $d_{\text{edit}}$: edit distance (Levenshtein)
\end{itemize}

%%%%%%%%%%%%%%%%%%%%%%%%%%%%%%%%%%%%%%%%%%%%%%%%%%%%%%%%%%%%%%%%%%%%%%%%
\section{Single-Molecule Sequencing and Basecalling Model}
\label{sec:app-f-sms-basecalling}

We consider a single-molecule sequencing instrument that produces a raw signal and a basecaller that maps that signal into reads with quality scores.

\subsection{Raw signal and segmentation}

The instrument outputs a continuous time series
\begin{equation}
X = (x_1, x_2, \ldots, x_t),
\end{equation}
where $x_j$ is the physical measurement at time index $j$ and $t$ is the total number of measurements in a run.

A single-molecule signal (one binding event) is a contiguous subsequence
\begin{equation}
x^{(i)} = (x_{a_i}, x_{a_i+1}, \ldots, x_{b_i}),
\end{equation}
with length $\ell_i = b_i - a_i + 1$.

A segmentation model $S$ maps the full signal into $n$ single-molecule events
\begin{equation}
(x^{(1)}, \ldots, x^{(n)}) = S(X).
\end{equation}

This segmentation is the operational realization of the transition $\Prob(\sigma \mid l)$ in the pipeline factorization (Section~\ref{sec:app-f-pipeline-factorization}).

\subsection{Basecalling and quality scores}

Let $\mathcal{A}$ denote the nucleotide alphabet, e.g.,
\begin{equation}
\mathcal{A} = \{A, C, G, T\} \quad\text{or}\quad \{A, C, G, T, N\}.
\end{equation}

A basecalling function
\begin{equation}
f_{\text{basecaller}} : \mathcal{S} \to \mathcal{A}^{*}
\end{equation}
maps a single-molecule signal $x^{(i)}$ to a predicted read $r^{(i)}$ of length $L_i$:
\begin{equation}
r^{(i)} = f_{\text{basecaller}}(x^{(i)}) \in \mathcal{A}^{L_i}.
\end{equation}

For each base $r^{(i)}_j$, the basecaller emits a Phred quality score $Q^{(i)}_j$, with implied error probability
\begin{equation}
Q^{(i)}_j = -10 \log_{10} p^{(i)}_j,
\quad
p^{(i)}_j = 10^{-Q^{(i)}_j/10}.
\end{equation}

Collecting all reads and qualities from a run:
\begin{equation}
R = \{r^{(1)}, \ldots, r^{(n)}\},
\quad
Q = \{Q^{(1)}, \ldots, Q^{(n)}\}.
\end{equation}

\subsection{Pipeline Factorization Theorem and signal-to-read mapping}
\label{sec:app-f-pipeline-factorization}

We model the full single-molecule sequencing pipeline via a hierarchy of measurable spaces:
\begin{equation}
\mathcal{H}, \mathcal{G}, \mathcal{U}, \mathcal{D}, \mathcal{L}, \mathcal{S}, \mathcal{R},
\end{equation}
representing (respectively) haplotypes, genomic molecules, post-mutation sequences, DNA fragments, library molecules, signals, and reads.

\begin{theorem}[Pipeline Factorization Theorem]
\label{thm:app-f-pipeline-factorization}
The joint distribution over all pipeline variables factorizes as
\begin{equation}
\Prob(h,g,u,d,l,\sigma,r)
=
\Prob(h)\,
\Prob(g\mid h)\,
\Prob(u\mid g)\,
\Prob(d\mid u)\,
\Prob(l\mid d)\,
\Prob(\sigma\mid l)\,
\Prob(r\mid \sigma).
\end{equation}
\end{theorem}

Each factor corresponds to a physical or computational transformation:
\begin{itemize}
\item $\Prob(g\mid h)$: genomic molecules from haplotype
\item $\Prob(u\mid g)$: mutations over cell divisions
\item $\Prob(d\mid u)$: fragmentation
\item $\Prob(l\mid d)$: library preparation and adapter ligation
\item $\Prob(\sigma\mid l)$: instrument signal generation
\item $\Prob(r\mid \sigma)$: basecalling
\end{itemize}

Operationally, segmentation and basecalling implement the composite map
\begin{equation}
f_{\text{basecaller}}\bigl(f_{\text{segmentation}}(X)\bigr)
\to (R_{\text{pred}}, Q_{\text{pred}}),
\end{equation}
which realizes the last two conditional distributions $\Prob(\sigma\mid l)$ and $\Prob(r\mid \sigma)$ in the factorization. This makes explicit that ``improving the model'' is largely equivalent to improving the basecaller and its error model $\Prob(r\mid \sigma)$, a central goal of the SMA-seq / SEER loop in Part IV.

\subsection{Read-level average quality (predicted vs empirical)}

For read $r^{(i)}$ with basewise error probabilities $p^{(i)}_j$, the predicted per-read error rate is
\begin{equation}
\bar{p}^{(i)}_{\text{pred}}
=
\frac{1}{L_i}
\sum_{j=1}^{L_i} p^{(i)}_j
=
\frac{1}{L_i}
\sum_{j=1}^{L_i} 10^{-Q^{(i)}_j/10},
\end{equation}
with corresponding average predicted Phred
\begin{equation}
\bar{Q}^{(i)}_{\text{pred}}
=
-10 \log_{10}\bigl(\bar{p}^{(i)}_{\text{pred}}\bigr).
\end{equation}

If the true generating sequence for $x^{(i)}$ is $s^{(i)}$, the empirical error rate is
\begin{equation}
\bar{p}^{(i)}_{\text{emp}}
=
\frac{d_{\text{edit}}(r^{(i)}, s^{(i)})}{|s^{(i)}|},
\end{equation}
with empirical quality
\begin{equation}
\bar{Q}^{(i)}_{\text{emp}}
=
-10 \log_{10}\bigl(\bar{p}^{(i)}_{\text{emp}}\bigr),
\end{equation}
where $d_{\text{edit}}$ is the Levenshtein edit distance.

%%%%%%%%%%%%%%%%%%%%%%%%%%%%%%%%%%%%%%%%%%%%%%%%%%%%%%%%%%%%%%%%%%%%%%%%
\section{Sequence Counts, Experiments, and Confusion Matrix}
\label{sec:app-f-confusion-matrix}

We formalize sequence-level empirical error models used by SEER and SMA-seq.

\subsection{Individual experiments}

For a single experiment $E$ producing reads $R = \{r_1,\ldots,r_N\}$, let
\begin{equation}
S_E = \{s_1,\ldots,s_{m_E}\}
\end{equation}
be the set of unique sequences after collapsing identical reads. Define a count vector $c^{(E)} \in \mathbb{N}^{m_E}$ by
\begin{equation}
c^{(E)}_j = \bigl|\{r \in R : r = s_j\}\bigr|,
\quad j = 1,\ldots,m_E,
\end{equation}
so that the total number of reads is $N = \sum_{j=1}^{m_E} c^{(E)}_j$.

\subsection{Multiple experiments and global sequence index}

For experiments $\{E_1,\ldots,E_K\}$ performed on the same platform, define the global set of unique sequences
\begin{equation}
S = \bigcup_{k=1}^K S_{E_k} = \{s_1,\ldots,s_M\}.
\end{equation}

Each experiment $E_k$ induces a count vector $c^{(k)} \in \mathbb{N}^M$, with $c^{(k)}_j$ the count of sequence $s_j$ in $E_k$.

\subsection{Confusion matrix and empirical error model}

Using high-purity standards with known ground-truth sequence labels, we construct an $M \times M$ confusion matrix $C$, where
\begin{equation}
C_{ij} = \text{\# times a molecule with true sequence } s_i
\text{ is observed as } s_j.
\end{equation}

For a fixed true sequence $s_i$, the total number of molecules is
\begin{equation}
N_i = \sum_{j=1}^{M} C_{ij}.
\end{equation}

\begin{itemize}
\item \textbf{True positive rate (TPR)} for $s_i$:
\begin{equation}
\mathrm{TPR}_i = \Prob(\hat{s} = s_i \mid s_i)
= \frac{C_{ii}}{N_i}.
\end{equation}

\item \textbf{Misclassification probability}:
\begin{equation}
\varepsilon_i = \Prob(\hat{s} \neq s_i \mid s_i)
= 1 - \mathrm{TPR}_i
= \frac{\sum_{j\neq i} C_{ij}}{N_i}.
\end{equation}

\item \textbf{Pairwise misclassification probability} (for $i\neq j$):
\begin{equation}
\Prob(\hat{s}=s_j \mid s_i)
=
\frac{C_{ij}}{N_i}.
\end{equation}
\end{itemize}

These probabilities define a sequence-level empirical error model used in the likelihood calculations for haplotypes and diplotypes (Section~\ref{sec:app-f-haplotype-classification}).

\subsection{Single Molecule Accuracy (SMA)}
\label{sec:app-f-sma-definition}

Following the Enhancement Plan, we define Single Molecule Accuracy (SMA) as the true positive rate for a complete molecular read of a given standard.

\begin{definition}[Single Molecule Accuracy]
\label{def:app-f-sma}
For a standard with true sequence $s_i$, the Single Molecule Accuracy is
\begin{equation}
\mathrm{SMA}(s_i) := \mathrm{TPR}_i = \frac{C_{ii}}{N_i}.
\end{equation}

At the standard level, the SMA of an assay or basecaller is the appropriate average of $\mathrm{SMA}(s_i)$ across all standard sequences of interest.
\end{definition}

In SMA-seq, SMA is the primary performance metric that the protocol is designed to measure; SEER provides the confusion matrices from which $\mathrm{SMA}(s_i)$ is estimated.

%%%%%%%%%%%%%%%%%%%%%%%%%%%%%%%%%%%%%%%%%%%%%%%%%%%%%%%%%%%%%%%%%%%%%%%%
\section{Mean Phred vs. Phred of Mean Error Probability}
\label{sec:app-f-phred-inequality}

Let $p_1,\ldots,p_n$ be basewise error probabilities with corresponding Phred scores
\begin{equation}
Q_i = -10\log_{10} p_i.
\end{equation}

Define arithmetic means
\begin{equation}
\bar{p} = \frac{1}{n}\sum_{i=1}^n p_i,
\quad
\bar{Q} = \frac{1}{n}\sum_{i=1}^n Q_i.
\end{equation}

\begin{theorem}[Phred Averaging Inequality]
\label{thm:app-f-phred-inequality}
\begin{equation}
\bar{Q} \;\ge\; -10 \log_{10}(\bar{p}),
\end{equation}
with equality if and only if all $p_i$ are equal (or $n=1$).
\end{theorem}

\begin{proof}[Sketch]
Since $\log_{10}(x)$ is concave, Jensen's inequality gives
\begin{equation}
\log_{10}(\bar{p})
\ge
\frac{1}{n}\sum_{i=1}^n \log_{10}(p_i).
\end{equation}
Multiplying by $-10$ reverses the inequality and yields the stated result.
\end{proof}

\textbf{Interpretation:} Averaging in log-space produces a more optimistic quality score than converting the average error probability to Phred; this matters for how aggregate quality metrics are reported.

%%%%%%%%%%%%%%%%%%%%%%%%%%%%%%%%%%%%%%%%%%%%%%%%%%%%%%%%%%%%%%%%%%%%%%%%
\section{Alignment-Based Quality Metric}
\label{sec:app-f-alignment-quality}

Given a ground-truth sequence $G = (g_1,\ldots,g_N)$ and an aligned basecalled sequence $B = (b_1,\ldots,b_N)$ with per-base error probabilities $p_i$, we assign a score $s_i$ per alignment column:
\begin{equation}
s_i =
\begin{cases}
1 - p_i, & g_i = b_i,\\[3pt]
p_i, & g_i \neq b_i,\\[3pt]
0, & \text{if } g_i \text{ or } b_i \text{ is a gap.}
\end{cases}
\end{equation}

Define mean correctness
\begin{equation}
M = \frac{1}{N}\sum_{i=1}^N s_i \in [0,1],
\end{equation}
and effective error probability
\begin{equation}
p_{\text{err}} = 1 - M.
\end{equation}

The alignment-based Phred-like score is
\begin{equation}
Q_{\text{align}} = -10 \log_{10}(1 - M).
\end{equation}

This provides a single-number summary incorporating both predicted and empirical correctness over an alignment.

%%%%%%%%%%%%%%%%%%%%%%%%%%%%%%%%%%%%%%%%%%%%%%%%%%%%%%%%%%%%%%%%%%%%%%%%
\section{Per-Base Variant Likelihood from Basewise Error Rates}
\label{sec:app-f-variant-likelihood}

For a candidate haplotype (or reference) $g = (g_1,\ldots,g_L)$ and an aligned read $r = (r_1,\ldots,r_L)$, assume per-position error rate $e_i$. Then
\begin{equation}
\Prob(r_i = g_i \mid g_i,e_i) = 1 - e_i,
\qquad
\Prob(r_i = b \neq g_i \mid g_i,e_i) = \frac{e_i}{|\mathcal{A}|-1}.
\end{equation}

Assuming conditional independence across positions,
\begin{equation}
\Prob(r \mid g,e)
=
\prod_{i=1}^L
\Bigl[
(1-e_i)\,\mathbb{I}\{r_i=g_i\}
+
\frac{e_i}{|\mathcal{A}|-1}\,\mathbb{I}\{r_i\neq g_i\}
\Bigr].
\end{equation}

This per-read likelihood is used within haplotype and molecule-of-origin models (Chapters~\ref{chap:classification-model} and \ref{chap:posteriors}).

%%%%%%%%%%%%%%%%%%%%%%%%%%%%%%%%%%%%%%%%%%%%%%%%%%%%%%%%%%%%%%%%%%%%%%%%
\section{Haplotype Classification with Unknown Haplotype}
\label{sec:app-f-haplotype-classification}

Let
\begin{equation}
\mathcal{H} = \{h_1,\ldots,h_P\}
\end{equation}
be the set of candidate haplotypes, with priors $\Prob(h_i)$. For each $h_i$, we conceptually define:
\begin{itemize}
\item $M(h_i) = \{m^{(i)}_1,\ldots,m^{(i)}_{v_i}\}$: genomic molecules in the haplotype
\item $U(h_i)$: unique sequences after mutation
\item $D(h_i)$: fragments after fragmentation
\item $L(h_i)$: library molecules that receive adaptors
\end{itemize}

For a single read $r$, the full generative model marginalizes over unobserved stages:
\begin{equation}
\Prob(r \mid h_i)
=
\sum_{u\in U(h_i)}
\sum_{d\in D(h_i)}
\sum_{\ell\in L(h_i)}
\Prob(r \mid \ell)\,
\Prob(\ell\mid d)\,
\Prob(d\mid u)\,
\Prob(u\mid h_i).
\end{equation}

Assuming reads are conditionally independent given $h_i$,
\begin{equation}
\Prob(R \mid h_i) = \prod_{r\in R} \Prob(r \mid h_i).
\end{equation}

By Bayes' rule, the posterior over haplotypes is
\begin{equation}
\Prob(h_i\mid R)
=
\frac{\Prob(R\mid h_i)\,\Prob(h_i)}
{\sum_{j=1}^P \Prob(R\mid h_j)\,\Prob(h_j)}.
\end{equation}

We may define a likelihood ratio
\begin{equation}
\mathrm{LR}_i(R)
=
\frac{\Prob(h_i\mid R)}{1 - \Prob(h_i\mid R)},
\end{equation}
and accept $h_i$ as the sample's haplotype if $\mathrm{LR}_i(R) \ge \tau$ for a threshold $\tau$, or alternatively choose the MAP haplotype
\begin{equation}
\hat{h} = \arg\max_{1\le i\le P} \Prob(h_i\mid R),
\end{equation}
subject to a minimum posterior probability requirement.

%%%%%%%%%%%%%%%%%%%%%%%%%%%%%%%%%%%%%%%%%%%%%%%%%%%%%%%%%%%%%%%%%%%%%%%%
\section{Diplotypes, Polyploidy, and Cost-Based Decision Rules}
\label{sec:app-f-diplotype-classification}

For diploid loci (and more generally polyploidy), we let $\mathcal{D}$ be the space of possible diplotypes, with prior $\pi_d = \Prob(d)$. For a decision policy that may incorrectly call $d'\neq d$ or request resequencing, we define:
\begin{itemize}
\item $\varepsilon_{d\to d'}(\gamma,N)$: misclassification rate from true diplotype $d$ to $d'$ using $N$ reads and posterior threshold $\gamma$
\item $\psi_d(\gamma,N)$: probability that a sample with true diplotype $d$ is flagged for resequencing
\end{itemize}

Let $C_{d\to d'}$ be the cost of miscalling $d$ as $d'$, and $C_{\text{res},d}$ the cost of resequencing when the true diplotype is $d$. Then the expected cost is
\begin{equation}
C(\gamma,N)
=
\sum_{d\in \mathcal{D}}
\pi_d
\left[
\sum_{d'\neq d}
C_{d\to d'}\,\varepsilon_{d\to d'}(\gamma,N)
+
C_{\text{res},d}\,\psi_d(\gamma,N)
\right].
\end{equation}

We can define optimal $(\gamma^\ast,N^\ast)$ as the minimizer of $C(\gamma,N)$ under constraints (e.g., minimal sensitivity, budget).

%%%%%%%%%%%%%%%%%%%%%%%%%%%%%%%%%%%%%%%%%%%%%%%%%%%%%%%%%%%%%%%%%%%%%%%%
\section{Read-Level Haplotagging with Known Haplotype}
\label{sec:app-f-haplotagging}

Assume a known haplotype $h$ with molecules $M(h)=\{m_1,\ldots,m_v\}$ and priors $\Prob(m_j\mid h)$ reflecting stoichiometric ratios. For a read $r$, we compute
\begin{equation}
\Prob(m_j\mid r,h)
=
\frac{\Prob(r\mid m_j,h)\,\Prob(m_j\mid h)}
{\sum_{k=1}^v \Prob(r\mid m_k,h)\,\Prob(m_k\mid h)}.
\end{equation}

Define $\Prob(\text{other}\mid r,h) = 1 - \Prob(m_j\mid r,h)$ and likelihood ratio
\begin{equation}
\mathrm{LR}_j(r) = \frac{\Prob(m_j\mid r,h)}{\Prob(\text{other}\mid r,h)}.
\end{equation}

We assign read $r$ to $m_j$ if $\mathrm{LR}_j(r) \ge \tau$; otherwise the read is left unphased. For a given $\tau$ and total read count $N$, we can trade off unphased fraction vs misphasing via a cost function
\begin{equation}
C(\tau,N) = C_{\text{mis}} N P_{\text{mis}}(\tau) + C_{\text{unph}} N P_{\text{unph}}(\tau).
\end{equation}

%%%%%%%%%%%%%%%%%%%%%%%%%%%%%%%%%%%%%%%%%%%%%%%%%%%%%%%%%%%%%%%%%%%%%%%%
\section{Plasmid Replication, Mutation, and Purity Bounds}
\label{sec:app-f-plasmid-purity}

Let:
\begin{itemize}
\item $r$: per-base replication error rate
\item $L$: plasmid length in bp
\item $k$: number of replication cycles
\end{itemize}

A single base remains error-free after $k$ replications with probability $(1-r)^k$. Assuming independence across bases, the probability that the entire $L$-bp plasmid remains identical to the original is
\begin{equation}
P_{\text{pure}}(k) = (1-r)^{Lk} \approx \exp(-rLk)
\end{equation}
for small $r$. This upper-bounds the fraction of molecules that remain perfect copies after $k$ cycles.

Define the mutated fraction $P_{\text{mut}} = 1 - P_{\text{pure}}(k)$. A Phred-like purity Q-value is
\begin{equation}
Q_{\text{pur}} = -10 \log_{10}\bigl(P_{\text{mut}}\bigr).
\end{equation}

Experimental lower bounds on purity are obtained from capillary electrophoresis:
\begin{equation}
P_{\text{low}} = \frac{C_{\text{major}}}{C_{\text{major}} + C_{\text{other}}},
\end{equation}
and empirically from clonal expansion and Sanger sequencing via proportions of colonies matching the intended sequence.

%%%%%%%%%%%%%%%%%%%%%%%%%%%%%%%%%%%%%%%%%%%%%%%%%%%%%%%%%%%%%%%%%%%%%%%%
\section{Dual Cas9 Cutting and Gene-Isolation Probability}
\label{sec:app-f-dual-cas9}

% Placeholder for future figure showing read length distributions by end reason
\label{fig:end-reason-lengths}

\begin{itemize}
\item $G$: distance in bp between two Cas9 target sites (length of desired fragment)
\item $L$: fragment length random variable with pdf $f_L(\ell)$ and cdf $F_L(\ell)$
\item $e_1,e_2$: cutting efficiencies at the two Cas9 sites
\end{itemize}

The probability that a fragment is at least as long as the gene is
\begin{equation}
p_{\text{frag}}(G) = \Prob(L \ge G) = 1 - F_L(G^-).
\end{equation}

Assuming independence of Cas9 cuts and fragmentation, the conditional probability that both cuts succeed, given that the fragment is long enough, is $e_1 e_2$. Thus the overall probability of successfully isolating the gene is
\begin{equation}
p_{\text{dual}}(G) = p_{\text{frag}}(G) \, e_1 e_2.
\end{equation}

For exponentially distributed fragment lengths ($f_L(\ell)=\lambda e^{-\lambda \ell}$), we obtain
\begin{equation}
p_{\text{dual}}(G) = e^{-\lambda G} e_1 e_2,
\end{equation}
showing the exponential decay of success probability as a function of target length $G$.

%%%%%%%%%%%%%%%%%%%%%%%%%%%%%%%%%%%%%%%%%%%%%%%%%%%%%%%%%%%%%%%%%%%%%%%%
\section{Worked Examples and Computational Protocols}
\label{sec:app-f-worked-examples}

This section provides detailed computational walkthroughs of key mathematical models, enabling readers to reproduce calculations and implement the framework in practice.

\subsection{Example 1: Confusion Matrix Construction and SMA Calculation}

\textbf{Scenario:} A SEER experiment sequences three plasmid standards (S1, S2, S3) representing CYP2D6 *1, *2, and *10 alleles. Each standard is sequenced to $N=1000$ reads. The basecaller outputs unique sequences $s_1,\ldots,s_M$ with counts.

\textbf{Data:} For standard S1 (true sequence $s_1$):
\begin{itemize}
\item 950 reads correctly called as $s_1$
\item 30 reads miscalled as $s_2$ (single SNP error)
\item 15 reads miscalled as $s_4$ (insertion error)
\item 5 reads miscalled as $s_7$ (deletion error)
\end{itemize}

\textbf{Confusion matrix row for S1:}
\begin{equation}
C_{1j} = \begin{cases}
950, & j=1 \text{ (correct)}\\
30, & j=2\\
15, & j=4\\
5, & j=7\\
0, & \text{otherwise}
\end{cases}
\end{equation}

Total for S1: $N_1 = 950 + 30 + 15 + 5 = 1000$.

\textbf{True positive rate:}
\begin{equation}
\mathrm{TPR}_1 = \frac{C_{11}}{N_1} = \frac{950}{1000} = 0.950 = 95.0\%
\end{equation}

\textbf{Single Molecule Accuracy:}
\begin{equation}
\mathrm{SMA}(s_1) = \mathrm{TPR}_1 = 95.0\%
\end{equation}

\textbf{Empirical error rate:}
\begin{equation}
\varepsilon_1 = 1 - \mathrm{TPR}_1 = 5.0\%
\end{equation}

\textbf{Empirical Phred quality:}
\begin{equation}
Q_{\text{emp}}(s_1) = -10\log_{10}(0.05) \approx 13.0
\end{equation}

\textbf{Comparison to predicted quality:} If basecaller reports mean predicted $Q_{\text{pred}} = 20$ for these reads, the quality overstatement is:
\begin{equation}
\Delta Q = Q_{\text{pred}} - Q_{\text{emp}} = 20 - 13 = 7 \text{ Phred points (overconfident)}
\end{equation}

\textbf{Pairwise misclassification probabilities:}
\begin{align}
\Prob(\hat{s}=s_2\mid s_1) &= \frac{C_{12}}{N_1} = \frac{30}{1000} = 3.0\%\\
\Prob(\hat{s}=s_4\mid s_1) &= \frac{C_{14}}{N_1} = \frac{15}{1000} = 1.5\%\\
\Prob(\hat{s}=s_7\mid s_1) &= \frac{C_{17}}{N_1} = \frac{5}{1000} = 0.5\%
\end{align}

These probabilities directly populate the likelihood $\Prob(r\mid h)$ for haplotype classification (Section~\ref{sec:app-f-haplotype-classification}).

\subsection{Example 2: Haplotype Classification with Bayesian Posterior}

\textbf{Scenario:} A sample of unknown CYP2D6 diplotype yields $N=500$ reads. Candidate haplotypes are $\mathcal{H} = \{*1, *2, *10\}$ with confusion matrix from Example 1.

\textbf{Observed data:}
\begin{itemize}
\item 220 reads match *1 sequence exactly
\item 250 reads match *2 sequence exactly
\item 30 reads have ambiguous mapping
\end{itemize}

\textbf{Simplified model (ignoring ambiguous reads):} We have effectively 470 informative reads.

\textbf{Likelihood calculation for $h=*1$:}
Assuming conditional independence and using confusion matrix:
\begin{equation}
\Prob(R\mid h=*1) = \mathrm{TPR}_1^{220} \cdot \Prob(\hat{s}=s_2\mid s_1)^{250}
= 0.950^{220} \cdot 0.030^{250}
\end{equation}

This is extremely small; taking log-likelihood:
\begin{equation}
\log\Prob(R\mid *1) = 220\log(0.950) + 250\log(0.030) \approx -11.3 - 380.2 = -391.5
\end{equation}

\textbf{Likelihood for $h=*2$:}
\begin{equation}
\log\Prob(R\mid *2) = 220\log(0.03) + 250\log(0.95) \approx -334.8 - 12.8 = -347.6
\end{equation}

\textbf{Likelihood for $h=*10$:} Assuming *10 has distinct sequence and zero matches:
\begin{equation}
\log\Prob(R\mid *10) \approx -1000 \text{ (essentially zero)}
\end{equation}

\textbf{Posterior calculation (uniform prior):}
Using log-sum-exp trick:
\begin{equation}
\Prob(h=*2\mid R) = \frac{\exp(-347.6)}{\exp(-391.5) + \exp(-347.6) + \exp(-1000)} \approx \frac{1}{1 + \exp(-43.9) + 0} \approx 1.0
\end{equation}

\textbf{Result:} Posterior $\Prob(*2\mid R) \approx 0.9999$. MAP diplotype is *2 with overwhelming confidence.

\subsection{Example 3: Plasmid Purity Degradation Over Replication Cycles}

\textbf{Parameters:}
\begin{itemize}
\item Plasmid length: $L = 5000$ bp
\item Bacterial replication error rate: $r = 10^{-9}$ per base per replication
\item Number of replication cycles: $k = 30$ (overnight culture)
\end{itemize}

\textbf{Probability a single base remains error-free:}
\begin{equation}
p_{\text{base,pure}}(k) = (1-r)^k = (1-10^{-9})^{30} \approx 0.99999997
\end{equation}

\textbf{Probability entire plasmid remains error-free:}
\begin{equation}
P_{\text{pure}}(k) = (1-r)^{Lk} = (1-10^{-9})^{5000 \times 30} = (1-10^{-9})^{150000}
\end{equation}

Using approximation $(1-x)^n \approx e^{-nx}$ for small $x$:
\begin{equation}
P_{\text{pure}}(30) \approx \exp(-10^{-9} \times 150000) = \exp(-0.00015) \approx 0.99985 = 99.985\%
\end{equation}

\textbf{Mutated fraction:}
\begin{equation}
P_{\text{mut}} = 1 - P_{\text{pure}} = 0.00015 = 0.015\%
\end{equation}

\textbf{Purity Q-value:}
\begin{equation}
Q_{\text{pur}} = -10\log_{10}(0.00015) \approx 38.2
\end{equation}

\textbf{Interpretation:} After 30 replication cycles, purity remains $>99.98\%$, exceeding the SEER requirement $\pi \ge 0.95$ (Chapter~\ref{chap:purity}). This validates bacterial amplification as a reliable source of high-purity standards.

\subsection{Example 4: Dual Cas9 Cutting Probability for 15 kb Gene}

\textbf{Parameters:}
\begin{itemize}
\item Target gene length: $G = 15000$ bp
\item Cas9 cutting efficiencies: $e_1 = 0.85$, $e_2 = 0.90$
\item Fragment length distribution: Exponential with mean $\lambda^{-1} = 20000$ bp
\end{itemize}

\textbf{Probability fragment is long enough:}
\begin{equation}
p_{\text{frag}}(15000) = \Prob(L \ge 15000) = e^{-\lambda G} = \exp\left(-\frac{15000}{20000}\right) = \exp(-0.75) \approx 0.472 = 47.2\%
\end{equation}

\textbf{Probability both Cas9 sites cut:}
\begin{equation}
p_{\text{cut}} = e_1 e_2 = 0.85 \times 0.90 = 0.765 = 76.5\%
\end{equation}

\textbf{Overall success probability:}
\begin{equation}
p_{\text{dual}}(15000) = p_{\text{frag}}(15000) \cdot p_{\text{cut}} = 0.472 \times 0.765 \approx 0.361 = 36.1\%
\end{equation}

\textbf{Expected coverage:} If sequencing depth is $100\times$ for the full genome, effective coverage for the isolated 15 kb gene is:
\begin{equation}
\text{Effective coverage} = 100 \times p_{\text{dual}}(15000) = 100 \times 0.361 = 36.1\times
\end{equation}

\textbf{Optimization insight:} Success probability is dominated by fragmentation, not Cas9 efficiency. Increasing Cas9 efficiency from 85\% to 95\% ($e_1 = 0.95$) yields:
\begin{equation}
p_{\text{dual}}^{\text{new}} = 0.472 \times (0.95 \times 0.90) \approx 0.404 = 40.4\%
\end{equation}
a modest $\sim$12\% relative improvement. In contrast, increasing mean fragment length to 30 kb yields:
\begin{equation}
p_{\text{dual}}^{\text{long}} = \exp(-15000/30000) \times 0.765 = 0.606 \times 0.765 \approx 0.464 = 46.4\%
\end{equation}
a $\sim$29\% relative improvement. This suggests that optimizing library preparation (gentler DNA extraction, reduced shearing) is more effective than improving Cas9 reagents for large target isolation.

\subsection{Example 5: Quality Score Calibration Assessment}

\textbf{Scenario:} A basecaller reports per-base quality scores. For reads aligned to a standard with known sequence, we bin reads by predicted $Q_{\text{pred}}$ and compute empirical error rates.

\textbf{Data:}
\begin{center}
\begin{tabular}{ccccc}
\toprule
$Q_{\text{pred}}$ bin & \# Bases & \# Errors & $p_{\text{emp}}$ & $Q_{\text{emp}}$ \\
\midrule
Q10--Q15 & 50,000 & 5,200 & 0.104 & 9.8 \\
Q15--Q20 & 120,000 & 6,000 & 0.050 & 13.0 \\
Q20--Q25 & 200,000 & 4,000 & 0.020 & 17.0 \\
Q25--Q30 & 150,000 & 1,500 & 0.010 & 20.0 \\
Q30+ & 80,000 & 400 & 0.005 & 23.0 \\
\bottomrule
\end{tabular}
\end{center}

\textbf{Predicted vs. empirical comparison:}
\begin{itemize}
\item \textbf{Q10--Q15 bin (midpoint Q12.5):} Predicted $p_{\text{pred}} = 10^{-1.25} \approx 0.056$; Empirical $p_{\text{emp}} = 0.104$. \textbf{Underestimated quality} (worse than predicted).
\item \textbf{Q20--Q25 bin (midpoint Q22.5):} Predicted $p_{\text{pred}} = 10^{-2.25} \approx 0.0056$; Empirical $p_{\text{emp}} = 0.020$. \textbf{Overestimated quality} (basecaller overconfident).
\item \textbf{Q30+ bin (midpoint Q32):} Predicted $p_{\text{pred}} = 10^{-3.2} \approx 0.00063$; Empirical $p_{\text{emp}} = 0.005$. \textbf{Overestimated quality}.
\end{itemize}

\textbf{Calibration curve:} Plotting $Q_{\text{emp}}$ vs. $Q_{\text{pred}}$ should yield a diagonal line if perfectly calibrated. This dataset shows systematic overconfidence at high predicted Q (points fall below diagonal).

\textbf{Correction strategy:} Fit a calibration function $Q_{\text{emp}} = f(Q_{\text{pred}})$ using logistic regression or isotonic regression. For downstream haplotype classification, replace reported $Q$ with $f(Q)$ to obtain calibrated likelihoods.

%%%%%%%%%%%%%%%%%%%%%%%%%%%%%%%%%%%%%%%%%%%%%%%%%%%%%%%%%%%%%%%%%%%%%%%%
\section{Summary and Integration with Framework}
\label{sec:app-f-summary}

This appendix provides the canonical mathematical reference for the SMS Haplotype Classification Framework, consolidating:
\begin{itemize}
\item The Pipeline Factorization Theorem (Theorem~\ref{thm:app-f-pipeline-factorization}), which decomposes the full generative model
\item The formal definition of Single Molecule Accuracy (Definition~\ref{def:app-f-sma})
\item Quality score hierarchies and the critical distinction between predicted and empirical error rates
\item Confusion matrix construction and sequence-level error models that populate likelihood functions
\item Cost-based decision rules for diplotype classification and haplotagging
\item Purity bounds and dual Cas9 probability models for experimental design
\end{itemize}

These mathematical constructions are referenced throughout the main text:
\begin{itemize}
\item Chapter~\ref{chap:classification-model}: Pipeline factorization and state-space hierarchy
\item Chapter~\ref{chap:purity}: Purity theory and replication error models
\item Chapter~\ref{chap:posteriors}: Bayesian classification using confusion-matrix likelihoods
\item Chapter~\ref{chap:sma-seq}: SMA-seq methodology and SEER framework
\item Chapter~\ref{chap:experimental-design}: Dual Cas9 models and experimental optimization
\end{itemize}

The unified notation and canonical formulations in this appendix ensure consistency across all mathematical reasoning in the framework, enabling readers to trace derivations from foundational principles to practical implementations.


% Appendix G: Master Variable Reference Table (NEW - v6.2)
% Comprehensive catalog of all variables with definitions and cross-references
%%%%%%%%%%%%%%%%%%%%%%%%%%%%%%%%%%%%%%%%%%%%%%%%%%%%%%%%%%%%%%%%%%%%%%%%
%% Appendix G: Master Variable Reference Table
%% Version 6.2 - Boxed Reference Cards (November 2025)
%% Comprehensive cross-reference for all variables in the framework
%%%%%%%%%%%%%%%%%%%%%%%%%%%%%%%%%%%%%%%%%%%%%%%%%%%%%%%%%%%%%%%%%%%%%%%%

\chapter{Master Variable Reference Table}
\label{app:variable-master}
\label{app:variables}

This appendix provides a comprehensive, alphabetically organized reference table for all mathematical variables, symbols, and notation used throughout the SMS Haplotype Classification Framework. Each entry is presented as a boxed reference card including:
\begin{itemize}
\item \textbf{Symbol}: The mathematical notation
\item \textbf{Name/Description}: Full name and meaning
\item \textbf{Definition}: Full prose definition of the variable
\item \textbf{Type}: Scalar, vector, matrix, set, random variable, etc.
\item \textbf{Domain/Range}: Valid values or mathematical space
\item \textbf{Units}: Physical or mathematical units where applicable
\item \textbf{First Definition}: Chapter and section where formally introduced
\item \textbf{Key Uses}: Primary chapters where the variable plays a central role
\end{itemize}

%%%%%%%%%%%%%%%%%%%%%%%%%%%%%%%%%%%%%%%%%%%%%%%%%%%%%%%%%%%%%%%%%%%%%%%%
\section{Latin Alphabet (Uppercase)}
\label{sec:variables-latin-upper}

\MasterVarBox
  {V-A}% ID
  {A}% Symbol
  {Accuracy}% Name
  {Overall fraction of correct classifications in a given evaluation set, often measured at the diplotype or haplotype level.}% Definition
  {Scalar}% Type
  {[0, 1]}% Domain
  {dimensionless}% Units
  {Ch.~4.2}% First defined
  {Ch.~4, 11, 14}% Key uses

\MasterVarBox
  {V-AS}% ID
  {\text{AS}}% Symbol
  {Activity Score}% Name
  {Composite measure of CYP2D6 metabolic capacity derived from star allele genotypes, used to define phenotype categories such as normal or poor metabolizer.}% Definition
  {Scalar}% Type
  {\mathbb{R}_{\geq 0}}% Domain
  {dimensionless}% Units
  {Ch.~17.3, 18.2}% First defined
  {Ch.~17, 18}% Key uses

\MasterVarBox
  {V-AS-EFF}% ID
  {\text{AS}_{\text{eff}}}% Symbol
  {Effective Activity Score (phenoconverted)}% Name
  {Activity score adjusted for drug-drug interactions or enzyme inhibition, representing the actual metabolic capacity in the presence of inhibitors.}% Definition
  {Scalar}% Type
  {\mathbb{R}_{\geq 0}}% Domain
  {dimensionless}% Units
  {Ch.~17.3}% First defined
  {Ch.~17}% Key uses

\MasterVarBox
  {V-AS-GENO}% ID
  {\text{AS}_{\text{geno}}}% Symbol
  {Genotypic Activity Score}% Name
  {Activity score derived solely from genotype without environmental or drug interaction adjustments.}% Definition
  {Scalar}% Type
  {\mathbb{R}_{\geq 0}}% Domain
  {dimensionless}% Units
  {Ch.~17.3}% First defined
  {Ch.~17}% Key uses

\MasterVarBox
  {V-C-IJ}% ID
  {C_{ij}}% Symbol
  {Confusion Matrix Entry}% Name
  {Individual entry in confusion matrix representing count or probability of classifying true class $i$ as predicted class $j$.}% Definition
  {Matrix entry}% Type
  {\mathbb{N}}% Domain
  {count}% Units
  {Ch.~4.3}% First defined
  {Ch.~4, 6, 11}% Key uses

\MasterVarBox
  {V-C-MATRIX}% ID
  {\mathbf{C}}% Symbol
  {Confusion Matrix}% Name
  {Matrix representation of classification performance where entry $(i,j)$ represents the count or probability of true class $i$ being classified as class $j$.}% Definition
  {Matrix}% Type
  {\mathbb{R}^{P \times P}}% Domain
  {prob. or count}% Units
  {Ch.~4.3}% First defined
  {Ch.~4, 6, 11, 14}% Key uses

\MasterVarBox
  {V-C-STRAIN}% ID
  {C_{\text{strain}}}% Symbol
  {Bacterial Strain Identifier}% Name
  {Categorical identifier for bacterial strain used in validation experiments or antimicrobial resistance typing.}% Definition
  {Categorical}% Type
  {---}% Domain
  {---}% Units
  {Ch.~16.1}% First defined
  {Ch.~16}% Key uses

\MasterVarBox
  {V-C-TEST}% ID
  {C_{\text{test}}}% Symbol
  {Cost of Genotyping Test}% Name
  {Dollar cost of performing a single genotyping test, used in cost-effectiveness analyses.}% Definition
  {Scalar}% Type
  {\mathbb{R}_{> 0}}% Domain
  {USD}% Units
  {Ch.~17.5, 20.2}% First defined
  {Ch.~17, 20}% Key uses

\MasterVarBox
  {V-C-TOTAL}% ID
  {C_{\text{total}}}% Symbol
  {Total Cost}% Name
  {Total cost including testing, treatment, and adverse event management in health economic analyses.}% Definition
  {Scalar}% Type
  {\mathbb{R}}% Domain
  {USD}% Units
  {Ch.~17.5}% First defined
  {Ch.~17, 20}% Key uses

\MasterVarBox
  {V-D}% ID
  {D}% Symbol
  {Diplotype}% Name
  {Ordered or unordered pair of haplotypes representing the complete genotype at a locus, written as $(h_1, h_2)$ or $h_1/h_2$.}% Definition
  {Tuple}% Type
  {\mathcal{H} \times \mathcal{H}}% Domain
  {---}% Units
  {Ch.~6.4, 18.4}% First defined
  {Ch.~6, 14, 17, 18}% Key uses

\MasterVarBox
  {V-D-AMR}% ID
  {D_{\text{AMR}}}% Symbol
  {Depth Threshold for AMR Locus}% Name
  {Minimum read depth required at antimicrobial resistance loci for confident variant calling.}% Definition
  {Scalar}% Type
  {\mathbb{N}}% Domain
  {reads}% Units
  {Ch.~16.1}% First defined
  {Ch.~16}% Key uses

\MasterVarBox
  {V-D-KL}% ID
  {D_{\text{KL}}}% Symbol
  {Kullback-Leibler Divergence}% Name
  {Statistical distance measure between two probability distributions, quantifying information loss when one distribution is used to approximate another.}% Definition
  {Scalar}% Type
  {\mathbb{R}_{\geq 0}}% Domain
  {bits or nats}% Units
  {Ch.~7.4}% First defined
  {Ch.~7, 14}% Key uses

\MasterVarBox
  {V-D-REPORTED}% ID
  {D_{\text{reported}}}% Symbol
  {Reported Diplotype}% Name
  {Diplotype classification reported by the haplotype calling algorithm, which may differ from the true diplotype due to classification errors.}% Definition
  {Tuple}% Type
  {\mathcal{H} \times \mathcal{H}}% Domain
  {---}% Units
  {Ch.~17.2}% First defined
  {Ch.~17, 18}% Key uses

\MasterVarBox
  {V-D-TRUE}% ID
  {D_{\text{true}}}% Symbol
  {True Diplotype}% Name
  {Ground truth diplotype for a sample, typically established by orthogonal validation methods or known synthetic standards.}% Definition
  {Tuple}% Type
  {\mathcal{H} \times \mathcal{H}}% Domain
  {---}% Units
  {Ch.~17.2}% First defined
  {Ch.~17, 18}% Key uses

\MasterVarBox
  {V-d}% ID
  {d}% Symbol
  {DNA Fragment}% Name
  {DNA sequence fragment resulting from physical fragmentation or enzymatic digestion of genomic molecules.}% Definition
  {Sequence}% Type
  {\mathcal{A}^*}% Domain
  {---}% Units
  {Ch.~4.1}% First defined
  {Ch.~4, 8}% Key uses

\MasterVarBox
  {V-d-I}% ID
  {d^{(i)}}% Symbol
  {Fragment $i$}% Name
  {The $i$-th DNA fragment in a collection of fragments, indexed by $i$.}% Definition
  {Sequence}% Type
  {\mathcal{A}^{L_i}}% Domain
  {---}% Units
  {Ch.~4.1}% First defined
  {Ch.~4, 8}% Key uses

\MasterVarBox
  {V-d-EDIT}% ID
  {d_{\text{edit}}}% Symbol
  {Levenshtein Edit Distance}% Name
  {Minimum number of single-character edits (insertions, deletions, or substitutions) required to transform one string into another.}% Definition
  {Scalar}% Type
  {\mathbb{N}}% Domain
  {bases}% Units
  {Ch.~4.2}% First defined
  {Ch.~4, 11, 14}% Key uses

\MasterVarBox
  {V-E}% ID
  {E}% Symbol
  {Expected Value Operator}% Name
  {Mathematical expectation operator computing the average value of a random variable over its distribution.}% Definition
  {Operator}% Type
  {---}% Domain
  {---}% Units
  {Ch.~5.3, 7.2}% First defined
  {Ch.~5, 7, 14}% Key uses

\MasterVarBox
  {V-E-I}% ID
  {E_{\text{I}}}% Symbol
  {Type I Error Indicator}% Name
  {Binary indicator variable equal to 1 if a false positive classification occurs (normal metabolizer called as poor metabolizer).}% Definition
  {Binary}% Type
  {\{0, 1\}}% Domain
  {---}% Units
  {Ch.~17.2}% First defined
  {Ch.~17}% Key uses

\MasterVarBox
  {V-E-II}% ID
  {E_{\text{II}}}% Symbol
  {Type II Error Indicator}% Name
  {Binary indicator variable equal to 1 if a false negative classification occurs (poor metabolizer called as normal metabolizer).}% Definition
  {Binary}% Type
  {\{0, 1\}}% Domain
  {---}% Units
  {Ch.~17.2}% First defined
  {Ch.~17}% Key uses

\MasterVarBox
  {V-E-CLINICAL}% ID
  {E_{\text{clinical}}}% Symbol
  {Clinically Actionable Error Indicator}% Name
  {Binary indicator for genotyping errors that would alter clinical treatment decisions or dosing recommendations.}% Definition
  {Binary}% Type
  {\{0, 1\}}% Domain
  {---}% Units
  {Ch.~17.2}% First defined
  {Ch.~17}% Key uses

\MasterVarBox
  {V-E-PHENO}% ID
  {E_{\text{pheno}}}% Symbol
  {Phenotype Misclassification Indicator}% Name
  {Binary indicator equal to 1 if diplotype errors result in incorrect metabolizer phenotype assignment.}% Definition
  {Binary}% Type
  {\{0, 1\}}% Domain
  {---}% Units
  {Ch.~17.2}% First defined
  {Ch.~17}% Key uses

\MasterVarBox
  {V-F}% ID
  {F}% Symbol
  {Fragmentation Distribution}% Name
  {Probability distribution governing DNA fragment sizes after physical or enzymatic fragmentation.}% Definition
  {Function}% Type
  {---}% Domain
  {---}% Units
  {Ch.~4.1}% First defined
  {Ch.~4, 8}% Key uses

\MasterVarBox
  {V-f}% ID
  {f}% Symbol
  {Basecaller Function}% Name
  {Function mapping raw sequencing signals (in signal space $\mathcal{S}$) to basecalled reads (in read space $\mathcal{R}$).}% Definition
  {Function}% Type
  {\mathcal{S} \to \mathcal{R}}% Domain
  {---}% Units
  {Ch.~4.2}% First defined
  {Ch.~4, 11}% Key uses

\MasterVarBox
  {V-f-INFORM}% ID
  {f_{\text{inform}}}% Symbol
  {Informative Read Fraction}% Name
  {Fraction of sequencing reads that span variant positions and can contribute to haplotype discrimination.}% Definition
  {Scalar}% Type
  {[0, 1]}% Domain
  {dimensionless}% Units
  {Ch.~7.4}% First defined
  {Ch.~7, 14}% Key uses

\MasterVarBox
  {V-f-POP}% ID
  {f_{\text{pop}}}% Symbol
  {Population Haplotype Frequency}% Name
  {Frequency of a specific haplotype in the reference population, used as prior probability in Bayesian haplotype calling.}% Definition
  {Scalar}% Type
  {[0, 1]}% Domain
  {dimensionless}% Units
  {Ch.~6.2}% First defined
  {Ch.~6, 17, 18}% Key uses

\MasterVarBox
  {V-G}% ID
  {G}% Symbol
  {Genomic Locus}% Name
  {Specific region or coordinate range in the genome targeted for sequencing and haplotype analysis.}% Definition
  {Set}% Type
  {---}% Domain
  {---}% Units
  {Ch.~4.1}% First defined
  {Ch.~4}% Key uses

\MasterVarBox
  {V-g}% ID
  {g}% Symbol
  {Genomic Molecule}% Name
  {DNA molecule extracted from genomic DNA prior to library preparation, representing native sequence content.}% Definition
  {Sequence}% Type
  {\mathcal{A}^*}% Domain
  {---}% Units
  {Ch.~4.1}% First defined
  {Ch.~4, 8}% Key uses

\MasterVarBox
  {V-H}% ID
  {H}% Symbol
  {Haplotype Set}% Name
  {Complete set of candidate haplotypes considered in classification, typically defined by a reference panel such as PharmVar.}% Definition
  {Set}% Type
  {\mathcal{H}}% Domain
  {---}% Units
  {Ch.~4.1, 6.1}% First defined
  {Ch.~4--6, 17, 18}% Key uses

\MasterVarBox
  {V-H-0}% ID
  {H_0}% Symbol
  {Null Hypothesis}% Name
  {Baseline hypothesis in statistical hypothesis testing, typically representing no effect or no difference.}% Definition
  {Hypothesis}% Type
  {---}% Domain
  {---}% Units
  {Ch.~7.3, 14.3}% First defined
  {Ch.~7, 14, 17, 18}% Key uses

\MasterVarBox
  {V-H-1}% ID
  {H_1}% Symbol
  {Alternative Hypothesis}% Name
  {Alternative hypothesis in statistical testing, representing the effect or difference of interest.}% Definition
  {Hypothesis}% Type
  {---}% Domain
  {---}% Units
  {Ch.~7.3, 14.3}% First defined
  {Ch.~7, 14, 17, 18}% Key uses

\MasterVarBox
  {V-h}% ID
  {h}% Symbol
  {Haplotype}% Name
  {Single linear sequence of nucleotides at a genomic locus, representing one of the two copies inherited from parents.}% Definition
  {Sequence}% Type
  {\mathcal{A}^L}% Domain
  {---}% Units
  {Ch.~4.1}% First defined
  {Ch.~4--6, 14, 17, 18}% Key uses

\MasterVarBox
  {V-h-I}% ID
  {h_i}% Symbol
  {Haplotype $i$}% Name
  {The $i$-th haplotype in the reference panel or candidate set, indexed by $i$.}% Definition
  {Sequence}% Type
  {\mathcal{A}^L}% Domain
  {---}% Units
  {Ch.~4.1}% First defined
  {Ch.~4--6, 14, 17, 18}% Key uses

\MasterVarBox
  {V-h-CLOSEST}% ID
  {h_{\text{closest}}}% Symbol
  {Closest Competing Haplotype}% Name
  {Haplotype in the panel with minimum edit distance to the true haplotype, representing the most likely misclassification target.}% Definition
  {Sequence}% Type
  {\mathcal{A}^L}% Domain
  {---}% Units
  {Ch.~7.4}% First defined
  {Ch.~7}% Key uses

\MasterVarBox
  {V-h-MAJOR}% ID
  {h_{\text{major}}}% Symbol
  {Major Haplotype in Mixture}% Name
  {Haplotype present at higher frequency in a diploid mixture or contaminated sample.}% Definition
  {Sequence}% Type
  {\mathcal{A}^L}% Domain
  {---}% Units
  {Ch.~7.5, 14.1}% First defined
  {Ch.~7, 14}% Key uses

\MasterVarBox
  {V-h-MINOR}% ID
  {h_{\text{minor}}}% Symbol
  {Minor Haplotype in Mixture}% Name
  {Haplotype present at lower frequency in a diploid mixture, often more difficult to detect and classify.}% Definition
  {Sequence}% Type
  {\mathcal{A}^L}% Domain
  {---}% Units
  {Ch.~7.5, 14.1}% First defined
  {Ch.~7, 14}% Key uses

\MasterVarBox
  {V-h-TRUE}% ID
  {h_{\text{true}}}% Symbol
  {True Haplotype}% Name
  {Ground truth haplotype for a sample, established by independent validation or known synthetic controls.}% Definition
  {Sequence}% Type
  {\mathcal{A}^L}% Domain
  {---}% Units
  {Ch.~6.3, 7.4}% First defined
  {Ch.~6, 7, 14}% Key uses

\MasterVarBox
  {V-I}% ID
  {I}% Symbol
  {Enzyme Inhibition Fraction}% Name
  {Fraction of enzyme activity lost due to competitive or non-competitive inhibition by co-administered drugs.}% Definition
  {Scalar}% Type
  {[0, 1]}% Domain
  {dimensionless}% Units
  {Ch.~17.3}% First defined
  {Ch.~17}% Key uses

\MasterVarBox
  {V-K}% ID
  {K}% Symbol
  {Number of Events/Reads}% Name
  {Random variable representing count of events (reads, errors, or successes) in binomial or Poisson models.}% Definition
  {Random Var.}% Type
  {\mathbb{N}}% Domain
  {count}% Units
  {Ch.~7.3, 14.3}% First defined
  {Ch.~7, 14}% Key uses

\MasterVarBox
  {V-k-DIST}% ID
  {k}% Symbol
  {Number of Distinguishing Positions}% Name
  {Number of base positions at which two haplotypes differ, determining their discriminability given sequencing errors.}% Definition
  {Scalar}% Type
  {\mathbb{N}}% Domain
  {bases}% Units
  {Ch.~7.4}% First defined
  {Ch.~7, 14}% Key uses

\MasterVarBox
  {V-k-CYCLE}% ID
  {k}% Symbol
  {Replication Cycle Count}% Name
  {Number of PCR or plasmid replication cycles, affecting error accumulation in library preparation.}% Definition
  {Scalar}% Type
  {\mathbb{N}}% Domain
  {cycles}% Units
  {Ch.~5.2}% First defined
  {Ch.~5, 8}% Key uses

\MasterVarBox
  {V-k-ALPHA}% ID
  {k_\alpha}% Symbol
  {Critical Value at Significance $\alpha$}% Name
  {Critical threshold value for hypothesis test at significance level $\alpha$, used to determine rejection regions.}% Definition
  {Scalar}% Type
  {\mathbb{R} \text{ or } \mathbb{N}}% Domain
  {---}% Units
  {Ch.~7.3, 14.3}% First defined
  {Ch.~7, 14}% Key uses

\MasterVarBox
  {V-L}% ID
  {L}% Symbol
  {Sequence Length}% Name
  {Length of haplotype, read, or genomic region in base pairs.}% Definition
  {Scalar}% Type
  {\mathbb{N}}% Domain
  {bases}% Units
  {Ch.~4.1}% First defined
  {Ch.~4--7, 11, 14}% Key uses

\MasterVarBox
  {V-L-I}% ID
  {L_i}% Symbol
  {Length of Read $i$}% Name
  {Length in bases of the $i$-th sequencing read, which may vary across reads.}% Definition
  {Scalar}% Type
  {\mathbb{N}}% Domain
  {bases}% Units
  {Ch.~4.2}% First defined
  {Ch.~4, 7, 11}% Key uses

\MasterVarBox
  {V-L-FRAG}% ID
  {L_{\text{frag}}}% Symbol
  {Mean Fragment Length}% Name
  {Average length of DNA fragments after physical or enzymatic fragmentation.}% Definition
  {Scalar}% Type
  {\mathbb{R}_{> 0}}% Domain
  {bases}% Units
  {Ch.~7.4, 8.2}% First defined
  {Ch.~7, 8, 14}% Key uses

\MasterVarBox
  {V-L-MOL}% ID
  {L_{\text{mol}}}% Symbol
  {Molecule Length}% Name
  {Length of intact genomic or plasmid DNA molecule prior to fragmentation.}% Definition
  {Scalar}% Type
  {\mathbb{N}}% Domain
  {bases}% Units
  {Ch.~4.1, 5.2}% First defined
  {Ch.~4, 5, 8}% Key uses

\MasterVarBox
  {V-M}% ID
  {M}% Symbol
  {Confusion Matrix (SMA context)}% Name
  {Confusion matrix specifically for single-molecule accuracy evaluation, tracking read-level correctness.}% Definition
  {Matrix}% Type
  {\mathbb{R}^{P \times P}}% Domain
  {probability}% Units
  {Ch.~11.2}% First defined
  {Ch.~11, 14}% Key uses

\MasterVarBox
  {V-M-AB}% ID
  {M_{ab}}% Symbol
  {SEER Matrix Entry}% Name
  {Entry in SEER (Sequence Error Rate Estimate) matrix representing transition probability from nucleotide $a$ to $b$.}% Definition
  {Matrix entry}% Type
  {[0, 1]}% Domain
  {probability}% Units
  {Ch.~11.2}% First defined
  {Ch.~11}% Key uses

\MasterVarBox
  {V-N}% ID
  {N}% Symbol
  {Total Read Count}% Name
  {Total number of sequencing reads aligned to the target locus.}% Definition
  {Scalar}% Type
  {\mathbb{N}}% Domain
  {reads}% Units
  {Ch.~4.2, 6.3}% First defined
  {Ch.~4, 6, 7, 14}% Key uses

\MasterVarBox
  {V-N-CORRECT}% ID
  {N_{\text{correct}}}% Symbol
  {Count of Correct Classifications}% Name
  {Number of correctly classified reads or molecules in accuracy evaluation.}% Definition
  {Scalar}% Type
  {\mathbb{N}}% Domain
  {reads}% Units
  {Ch.~11.2, 14.2}% First defined
  {Ch.~11, 14}% Key uses

\MasterVarBox
  {V-N-EFF}% ID
  {N_{\text{eff}}}% Symbol
  {Effective Coverage}% Name
  {Effective number of informative reads after filtering for quality and informativeness criteria.}% Definition
  {Scalar}% Type
  {\mathbb{R}_{\geq 0}}% Domain
  {reads}% Units
  {Ch.~7.4}% First defined
  {Ch.~7, 14}% Key uses

\MasterVarBox
  {V-N-PERFECT}% ID
  {N_{\text{perfect}}}% Symbol
  {Count of Perfect Reads}% Name
  {Number of error-free reads with perfect match to reference haplotype.}% Definition
  {Scalar}% Type
  {\mathbb{N}}% Domain
  {reads}% Units
  {Ch.~7.1}% First defined
  {Ch.~7}% Key uses

\MasterVarBox
  {V-N-REQUIRED}% ID
  {N_{\text{required}}}% Symbol
  {Required Read Count}% Name
  {Minimum number of reads needed to achieve target classification confidence or statistical power.}% Definition
  {Scalar}% Type
  {\mathbb{N}}% Domain
  {reads}% Units
  {Ch.~7.3, 7.4}% First defined
  {Ch.~7, 14}% Key uses

\MasterVarBox
  {V-N-TOTAL}% ID
  {N_{\text{total}}}% Symbol
  {Total Molecules/Reads}% Name
  {Total count of molecules or reads in dataset, used as denominator in rate calculations.}% Definition
  {Scalar}% Type
  {\mathbb{N}}% Domain
  {count}% Units
  {Ch.~5.1, 7.4, 14.2}% First defined
  {Ch.~5, 7, 11, 14}% Key uses

\MasterVarBox
  {V-N-I}% ID
  {N_i}% Symbol
  {Row Sum of Confusion Matrix}% Name
  {Total count of true class $i$ instances, equal to the sum of row $i$ in confusion matrix.}% Definition
  {Scalar}% Type
  {\mathbb{N}}% Domain
  {count}% Units
  {Ch.~4.3}% First defined
  {Ch.~4, 11}% Key uses

\MasterVarBox
  {V-P}% ID
  {P}% Symbol
  {Number of Haplotypes in Panel}% Name
  {Cardinality of the haplotype reference panel, determining the size of the classification problem.}% Definition
  {Scalar}% Type
  {\mathbb{N}}% Domain
  {count}% Units
  {Ch.~6.1, 7.4}% First defined
  {Ch.~6, 7, 17, 18}% Key uses

\MasterVarBox
  {V-P-A}% ID
  {P(A)}% Symbol
  {Probability of Event $A$}% Name
  {Probability measure of event $A$, fundamental to all probabilistic models in the framework.}% Definition
  {Scalar}% Type
  {[0, 1]}% Domain
  {dimensionless}% Units
  {Ch.~4.1}% First defined
  {All chapters}% Key uses

\MasterVarBox
  {V-P-0}% ID
  {P_0}% Symbol
  {Probability Under Null Hypothesis}% Name
  {Probability or parameter value assumed true under the null hypothesis in hypothesis testing.}% Definition
  {Scalar}% Type
  {[0, 1]}% Domain
  {dimensionless}% Units
  {Ch.~7.3}% First defined
  {Ch.~7}% Key uses

\MasterVarBox
  {V-P-1}% ID
  {P_1}% Symbol
  {Probability Under Alternative}% Name
  {Probability or parameter value assumed true under the alternative hypothesis.}% Definition
  {Scalar}% Type
  {[0, 1]}% Domain
  {dimensionless}% Units
  {Ch.~7.3}% First defined
  {Ch.~7}% Key uses

\MasterVarBox
  {V-P-ADR}% ID
  {P_{\text{ADR}}}% Symbol
  {Probability of Adverse Drug Reaction}% Name
  {Probability of experiencing an adverse drug reaction given genotype and drug regimen.}% Definition
  {Scalar}% Type
  {[0, 1]}% Domain
  {dimensionless}% Units
  {Ch.~17.5}% First defined
  {Ch.~17}% Key uses

\MasterVarBox
  {V-P-FAILURE}% ID
  {P_{\text{failure}}}% Symbol
  {Probability of Treatment Failure}% Name
  {Probability of therapeutic failure due to genotype-drug interaction or metabolic phenotype.}% Definition
  {Scalar}% Type
  {[0, 1]}% Domain
  {dimensionless}% Units
  {Ch.~17.5}% First defined
  {Ch.~17}% Key uses

\MasterVarBox
  {V-p}% ID
  {p}% Symbol
  {Error Probability (per base)}% Name
  {Per-base error probability in sequencing, typically derived from Phred quality scores.}% Definition
  {Scalar}% Type
  {[0, 1]}% Domain
  {dimensionless}% Units
  {Ch.~4.2}% First defined
  {Ch.~4, 7, 11}% Key uses

\MasterVarBox
  {V-p-I}% ID
  {p_i}% Symbol
  {Error Probability at Position $i$}% Name
  {Position-specific error probability at base position $i$ in a read.}% Definition
  {Scalar}% Type
  {[0, 1]}% Domain
  {dimensionless}% Units
  {Ch.~4.2}% First defined
  {Ch.~4, 7}% Key uses

\MasterVarBox
  {V-Q}% ID
  {Q}% Symbol
  {Phred Quality Score}% Name
  {Phred-scaled measure of per-base error probability, defined by $Q = -10 \log_{10} p$, where $p$ is the probability that the base call is incorrect.}% Definition
  {Scalar}% Type
  {\mathbb{R}_{\geq 0}}% Domain
  {Phred units}% Units
  {Ch.~4.2}% First defined
  {Ch.~4, 7, 11}% Key uses

\MasterVarBox
  {V-Q-EMP}% ID
  {Q_{\text{emp}}}% Symbol
  {Empirical Quality Score}% Name
  {Phred quality score derived from observed error rates in validation data, representing actual rather than predicted accuracy.}% Definition
  {Scalar}% Type
  {\mathbb{R}_{\geq 0}}% Domain
  {Phred units}% Units
  {Ch.~7.2, 11.2}% First defined
  {Ch.~7, 11}% Key uses

\MasterVarBox
  {V-Q-PRED}% ID
  {Q_{\text{pred}}}% Symbol
  {Predicted Quality Score}% Name
  {Quality score predicted by basecaller model, which may overestimate or underestimate true accuracy.}% Definition
  {Scalar}% Type
  {\mathbb{R}_{\geq 0}}% Domain
  {Phred units}% Units
  {Ch.~4.2, 7.3}% First defined
  {Ch.~4, 7, 11}% Key uses

\MasterVarBox
  {V-Q-WORKFLOW}% ID
  {Q_{\text{workflow}}}% Symbol
  {Pipeline QC Score}% Name
  {Quality control score aggregating multiple metrics to assess overall pipeline performance.}% Definition
  {Scalar}% Type
  {\mathbb{R}_{\geq 0}}% Domain
  {---}% Units
  {Ch.~16.1}% First defined
  {Ch.~16}% Key uses

\MasterVarBox
  {V-Q-IJ}% ID
  {Q^{(i)}_j}% Symbol
  {Quality Score Base $j$ Read $i$}% Name
  {Phred quality score for base position $j$ in read $i$.}% Definition
  {Scalar}% Type
  {\mathbb{R}_{\geq 0}}% Domain
  {Phred units}% Units
  {Ch.~4.2}% First defined
  {Ch.~4, 11}% Key uses

\MasterVarBox
  {V-R}% ID
  {R}% Symbol
  {Read Set}% Name
  {Collection of $N$ sequencing reads aligned to target locus, written as $\mathbf{R} = \{r^{(1)}, r^{(2)}, \ldots, r^{(N)}\}$.}% Definition
  {Set}% Type
  {\mathcal{R}^N}% Domain
  {---}% Units
  {Ch.~4.2, 6.3}% First defined
  {Ch.~4, 6, 7, 18}% Key uses

\MasterVarBox
  {V-R-SPLUS}% ID
  {R_{S+}}% Symbol
  {Signal-Positive Read Set}% Name
  {Subset of reads passing signal quality thresholds in SEER workflow.}% Definition
  {Set}% Type
  {\mathcal{R}^*}% Domain
  {---}% Units
  {Ch.~11.3}% First defined
  {Ch.~11}% Key uses

\MasterVarBox
  {V-r}% ID
  {r}% Symbol
  {Read (basecalled sequence)}% Name
  {Basecalled DNA sequence produced by sequencing instrument, potentially containing sequencing errors.}% Definition
  {Sequence}% Type
  {\mathcal{A}^L}% Domain
  {---}% Units
  {Ch.~4.2}% First defined
  {Ch.~4, 6, 11}% Key uses

\MasterVarBox
  {V-r-I}% ID
  {r^{(i)}}% Symbol
  {Read $i$}% Name
  {The $i$-th read in a read set, indexed by $i$.}% Definition
  {Sequence}% Type
  {\mathcal{A}^{L_i}}% Domain
  {---}% Units
  {Ch.~4.2}% First defined
  {Ch.~4, 6, 11, 18}% Key uses

\MasterVarBox
  {V-r-IJ}% ID
  {r^{(i)}_j}% Symbol
  {Base $j$ of Read $i$}% Name
  {Nucleotide at position $j$ in read $i$, taking values in $\{A, C, G, T\}$.}% Definition
  {Nucleotide}% Type
  {\mathcal{A}}% Domain
  {---}% Units
  {Ch.~4.2}% First defined
  {Ch.~4, 11}% Key uses

\MasterVarBox
  {V-r-ERROR}% ID
  {r}% Symbol
  {Per-Base Replication Error Rate}% Name
  {Per-base error rate during PCR amplification or plasmid replication, distinct from sequencing error rate.}% Definition
  {Scalar}% Type
  {[0, 1]}% Domain
  {dimensionless}% Units
  {Ch.~5.2}% First defined
  {Ch.~5, 8}% Key uses

\MasterVarBox
  {V-S}% ID
  {S}% Symbol
  {Signal Segmentation Model}% Name
  {Model for segmenting raw instrument signal into base-level events, used in basecalling.}% Definition
  {Function}% Type
  {---}% Domain
  {---}% Units
  {Ch.~4.2}% First defined
  {Ch.~4, 11}% Key uses

\MasterVarBox
  {V-s}% ID
  {s}% Symbol
  {True Sequence (Ground Truth)}% Name
  {Ground truth DNA sequence for validation, typically from synthetic plasmids or orthogonally validated samples.}% Definition
  {Sequence}% Type
  {\mathcal{A}^L}% Domain
  {---}% Units
  {Ch.~4.2, 5.1}% First defined
  {Ch.~4, 5, 11, 14}% Key uses

\MasterVarBox
  {V-s-I}% ID
  {s^{(i)}}% Symbol
  {True Sequence for Read $i$}% Name
  {Ground truth sequence corresponding to read $i$ before sequencing errors.}% Definition
  {Sequence}% Type
  {\mathcal{A}^{L_i}}% Domain
  {---}% Units
  {Ch.~4.2}% First defined
  {Ch.~4, 11}% Key uses

\MasterVarBox
  {V-s-HAT}% ID
  {\hat{s}}% Symbol
  {Predicted/Assigned Sequence}% Name
  {Sequence assigned by classification algorithm, which may differ from true sequence due to errors.}% Definition
  {Sequence}% Type
  {\mathcal{A}^L}% Domain
  {---}% Units
  {Ch.~4.3}% First defined
  {Ch.~4, 11}% Key uses

\MasterVarBox
  {V-TPR}% ID
  {\mathrm{TPR}}% Symbol
  {True Positive Rate}% Name
  {Fraction of true positives correctly identified, also known as sensitivity or recall.}% Definition
  {Scalar}% Type
  {[0, 1]}% Domain
  {dimensionless}% Units
  {Ch.~4.3, 5.2}% First defined
  {Ch.~4, 5, 11, 14}% Key uses

\MasterVarBox
  {V-SMA}% ID
  {\mathrm{SMA}}% Symbol
  {Single Molecule Accuracy}% Name
  {Probability that all bases in a single read are correctly called, representing read-level accuracy.}% Definition
  {Scalar}% Type
  {[0, 1]}% Domain
  {dimensionless}% Units
  {Ch.~11.2}% First defined
  {Ch.~11, 14}% Key uses

\MasterVarBox
  {V-X}% ID
  {X}% Symbol
  {Raw Signal Time Series}% Name
  {Raw electrical or optical signal trace from sequencing instrument before basecalling.}% Definition
  {Vector}% Type
  {\mathbb{R}^T}% Domain
  {instrument units}% Units
  {Ch.~4.2}% First defined
  {Ch.~4, 11}% Key uses

\MasterVarBox
  {V-x-I}% ID
  {x^{(i)}}% Symbol
  {Signal Segment $i$}% Name
  {Signal segment corresponding to the $i$-th base or event.}% Definition
  {Vector}% Type
  {\mathbb{R}^{\ell_i}}% Domain
  {instrument units}% Units
  {Ch.~4.2}% First defined
  {Ch.~4, 11}% Key uses

%%%%%%%%%%%%%%%%%%%%%%%%%%%%%%%%%%%%%%%%%%%%%%%%%%%%%%%%%%%%%%%%%%%%%%%%
\section{Greek Alphabet}
\label{sec:variables-greek}

\MasterVarBox
  {V-ALPHA}% ID
  {\alpha}% Symbol
  {Significance Level (Type I Error)}% Name
  {Probability of rejecting null hypothesis when it is true, typically set to 0.05 or 0.01.}% Definition
  {Scalar}% Type
  {(0, 1)}% Domain
  {dimensionless}% Units
  {Ch.~7.3, 14.3}% First defined
  {Ch.~7, 14, 17, 18}% Key uses

\MasterVarBox
  {V-ALPHA-I}% ID
  {\alpha_i}% Symbol
  {Dirichlet Concentration Parameter}% Name
  {Concentration parameter for Dirichlet prior on haplotype frequencies.}% Definition
  {Scalar}% Type
  {\mathbb{R}_{> 0}}% Domain
  {---}% Units
  {Ch.~6.2}% First defined
  {Ch.~6}% Key uses

\MasterVarBox
  {V-BETA}% ID
  {\beta}% Symbol
  {Type II Error Probability (1 - Power)}% Name
  {Probability of failing to reject null hypothesis when alternative is true, related to statistical power by $\text{Power} = 1 - \beta$.}% Definition
  {Scalar}% Type
  {(0, 1)}% Domain
  {dimensionless}% Units
  {Ch.~7.3, 14.3}% First defined
  {Ch.~7, 14, 17}% Key uses

\MasterVarBox
  {V-PI-VECTOR}% ID
  {\boldsymbol{\pi}}% Symbol
  {Haplotype Frequency Vector}% Name
  {Vector of population haplotype frequencies forming a probability distribution over the haplotype panel.}% Definition
  {Vector}% Type
  {\Delta^{P-1}}% Domain
  {dimensionless}% Units
  {Ch.~6.2}% First defined
  {Ch.~6}% Key uses

\MasterVarBox
  {V-DELTA}% ID
  {\delta}% Symbol
  {Confidence Parameter (1 - Confidence)}% Name
  {Complement of confidence level, where confidence level $= 1 - \delta$.}% Definition
  {Scalar}% Type
  {(0, 1)}% Domain
  {dimensionless}% Units
  {Ch.~7.4}% First defined
  {Ch.~7}% Key uses

\MasterVarBox
  {V-EPSILON-POST}% ID
  {\epsilon}% Symbol
  {Posterior Threshold (Misclass. Prob.)}% Name
  {Threshold on posterior probability below which classification is rejected as insufficiently confident.}% Definition
  {Scalar}% Type
  {(0, 1)}% Domain
  {dimensionless}% Units
  {Ch.~6.3, 7.4}% First defined
  {Ch.~6, 7, 14}% Key uses

\MasterVarBox
  {V-EPSILON-PREC}% ID
  {\epsilon}% Symbol
  {Precision Half-Width}% Name
  {Half-width of confidence interval, determining precision of parameter estimates.}% Definition
  {Scalar}% Type
  {(0, 1)}% Domain
  {dimensionless}% Units
  {Ch.~14.3}% First defined
  {Ch.~14}% Key uses

\MasterVarBox
  {V-VAREPSILON-I}% ID
  {\varepsilon_i}% Symbol
  {Misclassification Probability}% Name
  {Probability of misclassifying true class $i$, equal to $1 - C_{ii}$ in confusion matrix.}% Definition
  {Scalar}% Type
  {[0, 1]}% Domain
  {dimensionless}% Units
  {Ch.~4.3}% First defined
  {Ch.~4, 11}% Key uses

\MasterVarBox
  {V-THETA}% ID
  {\theta}% Symbol
  {Per-Base Accuracy}% Name
  {Probability of correct base call at a single position, equal to $1 - p$ where $p$ is error rate.}% Definition
  {Scalar}% Type
  {[0, 1]}% Domain
  {dimensionless}% Units
  {Ch.~7.1}% First defined
  {Ch.~7}% Key uses

\MasterVarBox
  {V-THETA-Q}% ID
  {\theta(Q)}% Symbol
  {Quality-Dependent Accuracy}% Name
  {Per-base accuracy as a function of Phred quality score $Q$, typically $\theta(Q) = 1 - 10^{-Q/10}$.}% Definition
  {Function}% Type
  {[0, 1]}% Domain
  {dimensionless}% Units
  {Ch.~7.1}% First defined
  {Ch.~7}% Key uses

\MasterVarBox
  {V-LAMBDA}% ID
  {\lambda}% Symbol
  {Mixture Fraction (Minor Component)}% Name
  {Fraction of minor haplotype in diploid mixture, with major fraction $= 1 - \lambda$.}% Definition
  {Scalar}% Type
  {[0, 1]}% Domain
  {dimensionless}% Units
  {Ch.~6.4, 7.5, 14.1}% First defined
  {Ch.~6, 7, 14}% Key uses

\MasterVarBox
  {V-LAMBDA-PENALTY}% ID
  {\lambda}% Symbol
  {Complexity Penalty Parameter}% Name
  {Regularization parameter controlling model complexity in penalized likelihood or Bayesian inference.}% Definition
  {Scalar}% Type
  {\mathbb{R}_{> 0}}% Domain
  {---}% Units
  {Ch.~6.2, 12.3}% First defined
  {Ch.~6, 12}% Key uses

\MasterVarBox
  {V-LAMBDA-MIN}% ID
  {\lambda_{\text{min}}}% Symbol
  {Minimum Detectable Fraction}% Name
  {Smallest mixture fraction that can be reliably detected given coverage and error rates.}% Definition
  {Scalar}% Type
  {(0, 1)}% Domain
  {dimensionless}% Units
  {Ch.~7.5, 14.3}% First defined
  {Ch.~7, 14}% Key uses

\MasterVarBox
  {V-LAMBDA-HAT}% ID
  {\hat{\lambda}}% Symbol
  {Observed Mixture Proportion}% Name
  {Estimated mixture fraction from read counts or posterior inference.}% Definition
  {Scalar}% Type
  {[0, 1]}% Domain
  {dimensionless}% Units
  {Ch.~14.3}% First defined
  {Ch.~14}% Key uses

\MasterVarBox
  {V-MU}% ID
  {\mu}% Symbol
  {Mean of Distribution}% Name
  {Expected value or population mean of a probability distribution.}% Definition
  {Scalar}% Type
  {\mathbb{R}}% Domain
  {variable}% Units
  {Ch.~5.3, 7.2}% First defined
  {Ch.~5, 7}% Key uses

\MasterVarBox
  {V-PI}% ID
  {\pi}% Symbol
  {Empirical Purity}% Name
  {Fraction of error-free reads in a sequencing dataset, defined as the long-run proportion of perfectly sequenced molecules when read count $N \to \infty$.}% Definition
  {Scalar}% Type
  {[0, 1]}% Domain
  {dimensionless}% Units
  {Ch.~5.1}% First defined
  {Ch.~5, 8, 11, 14}% Key uses

\MasterVarBox
  {V-PI-0}% ID
  {\pi_0}% Symbol
  {True Proportion Under Null}% Name
  {Population proportion assumed under null hypothesis in proportion testing.}% Definition
  {Scalar}% Type
  {[0, 1]}% Domain
  {dimensionless}% Units
  {Ch.~18.2}% First defined
  {Ch.~18}% Key uses

\MasterVarBox
  {V-PI-UPPER}% ID
  {\pi_{\text{upper}}}% Symbol
  {Theoretical Purity Upper Bound}% Name
  {Theoretical upper bound on achievable purity given experimental parameters.}% Definition
  {Function}% Type
  {[0, 1]}% Domain
  {dimensionless}% Units
  {Ch.~5.2}% First defined
  {Ch.~5}% Key uses

\MasterVarBox
  {V-PI-I}% ID
  {\pi_i}% Symbol
  {Prior Probability of Haplotype $i$}% Name
  {Prior probability assigned to haplotype $i$, typically from population frequency data.}% Definition
  {Scalar}% Type
  {[0, 1]}% Domain
  {dimensionless}% Units
  {Ch.~6.2}% First defined
  {Ch.~6}% Key uses

\MasterVarBox
  {V-RHO}% ID
  {\rho}% Symbol
  {Correlation Coefficient}% Name
  {Pearson correlation coefficient measuring linear association between two variables.}% Definition
  {Scalar}% Type
  {[-1, 1]}% Domain
  {dimensionless}% Units
  {Ch.~14.3}% First defined
  {Ch.~14}% Key uses

\MasterVarBox
  {V-SIGMA}% ID
  {\sigma}% Symbol
  {Standard Deviation}% Name
  {Standard deviation of a probability distribution, measuring spread around the mean.}% Definition
  {Scalar}% Type
  {\mathbb{R}_{\geq 0}}% Domain
  {variable}% Units
  {Ch.~7.3, 14.3}% First defined
  {Ch.~7, 14}% Key uses

\MasterVarBox
  {V-SIGMA-SIGNAL}% ID
  {\sigma}% Symbol
  {Instrument Signal}% Name
  {Raw signal trace from sequencing instrument (context-dependent with signal space meaning).}% Definition
  {Vector}% Type
  {\mathbb{R}^T}% Domain
  {instrument units}% Units
  {Ch.~4.2}% First defined
  {Ch.~4}% Key uses

\MasterVarBox
  {V-SIGMA-I}% ID
  {\sigma^{(i)}}% Symbol
  {Signal for Molecule $i$}% Name
  {Signal trace for the $i$-th molecule or read.}% Definition
  {Vector}% Type
  {\mathbb{R}^{\ell_i}}% Domain
  {instrument units}% Units
  {Ch.~4.2}% First defined
  {Ch.~4}% Key uses

\MasterVarBox
  {V-TAU}% ID
  {\tau}% Symbol
  {Threshold (General)}% Name
  {General threshold parameter used in classification or hypothesis testing.}% Definition
  {Scalar}% Type
  {\mathbb{R}}% Domain
  {variable}% Units
  {Ch.~6.3, 11.3}% First defined
  {Ch.~6, 11}% Key uses

\MasterVarBox
  {V-PHI}% ID
  {\Phi}% Symbol
  {Standard Normal CDF}% Name
  {Cumulative distribution function of standard normal distribution.}% Definition
  {Function}% Type
  {[0, 1]}% Domain
  {dimensionless}% Units
  {Ch.~7.3}% First defined
  {Ch.~7}% Key uses

\MasterVarBox
  {V-CHI-SQ}% ID
  {\chi^2}% Symbol
  {Chi-Square Statistic}% Name
  {Test statistic for chi-square goodness-of-fit or independence tests.}% Definition
  {Scalar}% Type
  {\mathbb{R}_{\geq 0}}% Domain
  {---}% Units
  {Ch.~14.4}% First defined
  {Ch.~14}% Key uses

%%%%%%%%%%%%%%%%%%%%%%%%%%%%%%%%%%%%%%%%%%%%%%%%%%%%%%%%%%%%%%%%%%%%%%%%
\section{Special Mathematical Symbols}
\label{sec:variables-special}

\MasterVarBox
  {V-E-X}% ID
  {\mathbb{E}[X]}% Symbol
  {Expectation of $X$}% Name
  {Expected value of random variable $X$, integrating over its distribution.}% Definition
  {Operator}% Type
  {\mathbb{R}}% Domain
  {variable}% Units
  {Ch.~5.3, 7.2}% First defined
  {Ch.~5, 7, 14}% Key uses

\MasterVarBox
  {V-IND-A}% ID
  {\mathbbm{1}\{A\}}% Symbol
  {Indicator Function of Event $A$}% Name
  {Function equal to 1 if event $A$ occurs and 0 otherwise.}% Definition
  {Function}% Type
  {\{0, 1\}}% Domain
  {---}% Units
  {Ch.~11.2, 17.2}% First defined
  {Ch.~11, 17}% Key uses

\MasterVarBox
  {V-ALPHABET}% ID
  {\mathcal{A}}% Symbol
  {Nucleotide Alphabet}% Name
  {Set of nucleotide symbols, typically $\{A, C, G, T\}$ for DNA.}% Definition
  {Set}% Type
  {finite}% Domain
  {---}% Units
  {Ch.~4.1}% First defined
  {All chapters}% Key uses

\MasterVarBox
  {V-A-STAR}% ID
  {\mathcal{A}^*}% Symbol
  {Set of All Sequences}% Name
  {Kleene closure of nucleotide alphabet, representing sequences of any length.}% Definition
  {Set}% Type
  {infinite}% Domain
  {---}% Units
  {Ch.~4.1}% First defined
  {Ch.~4, 6}% Key uses

\MasterVarBox
  {V-A-L}% ID
  {\mathcal{A}^L}% Symbol
  {Sequences of Length $L$}% Name
  {Set of all nucleotide sequences of fixed length $L$.}% Definition
  {Set}% Type
  {finite}% Domain
  {---}% Units
  {Ch.~4.1}% First defined
  {Ch.~4, 6}% Key uses

\MasterVarBox
  {V-D-I}% ID
  {\mathcal{D}_i}% Symbol
  {Fragment Set for Haplotype $i$}% Name
  {Set of DNA fragments derived from haplotype $i$ after fragmentation.}% Definition
  {Set}% Type
  {---}% Domain
  {---}% Units
  {Ch.~4.1}% First defined
  {Ch.~4}% Key uses

\MasterVarBox
  {V-G-I}% ID
  {\mathcal{G}_i}% Symbol
  {Genomic Molecule Set}% Name
  {Set of genomic DNA molecules for sample $i$ prior to library preparation.}% Definition
  {Set}% Type
  {---}% Domain
  {---}% Units
  {Ch.~4.1}% First defined
  {Ch.~4}% Key uses

\MasterVarBox
  {V-H-SET}% ID
  {\mathcal{H}}% Symbol
  {Haplotype Panel}% Name
  {Complete set of reference haplotypes used for classification (e.g., PharmVar panel).}% Definition
  {Set}% Type
  {finite}% Domain
  {---}% Units
  {Ch.~4.1, 6.1}% First defined
  {Ch.~4, 6, 17, 18}% Key uses

\MasterVarBox
  {V-L-I-SET}% ID
  {\mathcal{L}_i}% Symbol
  {Library Molecule Set}% Name
  {Set of library molecules after amplification and adapter ligation.}% Definition
  {Set}% Type
  {---}% Domain
  {---}% Units
  {Ch.~4.1}% First defined
  {Ch.~4, 8}% Key uses

\MasterVarBox
  {V-NORMAL}% ID
  {\mathcal{N}(\mu, \sigma^2)}% Symbol
  {Normal Distribution}% Name
  {Gaussian distribution with mean $\mu$ and variance $\sigma^2$.}% Definition
  {Distribution}% Type
  {\mathbb{R}}% Domain
  {---}% Units
  {Ch.~7.3, 14.3}% First defined
  {Ch.~7, 14}% Key uses

\MasterVarBox
  {V-BIG-O}% ID
  {\mathcal{O}(n)}% Symbol
  {Big-O Notation (Asymptotic)}% Name
  {Asymptotic upper bound on computational complexity or growth rate.}% Definition
  {Notation}% Type
  {---}% Domain
  {---}% Units
  {Ch.~6.3, 14.3}% First defined
  {Ch.~6, 14}% Key uses

\MasterVarBox
  {V-R-SPACE}% ID
  {\mathcal{R}}% Symbol
  {Read Space}% Name
  {Space of all possible basecalled reads, including those with errors.}% Definition
  {Set}% Type
  {infinite}% Domain
  {---}% Units
  {Ch.~4.2}% First defined
  {Ch.~4, 6}% Key uses

\MasterVarBox
  {V-S-SPACE}% ID
  {\mathcal{S}}% Symbol
  {Signal Space}% Name
  {Space of raw instrument signals before basecalling.}% Definition
  {Set}% Type
  {infinite}% Domain
  {---}% Units
  {Ch.~4.2}% First defined
  {Ch.~4}% Key uses

\MasterVarBox
  {V-U-I}% ID
  {\mathcal{U}_i}% Symbol
  {Mutated Sequence Set}% Name
  {Set of sequences resulting from replication errors or PCR mutations.}% Definition
  {Set}% Type
  {---}% Domain
  {---}% Units
  {Ch.~4.1}% First defined
  {Ch.~4}% Key uses

\MasterVarBox
  {V-PROB-A}% ID
  {\Prob(A)}% Symbol
  {Probability of Event $A$}% Name
  {Probability measure of event $A$ under the probability space $(\Omega, \mathcal{F}, P)$.}% Definition
  {Function}% Type
  {[0, 1]}% Domain
  {dimensionless}% Units
  {Ch.~4.1}% First defined
  {All chapters}% Key uses

\MasterVarBox
  {V-PROB-A-B}% ID
  {\Prob(A \mid B)}% Symbol
  {Conditional Probability}% Name
  {Probability of event $A$ given that event $B$ has occurred.}% Definition
  {Function}% Type
  {[0, 1]}% Domain
  {dimensionless}% Units
  {Ch.~4.1, 6.1}% First defined
  {All chapters}% Key uses

\MasterVarBox
  {V-NORM-2}% ID
  {\|\mathbf{x}\|_2}% Symbol
  {Euclidean Norm}% Name
  {Euclidean (L2) norm of vector $\mathbf{x}$, equal to $\sqrt{\sum_i x_i^2}$.}% Definition
  {Function}% Type
  {\mathbb{R}_{\geq 0}}% Domain
  {variable}% Units
  {Ch.~14.3}% First defined
  {Ch.~14}% Key uses

\MasterVarBox
  {V-NORM-F}% ID
  {\|\mathbf{M}\|_F}% Symbol
  {Frobenius Norm}% Name
  {Frobenius norm of matrix $\mathbf{M}$, equal to $\sqrt{\sum_{i,j} M_{ij}^2}$.}% Definition
  {Function}% Type
  {\mathbb{R}_{\geq 0}}% Domain
  {variable}% Units
  {Ch.~14.2}% First defined
  {Ch.~14}% Key uses

\MasterVarBox
  {V-z-ALPHA}% ID
  {z_{\alpha}}% Symbol
  {Standard Normal Quantile}% Name
  {Quantile of standard normal distribution at probability level $\alpha$.}% Definition
  {Scalar}% Type
  {\mathbb{R}}% Domain
  {---}% Units
  {Ch.~7.3, 14.3}% First defined
  {Ch.~7, 14, 17, 18}% Key uses

\MasterVarBox
  {V-z-1-ALPHA-2}% ID
  {z_{1-\alpha/2}}% Symbol
  {Two-Tailed Normal Quantile}% Name
  {Critical value for two-tailed hypothesis test at significance level $\alpha$.}% Definition
  {Scalar}% Type
  {\mathbb{R}}% Domain
  {---}% Units
  {Ch.~7.3, 14.3}% First defined
  {Ch.~7, 14, 17}% Key uses

\MasterVarBox
  {V-z-1-BETA}% ID
  {z_{1-\beta}}% Symbol
  {Power-Related Quantile}% Name
  {Standard normal quantile related to power $(1-\beta)$ in sample size calculations.}% Definition
  {Scalar}% Type
  {\mathbb{R}}% Domain
  {---}% Units
  {Ch.~7.3, 14.3}% First defined
  {Ch.~7, 14, 17}% Key uses

\MasterVarBox
  {V-z-1-DELTA}% ID
  {z_{1-\delta}}% Symbol
  {Confidence Quantile}% Name
  {Standard normal quantile for confidence level $1-\delta$.}% Definition
  {Scalar}% Type
  {\mathbb{R}}% Domain
  {---}% Units
  {Ch.~7.4}% First defined
  {Ch.~7}% Key uses

%%%%%%%%%%%%%%%%%%%%%%%%%%%%%%%%%%%%%%%%%%%%%%%%%%%%%%%%%%%%%%%%%%%%%%%%
\section{Domain-Specific Variables}
\label{sec:variables-domain-specific}

\subsection{CYP2D6 Pharmacogenomics (Chapters 17--18)}

\textbf{Note:} CYP2D6 allele nomenclature uses star (*) notation and is categorical rather than mathematical. Representative examples are provided below as boxed entries.

\MasterVarBox
  {V-ALLELE-1}% ID
  {*1}% Symbol
  {CYP2D6 *1 Allele (Reference/Wild-Type)}% Name
  {Reference allele with normal enzyme function, serving as the baseline for activity scoring.}% Definition
  {Categorical}% Type
  {PharmVar}% Domain
  {---}% Units
  {Ch.~17, 18}% First defined
  {Ch.~17, 18}% Key uses

\MasterVarBox
  {V-ALLELE-DUP}% ID
  {*1\times2}% Symbol
  {Gene Duplication (2 copies of *1)}% Name
  {Copy number variant with two functional copies of *1 allele, associated with ultrarapid metabolizer phenotype.}% Definition
  {Categorical}% Type
  {PharmVar}% Domain
  {---}% Units
  {Ch.~17, 18}% First defined
  {Ch.~17, 18}% Key uses

\MasterVarBox
  {V-ALLELE-FUSION}% ID
  {*36+*10}% Symbol
  {Tandem Fusion Allele}% Name
  {Hybrid allele resulting from recombination between *36 and *10, common in Asian populations.}% Definition
  {Categorical}% Type
  {PharmVar}% Domain
  {---}% Units
  {Ch.~17, 18}% First defined
  {Ch.~17, 18}% Key uses

\MasterVarBox
  {V-ALLELE-5}% ID
  {*5}% Symbol
  {Gene Deletion (Null Allele)}% Name
  {Complete gene deletion resulting in no enzyme activity.}% Definition
  {Categorical}% Type
  {PharmVar}% Domain
  {---}% Units
  {Ch.~17, 18}% First defined
  {Ch.~17, 18}% Key uses

\MasterVarBox
  {V-FREQ-36-10}% ID
  {\widehat{f}(*36+*10)}% Symbol
  {Observed Allele Frequency}% Name
  {Empirical frequency of *36+*10 fusion allele in study cohort.}% Definition
  {Scalar}% Type
  {[0, 1]}% Domain
  {dimensionless}% Units
  {Ch.~18.2}% First defined
  {Ch.~18}% Key uses

\MasterVarBox
  {V-PM}% ID
  {\text{PM}}% Symbol
  {Poor Metabolizer Phenotype}% Name
  {Phenotype category with no or severely reduced enzyme activity (AS $\approx$ 0).}% Definition
  {Categorical}% Type
  {CPIC}% Domain
  {---}% Units
  {Ch.~17, 18}% First defined
  {Ch.~17, 18}% Key uses

\MasterVarBox
  {V-IM}% ID
  {\text{IM}}% Symbol
  {Intermediate Metabolizer}% Name
  {Phenotype category with reduced enzyme activity (AS $\approx$ 0.5--1.0).}% Definition
  {Categorical}% Type
  {CPIC}% Domain
  {---}% Units
  {Ch.~17, 18}% First defined
  {Ch.~17, 18}% Key uses

\MasterVarBox
  {V-NM}% ID
  {\text{NM}}% Symbol
  {Normal Metabolizer}% Name
  {Phenotype category with normal enzyme activity (AS $\approx$ 1.0--2.0).}% Definition
  {Categorical}% Type
  {CPIC}% Domain
  {---}% Units
  {Ch.~17, 18}% First defined
  {Ch.~17, 18}% Key uses

\MasterVarBox
  {V-UM}% ID
  {\text{UM}}% Symbol
  {Ultrarapid Metabolizer}% Name
  {Phenotype category with increased enzyme activity due to gene duplication (AS $>$ 2.0).}% Definition
  {Categorical}% Type
  {CPIC}% Domain
  {---}% Units
  {Ch.~17, 18}% First defined
  {Ch.~17, 18}% Key uses

%%%%%%%%%%%%%%%%%%%%%%%%%%%%%%%%%%%%%%%%%%%%%%%%%%%%%%%%%%%%%%%%%%%%%%%%
\section{Cross-Reference Index}
\label{sec:variables-xref}

This section provides a reverse index: for each chapter, which variables are introduced or play a central role.

\subsection{Variables by Chapter}

\textbf{Chapter 4 (Haplotype Classification Model):} $\mathcal{H}$, $h$, $h_i$, $g$, $d$, $\ell$, $\sigma$, $r$, $s$, $\mathcal{A}$, $L$, $C_{ij}$, $\mathbf{C}$, $d_{\text{edit}}$, $Q$, $p$, TPR, SMA

\textbf{Chapter 5 (Purity Theory):} $\pi$, $\pi_{\text{upper}}$, $k$, $r$, $N_{\text{correct}}$, $N_{\text{total}}$, TPR

\textbf{Chapter 6 (Posterior Computation):} $\Prob(h_i \mid R)$, $\Prob(R \mid h_i)$, $\Prob(h_i)$, $\pi_i$, $f_{\text{pop}}$, $D$, $\lambda$, $\epsilon$

\textbf{Chapter 7 (Experimental Design):} $N$, $N_{\text{required}}$, $N_{\text{eff}}$, $N_{\text{perfect}}$, $L$, $\theta$, $Q_{\text{pred}}$, $Q_{\text{emp}}$, $d$, $\alpha$, $\beta$, $\delta$, $D_{\text{KL}}$, $P$, $k$, $\lambda_{\text{min}}$, $z_{\alpha}$, $p_0$, $p_1$

\textbf{Chapter 11 (Basecaller Quality Models):} $Q$, $Q_{\text{pred}}$, $Q_{\text{emp}}$, $\bar{Q}$, $p$, $d_{\text{edit}}$, $\mathbf{C}$, $M$, $M_{ab}$, SMA, SEER, $R_{S+}$, $\pi$

\textbf{Chapter 14 (Mixtures and Diplotype Accuracy):} $\lambda$, $\hat{\lambda}$, $D$, $h_{\text{major}}$, $h_{\text{minor}}$, $N_{\text{correct}}$, $N_{\text{total}}$, $\epsilon$, $\alpha$, $\beta$, $k_{\alpha}$, $\chi^2$, $\sigma$, $D_{\text{KL}}$

\textbf{Chapter 17 (CYP2D6 Pain/Psychiatry):} $E_{\text{I}}$, $E_{\text{II}}$, $E_{\text{pheno}}$, $E_{\text{clinical}}$, AS, $\text{AS}_{\text{geno}}$, $\text{AS}_{\text{eff}}$, $I$, $D_{\text{reported}}$, $D_{\text{true}}$, $P_{\text{ADR}}$, $P_{\text{failure}}$, $C_{\text{test}}$, $C_{\text{total}}$, $z_{\alpha}$, $\alpha$

\textbf{Chapter 18 (Singapore Tamoxifen Cohort):} $D$, $D_{\text{true}}$, $D_{\text{reported}}$, $P(D_k \mid \mathbf{R})$, $\widehat{f}(*36+*10)$, $\pi_0$, $p$, $H_0$, $H_1$, $\alpha$, AS

%%%%%%%%%%%%%%%%%%%%%%%%%%%%%%%%%%%%%%%%%%%%%%%%%%%%%%%%%%%%%%%%%%%%%%%%
\section{Usage Notes and Best Practices}
\label{sec:variables-usage}

\subsection{Avoiding Notation Conflicts}

\begin{enumerate}
\item \textbf{Context Sensitivity:} Some symbols have different meanings in different contexts:
\begin{itemize}
\item $L$: Sequence length (most chapters) vs. molecule length $L_{\text{mol}}$ (Ch.~5)
\item $\lambda$: Mixture fraction (Ch.~6, 14) vs. complexity penalty (Ch.~12)
\item $d$: DNA fragment (Ch.~4) vs. Phred overstatement (Ch.~7)
\item $\sigma$: Signal vector (Ch.~4) vs. standard deviation (Ch.~7, 14)
\end{itemize}
Always check the local context and chapter reference.

\item \textbf{Subscript Disambiguation:}
\begin{itemize}
\item $Q_{\text{pred}}$ vs. $Q_{\text{emp}}$: Always use text subscripts for predicted/empirical
\item $N_{\text{total}}$ vs. $N_{\text{correct}}$ vs. $N_i$: Use text for semantic meaning, numeric for index
\end{itemize}

\item \textbf{Probability Macro:} \textbf{Always use} \verb|\Prob(...)| for probability, never raw $P(...)$ or $\mathbb{P}(...)$
\end{enumerate}

\subsection{Variable Naming Conventions}

\begin{itemize}
\item \textbf{Observed/Estimated:} Use hat notation $\hat{\theta}$ for point estimates
\item \textbf{True/Ground Truth:} Use subscript "true" for ground truth: $h_{\text{true}}$, $D_{\text{true}}$
\item \textbf{Predicted:} Use subscript "pred": $Q_{\text{pred}}$
\item \textbf{Empirical:} Use subscript "emp": $Q_{\text{emp}}$, $\bar{p}^{(i)}_{\text{emp}}$
\end{itemize}

\subsection{Cross-Referencing This Table}

To cite this table in text:
\begin{verbatim}
See Appendix~\ref{app:variable-master},
Section~\ref{sec:variables-latin-upper} for variable definitions.
\end{verbatim}

For specific variable lookup, use the alphabetical organization in Sections~\ref{sec:variables-latin-upper} (Latin), \ref{sec:variables-greek} (Greek), and \ref{sec:variables-special} (special symbols).


% Appendix H: Master Equation Reference Table (NEW - v6.2)
% Comprehensive catalog of all numbered equations with cross-references
\chapter{Master Equation Reference Table}
\label{app:equation-master}

\section*{Overview}

This appendix provides a comprehensive catalog of all numbered equations in the SMS Haplotype Classification Framework. Equations are presented as boxed reference cards organized by functional category to enable systematic navigation across chapters. Use the equation labels (e.g., \texttt{eq:posterior-bayes}) with \verb|\ref{}| or \verb|\eqref{}| for citations.

\subsection*{Cross-Reference Codes}

Each equation is assigned a unique label following the pattern \texttt{eq:descriptive-name}. Core equations (CE\#1--15) use labels \texttt{eq:ce1} through \texttt{eq:ce14} and are cataloged in Appendix~B.

\subsection*{Organization}

Equations are grouped into seven functional categories:
\begin{enumerate}
\item \textbf{Core Equations (CE\#1--14):} Foundational mathematical models (Appendix~B)
\item \textbf{Pipeline Foundations (Ch.~4--6):} Bayesian haplotype classification framework
\item \textbf{Experimental Design (Ch.~7):} Coverage, sample size, power calculations
\item \textbf{Reference Standards (Ch.~8--9):} Plasmid purity, enrichment efficiency
\item \textbf{Computational Methods (Ch.~10--13):} Quality models, noisy labels, fine-tuning
\item \textbf{Validation \& Quality Control (Ch.~14, Appendices):} Mixture validation, QC gates
\item \textbf{Clinical Applications (Ch.~17--18):} Pharmacogenomic outcomes, diplotype resolution
\end{enumerate}

\clearpage

%==============================================================================
\section{Core Equations (Appendix B)}
\label{sec:eqmaster-core}
%==============================================================================

\MasterEqBox
  {CE-01}% ID
  {eq:ce1}% Label
  {Haplotype Posterior Probability}% Name
  {\Prob(h | \mathbf{r}) = \frac{\Prob(\mathbf{r} | h) \Prob(h)}{\Prob(\mathbf{r})}}% Equation
  {Defines the posterior probability of haplotype $h$ given observed reads $\mathbf{r}$ as the normalized product of likelihood and prior, fundamental to Bayesian haplotype classification.}% Definition/Role
  {Core Equation}% Category
  {App.~B.1}% First defined
  {$h$, $\mathbf{r}$, $\Prob(\mathbf{r}\mid h)$, $\Prob(h)$, $\Prob(\mathbf{r})$}% Key variables

\MasterEqBox
  {CE-02}% ID
  {eq:ce2}% Label
  {Per-Base Likelihood Model}% Name
  {\Prob(r | h) = \begin{cases} 1 - \epsilon_b & \text{if } r = h[j] \\ \epsilon_b/3 & \text{otherwise} \end{cases}}% Equation
  {Symmetric error model for per-base sequencing likelihood with uniform mismatch probabilities, where $\epsilon_b$ is the per-base error rate.}% Definition/Role
  {Core Equation}% Category
  {App.~B.2}% First defined
  {$r$, $h$, $\epsilon_b$}% Key variables

\MasterEqBox
  {CE-03}% ID
  {eq:ce3}% Label
  {Empirical Purity}% Name
  {\pi = \frac{\mathbb{E}[\text{perfect reads}]}{N}}% Equation
  {Defines empirical purity as the expected fraction of error-free reads in a dataset of size $N$, central to understanding classification performance ceilings.}% Definition/Role
  {Core Equation}% Category
  {App.~B.3}% First defined
  {$\pi$, $N$}% Key variables

\MasterEqBox
  {CE-04}% ID
  {eq:ce4}% Label
  {Purity Ceiling (TPR Bound)}% Name
  {\text{TPR}(h_i) \leq \pi}% Equation
  {Establishes that true positive rate for any haplotype cannot exceed empirical purity, creating fundamental limit on classification accuracy.}% Definition/Role
  {Core Equation}% Category
  {App.~B.4}% First defined
  {$\text{TPR}$, $\pi$, $h_i$}% Key variables

\MasterEqBox
  {CE-05}% ID
  {eq:ce5}% Label
  {Predicted Quality Score}% Name
  {Q_{\text{pred}} = -10 \log_{10} \mathbb{E}[\epsilon]}% Equation
  {Defines predicted Phred quality score as log-scaled expectation of per-base error rate $\epsilon$, as implied by basecaller's probabilistic model.}% Definition/Role
  {Core Equation}% Category
  {App.~B.5}% First defined
  {$Q_{\text{pred}}$, $\epsilon$}% Key variables

\MasterEqBox
  {CE-06}% ID
  {eq:ce6}% Label
  {Empirical Quality Score}% Name
  {Q_{\text{emp}} = -10 \log_{10} \widehat{\epsilon}}% Equation
  {Defines empirical quality score from observed error rate $\widehat{\epsilon}$ in validation data, enabling calibration assessment.}% Definition/Role
  {Core Equation}% Category
  {App.~B.6}% First defined
  {$Q_{\text{emp}}$, $\widehat{\epsilon}$}% Key variables

\MasterEqBox
  {CE-07}% ID
  {eq:ce7}% Label
  {Single Molecule Accuracy (Definition)}% Name
  {\text{SMA} = \Prob(\text{all bases correct})}% Equation
  {Defines single-molecule accuracy as the probability that every base in a read is correctly called, representing read-level correctness rather than base-level accuracy.}% Definition/Role
  {Core Equation}% Category
  {App.~B.7}% First defined
  {$\text{SMA}$, $L$}% Key variables

\MasterEqBox
  {CE-08}% ID
  {eq:ce8}% Label
  {SMA Factorization}% Name
  {\text{SMA} = \prod_{j=1}^{L} (1 - \epsilon_j)}% Equation
  {Factorizes single-molecule accuracy over base positions, assuming independence of base-calling errors across positions.}% Definition/Role
  {Core Equation}% Category
  {App.~B.8}% First defined
  {$\text{SMA}$, $L$, $\epsilon_j$}% Key variables

\MasterEqBox
  {CE-09}% ID
  {eq:ce9}% Label
  {Effective Coverage}% Name
  {N_{\text{eff}} = N \cdot \mathbb{E}[\text{informative fraction}]}% Equation
  {Defines effective coverage as total read count $N$ weighted by expected fraction of reads spanning discriminating positions, accounting for fragment length distribution.}% Definition/Role
  {Core Equation}% Category
  {App.~B.9}% First defined
  {$N_{\text{eff}}$, $N$}% Key variables

\MasterEqBox
  {CE-10}% ID
  {eq:ce10}% Label
  {Calibration Gap}% Name
  {\Delta Q = Q_{\text{emp}} - Q_{\text{pred}}}% Equation
  {Quantifies quality score calibration gap, with $\Delta Q < 0$ indicating basecaller overestimation of accuracy.}% Definition/Role
  {Core Equation}% Category
  {App.~B.10}% First defined
  {$\Delta Q$, $Q_{\text{emp}}$, $Q_{\text{pred}}$}% Key variables

\MasterEqBox
  {CE-12}% ID
  {eq:ce12}% Label
  {Mixture Proportion (Heterozygote)}% Name
  {\lambda = \frac{N_A}{N_A + N_B}}% Equation
  {Defines mixture proportion for diploid heterozygote where $N_A$ and $N_B$ are read counts supporting each haplotype, with theoretical value $\lambda = 0.5$ for balanced diploid.}% Definition/Role
  {Core Equation}% Category
  {App.~B.12}% First defined
  {$\lambda$, $N_A$, $N_B$}% Key variables

\MasterEqBox
  {CE-13}% ID
  {eq:ce13}% Label
  {Sequencing QC Gate}% Name
  {\text{SMA}_{\text{seq}} \geq \text{SMA}_{\text{min}}}% Equation
  {Defines sequencing quality control gate requiring single-molecule accuracy to exceed minimum threshold for data acceptance.}% Definition/Role
  {Core Equation}% Category
  {App.~B.13}% First defined
  {$\text{SMA}_{\text{seq}}$, $\text{SMA}_{\text{min}}$}% Key variables

\MasterEqBox
  {CE-14}% ID
  {eq:ce14}% Label
  {Binary Confusion Matrix}% Name
  {\mathbf{C} = \begin{bmatrix} \text{TP} & \text{FP} \\ \text{FN} & \text{TN} \end{bmatrix}}% Equation
  {Standard 2×2 confusion matrix for binary classification with true positives (TP), false positives (FP), false negatives (FN), and true negatives (TN).}% Definition/Role
  {Core Equation}% Category
  {App.~B.14}% First defined
  {$\mathbf{C}$, TP, FP, FN, TN}% Key variables

%==============================================================================
\section{Pipeline Foundations (Chapters 4--6)}
\label{sec:eqmaster-pipeline}
%==============================================================================

\MasterEqBox
  {EQ-PIPELINE-FACTOR}% ID
  {eq:pipeline-factorization-sma}% Label
  {Pipeline Factorization}% Name
  {P_{\text{correct}} = P_{\text{extraction}} \cdot P_{\text{sequencing}} \cdot P_{\text{classification}}}% Equation
  {Factorizes overall probability of correct haplotype assignment into independent extraction, sequencing, and classification stages.}% Definition/Role
  {Pipeline Foundations}% Category
  {Ch.~4.2}% First defined
  {$P_{\text{correct}}$, $P_{\text{extraction}}$, $P_{\text{sequencing}}$, $P_{\text{classification}}$}% Key variables

\MasterEqBox
  {EQ-PURITY-DEF}% ID
  {eq:purity-def}% Label
  {Asymptotic Purity Definition}% Name
  {\pi = \lim_{N \to \infty} \frac{1}{N} \sum_{n=1}^{N} \mathbbm{1}\{r_n \text{ error-free}\}}% Equation
  {Formal asymptotic definition of empirical purity as the long-run fraction of error-free reads when sample size approaches infinity.}% Definition/Role
  {Pipeline Foundations}% Category
  {Ch.~5.1}% First defined
  {$\pi$, $N$, $r_n$}% Key variables

\MasterEqBox
  {EQ-TPR-CEILING}% ID
  {eq:tpr-ceiling}% Label
  {True Positive Rate Ceiling}% Name
  {\text{TPR}(h_i) \leq \pi}% Equation
  {Restates purity bound: no haplotype can achieve TPR exceeding dataset purity, establishing fundamental performance limit.}% Definition/Role
  {Pipeline Foundations}% Category
  {Ch.~5.2}% First defined
  {$\text{TPR}$, $h_i$, $\pi$}% Key variables

\MasterEqBox
  {EQ-PURITY-BRACKET}% ID
  {eq:purity-bracket}% Label
  {Purity Bracketing Inequality}% Name
  {(1 - \epsilon)^L \leq \pi \leq e^{-L\epsilon}}% Equation
  {Provides tight lower and upper bounds on purity as function of read length $L$ and average error rate $\epsilon$, with exponential upper bound tighter for small $\epsilon$.}% Definition/Role
  {Pipeline Foundations}% Category
  {Ch.~5.4}% First defined
  {$\pi$, $L$, $\epsilon$}% Key variables

\MasterEqBox
  {EQ-LIKELIHOOD-FACTOR}% ID
  {eq:likelihood-factorization}% Label
  {Likelihood Factorization (Independence)}% Name
  {\Prob(\mathbf{r} | h) = \prod_{n=1}^{N} \Prob(r_n | h)}% Equation
  {Factorizes dataset likelihood over independent reads, enabling efficient log-likelihood computation.}% Definition/Role
  {Pipeline Foundations}% Category
  {Ch.~6.2}% First defined
  {$\mathbf{r}$, $h$, $N$, $r_n$}% Key variables

\MasterEqBox
  {EQ-PER-READ-LIKE}% ID
  {eq:per-read-likelihood}% Label
  {Per-Read Likelihood (Base Factorization)}% Name
  {\Prob(r | h) = \prod_{j=1}^{L} \Prob(r[j] | h[j])}% Equation
  {Further factorizes per-read likelihood over base positions, assuming independence of base-calling errors within a read.}% Definition/Role
  {Pipeline Foundations}% Category
  {Ch.~6.2}% First defined
  {$r$, $h$, $L$}% Key variables

\MasterEqBox
  {EQ-POSTERIOR-BAYES}% ID
  {eq:posterior-bayes}% Label
  {Normalized Bayesian Posterior}% Name
  {\Prob(h | \mathbf{r}) = \frac{\Prob(\mathbf{r} | h) \Prob(h)}{\sum_{h'} \Prob(\mathbf{r} | h') \Prob(h')}}% Equation
  {Computes posterior probability of haplotype $h$ given reads $\mathbf{r}$, normalizing over entire haplotype panel $H$ to ensure probabilities sum to unity.}% Definition/Role
  {Pipeline Foundations}% Category
  {Ch.~6.3}% First defined
  {$h$, $H$, $\mathbf{r}$, $\Prob(\mathbf{r}\mid h)$, $\Prob(h)$}% Key variables

\MasterEqBox
  {EQ-MAP}% ID
  {eq:map-classification}% Label
  {Maximum A Posteriori (MAP) Rule}% Name
  {\widehat{h}_{\text{MAP}} = \arg\max_{h \in \mathcal{H}} \Prob(h | \mathbf{r})}% Equation
  {Selects haplotype with maximum posterior probability as the classification decision, minimizing expected misclassification rate under 0-1 loss.}% Definition/Role
  {Pipeline Foundations}% Category
  {Ch.~6.4}% First defined
  {$\widehat{h}_{\text{MAP}}$, $\mathcal{H}$, $\mathbf{r}$}% Key variables

\MasterEqBox
  {EQ-BAYES-FACTOR}% ID
  {eq:bayes-factor}% Label
  {Bayes Factor (Likelihood Ratio)}% Name
  {\text{BF}_{12} = \frac{\Prob(\mathbf{r} | h_1)}{\Prob(\mathbf{r} | h_2)}}% Equation
  {Quantifies relative evidence for haplotype $h_1$ versus $h_2$ independent of priors, with $\text{BF}_{12} > 1$ favoring $h_1$.}% Definition/Role
  {Pipeline Foundations}% Category
  {Ch.~6.5}% First defined
  {$\text{BF}_{12}$, $h_1$, $h_2$, $\mathbf{r}$}% Key variables

%==============================================================================
\section{Experimental Design (Chapter 7)}
\label{sec:eqmaster-design}
%==============================================================================

\MasterEqBox
  {EQ-PERFECT-READ}% ID
  {eq:perfect-read}% Label
  {Error-Free Read Probability}% Name
  {p_{\text{perfect}} = (1 - \epsilon)^L}% Equation
  {Probability that a read of length $L$ has no errors under independent per-base error model with rate $\epsilon$.}% Definition/Role
  {Experimental Design}% Category
  {Ch.~7.2}% First defined
  {$p_{\text{perfect}}$, $L$, $\epsilon$}% Key variables

\MasterEqBox
  {EQ-EXPECTED-PERFECT}% ID
  {eq:expected-perfect}% Label
  {Expected Perfect Read Count}% Name
  {\mathbb{E}[N_{\text{perfect}}] = N \cdot (1 - \epsilon)^L}% Equation
  {Expected number of error-free reads in dataset of size $N$ with per-base error rate $\epsilon$ and read length $L$.}% Definition/Role
  {Experimental Design}% Category
  {Ch.~7.2}% First defined
  {$N_{\text{perfect}}$, $N$, $L$, $\epsilon$}% Key variables

\MasterEqBox
  {EQ-EXPECTED-ERRORS}% ID
  {eq:expected-errors}% Label
  {Expected Errors Per Read}% Name
  {\mathbb{E}[\text{errors per read}] = L \cdot \epsilon}% Equation
  {Expected number of base errors in a single read under independent error model, linear in both read length and error rate.}% Definition/Role
  {Experimental Design}% Category
  {Ch.~7.2}% First defined
  {$L$, $\epsilon$}% Key variables

\MasterEqBox
  {EQ-SAMPLE-SIZE-QUALITY}% ID
  {eq:sample-size-quality}% Label
  {Sample Size for Error Rate Estimation}% Name
  {N \geq \frac{z_{\alpha/2}^2 \cdot \epsilon(1 - \epsilon)}{\delta^2}}% Equation
  {Minimum sample size needed to estimate per-base error rate $\epsilon$ with precision half-width $\delta$ at significance level $\alpha$, based on normal approximation to binomial.}% Definition/Role
  {Experimental Design}% Category
  {Ch.~7.3}% First defined
  {$N$, $z_{\alpha/2}$, $\epsilon$, $\delta$}% Key variables

\MasterEqBox
  {EQ-EXACT-BINOM-PVALUE}% ID
  {eq:exact-binomial-pvalue}% Label
  {Exact Binomial P-Value}% Name
  {p = \sum_{k=k_{\text{obs}}}^{n} \binom{n}{k} \epsilon_0^k (1 - \epsilon_0)^{n-k}}% Equation
  {Exact $p$-value for one-sided binomial test of null hypothesis $H_0: \epsilon = \epsilon_0$ given observed $k_{\text{obs}}$ errors in $n$ bases.}% Definition/Role
  {Experimental Design}% Category
  {Ch.~7.3}% First defined
  {$p$, $k_{\text{obs}}$, $n$, $\epsilon_0$}% Key variables

\MasterEqBox
  {EQ-COVERAGE-CONFIDENCE}% ID
  {eq:coverage-confidence}% Label
  {Minimum Coverage for Confidence}% Name
  {N \geq \frac{\log(\epsilon_0/\epsilon_1) + z_{\alpha}\sqrt{2\log(\epsilon_1/\epsilon_0)}}{D_{\text{KL}}(h_{\text{true}} || h_{\text{closest}})}}% Equation
  {Minimum read count required to distinguish true haplotype from closest competitor with confidence $1-\alpha$, inversely proportional to KL divergence between haplotypes.}% Definition/Role
  {Experimental Design}% Category
  {Ch.~7.4}% First defined
  {$N$, $z_\alpha$, $D_{\text{KL}}$, $h_{\text{true}}$, $h_{\text{closest}}$}% Key variables

%==============================================================================
\section{Reference Standards (Chapters 8--9)}
\label{sec:eqmaster-standards}
%==============================================================================

\MasterEqBox
  {EQ-PLASMID-PURITY}% ID
  {eq:plasmid-purity-bound}% Label
  {Plasmid Purity Specification}% Name
  {\pi_{\text{plasmid}} \geq 0.95}% Equation
  {Minimum acceptable purity for plasmid reference standards, ensuring high-quality ground truth for basecaller training and validation.}% Definition/Role
  {Reference Standards}% Category
  {Ch.~8.5}% First defined
  {$\pi_{\text{plasmid}}$}% Key variables

\MasterEqBox
  {EQ-DUAL-CUT}% ID
  {eq:dual-cut-prob}% Label
  {Dual Enzymatic Cleavage Probability}% Name
  {P_{\text{dual cut}} = \left(1 - e^{-\lambda t}\right)^2}% Equation
  {Probability that both restriction sites are cut by time $t$ under exponential cleavage kinetics with rate $\lambda$, squared due to independence.}% Definition/Role
  {Reference Standards}% Category
  {Ch.~9.2}% First defined
  {$P_{\text{dual cut}}$, $\lambda$, $t$}% Key variables

\MasterEqBox
  {EQ-FRAGMENT-CDF}% ID
  {eq:fragment-cdf}% Label
  {Fragment Size CDF (Exponential)}% Name
  {F(x) = 1 - e^{-\lambda x}}% Equation
  {Cumulative distribution function for fragment size under exponential fragmentation model with rate parameter $\lambda$.}% Definition/Role
  {Reference Standards}% Category
  {Ch.~9.2}% First defined
  {$F(x)$, $\lambda$, $x$}% Key variables

\MasterEqBox
  {EQ-T7E1-EFF}% ID
  {eq:t7e1-efficiency}% Label
  {T7 Endonuclease I Cleavage Efficiency}% Name
  {E_{\text{T7E1}} = \frac{N_{\text{cleaved}}}{N_{\text{heteroduplex}}}}% Equation
  {Measures fraction of heteroduplex molecules successfully cleaved by T7 endonuclease I, critical for enrichment quality.}% Definition/Role
  {Reference Standards}% Category
  {Ch.~9.3}% First defined
  {$E_{\text{T7E1}}$, $N_{\text{cleaved}}$, $N_{\text{heteroduplex}}$}% Key variables

\MasterEqBox
  {EQ-ENRICHMENT-FOLD}% ID
  {eq:enrichment-fold}% Label
  {Fold-Enrichment Metric}% Name
  {\text{Enrichment fold} = \frac{\text{On-target fraction}}{\text{Genomic fraction}}}% Equation
  {Quantifies enrichment efficiency as ratio of on-target read fraction to target's genomic fraction, with values $>$ 1000× typical for successful targeted sequencing.}% Definition/Role
  {Reference Standards}% Category
  {Ch.~9.4}% First defined
  {On-target fraction, Genomic fraction}% Key variables

%==============================================================================
\section{Computational Methods (Chapters 10--13)}
\label{sec:eqmaster-computational}
%==============================================================================

\MasterEqBox
  {EQ-DATASET-LOGLIK}% ID
  {eq:dataset-loglik}% Label
  {Dataset Log-Likelihood Factorization}% Name
  {\log \Prob(\mathbf{r} | h_i) = \sum_{n=1}^{N} \log \Prob(r_n | h_i)}% Equation
  {Expresses dataset log-likelihood as sum over reads, enabling numerical stability and efficient computation in log-space.}% Definition/Role
  {Computational Methods}% Category
  {Ch.~10.3}% First defined
  {$\mathbf{r}$, $h_i$, $N$, $r_n$}% Key variables

\MasterEqBox
  {EQ-POSTERIOR-WORKFLOW}% ID
  {eq:posterior-workflow}% Label
  {Numerically Stable Posterior Computation}% Name
  {\Prob(h | \mathbf{r}) = \frac{\exp\left(\sum \log \Prob(r_n | h)\right) \Prob(h)}{\sum_{h'} \exp\left(\sum \log \Prob(r_n | h')\right) \Prob(h')}}% Equation
  {Computes posterior in log-space to avoid numerical underflow, critical for long reads or large read counts where raw probabilities become vanishingly small.}% Definition/Role
  {Computational Methods}% Category
  {Ch.~10.4}% First defined
  {$h$, $\mathbf{r}$, $\Prob(h)$}% Key variables

\MasterEqBox
  {EQ-PURITY-CEILING-SMA}% ID
  {eq:purity-ceiling-sma}% Label
  {Purity-SMA Relationship}% Name
  {\pi \leq \text{SMA}}% Equation
  {Establishes that empirical purity cannot exceed single-molecule accuracy, since perfect reads are necessary (but not sufficient) for correct classification.}% Definition/Role
  {Computational Methods}% Category
  {Ch.~11.2}% First defined
  {$\pi$, $\text{SMA}$}% Key variables

\MasterEqBox
  {EQ-SMA-EXACT-PROB}% ID
  {eq:sma-exact-probability}% Label
  {SMA Exact Definition}% Name
  {\text{SMA} = \Prob(\text{all } L \text{ bases correct})}% Equation
  {Formal definition of single-molecule accuracy as joint probability that all $L$ bases in a read are correctly called.}% Definition/Role
  {Computational Methods}% Category
  {Ch.~11.7}% First defined
  {$\text{SMA}$, $L$}% Key variables

\MasterEqBox
  {EQ-SMA-PREDICTED}% ID
  {eq:sma-predicted}% Label
  {Predicted SMA from Quality Scores}% Name
  {\text{SMA}_{\text{pred}} = \prod_{j=1}^{L} (1 - 10^{-Q_j/10})}% Equation
  {Computes predicted single-molecule accuracy from per-base Phred quality scores $Q_j$, assuming independence and accurate calibration.}% Definition/Role
  {Computational Methods}% Category
  {Ch.~11.8}% First defined
  {$\text{SMA}_{\text{pred}}$, $L$, $Q_j$}% Key variables

\MasterEqBox
  {EQ-SMA-CALIB-GAP}% ID
  {eq:sma-calibration-gap}% Label
  {SMA Calibration Gap}% Name
  {\Delta_{\text{SMA}} = \text{SMA}_{\text{pred}} - \text{SMA}_{\text{emp}}}% Equation
  {Quantifies read-level calibration gap, with $\Delta_{\text{SMA}} > 0$ indicating basecaller overconfidence in read-level accuracy.}% Definition/Role
  {Computational Methods}% Category
  {Ch.~11.9}% First defined
  {$\Delta_{\text{SMA}}$, $\text{SMA}_{\text{pred}}$, $\text{SMA}_{\text{emp}}$}% Key variables

\MasterEqBox
  {EQ-MIN-DATABASE-SIZE}% ID
  {eq:min-database-size}% Label
  {Minimum Reference Database Size}% Name
  {N_{\text{db}} \geq \frac{|\mathcal{H}| \cdot k}{\pi_{\text{min}}}}% Equation
  {Minimum database size required to achieve $k$-fold coverage per haplotype in panel $\mathcal{H}$ with minimum purity $\pi_{\text{min}}$, accounting for error-free read requirement.}% Definition/Role
  {Computational Methods}% Category
  {Ch.~12.3}% First defined
  {$N_{\text{db}}$, $|\mathcal{H}|$, $k$, $\pi_{\text{min}}$}% Key variables

\MasterEqBox
  {EQ-LABEL-TRANSITION}% ID
  {eq:label-transition}% Label
  {Label Noise Transition Model}% Name
  {\widetilde{y} = T(y) \text{ where } T_{ij} = \Prob(\widetilde{y} = j | y = i)}% Equation
  {Models label noise via transition matrix $T$ where noisy label $\widetilde{y}$ depends on true label $y$ through conditional probabilities $T_{ij}$.}% Definition/Role
  {Computational Methods}% Category
  {Ch.~12.4}% First defined
  {$\widetilde{y}$, $y$, $T$, $T_{ij}$}% Key variables

\MasterEqBox
  {EQ-OBSERVED-CONFUSION}% ID
  {eq:observed-confusion}% Label
  {Noisy Confusion Matrix Decomposition}% Name
  {\widetilde{\mathbf{C}} = \mathbf{C}_{\text{true}} \cdot \mathbf{T}}% Equation
  {Decomposes observed confusion matrix $\widetilde{\mathbf{C}}$ as product of true classifier confusion $\mathbf{C}_{\text{true}}$ and label noise transition matrix $\mathbf{T}$.}% Definition/Role
  {Computational Methods}% Category
  {Ch.~12.4}% First defined
  {$\widetilde{\mathbf{C}}$, $\mathbf{C}_{\text{true}}$, $\mathbf{T}$}% Key variables

\MasterEqBox
  {EQ-CE-LOSS}% ID
  {eq:ce-loss}% Label
  {Cross-Entropy Loss}% Name
  {L_{\text{CE}} = -\sum_{i=1}^{C} y_i \log(\widehat{y}_i)}% Equation
  {Standard cross-entropy loss for classification with $C$ classes, where $y$ is one-hot true label and $\widehat{y}$ is predicted probability distribution.}% Definition/Role
  {Computational Methods}% Category
  {Ch.~13.5}% First defined
  {$L_{\text{CE}}$, $C$, $y_i$, $\widehat{y}_i$}% Key variables

\MasterEqBox
  {EQ-FOCAL-LOSS}% ID
  {eq:focal-loss}% Label
  {Focal Loss (Hard Example Mining)}% Name
  {L_{\text{focal}} = -(1 - \widehat{y})^\gamma \log(\widehat{y})}% Equation
  {Down-weights easy examples (high $\widehat{y}$) via modulating factor $(1-\widehat{y})^\gamma$, focusing training on hard negatives where $\widehat{y}$ is low.}% Definition/Role
  {Computational Methods}% Category
  {Ch.~13.5}% First defined
  {$L_{\text{focal}}$, $\widehat{y}$, $\gamma$}% Key variables

\MasterEqBox
  {EQ-CALIBRATION-ERROR}% ID
  {eq:calibration-error}% Label
  {Expected Calibration Error (ECE)}% Name
  {\text{ECE} = \sum_{m=1}^{M} \frac{|B_m|}{N} |\text{acc}(B_m) - \text{conf}(B_m)|}% Equation
  {Measures calibration by partitioning predictions into $M$ bins $B_m$ and computing weighted average of accuracy-confidence gaps.}% Definition/Role
  {Computational Methods}% Category
  {Ch.~13.6}% First defined
  {$\text{ECE}$, $M$, $B_m$, $N$}% Key variables

%==============================================================================
\section{Validation \& Quality Control (Chapter 14, Appendices)}
\label{sec:eqmaster-validation}
%==============================================================================

\MasterEqBox
  {EQ-MIXTURE-PROP-SS}% ID
  {eq:mixture-proportion-sample-size}% Label
  {Sample Size for Mixture Proportion Estimation}% Name
  {N \geq \frac{z_{1-\delta/2}^2}{4\epsilon^2 \cdot [\lambda(1-\lambda)]}}% Equation
  {Minimum read count needed to estimate mixture proportion $\lambda$ with relative precision $\epsilon$ at confidence $1-\delta$, with denominator maximized when $\lambda = 0.5$.}% Definition/Role
  {Validation \& QC}% Category
  {Ch.~14.3}% First defined
  {$N$, $z_{1-\delta/2}$, $\epsilon$, $\lambda$}% Key variables

\MasterEqBox
  {EQ-MINOR-DETECTION}% ID
  {eq:minor-detection-limit}% Label
  {Minor Haplotype Detection Limit}% Name
  {N \geq \frac{(z_\alpha + z_\beta)^2}{\lambda_{\text{min}}^2}}% Equation
  {Minimum coverage required to detect minor haplotype at fraction $\lambda_{\text{min}}$ with significance $\alpha$ and power $1-\beta$.}% Definition/Role
  {Validation \& QC}% Category
  {Ch.~14.4}% First defined
  {$N$, $z_\alpha$, $z_\beta$, $\lambda_{\text{min}}$}% Key variables

\MasterEqBox
  {EQ-DIPLOTYPE-ERROR}% ID
  {eq:diplotype-error}% Label
  {Diplotype Classification Error Probability}% Name
  {P(\text{error}) = 1 - \Prob(\text{correct diplotype})}% Equation
  {Defines diplotype-level error probability as complement of correct diplotype posterior, accounting for phase ambiguity in heterozygotes.}% Definition/Role
  {Validation \& QC}% Category
  {Ch.~14.5}% First defined
  {$P(\text{error})$, $\Prob(\text{correct diplotype})$}% Key variables

%==============================================================================
\section{Clinical Applications (Chapters 17--18)}
\label{sec:eqmaster-clinical}
%==============================================================================

\MasterEqBox
  {EQ-PHENOCONV}% ID
  {eq:phenoconversion}% Label
  {Phenoconversion (Drug-Drug Interaction)}% Name
  {\text{AS}_{\text{eff}} = \text{AS}_{\text{geno}} \cdot (1 - I)}% Equation
  {Adjusts genotypic activity score for enzyme inhibition fraction $I$ due to co-administered drugs, modeling phenoconversion from normal to poor metabolizer.}% Definition/Role
  {Clinical Applications}% Category
  {Ch.~17.4}% First defined
  {$\text{AS}_{\text{eff}}$, $\text{AS}_{\text{geno}}$, $I$}% Key variables

\MasterEqBox
  {EQ-DIPLOTYPE-POSTERIOR}% ID
  {eq:diplotype-posterior}% Label
  {Diplotype Posterior (Ambiguity Resolution)}% Name
  {P(D_k | \mathbf{R}) = \frac{P(\mathbf{R} | D_k) \cdot P(D_k)}{\sum_{j=1}^{K} P(\mathbf{R} | D_j) \cdot P(D_j)}}% Equation
  {Computes posterior probability of diplotype $D_k$ among $K$ phase-ambiguous candidates, integrating read evidence with population frequency priors.}% Definition/Role
  {Clinical Applications}% Category
  {Ch.~18.3}% First defined
  {$D_k$, $K$, $\mathbf{R}$, $P(D_k)$}% Key variables

\MasterEqBox
  {EQ-ENDOXIFEN-PRED}% ID
  {eq:ch18-endoxifen-prediction}% Label
  {Endoxifen Concentration Model}% Name
  {[\text{endoxifen}] = \beta_0 + \beta_1 \cdot \text{AS} + \beta_2 \cdot [\text{tamoxifen}]}% Equation
  {Linear regression model predicting endoxifen plasma concentration from CYP2D6 activity score and parent tamoxifen concentration, validated in Singapore cohort.}% Definition/Role
  {Clinical Applications}% Category
  {Ch.~18.5}% First defined
  {$[\text{endoxifen}]$, AS, $[\text{tamoxifen}]$, $\beta_0$, $\beta_1$, $\beta_2$}% Key variables

%==============================================================================
\section{Cross-Reference Index by Chapter}
\label{sec:eqmaster-index}
%==============================================================================

\subsection*{Appendix B: Core Mathematical Models}
\texttt{eq:ce1}, \texttt{eq:ce2}, \texttt{eq:ce3}, \texttt{eq:ce4}, \texttt{eq:ce5}, \texttt{eq:ce6}, \texttt{eq:ce7}, \texttt{eq:ce8}, \texttt{eq:ce9}, \texttt{eq:ce10}, \texttt{eq:ce12}, \texttt{eq:ce13}, \texttt{eq:ce14}, \texttt{eq:bayes-posterior}

\subsection*{Chapter 4: Haplotype Classification Model}
\texttt{eq:pipeline-factorization-sma}

\subsection*{Chapter 5: Purity Theory}
\texttt{eq:purity-def}, \texttt{eq:tpr-ceiling}, \texttt{eq:purity-ce-lower}, \texttt{eq:purity-upper}, \texttt{eq:purity-upper-poisson}, \texttt{eq:purity-bracket}, \texttt{eq:qpurity}, \texttt{eq:mismatch-ambiguity}

\subsection*{Chapter 6: Posterior Computation}
\texttt{eq:likelihood-factorization}, \texttt{eq:per-read-likelihood}, \texttt{eq:posterior-bayes}, \texttt{eq:map-classification}, \texttt{eq:bayes-factor}

\subsection*{Chapter 7: Experimental Design}
\texttt{eq:perfect-read}, \texttt{eq:expected-perfect}, \texttt{eq:expected-errors}, \texttt{eq:sample-size-quality}, \texttt{eq:exact-binomial-pvalue}, \texttt{eq:coverage-confidence}

\subsection*{Chapter 8: Plasmid Standards}
\texttt{eq:plasmid-purity-bound}

\subsection*{Chapter 9: Targeted Enrichment}
\texttt{eq:dual-cut-prob}, \texttt{eq:fragment-cdf}, \texttt{eq:t7e1-efficiency}, \texttt{eq:on-target}, \texttt{eq:off-target}, \texttt{eq:nonspecific}, \texttt{eq:enrichment-fold}, \texttt{eq:spike-in-efficiency}

\subsection*{Chapter 10: Haplotype Mixtures}
\texttt{eq:dataset-loglik}, \texttt{eq:posterior-workflow}

\subsection*{Chapter 11: Basecaller Quality Models}
\texttt{eq:purity-ceiling-sma}, \texttt{eq:purity-upper-sma}, \texttt{eq:confusion-matrix-sma}, \texttt{eq:overestimation-sma}, \texttt{eq:seer-matrix-normalization}, \texttt{eq:indel-rates}, \texttt{eq:error-indicator}, \texttt{eq:empirical-error-rate}, \texttt{eq:empirical-q-read}, \texttt{eq:predicted-q-read-formal}, \texttt{eq:sma-exact-probability}, \texttt{eq:sma-exact-estimator}, \texttt{eq:wilson-interval-sma}, \texttt{eq:sma-base}, \texttt{eq:sma-base-decomposition}, \texttt{eq:sma-predicted}, \texttt{eq:sma-calibration-gap}, \texttt{eq:purity-bias-sma}

\subsection*{Chapter 12: Noisy Label Learning}
\texttt{eq:noise-bias}, \texttt{eq:min-database-size}, \texttt{eq:label-transition}, \texttt{eq:observed-confusion}, \texttt{eq:transition-estimate}, \texttt{eq:denoised-confusion}, \texttt{eq:confusion-se}, \texttt{eq:robust-loss}, \texttt{eq:gce-loss}, \texttt{eq:weighted-loss}

\subsection*{Chapter 13: Basecaller Fine-Tuning}
\texttt{eq:learning-rate-reduction}, \texttt{eq:fine-tuning}, \texttt{eq:ce-loss}, \texttt{eq:ctc-loss}, \texttt{eq:focal-loss}, \texttt{eq:quality-overestimation}, \texttt{eq:calibration-error}, \texttt{eq:linear-recal}, \texttt{eq:isotonic-recal}, \texttt{eq:seer-workflow-weight}, \texttt{eq:sma-seq-qc-gate}

\subsection*{Chapter 14: Library Preparation Quality Control}
\texttt{eq:mixture-proportion-sample-size}, \texttt{eq:minor-detection-limit}, \texttt{eq:diplotype-error}

\subsection*{Chapter 17: CYP2D6 Pain Management \& Psychiatry}
\texttt{eq:phenoconversion}

\subsection*{Chapter 18: Singapore Tamoxifen Cohort}
\texttt{eq:diplotype-posterior}, \texttt{eq:ch18-endoxifen-prediction}

\subsection*{Appendix D: Computational Protocols}
\texttt{eq:seer-matrix}

%==============================================================================
\section{Usage Guidelines}
\label{sec:eqmaster-usage}
%==============================================================================

\subsection*{Citing Equations}

Use \verb|\ref{}| for equation numbers only or \verb|\eqref{}| for parenthetical equation numbers:

\begin{verbatim}
As shown in Equation~\ref{eq:posterior-bayes}, the posterior probability...
The Bayesian update rule \eqref{eq:posterior-bayes} can be applied...
\end{verbatim}

For core equations with CE\# codes, use the \verb|\CEref{}| command:

\begin{verbatim}
The purity ceiling \CEref{4} limits classification performance...
\end{verbatim}

\subsection*{Related Equations}

\textbf{Bayesian Inference Chain:}
\begin{itemize}
\item \texttt{eq:ce1} $\to$ \texttt{eq:posterior-bayes} $\to$ \texttt{eq:map-classification}
\item \texttt{eq:likelihood-factorization} $\to$ \texttt{eq:per-read-likelihood}
\item \texttt{eq:dataset-loglik} $\to$ \texttt{eq:posterior-workflow}
\end{itemize}

\textbf{Purity Theory Chain:}
\begin{itemize}
\item \texttt{eq:ce3} $\to$ \texttt{eq:purity-def} $\to$ \texttt{eq:purity-bracket}
\item \texttt{eq:ce4} $\to$ \texttt{eq:tpr-ceiling} $\to$ \texttt{eq:purity-ceiling-sma}
\end{itemize}

\textbf{Quality Score Chain:}
\begin{itemize}
\item \texttt{eq:ce5} $\to$ \texttt{eq:ce6} $\to$ \texttt{eq:ce10}
\item \texttt{eq:empirical-q-read} $\to$ \texttt{eq:predicted-q-read-formal} $\to$ \texttt{eq:quality-overestimation}
\item \texttt{eq:linear-recal} $\to$ \texttt{eq:isotonic-recal}
\end{itemize}

\textbf{SMA (Single Molecule Accuracy) Chain:}
\begin{itemize}
\item \texttt{eq:ce7} $\to$ \texttt{eq:ce8} $\to$ \texttt{eq:sma-base}
\item \texttt{eq:sma-exact-probability} $\to$ \texttt{eq:sma-exact-estimator}
\item \texttt{eq:sma-predicted} $\to$ \texttt{eq:sma-calibration-gap}
\end{itemize}

\textbf{Sample Size \& Power Chain:}
\begin{itemize}
\item \texttt{eq:sample-size-quality} $\to$ \texttt{eq:coverage-confidence}
\item \texttt{eq:mixture-proportion-sample-size} $\to$ \texttt{eq:minor-detection-limit}
\item \texttt{eq:min-database-size}
\end{itemize}

\textbf{Noisy Label Learning Chain:}
\begin{itemize}
\item \texttt{eq:label-transition} $\to$ \texttt{eq:observed-confusion} $\to$ \texttt{eq:denoised-confusion}
\item \texttt{eq:robust-loss} $\to$ \texttt{eq:gce-loss} $\to$ \texttt{eq:weighted-loss}
\end{itemize}

\subsection*{Equation Naming Conventions}

\begin{itemize}
\item \textbf{Core equations:} \texttt{eq:ce1} through \texttt{eq:ce14} (reserved for Appendix~B)
\item \textbf{Descriptive names:} Use kebab-case: \texttt{eq:posterior-bayes}, \texttt{eq:sample-size-quality}
\item \textbf{Chapter-specific:} Prefix with chapter identifier if ambiguous: \texttt{eq:ch18-endoxifen-prediction}
\item \textbf{Avoid generic names:} Don't use \texttt{eq:equation1} or \texttt{eq:main} — use descriptive labels
\end{itemize}

\subsection*{Missing Equation Labels}

If you encounter an unlabeled equation that should be referenced, add a label following the conventions above. Example:

\begin{verbatim}
\begin{equation}
\text{New important equation}
\label{eq:descriptive-name}
\end{equation}
\end{verbatim}

Then add the equation to this master table in the appropriate category section.

%==============================================================================
\section{Equation Variable Cross-Reference}
\label{sec:eqmaster-var-xref}
%==============================================================================

For detailed variable definitions, see Appendix~G (Master Variable Reference Table). Key variable-equation relationships:

\begin{itemize}
\item \textbf{$h$ (haplotype):} Used in \texttt{eq:ce1}, \texttt{eq:ce2}, \texttt{eq:posterior-bayes}, \texttt{eq:likelihood-factorization}, \texttt{eq:per-read-likelihood}, \texttt{eq:map-classification}

\item \textbf{$\pi$ (purity):} Defined in \texttt{eq:ce3}, \texttt{eq:purity-def}; used in \texttt{eq:ce4}, \texttt{eq:tpr-ceiling}, \texttt{eq:purity-bracket}, \texttt{eq:purity-bias-sma}

\item \textbf{$Q$ (quality score):} Used in \texttt{eq:ce5}, \texttt{eq:ce6}, \texttt{eq:ce10}, \texttt{eq:empirical-q-read}, \texttt{eq:predicted-q-read-formal}, \texttt{eq:quality-overestimation}, \texttt{eq:linear-recal}, \texttt{eq:isotonic-recal}

\item \textbf{SMA (single molecule accuracy):} Defined in \texttt{eq:ce7}, \texttt{eq:ce8}; used in \texttt{eq:sma-exact-probability}, \texttt{eq:sma-base}, \texttt{eq:sma-predicted}, \texttt{eq:sma-calibration-gap}, \texttt{eq:sma-seq-qc-gate}

\item \textbf{$N$ (sample size):} Used in \texttt{eq:sample-size-quality}, \texttt{eq:coverage-confidence}, \texttt{eq:mixture-proportion-sample-size}, \texttt{eq:minor-detection-limit}, \texttt{eq:min-database-size}

\item \textbf{$\mathbf{C}$ (confusion matrix):} Defined in \texttt{eq:ce14}, \texttt{eq:confusion-matrix-sma}; used in \texttt{eq:observed-confusion}, \texttt{eq:denoised-confusion}, \texttt{eq:confusion-se}, \texttt{eq:seer-matrix-normalization}

\item \textbf{$\lambda$ (mixture proportion):} Defined in \texttt{eq:ce12}; used in \texttt{eq:mixture-proportion-sample-size}, \texttt{eq:minor-detection-limit}

\item \textbf{$D$ (diplotype):} Used in \texttt{eq:diplotype-posterior}, \texttt{eq:diplotype-error}

\item \textbf{AS (activity score):} Used in \texttt{eq:phenoconversion}, \texttt{eq:ch18-endoxifen-prediction}
\end{itemize}

%==============================================================================
% End of Appendix H
%==============================================================================


% Appendix I: Chapter-by-Chapter Mathematical Reference (NEW - v6.3)
% Database-generated comprehensive reference with all equations and variables by chapter
% Auto-generated from equations.yaml and variables.yaml (87 equations, 67 variables)
\chapter{Chapter-by-Chapter Mathematical Reference}
\label{app:chapter-math-reference}
\label{app:database-appendix}

\section*{Overview}

This appendix provides comprehensive chapter-by-chapter mathematical references automatically generated from the equation and variable database system. Each chapter section includes:

\begin{itemize}
\item Complete list of all equations introduced in that chapter with full LaTeX, descriptions, physical interpretations, and assumptions
\item Summary table of all variables defined or used in the chapter with units, domains, and typical ranges
\item Cross-references to related equations and variables
\item Examples demonstrating practical usage
\end{itemize}

This database-driven approach ensures consistency across the textbook and enables systematic updates as the mathematical framework evolves. All content is generated from the authoritative \texttt{equations.yaml} and \texttt{variables.yaml} databases (see repository root).

\bigskip

\noindent\textbf{Database Statistics:}
\begin{itemize}
\item Total equations: 87 across 11 chapters
\item Total variables: 67 with comprehensive specifications
\item Equation families: 9 interconnected groups identified
\item Hub variables: 30+ variables used in multiple equations
\item Data consistency: 100\% validated
\end{itemize}

\bigskip

\noindent\textbf{Using This Appendix:}
\begin{itemize}
\item Navigate to your chapter of interest
\item Review variable summary table for quick reference
\item See detailed equation boxes for complete mathematical specifications
\item Follow cross-references to related material
\item Use equation labels (e.g., \texttt{eq:posterior-bayes}) for citations
\end{itemize}

\clearpage

%%%%%%%%%%%%%%%%%%%%%%%%%%%%%%%%%%%%%%%%%%%%%%%%%%%%%%%%%%%%%%%%%%%%%%%%
%% Chapter 4: Haplotype Classification Model
%%%%%%%%%%%%%%%%%%%%%%%%%%%%%%%%%%%%%%%%%%%%%%%%%%%%%%%%%%%%%%%%%%%%%%%%

\section{Chapter 4: Haplotype Classification Model}
\label{sec:chapter4-math-ref}

\input{generated/appendices/chapter4_appendix.tex}

\clearpage

%%%%%%%%%%%%%%%%%%%%%%%%%%%%%%%%%%%%%%%%%%%%%%%%%%%%%%%%%%%%%%%%%%%%%%%%
%% Chapter 5: Purity Theory
%%%%%%%%%%%%%%%%%%%%%%%%%%%%%%%%%%%%%%%%%%%%%%%%%%%%%%%%%%%%%%%%%%%%%%%%

\section{Chapter 5: Purity Theory}
\label{sec:chapter5-math-ref}

\input{generated/appendices/chapter5_appendix.tex}

\clearpage

%%%%%%%%%%%%%%%%%%%%%%%%%%%%%%%%%%%%%%%%%%%%%%%%%%%%%%%%%%%%%%%%%%%%%%%%
%% Chapter 6: Posterior Computation
%%%%%%%%%%%%%%%%%%%%%%%%%%%%%%%%%%%%%%%%%%%%%%%%%%%%%%%%%%%%%%%%%%%%%%%%

\section{Chapter 6: Posterior Computation}
\label{sec:chapter6-math-ref}

\input{generated/appendices/chapter6_appendix.tex}

\clearpage

%%%%%%%%%%%%%%%%%%%%%%%%%%%%%%%%%%%%%%%%%%%%%%%%%%%%%%%%%%%%%%%%%%%%%%%%
%% Chapter 7: Experimental Design
%%%%%%%%%%%%%%%%%%%%%%%%%%%%%%%%%%%%%%%%%%%%%%%%%%%%%%%%%%%%%%%%%%%%%%%%

\section{Chapter 7: Experimental Design}
\label{sec:chapter7-math-ref}

\input{generated/appendices/chapter7_appendix.tex}

\clearpage

%%%%%%%%%%%%%%%%%%%%%%%%%%%%%%%%%%%%%%%%%%%%%%%%%%%%%%%%%%%%%%%%%%%%%%%%
%% Chapter 8: Plasmid Standards
%%%%%%%%%%%%%%%%%%%%%%%%%%%%%%%%%%%%%%%%%%%%%%%%%%%%%%%%%%%%%%%%%%%%%%%%

\section{Chapter 8: Plasmid Standards}
\label{sec:chapter8-math-ref}

\input{generated/appendices/chapter8_appendix.tex}

\clearpage

%%%%%%%%%%%%%%%%%%%%%%%%%%%%%%%%%%%%%%%%%%%%%%%%%%%%%%%%%%%%%%%%%%%%%%%%
%% Chapter 9: Targeted Enrichment
%%%%%%%%%%%%%%%%%%%%%%%%%%%%%%%%%%%%%%%%%%%%%%%%%%%%%%%%%%%%%%%%%%%%%%%%

\section{Chapter 9: Targeted Enrichment}
\label{sec:chapter9-math-ref}

\input{generated/appendices/chapter9_appendix.tex}

\clearpage

%%%%%%%%%%%%%%%%%%%%%%%%%%%%%%%%%%%%%%%%%%%%%%%%%%%%%%%%%%%%%%%%%%%%%%%%
%% Chapter 10: Haplotype Mixtures
%%%%%%%%%%%%%%%%%%%%%%%%%%%%%%%%%%%%%%%%%%%%%%%%%%%%%%%%%%%%%%%%%%%%%%%%

\section{Chapter 10: Haplotype Mixtures}
\label{sec:chapter10-math-ref}

\input{generated/appendices/chapter10_appendix.tex}

\clearpage

%%%%%%%%%%%%%%%%%%%%%%%%%%%%%%%%%%%%%%%%%%%%%%%%%%%%%%%%%%%%%%%%%%%%%%%%
%% Chapter 11: Basecaller Quality Models
%%%%%%%%%%%%%%%%%%%%%%%%%%%%%%%%%%%%%%%%%%%%%%%%%%%%%%%%%%%%%%%%%%%%%%%%

\section{Chapter 11: Basecaller Quality Models}
\label{sec:chapter11-math-ref}

\input{generated/appendices/chapter11_appendix.tex}

\clearpage

%%%%%%%%%%%%%%%%%%%%%%%%%%%%%%%%%%%%%%%%%%%%%%%%%%%%%%%%%%%%%%%%%%%%%%%%
%% Chapter 12: Noisy Label Learning
%%%%%%%%%%%%%%%%%%%%%%%%%%%%%%%%%%%%%%%%%%%%%%%%%%%%%%%%%%%%%%%%%%%%%%%%

\section{Chapter 12: Noisy Label Learning}
\label{sec:chapter12-math-ref}

\input{generated/appendices/chapter12_appendix.tex}

\clearpage

%%%%%%%%%%%%%%%%%%%%%%%%%%%%%%%%%%%%%%%%%%%%%%%%%%%%%%%%%%%%%%%%%%%%%%%%
%% Chapter 13: Basecaller Fine-Tuning
%%%%%%%%%%%%%%%%%%%%%%%%%%%%%%%%%%%%%%%%%%%%%%%%%%%%%%%%%%%%%%%%%%%%%%%%

\section{Chapter 13: Basecaller Fine-Tuning}
\label{sec:chapter13-math-ref}

\input{generated/appendices/chapter13_appendix.tex}

\clearpage

%%%%%%%%%%%%%%%%%%%%%%%%%%%%%%%%%%%%%%%%%%%%%%%%%%%%%%%%%%%%%%%%%%%%%%%%
%% Chapter 14: Library Preparation
%%%%%%%%%%%%%%%%%%%%%%%%%%%%%%%%%%%%%%%%%%%%%%%%%%%%%%%%%%%%%%%%%%%%%%%%

\section{Chapter 14: Library Preparation}
\label{sec:chapter14-math-ref}

\input{generated/appendices/chapter14_appendix.tex}

\clearpage

%%%%%%%%%%%%%%%%%%%%%%%%%%%%%%%%%%%%%%%%%%%%%%%%%%%%%%%%%%%%%%%%%%%%%%%%
%% End of Chapter-by-Chapter Mathematical Reference
%%%%%%%%%%%%%%%%%%%%%%%%%%%%%%%%%%%%%%%%%%%%%%%%%%%%%%%%%%%%%%%%%%%%%%%%


%%%%%%%%%%%%%%%%%%%%%%%%%%%%%%%%%%%%%%%%%%%%%%%%%%%%%%%%%%%%%%%%%%%%%%%%
%% DEMONSTRATION APPENDICES (Optional - can be commented out)
%%%%%%%%%%%%%%%%%%%%%%%%%%%%%%%%%%%%%%%%%%%%%%%%%%%%%%%%%%%%%%%%%%%%%%%%
% The following appendices demonstrate the eqboxV5 template system.
% These can be commented out after reviewing the examples.

\chapter{eqboxV5 Template Demonstration}
\label{app:eqbox-demo}

This appendix demonstrates the eqboxV5 template system integrated in this document. The examples below show the v5-style aesthetic and comprehensive documentation structure.

\section*{About These Examples}

The following examples are production-ready equation boxes that demonstrate:
\begin{itemize}
\item Clean gray aesthetic matching the v5 baseline style
\item Comprehensive mathematical exposition with all helper sections
\item Proper use of the \texttt{\textbackslash CEanchor} system for cross-referencing
\item Page break control using \texttt{\textbackslash needspace}
\item Professional formatting suitable for clinical documentation
\end{itemize}

For authoring instructions, see \texttt{docs/EQBOX\_AUTHORING\_INSTRUCTIONS\_V5.md}.

% TEMPORARILY COMMENTED OUT - causes TeX capacity exceeded error due to tcolorbox nesting
% \input{templates/eqbox_example_rigorous.tex}

\chapter{Table Template Demonstration}
\label{app:table-demo}

This appendix demonstrates the companion table templates matching the eqboxV5 style. These tables provide professional formatting for summary data, parameters, and experimental specifications.

\section*{About These Tables}

The tables demonstrate:
\begin{itemize}
\item Consistent gray color scheme matching eqboxV5
\item Professional formatting using booktabs
\item Alternating row colors for readability
\item Realistic data appropriate for haplotype classification
\item Reusable templates for custom tables
\end{itemize}

\input{templates/table_example_realistic.tex}

%%%%%%%%%%%%%%%%%%%%%%%%%%%%%%%%%%%%%%%%%%%%%%%%%%%%%%%%%%%%%%%%%%%%%%%%
%% BACK MATTER
%%%%%%%%%%%%%%%%%%%%%%%%%%%%%%%%%%%%%%%%%%%%%%%%%%%%%%%%%%%%%%%%%%%%%%%%

\backmatter

% Bibliography
\bibliographystyle{plain}
\bibliography{references}

\end{document}
