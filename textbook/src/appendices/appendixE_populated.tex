%%%%%%%%%%%%%%%%%%%%%%%%%%%%%%%%%%%%%%%%%%%%%%%%%%%%%%%%%%%%%%%%%%%%%%%%
%% Appendix E: Version History
%% Part: Appendices
%% Version 6.0 - Migrated from v5.tex Appendix D (lines 2253-2344) + NEW v6.0 entry
%%%%%%%%%%%%%%%%%%%%%%%%%%%%%%%%%%%%%%%%%%%%%%%%%%%%%%%%%%%%%%%%%%%%%%%%

\chapter{Version History}
\label{app:version_history}
\label{app:version-history}

This version history documents the evolution of the mathematical framework, tracking major enhancements, refinements, and structural improvements. Each version represents significant methodological advances or documentation improvements that affect framework implementation or interpretation.

\section*{Version 6.0 (Current - Seven-Part Book Architecture)}

\textbf{Release Date:} October 16, 2025

\textbf{Last Updated:} November 16, 2025

\textbf{Major Reorganization:} Version 6.0 represents a comprehensive restructuring from a five-part research document into a seven-part integrated methods and applications volume. This reorganization maintains all mathematical rigor while dramatically enhancing practical utility for clinical deployment, regulatory compliance, and reproducible research.

\subsection*{Key Changes in Version 6.0}

\begin{itemize}
\item \textbf{Seven-Part Book Structure:} Reorganized entire framework into seven cohesive parts with clear narrative arc: Part I (Clinical Motivation and Technical Background), Part II (Mathematical Foundations), Part III (Physical Standards and Laboratory Workflows), Part IV (SMA-seq and Model Improvement), Part V (Validation of Genetic Tests), Part VI (Clinical Applications), Part VII (Operational Excellence). This structure provides dual accessibility for clinical/assay-focused and methods/theory-focused readers.

\item \textbf{Explicit Clinical Motivation (Part I):} Added new Part I (3 chapters) providing clinical context, pharmacogenomics motivation, and genomic complexity overview. Establishes clear rationale for single-molecule sequencing approaches before diving into technical details.

\item \textbf{Expanded Laboratory Workflows (Part III):} Created comprehensive Part III (3 chapters) documenting plasmid standard construction, Cas9 enrichment protocols, and complete sample-to-analysis workflows. Provides reproducible laboratory protocols with accept/reject criteria.

\item \textbf{SMA-seq Methods Integration (Part IV):} Consolidated SMA-seq methodology, noisy label training, and basecaller fine-tuning into cohesive Part IV (3 chapters). Provides complete methodology for model improvement using empirical data.

\item \textbf{Dedicated Clinical Applications (Part VI):} Created Part VI (3 chapters) showcasing bacterial strain typing, CYP2D6 pharmacogenomics, and 75-patient cohort study. Demonstrates real-world utility and validation across diverse applications.

\item \textbf{Operational Excellence Framework (Part VII):} Added Part VII (2 chapters) covering standard operating procedures and economic analysis. Addresses deployment considerations for clinical production environments.

\item \textbf{Enhanced Appendices:} Reorganized appendices into five focused sections: (A) Notation and Symbol Reference, (B) Core Equations Reference, (C) Laboratory Protocols and Quality Control, (D) Software Tools and Implementation, (E) Version History. Provides comprehensive reference material for all framework components.

\item \textbf{Improved Chapter Organization:} Expanded from 10 chapters to 20 chapters with clear, focused scopes. Each chapter includes explicit learning objectives, detailed content, practical examples, and integration with broader framework.

\item \textbf{Future-Proof Structure:} Architecture accommodates planned additions including technology-specific optimization chapters, expanded cohort studies, and regulatory documentation templates.
\end{itemize}

\subsection*{Migration from v5 to v6}

Version 6.0 content was systematically migrated from v5.0 using the following chapter mapping:

\begin{table}[H]
\centering
\caption{v5 to v6 Chapter Migration Map}
\label{tab:v5_to_v6_mapping}
\begin{tabular}{lll}
\toprule
\textbf{v5 Chapter} & \textbf{v6 Chapter(s)} & \textbf{Notes} \\
\midrule
Ch 1: Introduction & Ch 1-3 (Part I) & Expanded into 3-chapter intro \\
Ch 2: Foundations & Ch 4-7 (Part II) & Split into focused chapters \\
Ch 3: Models & Ch 4-7 (Part II) & Integrated into foundations \\
Ch 4: SEER Framework & Ch 11 (Part IV) & Moved to SMA-seq section \\
Ch 5: Likelihood & Ch 6 (Part II) & Integrated into posteriors \\
Ch 6: Bayesian Inference & Ch 6 (Part II) & Combined with likelihood \\
Ch 7: Advanced Topics & Ch 9, 12-13 (Parts III-IV) & Distributed appropriately \\
Ch 8: Implementation & Ch 10 (Part III) & Workflow chapter \\
Ch 9: Validation & Ch 14-15 (Part V) & Expanded validation \\
Ch 10: Case Studies & Ch 16-18 (Part VI) & Split into 3 applications \\
\bottomrule
\end{tabular}
\end{table}

\subsection*{Document Statistics}

\begin{itemize}
\item \textbf{Structure:} 7 Parts, 20 Chapters, 5 Appendices
\item \textbf{Expected Length:} 200-250 pages (fully populated)
\item \textbf{Current Completion:} 75\% (15 of 20 chapters populated as of November 16, 2025)
\item \textbf{Core Equations:} 16 equations (CE.1-CE.16) preserved from v5
\item \textbf{Notation:} 50+ symbols with comprehensive descriptions
\item \textbf{Tables:} 30+ tables with protocols, thresholds, and metrics
\end{itemize}

\section*{Version 5.0 (Five-Part Book Structure)}

\textbf{Date:} 2024

\begin{itemize}
\item \textbf{Comprehensive Book-Level Restructuring:} Transformed document from article format to professional book structure with five integrated parts, enabling more logical organization and improved navigation. Changed document class to book with proper chapter and part divisions. Updated all theorem environments to use chapter-based numbering. Enhanced headers and footers for book format with alternating page layouts.

\item \textbf{Five-Part Organizational Framework:} Reorganized entire content into cohesive parts that progress from theory through practice. Part I (Theoretical Foundations) establishes mathematical infrastructure. Part II (Empirical Measurement and Error Characterization) develops measurement-driven models including the new SEER framework. Part III (Inference and Classification Methods) presents likelihood computation and Bayesian inference. Part IV (Implementation and Technology-Specific Considerations) provides computational guidance. Part V (Validation and Clinical Applications) addresses regulatory compliance and real-world deployment.

\item \textbf{Major Addition: SEER Framework (Chapter 4):} Introduced comprehensive new chapter presenting the Sequencing Empirical Error Rate framework, a patented methodology for empirically quantifying technology-specific error patterns through controlled experiments with known-truth reference standards. Chapter includes confusion matrix formulation, true positive rate calculation, quality score calibration assessment, k-mer error profiling, basecaller optimization strategies, and clinical validation protocols. SEER provides data-driven foundation for confident haplotype classification with quantifiable accuracy guarantees.

\item \textbf{New Core Equations CE.17-CE.18:} Extended Core Equations reference with two essential SEER metrics. CE.17 formalizes true positive rate calculation from confusion matrices with purity constraints. CE.18 quantifies quality score overestimation for systematic miscalibration detection. Both equations integrate seamlessly with existing quality control framework and include comprehensive parameter descriptions, usage guidelines, and cross-references to detailed derivations.

\item \textbf{Enhanced Chapter Organization:} Converted all major sections to proper chapters with introductory context paragraphs. Chapter 1 (Introduction and Overview) sets framework scope. Chapter 2 (Mathematical Foundations) establishes notation and measurable spaces. Chapter 3 (Stage-Specific Probabilistic Models) develops mutation, fragmentation, and sequencing models. Chapter 4 (SEER Framework) characterizes empirical errors. Chapter 5 (Likelihood Computation Framework) presents tractable algorithms. Chapter 6 (Bayesian Inference and Haplotype Classification) enables probabilistic decision-making. Chapter 7 (Advanced Topics) addresses Cas9 enrichment, CNV, and pangenome extensions. Chapter 8 (Computational Implementation) provides workflow guidance. Chapter 9 (Validation Framework) establishes testing protocols. Chapter 10 (Case Studies) demonstrates practical applications.

\item \textbf{Future-Ready Structure:} Established organizational framework supporting planned additions including technology-specific chapters for Illumina, PacBio, and Oxford Nanopore platforms (Part IV), targeted enrichment method formalization (Part IV), and clinical performance benchmarking with real-world case studies such as the 75-patient Singapore cohort analysis (Part V).
\end{itemize}

\section*{Version 4.2 (Cross-Reference Enhancement)}

\begin{itemize}
\item \textbf{Implemented Comprehensive Cross-Reference System:} Added \texttt{\textbackslash CEref} and \texttt{\textbackslash CEanchor} commands throughout the document, enabling seamless bidirectional navigation between Core Equations and detailed sections.

\item \textbf{Enhanced Log-Base Convention Documentation:} Added explicit log-base reminder before first KL divergence definition, clarifying that all Kullback-Leibler divergences use $\log_2$ and are reported in bits.

\item \textbf{Expanded Library Preparation Documentation:} Added archival guidance for documenting adapter-ligated fraction alongside empirical fragment distribution in run manifests.

\item \textbf{Strengthened Log-Space Implementation Guidance:} Added explicit implementation note emphasizing log-space computation throughout likelihood calculations for numerical stability.

\item \textbf{Created Comprehensive QC Summary Table:} Added formal quality control gate table in Appendix C summarizing four critical checkpoints with explicit thresholds and action criteria.
\end{itemize}

\section*{Version 4.1 (Navigation and Glossary Enhancement)}

\begin{itemize}
\item \textbf{Fixed Hyperlink Navigation System:} Completely restructured all hypertarget placements for CE.1-CE.16, resolving navigation issues.

\item \textbf{Enhanced Page Layout Control:} Systematically increased needspace values for all Core Equation boxes, preventing page breaks.

\item \textbf{Comprehensive Glossary Enhancement:} Significantly expanded Appendix A (Notation Summary) with 3-5× more detailed content including typical value ranges, measurement protocols, and cross-references.

\item \textbf{Added Cross-Reference Network:} Systematically integrated cross-references throughout glossary entries, creating comprehensive navigation network.
\end{itemize}

\section*{Version 4.0 (Core Equations Framework)}

\begin{itemize}
\item \textbf{Enhanced Core Equations:} Expanded all 16 core equations with comprehensive parameter descriptions, usage guidelines, and examples.
\item \textbf{Fixed Hyperlink Navigation:} Added complete hypertarget tags for CE.7-CE.16.
\item \textbf{Improved Page Layout:} Implemented needspace commands before all equation boxes.
\item \textbf{Expanded Appendix Content:} Enhanced Appendices A-C with contextual introductions and implementation guidance.
\end{itemize}

\section*{Version 3.0 (Visual Enhancement)}

\begin{itemize}
\item Enhanced visual formatting with color-coded section headers
\item Improved table of contents with hierarchical numbering
\item Added comprehensive cross-referencing between Core Equations and detailed sections
\item Refined equation box styling with non-breaking containers
\end{itemize}

\section*{Version 2.0 (Core Equations Introduction)}

\begin{itemize}
\item Added Core Equations (Quick Reference) section with 16 essential formulas
\item Integrated empirical fragmentation as primary framework (CE.7-CE.10)
\item Added quality control gates: fragment model KL divergence (CE.9), read-length sanity (CE.10)
\item Formalized quality overstatement metric (CE.14) and purity constraints (CE.15)
\item Aligned notation with Perfect Reads – Ultra tool
\end{itemize}

\section*{Version 1.0 (Initial Release)}

\begin{itemize}
\item Initial release with complete mathematical framework
\item Bayesian hierarchical modeling approach
\item Support for multiple sequencing technologies
\item Clinical validation guidelines
\item Theoretical foundations for all core concepts
\end{itemize}

\section{Summary}

This version history demonstrates the framework's continuous evolution from initial mathematical foundations through comprehensive book-level organization with extensive practical guidance. Version 6.0 represents the most significant reorganization to date, transforming the framework from a research document into an integrated methods and applications volume suitable for clinical deployment, regulatory review, and educational use.
