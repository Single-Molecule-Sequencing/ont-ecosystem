%%%%%%%%%%%%%%%%%%%%%%%%%%%%%%%%%%%%%%%%%%%%%%%%%%%%%%%%%%%%%%%%%%%%%%%%
%% Chapter 16: Bacterial Strain Typing and Microbial Genomics
%% Part VI: Clinical Applications and Case Studies
%% Status: FORTHCOMING
%%%%%%%%%%%%%%%%%%%%%%%%%%%%%%%%%%%%%%%%%%%%%%%%%%%%%%%%%%%%%%%%%%%%%%%%

\chapter{Bacterial Strain Typing and Microbial Genomics}
\label{chap:bacterial}
\label{chap:clinical-applications}

\ChapterForthcomingNotice{This chapter will demonstrate framework application to bacterial genomics and highlight assay adaptations required for microbial laboratories.}{%
\item 16S and 23S rRNA haplotyping for strain discrimination
\item Clinical microbiology workflows and reporting
\item Antibiotic resistance gene phasing and catalog integration
\item Validation results on bacterial reference standards
\item Performance comparison with traditional culture and sequencing
\item Case studies: \textit{E.~coli}, \textit{S.~aureus}, \textit{M.~tuberculosis}
\item Public health surveillance applications and outbreak tracing
}{5--7}

\noindent\textbf{Chapter Objectives}
\begin{itemize}
\item Apply framework to bacterial genomics applications
\item Demonstrate strain-level discrimination capability
\item Validate methodology on microbial reference materials
\item Show clinical utility beyond human pharmacogenomics
\item Establish performance benchmarks for microbial applications
\end{itemize}

\noindent\textbf{Integration with Framework} The methods developed in Parts~II--V generalize beyond pharmacogenes to microbial applications. SMA-seq procedures (Part~IV) and validation protocols (Part~V) apply directly to bacterial reference standards, enabling rigorous accuracy assessment within CLIA-aligned microbiology laboratories.

\section{Bacterial Target Selection and Panel Design}
\label{sec:ch16-panel-design}
\noindent\textbf{Status: Outline.} Specify how to choose loci (16S, 23S, AMR genes) and design assays compatible with the framework. Plan inclusion of tables mapping targets to clinical use cases.
\begin{itemize}
\item Enumerate priority pathogens (CLSI, CDC threat lists) and associated genetic markers.
\item Outline design criteria for amplicon vs. adaptive sampling workflows.
\item Flag requirement to include references to Chapter~\ref{chap:workflow} for pipeline adaptations.
\end{itemize}

\begin{definition}[Microbial Panel Variables]
\textit{Placeholder: Define $C_{\text{strain}}$ (strain identifier), $R_{g}$ (resistance presence indicator), and $D_{\text{AMR}}$ (coverage depth for antimicrobial resistance loci) using the notation rules in Appendix~\ref{app:notation}. Document where each variable is captured in Appendix~\ref{app:protocols} intake forms and what thresholds align with \CEref{11}.}
\end{definition}

\begin{table}[htbp]
\centering
\caption{Placeholder --- Candidate Targets and Validation Requirements}
\label{tab:ch16-target-summary}
\begin{tabular}{lll}
\toprule
\textbf{Target} & \textbf{Clinical Use Case} & \textbf{Planned Actions} \\
\midrule
\textit{16S Hypervariable Loci} & \textit{Baseline strain discrimination} & \textit{Document primer scheme; link to Appendix~\ref{app:protocols}} \\
\textit{AMR Gene $g$} & \textit{Resistance phenotype confirmation} & \textit{Capture QC thresholds via \CEref{11}} \\
\textit{Plasmid Marker $p$} & \textit{Outbreak tracing support} & \textit{Add coverage simulations referencing Chapter~\ref{chap:workflow}} \\
\bottomrule
\end{tabular}
\begin{flushleft}\footnotesize\textit{Replace italics with validated entries after assay design review concludes.}\end{flushleft}
\end{table}

\begin{table}[htbp]
\centering
\caption{Placeholder --- Variable Reference Map}
\label{tab:ch16-variable-map}
\begin{tabular}{llll}
\toprule
\textbf{Variable} & \textbf{Meaning} & \textbf{Appendix Link} & \textbf{Outstanding Task} \\
\midrule
\textit{$C_{\text{strain}}$} & \textit{Strain-level identifier used in concordance tables} & \textit{Appendix~\ref{app:notation}} & \textit{Align with epidemiology codes} \\
\textit{$D_{\text{AMR}}$} & \textit{Depth threshold for AMR locus acceptance} & \textit{Appendix~\ref{app:core-equations}} & \textit{Simulate depth across replicates} \\
\textit{$Q_{\text{workflow}}$} & \textit{Pipeline QC score derived from \CEref{11}} & \textit{Appendix~\ref{app:protocols}} & \textit{Publish calculation checklist} \\
\bottomrule
\end{tabular}
\end{table}


\section{Analytical Validation Strategy}
\label{sec:ch16-validation}
\noindent\textbf{Status: Drafting.} Describe reference materials (ATCC strains, NIST microbial standards), experimental design, and performance metrics (strain-level accuracy, AMR detection sensitivity).
\begin{itemize}
\item Lay out replicate structure (biological vs. technical) and tie accuracy thresholds to Appendix~\ref{app:protocols} QC checkpoints.
\item Specify statistical tests (paired concordance, confidence interval calculations) reusing \CEref{11} likelihood framing for per-strain agreement.
\item Plan tables summarizing AMR gene detection sensitivity with hyperlinks to Appendix~\ref{app:notation} for symbol consistency.
\item Reserve figure slots for Bland--Altman plots and confusion matrices comparing against Chapter~\ref{chap:workflow} pipeline outputs.
\end{itemize}

\begin{eqbox}{Tutorial Placeholder --- Extending \CEref{11} to Strain Concordance}
\textit{Describe how to parameterize likelihood ratios for microbial haplotypes, outlining steps for computing $\Prob(\text{correct strain}\mid D)$ and mapping each term back to Appendix~\ref{app:core-equations}. Include reminder to document variable definitions in Table~\ref{tab:ch16-variable-map}.}
\end{eqbox}

\begin{table}[htbp]
\centering
\caption{Placeholder --- Validation Metric Summary}
\label{tab:ch16-validation-metrics}
\begin{tabular}{llll}
\toprule
\textbf{Metric} & \textbf{Definition Reference} & \textbf{Target Value} & \textbf{Status} \\
\midrule
\textit{Strain Concordance} & \textit{Appendix~\ref{app:notation}, \CEref{11}} & \textit{$\geq 0.95$} & \textit{Pending hybrid validation} \\
\textit{AMR Gene Sensitivity} & \textit{Appendix~\ref{app:core-equations}} & \textit{$\geq 0.98$} & \textit{Add SIR panel data} \\
\textit{Run-to-Run Reproducibility} & \textit{Appendix~\ref{app:protocols}} & \textit{$\leq 5\%$ variance} & \textit{Collect replicate runs} \\
\bottomrule
\end{tabular}
\begin{flushleft}\footnotesize\textit{Update target thresholds once regulatory acceptance criteria are finalized.}\end{flushleft}
\end{table}

\begin{example}[Resistance Coverage Walkthrough]
\textit{Placeholder: Provide a worked example computing $D_{\text{AMR}}$ for a \textit{Klebsiella} isolate, showing each step from raw pileup to Appendix~\ref{app:core-equations} derived confidence intervals.}
\end{example}
\noindent\textbf{Pending Inputs:} Finalize strain manifest, obtain sequencing run metadata from validation teams, and confirm acceptance criteria with clinical microbiology stakeholders.

\section{Clinical and Public Health Case Studies}
\label{sec:ch16-case-studies}
\noindent\textbf{Status: Outline.} Plan narrative vignettes covering hospital outbreak investigation, TB resistance profiling, and foodborne surveillance. Identify figures and tables to summarize timelines and actionable outcomes.
\begin{itemize}
\item Draft subsections for each case study with data requirements (coverage, turnaround, concordance).
\item Note dependencies on forthcoming figures illustrating phylogenetic resolution.
\item Highlight regulatory reporting considerations (CLIA, public health notifications).
\end{itemize}

\begin{table}[htbp]
\centering
\caption{Placeholder --- Case Study Timeline Snapshot}
\label{tab:ch16-case-timeline}
\begin{tabular}{llll}
\toprule
\textbf{Scenario} & \textbf{Key Milestone} & \textbf{Data Artifact} & \textbf{Owner} \\
\midrule
\textit{Hospital Outbreak} & \textit{Day 3 cluster report} & \textit{Phylogenetic tree (Chapter~\ref{chap:workflow})} & \textit{Infection prevention} \\
\textit{TB Resistance} & \textit{Resistance profile sign-out} & \textit{Appendix~\ref{app:protocols} QC log} & \textit{Clinical lab director} \\
\textit{Foodborne Surveillance} & \textit{Regulatory notification} & \textit{Appendix~\ref{app:notation} coding} & \textit{Public health liaison} \\
\bottomrule
\end{tabular}
\end{table}

\section{Operational Considerations for Microbiology Labs}
\label{sec:ch16-operations}
\noindent\textbf{Status: Outline.} Detail workflow modifications, biosafety requirements, and integration with LIMS systems specific to microbiology labs. Reserve tables for SOP alignment with Chapter~\ref{chap:sops}.
\begin{itemize}
\item Draft subsections for biosafety level mapping, contamination control, and waste handling protocols referencing Appendix~\ref{app:protocols} checklists.
\item Capture LIMS interface requirements, including data fields necessary for Appendix~\ref{app:core-equations} driven metrics (coverage, error rates).
\item Prepare staffing matrices linking competencies to Chapter~\ref{chap:sops} training modules.
\end{itemize}
\noindent\textbf{Action Items:} Coordinate with infection prevention to validate biosafety notes and gather example LIMS screenshots for inclusion.

\section{Reference Standards and Appendix Integration}
\label{sec:ch16-standards}
\noindent\textbf{Status: Outline.} List the microbial reference materials, plasmid constructs, and resistance gene panels necessary for validation. Cross-reference Appendix~\ref{app:protocols} for QC checkpoints and Appendix~\ref{app:notation} for symbol consistency when extending \CEref{11} likelihood calculations to microbial targets.
\begin{itemize}
\item Document source repositories (ATCC, BEI Resources) and accession metadata.
\item Plan data tables linking loci to resistance phenotypes and interpretive categories.
\item Track outstanding tasks such as sequencing additional reference strains and validating pipeline parameter presets.
\end{itemize}

\section{Outstanding Tasks and Data Dependencies}
\label{sec:ch16-tasks}
\noindent\textbf{Status: Tracking.} Summarize remaining work items including figure drafts for Section~\ref{sec:ch16-case-studies}, collaboration with infection prevention teams, and gap analyses for appendix cross-links. Maintain a rolling timeline to ensure content readiness aligns with Part VI publication milestones.
\begin{enumerate}[label=\textbf{T\arabic*}]
\item Draft phylogenetic resolution figure (Section~\ref{sec:ch16-case-studies}); assign data scientist and due date.
\item Compile biosafety documentation outlined in Section~\ref{sec:ch16-operations}; confirm with compliance officer.
\item Verify appendix hyperlinks (\CEref{11}, Appendix~\ref{app:protocols}, Appendix~\ref{app:notation}) once content is populated.
\item Collect turnaround-time metrics from pilot sites to populate operational tables.
\end{enumerate}
\noindent\textbf{Risks:} Outstanding IRB approval for hospital outbreak case study and pending delivery of additional reference strain sequencing runs.

\clearpage
