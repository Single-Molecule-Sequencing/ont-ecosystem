%%%%%%%%%%%%%%%%%%%%%%%%%%%%%%%%%%%%%%%%%%%%%%%%%%%%%%%%%%%%%%%%%%%%%%%%
%% Chapter 8: Building Plasmid Standards and Quality Control
%% Part III: Physical Standards, Library Preparation, and Workflows
%% Version 6.0 - Populated from v5 + NEW content
%%%%%%%%%%%%%%%%%%%%%%%%%%%%%%%%%%%%%%%%%%%%%%%%%%%%%%%%%%%%%%%%%%%%%%%%

\chapter{Building Plasmid Standards and Quality Control}
\label{chap:plasmid-standards}
\label{chap:standards}
\label{chap:plasmids}

Physical standards with known sequence identity form the foundation of empirical error measurement in the SMA-seq methodology (Chapter~\ref{chap:sma-seq}). The ability to confidently assert that a population of molecules possesses a specific sequence enables direct measurement of sequencing platform systematic errors through the SEER framework. This chapter details the complete workflow for constructing, verifying, and validating plasmid-based standards that serve as purity-bounded ground truth for confusion matrix estimation.

The choice of plasmid-based standards offers multiple practical advantages: (1) bacterial amplification produces unlimited quantities of identical molecules, (2) circular topology enables complete coverage in long-read sequencing without end-bias artifacts, (3) Sanger sequencing of individual colonies provides independent verification of sequence identity, and (4) standardized cloning workflows ensure reproducibility across laboratories. By constructing standards representing each target haplotype, we create the reference materials necessary for measuring platform-specific error rates.

\begin{learningobjectives}
By the end of this chapter, you will be able to:
\begin{itemize}
\item Design and execute Golden Gate assembly for scarless construction of haplotype standards
\item Select appropriate Type IIS restriction enzymes (BsaI, BsmBI) based on insert compatibility
\item Implement multi-level sequence verification protocols (Sanger, restriction digest, technical replication)
\item Choose optimal bacterial strains (e.g., JM109, DH5$\alpha$) to minimize methylation artifacts
\item Quantify plasmid purity $\pi \geq 0.995$ using replication-based bounds $\pi_{\text{upper}}(k,L,r)$
\item Integrate plasmid standards into the SEER confusion matrix estimation workflow
\item Troubleshoot common cloning failures and sequence verification discrepancies
\end{itemize}
\end{learningobjectives}

\section{Golden Gate Assembly Strategy}

Golden Gate assembly exploits Type IIS restriction enzymes that cleave DNA outside their recognition sequences, enabling scar-free assembly of multiple fragments with defined junction sequences. This methodology provides superior flexibility compared to traditional restriction cloning while maintaining high efficiency and specificity.

\subsection{Design Principles for Haplotype Inserts}

When constructing standards for haplotype classification, the insert sequence should encompass:

\begin{enumerate}
\item \textbf{Complete variant region:} Include all single nucleotide variants (SNVs), insertions, and deletions that distinguish the haplotype from reference and other alleles. For pharmacogenes like \textit{CYP2D6}, this typically spans 2-5 kb including exons and critical intronic regions.

\item \textbf{Flanking sequence context:} Extend 200-500 bp beyond the variant region on each end to provide alignment anchors and enable validation of junction integrity.

\item \textbf{Avoidance of internal Type IIS sites:} Survey the haplotype sequence for BsaI (5'-GGTCTC-3') and BsmBI (5'-CGTCTC-3') recognition sequences. If present within the target region, either:
\begin{itemize}
\item Select an alternative Type IIS enzyme (e.g., BspQI, SapI)
\item Use site-directed mutagenesis to create synonymous variants
\item Employ a two-step assembly strategy with intervening purification
\end{itemize}

\item \textbf{Optimal insert length:} For single-molecule sequencing, target 3-6 kb inserts that match typical read lengths. Longer inserts may be desirable for capturing distant variants but complicate synthesis and verification.
\end{enumerate}

\subsection{Type IIS Restriction Enzyme Selection}

The two most common enzymes for Golden Gate assembly are:

\begin{table}[!htbp]
\centering
\caption{Type IIS Enzyme Comparison for Golden Gate Assembly}
\label{tab:typeIIS-enzymes}
\begin{tabular}{lll}
\toprule
\textbf{Enzyme} & \textbf{Recognition Site} & \textbf{Characteristics} \\
\midrule
BsaI & 5'-GGTCTC(N)$_1$-3' & Most common; cleaves 1 bp downstream \\
BsmBI & 5'-CGTCTC(N)$_1$-3' & Alternative to BsaI; similar activity \\
BbsI & 5'-GAAGAC(N)$_2$-3' & Cleaves 2 bp downstream; longer overhangs \\
SapI & 5'-GCTCTTC(N)$_1$-3' & High fidelity; 7 bp recognition \\
\bottomrule
\end{tabular}
\end{table}

\textbf{Recommended protocol:} Use BsaI for initial attempts unless the insert contains internal BsaI sites. BsaI is well-established, inexpensive, and compatible with most commercial Golden Gate kits.

\subsection{Vector Selection and Linearization}

Select a destination vector containing:
\begin{itemize}
\item Type IIS recognition sites flanking the insertion site
\item Antibiotic resistance gene (typically ampicillin or kanamycin)
\item High-copy origin of replication (e.g., pUC or ColE1)
\item Negative selection marker (e.g., \textit{ccdB}, \textit{lacZ$\alpha$}) for background reduction
\end{itemize}

The pGGA vector series and commercial kits (e.g., NEB Golden Gate Assembly Kit, Addgene's pGGAselect) provide validated backbones. Linearize the vector through a Golden Gate digestion-ligation reaction that excises the negative selection marker and creates overhangs complementary to the insert.

\subsection{Assembly Reaction Optimization}

\textbf{Standard Golden Gate Reaction (20 µL):}

\begin{table}[!htbp]
\centering
\caption{Golden Gate Assembly Reaction Components}
\begin{tabular}{ll}
\toprule
\textbf{Component} & \textbf{Amount} \\
\midrule
Linearized vector & 50-100 ng \\
Insert PCR product or gene fragment & 2-3:1 molar ratio to vector \\
T4 DNA ligase buffer (10×) & 2 µL \\
BsaI-HF v2 (20 U/µL) & 0.5 µL (10 U total) \\
T4 DNA ligase (400 U/µL) & 0.5 µL (200 U total) \\
Nuclease-free water & to 20 µL \\
\bottomrule
\end{tabular}
\end{table}

\textbf{Thermocycler program:}
\begin{enumerate}
\item 37°C for 5 minutes (digestion)
\item 16°C for 5 minutes (ligation)
\item Repeat steps 1-2 for 25-50 cycles
\item 60°C for 5 minutes (enzyme inactivation)
\item 80°C for 10 minutes (heat kill - optional)
\item Hold at 4°C
\end{enumerate}

The cycling conditions allow iterative digestion and ligation, enabling assembly even when initial product is imperfect. More cycles generally improve yield but may increase off-target products.

\subsection{Transformation and Colony Screening}

\textbf{Transformation Protocol:}
\begin{enumerate}
\item Add 2-5 µL of Golden Gate reaction product to 50 µL chemically competent cells
\item Incubate on ice for 30 minutes
\item Heat shock at 42°C for 30-45 seconds
\item Return to ice for 2 minutes
\item Add 250 µL SOC medium
\item Recover at 37°C with shaking for 1 hour
\item Plate 50-200 µL on selective LB-agar plates
\item Incubate overnight at 37°C
\end{enumerate}

\textbf{Colony PCR Screening:}
\begin{itemize}
\item Design primers flanking the insert (one in vector backbone, one in insert)
\item Expected product size = insert length + distance between primers
\item Screen 8-16 colonies by colony PCR
\item Select 3-4 clones showing correct insert size for miniprep and Sanger verification
\end{itemize}

\section{Sequence Verification Protocols}

\subsection{Sanger Sequencing Coverage Strategy}

Complete verification of insert identity requires Sanger sequencing with redundant coverage:

\begin{protocol}[Sanger Verification Protocol]
\textbf{Primer Design:}
\begin{enumerate}
\item Design primers every 700-800 bp along insert
\item Include reverse primers offset by ~400 bp for bidirectional coverage
\item Extend primers 200 bp into vector backbone at junctions
\item Ensure 2× coverage minimum (forward + reverse reads)
\end{enumerate}

\textbf{Quality Criteria:}
\begin{itemize}
\item Phred20 quality or better across variant positions
\item No ambiguous base calls (N) in target region
\item Perfect match to intended haplotype sequence
\item Junction sequences match design (no mutations introduced)
\end{itemize}

\textbf{Interpretation:}
\begin{itemize}
\item Single SNV from expected: likely PCR error during amplification; re-sequence or discard clone
\item Multiple differences: wrong template or contamination; discard clone
\item Perfect match across 3+ independent reads: high confidence in sequence identity
\end{itemize}
\end{protocol}

\textbf{Recommendation:} Sequence 3-5 independent clones for each haplotype. Even with careful verification, low-frequency variants may exist below Sanger detection limits (~10-20\%). Sequencing multiple clones and selecting those with perfect concordance minimizes this risk.

\subsection{Restriction Enzyme Verification}

Orthogonal QC through restriction digest analysis provides independent confirmation:

\begin{enumerate}
\item \textbf{In silico prediction:} Use tools like NEBcutter or SnapGene to predict restriction digest patterns for both:
\begin{itemize}
\item Correct haplotype insert
\item Common off-target sequences (e.g., other alleles, partial deletions)
\end{itemize}

\item \textbf{Enzyme selection:} Choose 2-3 enzymes that produce diagnostic digest patterns:
\begin{itemize}
\item Different fragment sizes for target vs. off-target
\item 3-6 fragments for resolution on standard agarose gels
\item Avoid enzymes sensitive to methylation (see Section~\ref{sec:strain-selection})
\end{itemize}

\item \textbf{Digest protocol:} Follow manufacturer recommendations for each enzyme
\begin{itemize}
\item Use 500 ng - 1 µg plasmid DNA
\item Complete digestion (2-4 hours or overnight)
\item Include uncut control for size comparison
\end{itemize}

\item \textbf{Gel analysis:} Run on 0.8-1.2\% agarose gel with appropriate ladder
\begin{itemize}
\item Verify fragment number and sizes match prediction
\item Estimate band intensities (should be proportional to fragment length)
\item Compare across multiple clones for consistency
\end{itemize}
\end{enumerate}

\textbf{Acceptance criteria:} Digest pattern must match in silico prediction for all tested enzymes. Any discrepancies require investigation through additional Sanger sequencing or whole-plasmid sequencing.

\section{E. coli Strain Selection}
\label{sec:strain-selection}

Bacterial methylation systems can interfere with single-molecule sequencing basecalling by introducing modified bases that alter electrical or optical signals.

\subsection{Methylation Considerations}

Standard \textit{E. coli} laboratory strains contain two methylase systems:

\begin{table}[!htbp]
\centering
\caption{E. coli Methylation Systems}
\label{tab:ecoli-methylation}
\begin{tabular}{llll}
\toprule
\textbf{System} & \textbf{Modification} & \textbf{Sequence} & \textbf{Frequency} \\
\midrule
Dam methylase & N$^6$-methyladenine (6mA) & 5'-G$\underline{A}$TC-3' & ~1 per 256 bp \\
Dcm methylase & 5-methylcytosine (5mC) & 5'-C$\underline{C}$WGG-3' & ~1 per 400-600 bp \\
\bottomrule
\end{tabular}
\end{table}

\subsection{Strain Comparison}

\begin{table}[!htbp]
\centering
\caption{Recommended E. coli Strains for Plasmid Standards}
\begin{tabular}{lll}
\toprule
\textbf{Strain} & \textbf{Genotype} & \textbf{Considerations} \\
\midrule
DH5$\alpha$ & Dam$^+$ Dcm$^+$ & Standard cloning; may affect basecalling \\
JM110 & Dam$^-$ Dcm$^-$ & No methylation; lower yield \\
SCS110 & Dam$^-$ Dcm$^-$ & Commercial; good yield \\
TOP10 & Dam$^+$ Dcm$^+$ & High transformation efficiency \\
\bottomrule
\end{tabular}
\end{table}

\textbf{Recommendation for SMS applications:}
\begin{itemize}
\item \textbf{Oxford Nanopore:} Use Dam$^-$/Dcm$^-$ strains (JM110, SCS110) if basecaller was not trained on methylated DNA. Current ONT basecallers can handle methylation but older models may show systematic errors at GATC and CCWGG motifs.

\item \textbf{PacBio HiFi:} Standard strains acceptable. PacBio polymerase passes through methylated bases with minimal signal perturbation due to processive kinetics.

\item \textbf{Unknown/mixed samples:} Use Dam$^-$/Dcm$^-$ strains to avoid potential confounding. The yield penalty is acceptable for small-scale standard production.
\end{itemize}

\subsection{Growth and Isolation Protocol}

\begin{protocol}[Plasmid Standard Production]
\textbf{Culture Conditions:}
\begin{enumerate}
\item Inoculate verified clone into 5 mL LB + antibiotic
\item Grow overnight at 37°C with shaking (250 rpm)
\item Dilute 1:1000 into 100-500 mL fresh LB + antibiotic
\item Grow to mid-log phase (OD$_{600}$ = 0.4-0.6)
\item Harvest by centrifugation
\end{enumerate}

\textbf{Plasmid Isolation:}
\begin{itemize}
\item Use commercial midiprep or maxiprep kits (Qiagen, Promega, Zymo)
\item Follow manufacturer protocols for high-copy plasmids
\item Expected yield: 50-200 µg from 100 mL culture
\item Store at -20°C in 10 mM Tris-HCl pH 8.0 or TE buffer
\end{itemize}

\textbf{Quality Control:}
\begin{itemize}
\item Measure concentration by fluorometry (e.g., Qubit dsDNA assay)
\item Assess purity: A260/A280 ratio should be 1.8-2.0
\item Run uncut plasmid on agarose gel: should show primarily supercoiled form
\item Optional: Whole-plasmid Sanger sequencing or long-read sequencing for ultimate validation
\end{itemize}
\end{protocol}

\section{Purity Estimation for Standards}

While plasmid standards provide high sequence homogeneity, perfect purity cannot be assumed. Chapter~\ref{chap:purity} established that purity $\pi$ represents the fraction of molecules matching the intended sequence, and that classification accuracy cannot exceed purity (Theorem~\ref{thm:purity-ceiling}).

\subsection{Sources of Impurity in Plasmid Standards}

\begin{enumerate}
\item \textbf{PCR errors during insert amplification:} Error rate $\sim$10$^{-6}$ per base per cycle; 30 cycles introduces $\sim$0.003\% variants

\item \textbf{Spontaneous mutations during bacterial growth:} \textit{E. coli} mutation rate $\sim$10$^{-9}$ per base per generation; 30 generations introduces $\sim$0.00003\% variants

\item \textbf{Residual parental plasmid:} If negative selection is incomplete, background may contain 0.1-1\% original vector

\item \textbf{Contamination during handling:} Cross-contamination between standards during quantification or library preparation
\end{enumerate}

\subsection{Lower Bound via Technical Replication}

Following the methodology in Section~\ref{sec:replication-purity} (Chapter~\ref{chap:purity}), technical replicates provide an empirical lower bound on purity:

\begin{equation}
\pi \geq \frac{\text{\# reads agreeing across replicates}}{\text{Total reads}}
\label{eq_8_1}
\end{equation}

For plasmid standards, sequence 3-5 technical replicates (separate library preparations from same plasmid prep). Agreement rate typically exceeds 99.9\%, providing $\pi \geq 0.999$.

\subsection{Upper Bound via Sanger Sequencing}

Sanger sequencing detection limit provides an upper bound. If no variants are detected in Sanger traces:
\begin{itemize}
\item Single-base substitution detection: ~10-20\% minor allele
\item Implies purity $\pi \leq 0.8$-0.9 based on Sanger alone
\end{itemize}

However, sequencing multiple independent clones and selecting those with perfect Sanger concordance effectively pushes this bound higher. If 5 independent clones all show perfect match, probability that a common contaminant exists at 20\% is extremely low ($<$0.2$^5$ = 0.00032).

\textbf{Conservative estimate:} Combine Sanger verification of multiple clones with technical replication to assert $\pi \geq 0.995$ for well-constructed plasmid standards.

\section{Using Standards as Ground Truth}

Standards enable confusion matrix estimation through the SEER framework (Chapter~\ref{chap:sma-seq}). The workflow is:

\begin{enumerate}
\item \textbf{Construct standards:} One plasmid for each target haplotype $h_i$

\item \textbf{Sequence standards:} Process through complete experimental workflow (fragmentation, library prep, sequencing, basecalling) using the same protocol as clinical samples

\item \textbf{Extract observed sequences:} For each read $r$ from standard $h_i$, record the basecalled sequence

\item \textbf{Compute confusion matrix element:}
\begin{equation}
C_{ij} = \Prob(\text{observe } r \in R_j \mid \text{true sequence from } h_i)
\end{equation}
Empirically estimated as:
\begin{equation}
\hat{C}_{ij} = \frac{\text{\# reads from standard } h_i \text{ classified as } h_j}{\text{Total reads from standard } h_i}
\end{equation}

\item \textbf{Validate purity constraint:} For each standard $i$, verify $\hat{C}_{ii} \leq \pi_i$ where $\pi_i$ is the estimated purity. Violations indicate systematic problems (see Chapter~\ref{chap:qc-gates}, Gate 4).
\end{enumerate}

The quality of confusion matrix estimates depends directly on standard purity. Impurities introduce systematic bias: apparent off-diagonal elements $C_{ij}$ may actually reflect true variants in the standard rather than sequencing errors. This motivates the investment in high-quality standard construction and validation.

\section{Troubleshooting Guide}

\begin{table}[!htbp]
\centering
\caption{Common Issues in Standard Construction}
\label{tab:standard-troubleshooting}
\small
\begin{tabular}{p{3.5cm}p{4cm}p{6cm}}
\toprule
\textbf{Problem} & \textbf{Likely Cause} & \textbf{Solution} \\
\midrule
No colonies after transformation & Inefficient ligation; incorrect antibiotic & Check insert:vector ratio (should be 2-3:1 molar); verify antibiotic matches plasmid resistance gene \\
\midrule
Colonies but all negative by PCR & Self-ligation of vector; incorrect primers & Increase insert:vector ratio; verify primers by BLAST; use vector with negative selection marker \\
\midrule
Insert present but wrong size & Off-target amplification; partial deletion & Re-design primers with higher specificity; use high-fidelity polymerase; gel-purify PCR product \\
\midrule
Sanger sequence doesn't match expected & Wrong template; PCR error; synthesis error & Sequence multiple clones; if consistent across clones, verify input sequence was correct \\
\midrule
Restriction digest unexpected pattern & Methylation blocking enzyme; partial digest & Use methylation-insensitive enzyme; increase enzyme amount or incubation time; verify enzyme is active \\
\midrule
Low plasmid yield from Dam$^-$ strains & Strain-specific growth deficiency & Increase culture volume; use richer media; try alternative Dam$^-$ strain (JM110 vs. SCS110) \\
\bottomrule
\end{tabular}
\end{table}

\section{Chapter Summary}

\begin{keytakeaways}
This chapter established comprehensive protocols for constructing high-purity plasmid standards:

\textbf{Assembly Methodology:}
\begin{itemize}
\item \textbf{Golden Gate Strategy:} Type IIS restriction enzymes (BsaI at 37°C, BsmBI at 55°C) enable scarless assembly with 4-bp overhangs, achieving >90\% assembly efficiency when properly designed

\item \textbf{Insert Design Principles:} Complete target region coverage (all variant positions), flanking sequences for primer design, avoid internal Type IIS sites, optimize overhang compatibility for directional assembly

\item \textbf{Reaction Optimization:} One-pot assembly with T4 DNA ligase (37°C, 2-4 hours with 5-minute cycles), followed by heat inactivation and transformation
\end{itemize}

\textbf{Multi-Level Verification:}
\begin{itemize}
\item \textbf{Sanger Sequencing:} Primary verification of individual clones (5-10 colonies per construct) with bidirectional coverage ensuring complete insert accuracy

\item \textbf{Restriction Digest:} Orthogonal size verification using diagnostic restriction patterns, distinguishing correct inserts from common artifacts

\item \textbf{Technical Replication:} Independent clone selection and sequencing confirms consistency, detecting rare systematic errors in initial verification
\end{itemize}

\textbf{Bacterial Strain Selection:}
\begin{itemize}
\item \textbf{Methylation Considerations:} Dam/Dcm methylation in common strains affects nanopore basecalling at GATC/CCWGG motifs

\item \textbf{Recommended Strains:} JM109 (dam$^{-}$ dcm$^{-}$) for methylation-sensitive work, DH5$\alpha$ (dam$^{+}$ dcm$^{+}$) for standard applications with methylation calibration

\item \textbf{Platform-Specific Optimization:} Match strain methylation status to target application (clinical samples are naturally methylated)
\end{itemize}

\textbf{Purity Quantification:}
\begin{itemize}
\item \textbf{Conservative Bounds:} Replication model $\pi_{\text{upper}}(k,L,r) = (1-r)^{kL}$ provides theoretical upper bound

\item \textbf{Practical Estimates:} For 20 replication cycles, 5 kb insert, $r = 10^{-9}$ per-base error rate: $\pi \geq 0.9999$ (Equation~\ref{eq_8_1})

\item \textbf{Validation Requirements:} Purity $\pi \geq 0.995$ ensures standards support high-confidence confusion matrix estimation
\end{itemize}

\textbf{Integration with SEER Framework:}
\begin{itemize}
\item \textbf{Confusion Matrix Estimation:} High-purity standards enable direct measurement of basecaller error rates (Chapter~\ref{chap:sma-seq})

\item \textbf{Likelihood Calibration:} Accurate confusion matrices improve posterior computation (Chapter~\ref{chap:posteriors})

\item \textbf{Quality Control Gates:} Standards validate QC thresholds and detect systematic platform drift (Chapter~\ref{chap:qc-gates})
\end{itemize}

\textbf{Investment Payoff:} High-quality standard construction is labor-intensive but pays dividends throughout the framework. Accurate confusion matrices improve classification accuracy, enable detection of basecaller miscalibration, and provide empirical foundations for clinical validation. Without well-characterized standards, the entire inference pipeline rests on unjustified assumptions rather than measured reality.

\textbf{Troubleshooting:} Common failures include Type IIS site compatibility issues (check insert sequences), low transformation efficiency (optimize competent cell quality), and sequence verification discrepancies (ensure bidirectional coverage, check for PCR errors).

\textbf{Mathematical reference:} For purity bound derivations and replication models, see Appendices~\ref{app:mathematical-models} (Section 7) and \ref{app:core-equations}. For variable definitions, see Appendix~\ref{app:variable-master}.
\end{keytakeaways}
