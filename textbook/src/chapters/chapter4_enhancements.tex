%%%%%%%%%%%%%%%%%%%%%%%%%%%%%%%%%%%%%%%%%%%%%%%%%%%%%%%%%%%%%%%%%%%%%%%%
%% Chapter 4 Enhancements: Tables, Examples, and Tutorials
%% TO BE INTEGRATED INTO chapter4_populated.tex
%%%%%%%%%%%%%%%%%%%%%%%%%%%%%%%%%%%%%%%%%%%%%%%%%%%%%%%%%%%%%%%%%%%%%%%%

\section{Quick Reference: Key Variables and Notation}
\label{sec:ch4-notation}

Table~\ref{tab:ch4-variables} provides a quick reference for all variables introduced in this chapter, with links to detailed definitions and the appendix notation guide (Appendix~\ref{app:notation}).

\begin{table}[H]
\centering
\caption{Quick Reference: Variables in the Classification Model}
\label{tab:ch4-variables}
\small
\begin{tabular}{clp{5cm}c}
\toprule
\textbf{Symbol} & \textbf{Name} & \textbf{Definition} & \textbf{Reference} \\
\midrule
$\mathcal{H}$ & Haplotype space & Set of candidate haplotype sequences & Def.~\ref{def:state-space-hierarchy} \\
$h_i$ & Haplotype $i$ & A specific candidate sequence & -- \\
$P$ & Haplotype count & Total number of candidates & -- \\
\midrule
$\mathcal{R}$ & Read space & Set of all observed sequencing reads & Def.~\ref{def:state-space-hierarchy} \\
$r_j$ & Read $j$ & A single basecalled sequence & -- \\
$n$ & Read count & Total number of reads observed & -- \\
\midrule
$\sigma$ & Signal & Continuous sensor output (current or fluorescence) & Def.~\ref{def:signal-emission} \\
$x_t$ & Signal at time $t$ & Measurement at specific timepoint & -- \\
$T$ & Signal length & Number of timepoints in signal & -- \\
\midrule
$Q_j$ & Quality score & Phred-scaled error probability for position $j$ & Def.~\ref{def:quality-score-model} \\
$\mathbf{Q}$ & Quality vector & Quality scores for all positions in a read & -- \\
\midrule
$\mathbf{C}$ & Confusion matrix & Empirical error probabilities $\Prob(\text{obs}|\text{true})$ & Def.~\ref{def:confusion-matrix} \\
$C_{ij}$ & Confusion entry & Probability of observing $i$ given true $j$ & -- \\
\midrule
$d_{\text{edit}}(r,t)$ & Edit distance & Minimum edits to transform $r$ to $t$ & Def.~\ref{def:levenshtein-distance} \\
$|r|$ & Sequence length & Number of bases in read $r$ & -- \\
\midrule
$\mathcal{A}$ & Nucleotide alphabet & $\{A, C, G, T\}$ & -- \\
$\mathbb{I}[\cdot]$ & Indicator function & 1 if condition true, 0 otherwise & -- \\
\bottomrule
\end{tabular}
\end{table}

\noindent\textbf{Cross-Reference:} For comprehensive notation including mathematical operators, see Appendix~\ref{app:notation}. For core equation quick links, use \CEref{1} through \CEref{15}.

\section{Worked Example: Complete Classification Pipeline}
\label{sec:ch4-worked-example}

\noindent This section presents a numerical walkthrough of the complete pipeline from a true haplotype to observed read, illustrating every transformation with concrete values.

\subsection{Setup: Simple Two-Haplotype System}

Consider a simple pharmacogene locus with two candidate haplotypes:
\begin{itemize}
    \item $h_1$: Reference allele with sequence \texttt{ACGTACGT} (8 bases)
    \item $h_2$: Variant allele with sequence \texttt{ACGTCCGT} (SNP: A$\to$C at position 5)
\end{itemize}

Our task: given an observed read $r = \texttt{ACGTCCGT}$, compute the likelihood $\Prob(r|h_1)$ and $\Prob(r|h_2)$.

\subsection{Step 1: Signal Generation}

For this example, assume nanopore sequencing where each base generates a characteristic current level. Suppose the signal model (Definition~\ref{def:signal-emission}) produces:

\begin{itemize}
    \item True sequence \texttt{ACGTACGT} $\to$ expected signal: [110, 95, 130, 105, 110, 95, 130, 105] pA
    \item True sequence \texttt{ACGTCCGT} $\to$ expected signal: [110, 95, 130, 105, 95, 95, 130, 105] pA
\end{itemize}

The difference occurs at position 5, where A (110 pA) is replaced by C (95 pA). Real signals include noise:

\vspace{0.3cm}
\noindent Observed signal (with Gaussian noise, $\sigma = 5$ pA):
\[
\sigma_{\text{obs}} = [112, 93, 128, 107, 97, 94, 131, 103] \text{ pA}
\]

\subsection{Step 2: Basecalling}

The basecaller (e.g., Dorado) processes $\sigma_{\text{obs}}$ and produces:
\begin{itemize}
    \item Called sequence: $r = \texttt{ACGTCCGT}$
    \item Quality scores: $\mathbf{Q} = [30, 28, 32, 29, 25, 27, 31, 30]$
\end{itemize}

Position 5 has lower quality (Q25) because the observed signal (97 pA) falls between expected values for A (110 pA) and C (95 pA), creating ambiguity.

\subsection{Step 3: Likelihood Calculation via Quality Scores}

Using the quality score model (Definition~\ref{def:quality-score-model}), we compute the likelihood of observing $r$ given each true haplotype.

\textbf{For $h_1$ (\texttt{ACGTACGT}):}

Positions 1-4 match, position 5 is a mismatch (observed C vs. true A), positions 6-8 match.

\begin{align*}
\Prob(r|h_1) &= \prod_{j=1}^{8} \Prob(r_j | h_{1,j}, Q_j) \\
&= (1 - 10^{-30/10})(1 - 10^{-28/10})(1 - 10^{-32/10})(1 - 10^{-29/10}) \\
&\quad \times \left(\frac{10^{-25/10}}{3}\right) \times (1 - 10^{-27/10})(1 - 10^{-31/10})(1 - 10^{-30/10}) \\
&= (0.999)(0.998)(0.9994)(0.999) \times (0.00105) \times (0.998)(0.9992)(0.999) \\
&\approx 0.001042
\end{align*}

The low probability is driven by the mismatch at position 5, where $\Prob(\text{error}) = 10^{-25/10} = 0.00316$ is divided by 3 (assuming uniform error distribution over incorrect bases).

\textbf{For $h_2$ (\texttt{ACGTCCGT}):}

Perfect match at all positions!

\begin{align*}
\Prob(r|h_2) &= \prod_{j=1}^{8} (1 - 10^{-Q_j/10}) \\
&= (0.999)(0.998)(0.9994)(0.999)(0.9968)(0.998)(0.9992)(0.999) \\
&\approx 0.9898
\end{align*}

\subsection{Step 4: Likelihood Ratio and Interpretation}

The likelihood ratio strongly favors $h_2$:
\[
\frac{\Prob(r|h_2)}{\Prob(r|h_1)} = \frac{0.9898}{0.001042} \approx 950
\]

This read provides 950:1 evidence in favor of the variant haplotype $h_2$ versus the reference $h_1$. In a Bayesian framework (Chapter~\ref{chap:posteriors}), this likelihood ratio updates our prior belief $\Prob(h_2)/\Prob(h_1)$ to obtain the posterior $\Prob(h_2|r)/\Prob(h_1|r)$.

If the prior odds were 1:1 (equal population frequency), the posterior odds would be 950:1, translating to:
\[
\Prob(h_2|r) = \frac{950}{950 + 1} \approx 0.999
\]

A single high-quality read is nearly decisive evidence for the variant haplotype in this simple two-candidate case.

\subsection{Step 5: Effect of Quality Scores}

The importance of quality scores becomes clear when we compare the same read with degraded quality:

\vspace{0.3cm}
\noindent\textbf{Scenario: Lower quality at position 5} ($Q_5 = 15$ instead of 25)

For $h_1$ (mismatch at position 5):
\[
\Prob(r|h_1) = 0.994^4 \times \frac{10^{-15/10}}{3} \times 0.998^3 \approx 0.0104
\]

For $h_2$ (perfect match):
\[
\Prob(r|h_2) = 0.994^4 \times (1 - 10^{-15/10}) \times 0.998^3 \approx 0.9586
\]

Likelihood ratio:
\[
\frac{\Prob(r|h_2)}{\Prob(r|h_1)} \approx \frac{0.9586}{0.0104} \approx 92
\]

The evidence weakens dramatically (950:1 $\to$ 92:1) when quality degrades from Q25 to Q15, demonstrating why quality-aware models are essential.

\section{Tutorial: Constructing Confusion Matrices}
\label{sec:ch4-confusion-tutorial}

\subsection{Single-Base Confusion Matrix}

Suppose we sequence a plasmid standard with known sequence \texttt{AAACCCGGGTTTT} (4 of each base) 1,000 times. We count observed vs. true bases:

\begin{table}[H]
\centering
\caption{Single-Base Confusion Matrix Example (counts from 1,000 reads)}
\label{tab:ch4-confusion-counts}
\begin{tabular}{l|cccc|c}
\toprule
\textbf{Observed $\backslash$ True} & \textbf{A} & \textbf{C} & \textbf{G} & \textbf{T} & \textbf{Total Obs.} \\
\midrule
A & 3,920 & 5 & 8 & 12 & 3,945 \\
C & 8 & 3,880 & 6 & 7 & 3,901 \\
G & 10 & 9 & 3,890 & 5 & 3,914 \\
T & 12 & 6 & 6 & 3,876 & 3,900 \\
\midrule
\textbf{Total True} & 3,950 & 3,900 & 3,910 & 3,900 & 15,660 \\
\bottomrule
\end{tabular}
\end{table}

Normalizing columns to sum to 1 gives the confusion matrix $\mathbf{C}$:

\begin{table}[H]
\centering
\caption{Normalized Single-Base Confusion Matrix $\mathbf{C}$}
\label{tab:ch4-confusion-matrix}
\begin{tabular}{l|cccc}
\toprule
\textbf{Observed $\backslash$ True} & \textbf{A} & \textbf{C} & \textbf{G} & \textbf{T} \\
\midrule
A & 0.9924 & 0.0013 & 0.0020 & 0.0031 \\
C & 0.0020 & 0.9949 & 0.0015 & 0.0018 \\
G & 0.0025 & 0.0023 & 0.9949 & 0.0013 \\
T & 0.0030 & 0.0015 & 0.0015 & 0.9938 \\
\bottomrule
\end{tabular}
\end{table}

\textbf{Interpretation:}
\begin{itemize}
    \item Diagonal elements ($\sim$0.99) represent accuracy for each base
    \item Off-diagonal elements ($\sim$0.001--0.003) show substitution rates
    \item Row sums $\neq 1$ because the matrix is column-normalized (each column is $\Prob(\text{obs}|\text{true})$)
\end{itemize}

\subsection{Using the Confusion Matrix for Likelihood}

Given a read $r = \texttt{ACG}$ and true sequence $h = \texttt{ACG}$, the likelihood is:

\begin{align*}
\Prob(r|h) &= C_{AA} \times C_{CC} \times C_{GG} \\
&= 0.9924 \times 0.9949 \times 0.9949 \\
&\approx 0.9823
\end{align*}

For a mismatched read $r' = \texttt{ACT}$ versus $h = \texttt{ACG}$:

\begin{align*}
\Prob(r'|h) &= C_{AA} \times C_{CC} \times C_{TG} \\
&= 0.9924 \times 0.9949 \times 0.0015 \\
&\approx 0.00148
\end{align*}

The likelihood ratio is:
\[
\frac{\Prob(r|h)}{\Prob(r'|h)} = \frac{0.9823}{0.00148} \approx 664
\]

The perfect-match read is 664× more likely than the single-mismatch read under this confusion matrix.

\subsection{Context-Dependent Errors: Homopolymer Example}

Single-base confusion matrices miss context-dependent errors. For homopolymers, consider sequencing \texttt{AAAA} (4 A's):

\begin{table}[H]
\centering
\caption{Homopolymer Confusion: Observed Lengths for True \texttt{AAAA}}
\label{tab:ch4-homopolymer}
\begin{tabular}{lcc}
\toprule
\textbf{Observed Sequence} & \textbf{Count (out of 100)} & \textbf{Probability} \\
\midrule
\texttt{AAA} (deletion) & 8 & 0.08 \\
\texttt{AAAA} (correct) & 78 & 0.78 \\
\texttt{AAAAA} (insertion) & 12 & 0.12 \\
\texttt{AAAAAA} (2 insertions) & 2 & 0.02 \\
\bottomrule
\end{tabular}
\end{table}

Homopolymer indel rate (20\%) far exceeds single-base substitution rate ($\sim$1\%), illustrating why k-mer confusion matrices (capturing context) outperform single-base models for nanopore data.

\section{Summary Table: Pipeline Transformations}
\label{sec:ch4-pipeline-summary}

Table~\ref{tab:ch4-pipeline-summary} consolidates all transformations in the sequencing pipeline with their mathematical representations and key parameters.

\begin{table}[H]
\centering
\caption{Complete Sequencing Pipeline Transformations}
\label{tab:ch4-pipeline-summary}
\small
\begin{tabular}{p{2.5cm}p{3cm}p{5.5cm}p{2.5cm}}
\toprule
\textbf{Stage} & \textbf{Transformation} & \textbf{Probabilistic Model} & \textbf{Key Parameters} \\
\midrule
Haplotype Prior & Population $\to$ Sample & $\Prob(h)$ from allele frequency databases & Population freq., ancestry \\
\midrule
Genomic Molecules & Haplotype $\to$ Molecules & $\Prob(\mathbf{g}|h)$ replication & Ploidy, mosaicism \\
\midrule
Somatic Mutation & Molecules $\to$ Mutated & $\Prob(\mathbf{u}|\mathbf{g})$ mutation rate & Mutation rate $\mu$ \\
\midrule
Fragmentation & DNA $\to$ Fragments & $\Prob(\mathbf{d}|\mathbf{u})$ shearing & Shear size dist., enzyme kinetics \\
\midrule
Library Prep & Fragments $\to$ Library & $\Prob(\mathbf{l}|\mathbf{d})$ ligation, enrichment & Adapter efficiency, enrichment fold \\
\midrule
Signal Generation & Library $\to$ Signal & $\Prob(\boldsymbol{\sigma}|\mathbf{l})$ sensor physics & Noise $\sigma$, drift, k-mer model \\
\midrule
Basecalling & Signal $\to$ Reads & $\Prob(\mathbf{r}|\boldsymbol{\sigma})$ neural network & Basecaller model, Q-score calibration \\
\bottomrule
\end{tabular}
\end{table}

\noindent\textbf{Computational Strategy:} Direct computation of the full joint distribution is intractable. Practical implementations (Chapter~\ref{chap:posteriors}) approximate the pipeline by:
\begin{enumerate}
    \item Collapsing upstream stages (haplotype $\to$ library) into empirical fragment distributions
    \item Modeling signal $\to$ reads via confusion matrices or quality scores
    \item Assuming conditional independence of reads given haplotype
\end{enumerate}

These approximations enable real-time classification on standard computational hardware while maintaining mathematical rigor through explicit modeling of uncertainties.

\section{Equation Index and Cross-References}
\label{sec:ch4-equation-index}

For rapid lookup, Table~\ref{tab:ch4-equation-index} indexes all key equations in this chapter with hyperlinks to full derivations and related content.

\begin{table}[H]
\centering
\caption{Equation Quick Reference for Chapter 4}
\label{tab:ch4-equation-index}
\small
\begin{tabular}{p{2cm}p{6cm}p{4.5cm}}
\toprule
\textbf{Equation} & \textbf{Description} & \textbf{See Also} \\
\midrule
Pipeline Factorization & $\Prob(h, \mathbf{g}, \ldots, \mathbf{r}) = \Prob(h) \prod \Prob(\text{next}|\text{prev})$ & Thm.~\ref{thm:pipeline-factorization}, \CEref{1} \\
\midrule
Signal Emission & $\Prob(\sigma|l) = \prod_t \Prob(x_t|s_t)$ & Def.~\ref{def:signal-emission} \\
\midrule
Phred Quality Score & $\Prob(\text{error}|Q) = 10^{-Q/10}$ & Def.~\ref{def:quality-score-model} \\
\midrule
Quality-Based Likelihood & $\Prob(\hat{s}=s|\mathbf{Q}) = \prod_j (\cdots)$ & Def.~\ref{def:quality-score-model}, \CEref{2} \\
\midrule
Edit Distance & $d_{\text{edit}}(r,t) = \min\{\#\text{edits}\}$ & Def.~\ref{def:levenshtein-distance} \\
\midrule
Read Accuracy & $\text{Acc} = 1 - d_{\text{edit}}/\max(|r|,|t|)$ & Example~\ref{ex:edit-distance} \\
\midrule
Confusion Matrix & $C_{ij} = \Prob(\text{obs } i | \text{true } j)$ & Def.~\ref{def:confusion-matrix}, Chapter~\ref{chap:sma-seq} \\
\bottomrule
\end{tabular}
\end{table}

\noindent\textbf{Navigation:} All core equations referenced in this framework are cataloged in Appendix~\ref{app:core-equations} with full derivations, numerical examples, and software implementation notes. Use \CEref{N} commands to hyperlink directly to core equation N.

\clearpage
