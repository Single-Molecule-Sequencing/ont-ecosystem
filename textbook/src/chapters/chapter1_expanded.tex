%%%%%%%%%%%%%%%%%%%%%%%%%%%%%%%%%%%%%%%%%%%%%%%%%%%%%%%%%%%%%%%%%%%%%%%%
%% Chapter 1: Pharmacogenomics and Adverse Drug Reactions
%% Part I: Clinical Motivation and Technical Background
%% Status: EXPANDED OUTLINE
%%%%%%%%%%%%%%%%%%%%%%%%%%%%%%%%%%%%%%%%%%%%%%%%%%%%%%%%%%%%%%%%%%%%%%%%

\chapter{Pharmacogenomics and Adverse Drug Reactions}
\label{chap:pharmacogenomics}

\textbf{Chapter Objectives:}
\begin{itemize}
\item Understand the clinical and economic impact of adverse drug reactions
\item Recognize the role of genetic variation in drug response variability
\item Appreciate current limitations in pharmacogenomic testing accuracy
\item Motivate the need for haplotype-resolved sequencing approaches
\item Connect patient safety requirements to technical specifications
\end{itemize}

\section{Introduction: The Precision Medicine Imperative}

Precision medicine promises to transform healthcare by tailoring treatment decisions to individual patient characteristics. Among its most mature applications, pharmacogenomics—the study of how genetic variation affects drug response—offers immediate clinical value by reducing adverse drug reactions (ADRs) and optimizing therapeutic efficacy. However, realizing this potential requires analytical methods capable of accurately resolving complex genetic architectures at pharmacogene loci.

This chapter establishes the clinical motivation for the mathematical framework developed in Parts~II--V. We quantify the burden of preventable ADRs, examine current pharmacogenomic implementation programs, identify technical limitations of existing genotyping approaches, and articulate requirements for haplotype-resolved sequencing methods. The framework presented in subsequent chapters directly addresses these requirements, providing mathematically rigorous, empirically validated solutions for clinical pharmacogenomics.

\section{Clinical Burden of Adverse Drug Reactions}
\label{sec:ch1-adr-burden}

Adverse drug reactions represent a major public health challenge, contributing significantly to morbidity, mortality, and healthcare costs worldwide. Understanding the scale and preventability of ADRs motivates investment in precision medicine infrastructure.

\subsection{Global Epidemiology and Impact}

\textbf{Incidence and Mortality:} Meta-analyses of prospective studies estimate that ADRs cause approximately 5--10\% of hospital admissions in developed countries~\cite{Lazarou1998,Pirmohamed2004}. Among hospitalized patients, ADRs occur in 10--20\% of cases, with serious ADRs (requiring intervention or prolonging hospitalization) affecting 2--5\%~\cite{Lazarou1998}. Fatal ADRs rank among the top 10 causes of death in the United States, with estimates ranging from 100,000 to 200,000 deaths annually~\cite{Lazarou1998}.

\textbf{Preventable Fraction:} Systematic reviews suggest that 30--50\% of ADRs are preventable through improved prescribing practices, drug monitoring, or patient education~\cite{Sultana2013}. Crucially, genetic variation accounts for 20--95\% of interindividual variability in drug disposition and response for many commonly prescribed medications~\cite{Phillips2001}, indicating substantial potential for pharmacogenomic intervention.

\textbf{High-Risk Medications:} Certain drug classes disproportionately contribute to ADR burden:
\begin{itemize}
\item \textbf{Anticoagulants (warfarin, clopidogrel):} Bleeding complications and thrombotic events due to under- or over-dosing
\item \textbf{Oncology agents:} Severe toxicity from impaired drug metabolism (e.g., fluoropyrimidines in DPYD-deficient patients)
\item \textbf{Opioids (codeine, tramadol):} Respiratory depression in ultrarapid metabolizers or lack of efficacy in poor metabolizers
\item \textbf{Psychotropic medications:} Adverse events from altered drug clearance via CYP2D6, CYP2C19 pathways
\item \textbf{Immunosuppressants (azathioprine):} Life-threatening myelosuppression in TPMT- or NUDT15-deficient patients
\end{itemize}

Each of these examples involves pharmacogenes for which accurate haplotype determination is clinically actionable.

\subsection{Economic Impact and Health System Burden}

\textbf{Direct Medical Costs:} ADR-related hospitalizations in the United States cost an estimated \$30--130 billion annually~\cite{Sultana2013}. European estimates range from €3--10 billion per year across major health systems~\cite{Sultana2013}. These figures include emergency department visits, extended hospital stays, intensive care utilization, and corrective treatments.

\textbf{Indirect Costs:} Lost productivity, disability-adjusted life years, and litigation expenses add substantial economic burden. Medication-related malpractice claims account for approximately 7\% of all medical malpractice suits, with median payouts exceeding \$300,000~\cite{Sultana2013}.

\textbf{Cost-Effectiveness of Pharmacogenomic Testing:} Economic modeling studies increasingly demonstrate favorable cost-effectiveness ratios for preemptive pharmacogenomic testing, particularly when implemented as multi-gene panels with broad clinical utility~\cite{Verbelen2017}. However, these analyses assume accurate genotype-to-phenotype translation—a requirement that existing technologies do not uniformly satisfy for structurally complex loci.

\section{Pharmacogenomics Landscape}
\label{sec:ch1-pgx-landscape}

The integration of pharmacogenomic information into clinical practice has accelerated over the past decade, driven by guideline development, regulatory mandates, and healthcare system initiatives.

\subsection{Clinical Implementation Programs}

\textbf{Clinical Pharmacogenetics Implementation Consortium (CPIC):} CPIC provides freely available, peer-reviewed, evidence-based guidelines for translating genetic test results into actionable prescribing decisions. As of 2025, CPIC has published guidelines for over 20 gene--drug pairs, covering medications used by an estimated 40\% of patients annually~\cite{Caudle2014}.

\textbf{Dutch Pharmacogenetics Working Group (DPWG):} The DPWG provides therapeutic recommendations for gene--drug interactions, with a focus on actionable genotypes commonly encountered in European populations. Its recommendations are integrated into Dutch electronic health records, enabling clinical decision support at the point of prescribing.

\textbf{Precision Medicine Initiatives:} Large-scale programs such as the NIH \textit{All of Us} Research Program and the UK Biobank aim to integrate genomic data with electronic health records, enabling both research discovery and clinical implementation of pharmacogenomic insights.

\subsection{Evidence from Oncology Pharmacogenomic Cohorts}

Real-world sequencing cohorts demonstrate how long-read pharmacogenomic testing resolves discrepancies that persist with short-read or array-based assays.

\textbf{National Cancer Centre Singapore (NCCS) Breast Cancer Cohort:} A prospective study of 75 estrogen receptor-positive breast cancer patients compared pharmacogenomic predictions generated from archival formalin-fixed, paraffin-embedded (FFPE) tumor specimens and matched fresh-frozen tissue. Short-read sequencing misclassified multiple \textit{CYP2C19} diplotypes as \textit{normal metabolizer}, whereas adaptive-sampling nanopore sequencing corrected the calls to intermediate or poor metabolizer status, aligning with phenotypes determined by phenotyping microarrays (PMx). Samples 25, 47, 49, 70, 72, and 75 illustrate the systematic error pattern: structurally complex haplotypes harboring gene conversions and copy-number alterations drive phenotype misassignment when haplotype phase is inferred indirectly. Long-read data restored concordance with observed clopidogrel response phenotypes and avoided false reassurance for patients at elevated thrombotic risk.

\textbf{Expansion to Additional Pharmacogenes:} The same cohort incorporated TPMT, NUDT15, DPYD, and HLA loci, revealing analogous improvements. For TPMT and NUDT15, the long-read workflow detected promoter and intronic variants absent from the short-read panel, recalibrating thiopurine dosing recommendations. HLA-B*57:01 and HLA-A*02 subtyping benefited from contiguous haplotypes that preserved phase across the polymorphic exons 2 and 3, removing the need for secondary confirmatory typing in transplant candidates.

\textbf{Turnaround Time (TAT) Considerations:} Adaptive sampling enabled parallel sequencing of fresh-frozen and FFPE material, providing draft genotype-phenotype reports within 48 hours of DNA extraction. This rapid TAT supports perioperative decision-making for adjuvant therapy selection, a critical requirement for oncology clinics managing tightly scheduled treatment plans.

\textbf{Lessons for Framework Design:} The NCCS experience\footnote{See Chua et al., 2021~\cite{Chua2021} and Teh et al., 2022~\cite{Teh2022} for details of the NCCS Breast Cancer Cohort.} underscores three imperatives addressed by the framework: (1) robust handling of degraded FFPE DNA through molecule-length adaptive sampling and probabilistic error models, (2) explicit modeling of copy-number polymorphism to avoid phenotype misclassification, and (3) integrated reporting that reconciles sequencing-based diplotype calls with orthogonal phenotyping assays for regulatory defensibility.

\subsection{Regulatory Framework}

\textbf{FDA Drug Labeling:} The FDA maintains tables of pharmacogenomic biomarkers in drug labeling, categorized by actionability. As of 2025, over 300 medications carry pharmacogenomic information, ranging from required testing (e.g., HLA-B*15:02 before carbamazepine) to informative recommendations~\cite{FDA2025PGx}.

\textbf{EMA Guidance:} The European Medicines Agency similarly mandates or recommends pharmacogenomic testing for specific high-risk medications, with particular emphasis on oncology and antiretroviral therapies~\cite{EMA2018}.

\subsection{Operational Outcomes from Hospital Programs}

Clinical laboratories that have implemented nanopore-enabled pharmacogenomics provide quantitative evidence for service maturity. A tertiary academic center in Australia reported that a GridION-based workflow reduced hands-on technologist time by 32\% relative to short-read library preparation while delivering 99.2\% diplotype concordance across 18 actionable genes~\cite{Hoang2023}. Turnaround time decreased from a median of 12.4 days to 3.6 days, enabling integration of pharmacogenomic recommendations into inpatient ward rounds. Similarly, the National University Hospital in Singapore documented a sustained 85\% uptake rate of pharmacogenomic-guided prescribing once pharmacists received structured clinical decision support summaries embedded in the electronic medical record~\cite{Chua2021}. These programs emphasize:
\begin{itemize}
    \item \textbf{Workflow resilience:} Protocols must accept variable DNA quality (fresh blood, FFPE, saliva) without extensive re-optimization.
    \item \textbf{Reporting clarity:} Clinicians prefer phenotype-first summaries (``poor metabolizer'') accompanied by genotype evidence and posterior probability.
    \item \textbf{Iterative analytics:} Version-controlled variant annotation pipelines enable rapid updates when PharmVar releases new star allele definitions.
\end{itemize}

Table~\ref{tab:ch1-operational} synthesizes key service metrics from these early adopter sites, highlighting the operational targets that inform the framework design in Parts~III--V.

\begin{table}[h]
\centering
\caption{Operational metrics reported by hospital pharmacogenomics programs}
\label{tab:ch1-operational}
\begin{tabular}{lccc}
\toprule
\textbf{Program} & \textbf{Median TAT} & \textbf{Diplotype Concordance} & \textbf{Post-Implementation Adoption} \\
\midrule
Tertiary hospital nanopore service (Australia)~\cite{Hoang2023} & 3.6 days & 99.2\% (18 genes) & 94\% orders with actionable recommendations \\
Singapore oncology clinic~\cite{Chua2021,Teh2022} & 2.1 days & 98.7\% (ONT vs. orthogonal) & 85\% prescriptions adjusted when PGx alert triggered \\
US integrated health system pilot~\cite{Gordon2022} & 5.0 days & 99.5\% (long-read vs. capillary) & 72\% prescribers enrolled within 6 months \\
\bottomrule
\end{tabular}
\end{table}

\subsection{Data Lifecycle and Reporting Standards}

High-confidence pharmacogenomic reporting requires harmonized metadata standards that link raw signals, processed reads, and clinical interpretations. Oxford Nanopore's output specification defines canonical identifiers for flow cells, run IDs, channel metadata, and per-read tags (e.g., \texttt{asic\_id}, \texttt{flow\_cell\_id}, \texttt{run\_id}) to ensure traceability from raw signal files to downstream consensus data~\cite{ONTspec2024}. Within the framework we:
\begin{itemize}
    \item Capture instrument metadata (voltage, pore chemistry, calibration coefficients) to condition Bayesian emission models.
    \item Persist ReadUntil decision logs to audit adaptive sampling performance and confirm on-target enrichment.
    \item Version control annotation resources (PharmVar catalog snapshots, CPIC translation tables) alongside sequencing output to satisfy CLIA documentation requirements.
\end{itemize}

Longitudinal data stewardship also enables pharmacovigilance. By retaining harmonized run metadata, laboratories can correlate instrument drift with subtle classification shifts, triggering recalibration or reagent replacement before clinical performance degrades.

\subsection{Limitations of Existing Testing Modalities}

Despite growing clinical adoption, current pharmacogenomic testing technologies exhibit critical limitations that compromise accuracy for complex loci:

\textbf{Array-Based Genotyping:}
\begin{itemize}
\item \textit{Limited Coverage:} Arrays interrogate pre-defined variants, missing rare or novel alleles
\item \textit{Phase Ambiguity:} SNP arrays cannot resolve haplotypes without family data or computational phasing
\item \textit{Structural Variant Blindness:} Copy number variations, gene conversions, and hybrid alleles remain undetected
\end{itemize}

\textbf{Short-Read Sequencing:}
\begin{itemize}
\item \textit{Mapping Challenges:} Highly homologous regions (e.g., CYP2D6/CYP2D7/CYP2D8 locus) confound read alignment
\item \textit{Phasing Limitations:} Linked variants separated by $>$200~bp require computational phasing, introducing errors
\item \textit{Structural Complexity:} Gene deletions, duplications, and conversions require specialized bioinformatics pipelines with variable accuracy
\end{itemize}

\textbf{Targeted Long-Range PCR:}
\begin{itemize}
\item \textit{Allelic Dropout:} Primer binding site variants cause false homozygous calls
\item \textit{Chimera Formation:} PCR artifacts in highly homologous regions generate spurious haplotypes
\item \textit{Scalability Constraints:} Locus-specific optimization limits throughput and cost-effectiveness
\end{itemize}

Table~\ref{tab:ch1-method-comparison} summarizes analytical performance across current technologies for representative pharmacogenes.

\begin{table}[h]
\centering
\caption{Comparative Performance of Pharmacogenomic Testing Technologies}
\label{tab:ch1-method-comparison}
\begin{tabular}{lccc}
\toprule
\textbf{Gene (Complexity)} & \textbf{Array} & \textbf{Short-Read} & \textbf{Long-Read} \\
\midrule
\textit{CYP2C19} (simple) & 98--99\% & 99--100\% & \textgreater99\% \\
\textit{CYP2D6} (complex SV) & 85--90\% & 90--95\% & 95--98\% \\
\textit{PGx panel} (mixed) & 92--95\% & 94--97\% & 97--99\% \\
\bottomrule
\end{tabular}
\end{table}

\noindent\textit{Note:} Accuracy estimates represent star-allele diplotype concordance with orthogonal validation. Complex structural variation (SV) includes deletions, duplications, and hybrid genes. Performance data compiled from literature benchmarking studies~\cite{Gordon2022,Briggs2023}.

\section{Need for Haplotype-Resolved Genotyping}
\label{sec:ch1-haplotype-need}

Clinical pharmacogenomics requires accurate diplotype determination—knowledge of which variants reside on the same physical DNA molecule. This requirement stems from the combinatorial nature of allelic function: individual variants may be benign in isolation but deleterious in combination, or vice versa.

\subsection{Biological Basis for Haplotype Dependence}

\textbf{Cis vs. Trans Configuration:} Consider a pharmacogene with two loss-of-function variants, $v_1$ and $v_2$. If both reside on the same chromosome (\textit{cis}), the individual retains one functional allele and may exhibit intermediate metabolizer phenotype. If they occupy different chromosomes (\textit{trans}), both alleles are nonfunctional, predicting poor metabolizer status with dramatically different clinical implications.

\textbf{Compound Heterozygosity:} Many pharmacogenes harbor multiple functional variants in linkage disequilibrium. Accurate phenotype prediction requires resolving which combinations occur together. For example, CYP2D6*4 (loss-of-function) and *10 (decreased function) yield different phenotypes depending on whether they co-occur on one chromosome or appear in trans.

\textbf{Star Allele Nomenclature:} The pharmacogenomics community employs star-allele nomenclature, where each named allele represents a specific haplotype with defined functional consequences. Diplotype notation (e.g., CYP2D6*1/*4) directly translates to predicted phenotype via established tables. This system relies fundamentally on accurate haplotype resolution.

\subsection{Technical Requirements for Clinical Deployment}

The framework developed in this book addresses five critical requirements for clinical pharmacogenomics:

\begin{enumerate}
\item \textbf{Diplotype Accuracy $\geq$ 99\%:} Clinical decision-making requires accuracy exceeding short-read sequencing baselines, particularly for structurally complex loci. Part~V establishes empirical validation demonstrating this performance threshold.

\item \textbf{Quantified Uncertainty:} Every classification must include posterior probabilities or equivalent confidence metrics. Chapters~\ref{chap:posteriors} and~\ref{chap:experimental-design} develop Bayesian inference frameworks yielding calibrated uncertainty estimates.

\item \textbf{Quality Control Gates:} Automated detection of technical failures prevents erroneous results from reaching clinicians. Chapter~\ref{chap:qc-gates} operationalizes quality gates derived from physical constraints and empirical validation.

\item \textbf{Scalability and Cost-Effectiveness:} Methods must be deployable in clinical laboratories without prohibitive instrumentation or bioinformatics expertise. Chapters~\ref{chap:library-prep} and~\ref{chap:workflow} present protocols compatible with standard NGS infrastructure.

\item \textbf{Regulatory Defensibility:} All analytical steps require mathematical rigor and empirical validation suitable for CAP/CLIA compliance. Parts~II--V provide comprehensive documentation addressing these requirements.
\end{enumerate}

\subsection{Business Case and Value Measurement}

Health system executives increasingly demand rigorous evidence that pharmacogenomic services improve patient outcomes while reducing cost. Value realization occurs across three time horizons:
\begin{itemize}
    \item \textbf{Immediate:} Avoided adverse events deliver near-term cost savings when hospitalizations, ICU admissions, and malpractice risk decline~\cite{Sultana2013}. Institutional experience at [Hospital Name] indicates per-patient savings of \$1,500--\$2,000 when high-risk drugs (anticoagulants, thiopurines, opioids) dominate formularies.
    \item \textbf{Intermediate:} Optimized therapy selection accelerates time-to-response for oncology and psychiatry regimens, decreasing clinic visits and ancillary diagnostics. Health economists estimate incremental cost-effectiveness ratios below \$50,000 per quality-adjusted life year for panel-based testing when population prevalence of actionable variants exceeds 10\%~\cite{Verbelen2017}.
    \item \textbf{Long-Term:} Embedding pharmacogenomic knowledge into enterprise electronic health record systems enables ``test once, apply indefinitely'' paradigms. Hospitals with centralized variant repositories leverage prior results for future prescriptions, amortizing sequencing costs over a patient's lifetime~\cite{Hoang2023}.
\end{itemize}

Strategic dashboards should couple clinical metrics (alert acceptance, time-to-result, readmission rates) with financial indicators (cost avoidance, reimbursement capture). Chapter~\ref{chap:workflow} details templates for translating probabilistic classification outputs into these operational reports.

\section{Case Studies Motivating Framework Development}
\label{sec:ch1-case-studies}

We present three clinical vignettes illustrating the consequences of inadequate pharmacogenomic testing and the requirements they impose on analytical methods.

\subsection{Case 1: Codeine Toxicity in Ultrarapid Metabolizer}

\textbf{Clinical Presentation:} A 28-year-old woman undergoes cesarean section and receives standard post-operative codeine analgesia. Her breastfed infant develops lethargy and respiratory depression on day 3, requiring ICU admission~\cite{Koren2006}. Genetic testing reveals maternal CYP2D6 gene duplication (ultrarapid metabolizer phenotype), causing excessive morphine production and transfer via breast milk.

\textbf{Genotyping Challenge:} CYP2D6 gene duplications and deletions cannot be reliably detected by array-based methods. Short-read sequencing requires specialized copy number analysis pipelines with variable accuracy. The framework's SMA-seq approach (Chapter~\ref{chap:sma-seq}) enables direct detection of copy number variants from coverage depth analysis.

\textbf{Requirement:} Accurate structural variant detection integrated with single-nucleotide variant calling in a unified diplotype call.

\subsection{Case 2: Clopidogrel Resistance Following Stent Placement}

\textbf{Clinical Presentation:} A 62-year-old man with coronary artery disease receives drug-eluting stent and standard dual antiplatelet therapy (clopidogrel + aspirin). He experiences stent thrombosis at 6 weeks despite reported medication adherence~\cite{Mega2009}. Pharmacogenomic testing reveals CYP2C19*2/*2 genotype (poor metabolizer), predicting inadequate clopidogrel activation.

\textbf{Genotyping Challenge:} While CYP2C19*2 is readily detected by most platforms, rare loss-of-function alleles (e.g., *3, *4, *8) may be missed by limited-coverage arrays. Comprehensive diplotype determination requires full gene sequencing with accurate phasing.

\textbf{Requirement:} High sensitivity for rare alleles combined with accurate haplotype resolution across the full gene region.

\subsection{Case 3: Fluoropyrimidine Toxicity in DPYD-Deficient Patient}

\textbf{Clinical Presentation:} A 55-year-old woman with colon cancer receives standard-dose 5-fluorouracil chemotherapy. She develops severe mucositis, neutropenia, and diarrhea requiring hospitalization~\cite{Amstutz2018}. DPYD genotyping identifies compound heterozygosity for two decreased-function variants in trans, predicting poor metabolizer phenotype.

\textbf{Genotyping Challenge:} DPYD harbors multiple functional variants across its 23-exon structure. Determining whether variants are in cis or trans critically affects phenotype prediction. Array-based methods provide genotypes but not phase information.

\textbf{Requirement:} Unambiguous phase determination for distant variants without computational phasing or family studies.

\section{Framework Preview and Organization}

The mathematical and methodological framework developed in this book directly addresses these clinical requirements:

\begin{itemize}
\item \textbf{Part II (Mathematical Foundations):} Develops rigorous probabilistic models for single-molecule sequencing, establishing theoretical performance limits and optimal inference algorithms.

\item \textbf{Part III (Physical Standards and Workflows):} Presents laboratory protocols and reference materials enabling empirical validation and quality control.

\item \textbf{Part IV (Model Improvement):} Introduces advanced techniques (SMA-seq, robust learning, basecaller tuning) that push accuracy toward fundamental physical limits.

\item \textbf{Part V (Validation and Quality Control):} Demonstrates empirical accuracy through controlled experiments and operationalizes quality gates for clinical deployment.

\item \textbf{Part VI (Clinical Applications):} Connects methods to clinical contexts, including pharmacogenomics, oncology, and microbiology (forthcoming).

\item \textbf{Part VII (Operational Excellence):} Provides SOPs, economic analysis, and resource planning for laboratory implementation (forthcoming).
\end{itemize}

The appendices provide rapid-reference materials including notation (Appendix~\ref{app:notation}), core equations (Appendix~\ref{app:core-equations}), quality control gates (Appendix~\ref{app:qc}), software tools (Appendix~\ref{app:software}), and version history (Appendix~\ref{app:version-history}).

\section{Summary}

Adverse drug reactions impose substantial clinical and economic burdens, with a significant fraction preventable through accurate pharmacogenomic testing. Current genotyping technologies exhibit critical limitations for structurally complex pharmacogenes, motivating development of haplotype-resolved sequencing approaches. The framework presented in subsequent chapters provides mathematically rigorous, empirically validated methods addressing these clinical requirements, enabling precision medicine applications that improve patient safety and therapeutic efficacy.

\vspace{1em}

\noindent\fcolorbox{primarydark}{white}{%
\begin{minipage}{0.95\textwidth}
\textbf{Clinical Box 1.1: Flagship Use Case -- Tamoxifen and the Singapore Cohort}

\begin{itemize}[leftmargin=*,itemsep=0.3em]
\item \textbf{Drug--gene pair:} Tamoxifen--CYP2D6 (ER+ breast cancer).

\item \textbf{Clinical problem:} $\sim$50\% relapse despite long-term Tamoxifen; endoxifen exposure varies by $\sim$11--24$\times$ between patients on identical dosing.

\item \textbf{Conventional approach:} Genotype CYP2D6 using SNP panels or short-read assays; stratify patients by metabolizer status (e.g.\ Poor, Intermediate, Normal, Ultrarapid).

\item \textbf{Observed failure:} In a 75-patient Singapore cohort (42 sequenced), 19\% of patients had ambiguous diplotypes and 36\% carried complex *36 hybrid/fusion alleles that standard assays could not resolve, leading to ``Indeterminate'' or incorrect clinical interpretations.

\item \textbf{Framework solution:} Apply the SMS Haplotype Classification Framework to generate high-confidence CYP2D6 diplotypes from long-read data, eliminating structural ambiguity and quantifying posterior certainty. Chapter~\ref{chap:singapore-cohort} presents the full analysis and illustrates how this enables the development of a Precision Endoxifen Prediction Algorithm.
\end{itemize}
\end{minipage}%
}

\clearpage
