%%%%%%%%%%%%%%%%%%%%%%%%%%%%%%%%%%%%%%%%%%%%%%%%%%%%%%%%%%%%%%%%%%%%%%%%
%% Chapter 3: Single-Molecule Sequencing: Technologies and Capabilities
%% Part I: Clinical Motivation and Technical Background
%% Status: Expanded Outline
%%%%%%%%%%%%%%%%%%%%%%%%%%%%%%%%%%%%%%%%%%%%%%%%%%%%%%%%%%%%%%%%%%%%%%%%

\chapter{Single-Molecule Sequencing: Technologies and Capabilities}
\label{chap:single-molecule}
\label{chap:sms-overview}

\section{Introduction: The Long-Read Revolution}

The structural complexity and phasing challenges documented in Chapter~\ref{chap:genomic-complexity} necessitate a fundamentally different approach to DNA sequencing. While short-read technologies (Illumina, BGI) dominated genomics for two decades through scalability and per-base accuracy, they inherently fragment long-range information. Single-molecule sequencing technologies—Oxford Nanopore Technologies (ONT) and Pacific Biosciences (PacBio)—generate reads spanning tens to hundreds of kilobases, enabling direct observation of haplotypes without statistical inference~\cite{VanDijk2018,Logsdon2020}.

This chapter introduces the physical principles, signal characteristics, and operational parameters of single-molecule sequencing platforms. We establish the technological foundation for the probabilistic framework developed in Part~II and validated in Parts IV--V. Critically, we explain how platform-specific error profiles drive the mathematical modeling choices in Chapter~\ref{chap:basecaller} and how throughput-accuracy trade-offs inform experimental design (Chapter~\ref{chap:experimental-design}).

\noindent\textbf{Chapter Objectives:}
\begin{itemize}
\item Understand physical principles of nanopore and SMRT sequencing
\item Compare ONT and PacBio platforms for pharmacogenomic applications
\item Characterize error profiles and mitigation strategies
\item Assess operational parameters for clinical laboratory deployment
\item Connect platform capabilities to framework requirements
\end{itemize}

\section{Single-Molecule Sequencing: Core Principles}
\label{sec:ch3-principles}

\subsection{From Population Averages to Individual Molecules}

Traditional sequencing approaches (Sanger, Illumina) observe ensemble averages: millions of DNA molecules amplified from a single template yield a consensus signal. Heterozygous variants appear as overlapping peaks or mixed base calls, but phase information—which variants reside on the same chromosome—is lost unless molecules span both variant positions.

Single-molecule sequencing directly observes individual DNA molecules without amplification. Each read reports the unique sequence of one molecule, including:
\begin{itemize}
    \item \textbf{Haplotype structure}: All variants present on that chromosome copy
    \item \textbf{Epigenetic modifications}: Methylation patterns detectable via signal perturbations
    \item \textbf{Structural variants}: Deletions, insertions, inversions visible within single reads
    \item \textbf{Error profile}: Molecule-specific sequencing errors reflecting stochastic signal variation
\end{itemize}

The trade-off is reduced per-base accuracy: where Illumina achieves $>$99.9\% accuracy via consensus, single-molecule technologies initially delivered 85--95\% raw accuracy~\cite{Wenger2019}. Modern chemistries and basecalling algorithms (2024--2025) approach 99\% single-pass accuracy, with consensus methods pushing beyond 99.9\%~\cite{ONTDorado2024}. For haplotype classification, this accuracy is sufficient when coupled with probabilistic inference (\CEref{1}).

\subsection{The Haplotype Advantage}

For pharmacogenes, long reads provide decisive advantages:
\begin{enumerate}
    \item \textbf{\textit{CYP2D6} structural variants}: A 15~kb read spanning the entire gene definitively distinguishes gene from pseudogene, resolves hybrid allele breakpoints, and phases all coding variants simultaneously. Short reads cannot achieve this without statistical phasing assumptions that fail in complex regions.

    \item \textbf{Copy number determination}: Gene duplications appear as increased read depth \textit{and} haplotype-resolved read evidence. Distinguishing duplication of functional versus nonfunctional alleles requires observing which specific variant combinations occur in duplicate—only possible with haplotype-resolved reads.

    \item \textbf{Phasing across loci}: For multi-gene panels, ultra-long reads ($>$100~kb) can phase variants across multiple adjacent genes on the same chromosome, providing context for interpreting combined metabolizer status.
\end{enumerate}

\section{Oxford Nanopore Technologies}
\label{sec:ch3-nanopore}

\subsection{Platform Overview and Physical Principles}

Oxford Nanopore Technologies (ONT) sequences DNA by measuring electrical current changes as single-stranded DNA translocates through biological nanopores embedded in a synthetic membrane~\cite{Delahaye2021}. The technology rests on three core components:

\textbf{1. Nanopore Protein:} A biological protein (typically CsgG or MspA variants) forms a narrow channel approximately 1~nm in diameter. The pore's amino acid sequence creates a constriction region where 5--6 nucleotides simultaneously occupy the sensing zone.

\textbf{2. Synthetic Membrane and Voltage:} The nanopore protein sits in a synthetic lipid bilayer separating two chambers. Applying a potential difference (typically +180 mV) drives ionic current through the pore. Single-stranded DNA, prepared with a motor protein attached to the 5' end, is drawn through the pore by electrophoretic force.

\textbf{3. Current Measurement and Signal Processing:} As DNA translocates, different k-mer sequences (5--6 nucleotides in the sensing zone) modulate the ionic current distinctively. High-frequency measurement (4--5 kHz sampling) records current traces at sub-millisecond resolution. Neural network basecallers (Dorado, Guppy) convert raw current signals to nucleotide sequences.

\subsection{Sequencing Devices and Throughput}

ONT offers multiple device formats targeting different applications~\cite{Hoang2023}:

\textbf{MinION:} Portable USB-powered device with 512--2,048 nanopores per flow cell. Typical output: 10--50 Gb per flow cell depending on sample quality and run time. Suitable for targeted applications, small genome sequencing, or point-of-care diagnostics.

\textbf{GridION:} Benchtop instrument supporting five MinION flow cells run independently or in parallel. Designed for clinical laboratory automation with integrated compute and workflow management. Throughput: 50--250 Gb per device (5 flow cells).

\textbf{PromethION 24/48:} High-throughput platform with 24 or 48 flow cells, each containing 2,675 nanopores. Throughput: $>$10 Tb per device per 48-hour run. Clinical laboratories implementing pharmacogenomic panels at scale (thousands of patients annually) require PromethION throughput to achieve acceptable per-sample costs.

For targeted pharmacogenomic sequencing (\textit{CYP2D6}, \textit{CYP2C19}, \textit{CYP2C9}, etc.), adaptive sampling (Section~\ref{sec:ch3-adaptive}) enables $>$1,000$\times$ enrichment, reducing required instrument capacity. A GridION can process 50--100 patient samples per week with appropriate multiplexing and enrichment.

\subsection{Basecalling and Signal Processing}

Raw ONT data consists of current traces (measured in picoamperes) sampled at 4--5 kHz. Basecalling transforms these analog signals into discrete nucleotide sequences using deep neural networks:

\textbf{Dorado (Current Generation):} Transformer-based architecture trained on billions of nanopore reads. Dorado 7.0+ (2024) achieves~\cite{ONTDorado2024}:
\begin{itemize}
    \item \textbf{Accuracy (R10.4.1 chemistry):} Modal per-base accuracy 98--99\% single-pass, $>$99.5\% duplex
    \item \textbf{Throughput:} Real-time basecalling on NVIDIA GPU (A100, H100) at $>$100 Gb/hour
    \item \textbf{Quality scores:} Calibrated Phred scores (Phred+33 encoding, ASCII 33--126, range Q0--Q93) reflecting per-base confidence (Chapter~\ref{chap:basecaller}, \CEref{2})
\end{itemize}

\textbf{Data Output Formats:}
ONT sequencing generates multiple file formats organized by MinKNOW software:
\begin{itemize}
    \item \textbf{POD5 files:} Raw signal data stored as 16-bit integer arrays in ADC (analog-to-digital converter) space. Signals convert to picoamperes via calibration: $I_{\text{pA}} = (\text{ADC} - \text{offset}) \times \text{scale}$, where offset and scale are per-read calibration parameters stored in POD5 metadata. Based on Apache Arrow format for efficient access.
    \item \textbf{FASTQ files:} Basecalled sequences with Sanger-format quality scores (Phred+33). Quality $Q$ encodes error probability: $P(\text{error}) = 10^{-Q/10}$. ONT quality scores span Q0--Q93 (ASCII 33--126).
    \item \textbf{BAM files:} Aligned reads with rich metadata tags including barcode classification (BC), strand orientation (TS), modified base calls (MM/ML), and parent read IDs (pi) for split reads.
\end{itemize}

\textbf{MinKNOW Output Structure:}
MinKNOW organizes all sequencing artifacts under a single base output directory chosen at run start. Format-specific subdirectories include:
\begin{itemize}
    \item \texttt{pod5/}: Raw signal data (replaces legacy FAST5 format)
    \item \texttt{fastq\_pass/} and \texttt{fastq\_fail/}: Basecalled sequences separated by quality threshold
    \item \texttt{bam\_pass/} and \texttt{bam\_fail/}: Aligned reads (if alignment enabled)
    \item \texttt{sequencing\_summary.txt}: Per-read metadata including read ID, length, quality scores, channel number, and \texttt{end\_reason} (discussed in Chapter~\ref{chap:sma-seq})
\end{itemize}

The basecalling process introduces systematic and random errors that directly inform the emission probability $\Prob(r|h)$ (\CEref{2}) in the framework:
\begin{itemize}
    \item \textbf{Homopolymer errors}: Runs of identical bases (e.g., AAAAA) challenge current-level discrimination, yielding insertion/deletion errors at 2--5\% rate. The framework explicitly models homopolymer error modes.
    \item \textbf{Systematic biases}: Specific k-mer contexts (e.g., CpG dinucleotides with methylation) produce characteristic error patterns. Basecaller fine-tuning (Chapter~\ref{chap:basecaller-tuning}) addresses these biases.
    \item \textbf{Quality calibration}: Reported Q-scores require validation against empirical error rates using characterized reference materials (Chapter~\ref{chap:plasmid-standards}). Empirical observations suggest ONT quality scores between Q7--Q30 may systematically overestimate accuracy, necessitating empirical calibration via SMA-seq (Chapter~\ref{chap:sma-seq}).
\end{itemize}

\subsection{Adaptive Sampling and Targeted Sequencing}
\label{sec:ch3-adaptive}

A unique ONT capability enables selective sequencing: during translocation, the sequencer identifies the DNA molecule passing through each pore in real time. If the sequence matches an undesired region (e.g., off-target genomic loci), the system reverses voltage polarity, ejecting the molecule before complete sequencing. This \textbf{adaptive sampling} or \textbf{ReadUntil} approach concentrates sequencing effort on target regions.

For pharmacogenomic applications, adaptive sampling delivers:
\begin{itemize}
    \item \textbf{Enrichment:} 500--2,000$\times$ for targeted gene panels (Chapter~\ref{chap:library-prep})
    \item \textbf{Cost reduction:} Multiplex 50--100 patients per flow cell with acceptable coverage ($>$100$\times$ target depth)
    \item \textbf{Turnaround:} Complete sequencing in 6--12 hours versus 24--48 hours for whole-genome approaches
\end{itemize}

The framework's experimental design theory (Chapter~\ref{chap:experimental-design}) formalizes optimal depth allocation, determining when adaptive sampling provides marginal classification benefit versus uniform coverage.

\subsection{Nanopore Output Specification and Metadata Integrity}
\label{sec:ch3-ont-spec}

Robust clinical reporting requires deterministic metadata capture from raw nanopore signals through variant interpretation. Oxford Nanopore's open output specification enumerates regular expression patterns and contextual descriptions for every identifier emitted in \texttt{sequencing\_summary}, \texttt{pod5}, and basecalling reports~\cite{ONTspec2024}. Key elements include:
\begin{itemize}
    \item \texttt{asic\_id}: immutable identifier for the flow-cell application-specific integrated circuit, supporting retrospective correlation with instrument maintenance logs.
    \item \texttt{flow\_cell\_id} and \texttt{pore\_type}: enforceable naming conventions that distinguish R10.4.1 from legacy chemistries and prevent cross-run contamination during demultiplexing.
    \item \texttt{run\_id} and \texttt{exp\_start\_time}: timestamped UUIDs enabling reconstruction of sequencing batches and alignment with laboratory information systems.
    \item Channel-level attributes (\texttt{channel}, \texttt{well\_id}) that facilitate pore-health monitoring and flagging of underperforming wells for exclusion.
\end{itemize}

Table~\ref{tab:ch3-ont-metadata} highlights representative patterns directly excerpted from the specification and maps them to framework modules.

\begin{table}[h]
\centering
\caption{Representative ONT metadata patterns and framework usage}
\label{tab:ch3-ont-metadata}
\begin{tabular}{lll}
\toprule
\textbf{Field} & \textbf{Regex Pattern} & \textbf{Framework Application} \\
\midrule
\texttt{asic\_id} & \texttt{[A-F0-9]+} & Associates reads with flow-cell QC checks (Chapter~\ref{chap:qc-gates}) \\
\texttt{flow\_cell\_id} & \texttt{[A-Z0-9\_-]+} & Selects chemistry-specific emission parameters (Chapter~\ref{chap:basecaller}) \\
\texttt{run\_id} & \texttt{[0-9a-f-]{36}} & Seeds provenance graph for audit trails (Chapter~\ref{chap:workflow}) \\
\texttt{sample\_id} & \texttt{[A-Za-z0-9\_-]{1,32}} & Links sequencing output to LIS accession identifiers \\
\bottomrule
\end{tabular}
\end{table}

By validating incoming metadata against these patterns prior to analysis, laboratories detect mislabeled barcodes or malformed files before clinical interpretation commences. The framework's ingestion pipeline (Chapter~\ref{chap:data-pipeline}) enforces schema validation using the published YAML definitions, ensuring traceability from raw signal to signed clinical report.

\section{PacBio SMRT Sequencing}
\label{sec:ch3-pacbio}

\subsection{Platform Overview and Zero-Mode Waveguides}

Pacific Biosciences (PacBio) Single Molecule Real-Time (SMRT) sequencing observes DNA polymerase incorporating fluorescently labeled nucleotides in real time~\cite{Wenger2019}. The technology employs:

\textbf{Zero-Mode Waveguides (ZMWs):} Nano-wells 70~nm in diameter and 100~nm deep, etched into a metal film on a glass substrate. Each ZMW contains a single DNA polymerase molecule. The wells' nanoscale dimensions confine observation volume to approximately 20 zeptoliters (10$^{-21}$~L), enabling single-molecule detection despite micromolar nucleotide concentrations.

\textbf{Fluorescent Nucleotide Detection:} Four nucleotide types (A, C, G, T) carry distinct fluorophores. As polymerase incorporates a nucleotide, the fluorophore resides in the ZMW observation volume for milliseconds, emitting a characteristic color pulse. After incorporation, the fluorophore cleaves and diffuses away, resetting the ZMW for the next base.

\textbf{Circular Consensus Sequencing (CCS):} DNA templates are converted to closed circular molecules (SMRTbell libraries). Polymerase processively reads the insert multiple times (typically 10--20 passes), generating a high-accuracy consensus sequence from multiple observations of the same molecule. CCS reads are termed \textbf{HiFi reads}.

\subsection{HiFi Read Generation and Accuracy}

PacBio HiFi sequencing achieves exceptional accuracy through multiple observations of each template molecule~\cite{Wenger2019}.

\textbf{Raw Accuracy:} Single-pass (subread) accuracy is approximately 85--90\%, limited by enzyme kinetics and fluorophore signal-to-noise.

\textbf{HiFi Consensus:} Circular consensus from 10+ passes yields:
\begin{itemize}
    \item \textbf{Modal accuracy:} 99.5--99.9\% (Q30--Q40) per base
    \item \textbf{Read length:} 10--25 kb typical (limited by insert size and polymerase processivity)
    \item \textbf{Systematic error reduction:} Random errors cancel through consensus; systematic errors (e.g., at specific motifs) persist but occur at $<$0.5\% rate
\end{itemize}

This accuracy profile has profound implications for the framework~\cite{Wenger2019}:
\begin{itemize}
    \item Higher per-base accuracy reduces required read depth for equivalent classification confidence (\CEref{7})
    \item Systematic error modes differ from ONT, requiring platform-specific emission models $\Prob(r|h)$
    \item HiFi quality scores are well-calibrated, simplifying Bayesian posterior computation
\end{itemize}

\subsection{Sequencing Instruments}

\textbf{Sequel IIe:} Current-generation platform with 8 million ZMWs per SMRT Cell. Typical output: 160--400 Gb HiFi data per SMRT Cell (48-hour movie time), corresponding to 15--30 million HiFi reads at 15~kb mean length.

\textbf{Revio:} Next-generation system (2023) with 25 million ZMWs and faster polymerase chemistry. Output: 1--1.5 Tb HiFi per run, improving cost-effectiveness for high-throughput clinical applications.

For pharmacogenomics, a single Sequel IIe SMRT Cell can process 100--200 multiplexed patient samples with targeted enrichment, achieving $>$100$\times$ coverage on \textit{CYP2D6} and related genes. The challenge is library preparation complexity: PacBio requires larger DNA input quantities and more intricate workflows than ONT.

\section{Platform Comparison: ONT vs.\ PacBio for Pharmacogenomics}
\label{sec:ch3-comparison}

Table~\ref{tab:ch3-platform-comparison} summarizes key performance and operational parameters for clinical pharmacogenomic testing.

\begin{table}[h]
\centering
\caption{ONT and PacBio Platform Comparison for Clinical Pharmacogenomics}
\label{tab:ch3-platform-comparison}
\small
\begin{tabular}{lcc}
\toprule
\textbf{Parameter} & \textbf{ONT (PromethION)} & \textbf{PacBio (Sequel IIe)} \\
\midrule
\textbf{Accuracy} & & \\
Single-pass & 98--99\% & 85--90\% (subread) \\
Consensus & $>$99.5\% (duplex) & 99.5--99.9\% (HiFi) \\
\midrule
\textbf{Read Length} & & \\
Median & 20--50 kb & 15--25 kb (HiFi) \\
Ultra-long ($>$100 kb) & Routine & Rare \\
\midrule
\textbf{Throughput} & & \\
Per flow cell/SMRT Cell & 200--500 Gb & 160--400 Gb \\
Run time & 24--72 hr & 24--48 hr \\
\midrule
\textbf{Operational} & & \\
Sample input & 100--1000 ng & 1--5 \textmu g \\
Library prep time & 2--4 hr & 6--8 hr \\
Multiplexing capacity & 96--384 & 96--384 \\
\midrule
\textbf{Cost (2024 estimates)} & & \\
Per flow cell/SMRT Cell & \$900--\$1,200 & \$1,500--\$2,000 \\
Per sample (targeted) & \$15--\$30 & \$25--\$50 \\
\midrule
\textbf{Clinical Readiness} & & \\
CLIA validation & Multiple labs & Multiple labs \\
IVD reagents & Limited & Limited \\
Automation & High (GridION) & Moderate \\
\bottomrule
\end{tabular}
\end{table}

\subsection{Decision Framework for Platform Selection}

\textbf{Choose ONT when:}
\begin{itemize}
    \item Ultra-long reads ($>$50~kb) required for complex structural variant resolution
    \item Adaptive sampling critical for cost-effective targeted enrichment
    \item Rapid turnaround essential (real-time sequencing and basecalling)
    \item Laboratory workflow prioritizes simplicity (minimal sample input, streamlined library prep)
    \item Budget constraints favor lower per-sample costs
\end{itemize}

\textbf{Choose PacBio when:}
\begin{itemize}
    \item Maximum per-base accuracy required (HiFi 99.9\% for critical clinical variants)
    \item Regulatory pathway demands established error profiles (more mature validation literature)
    \item Laboratory has infrastructure for complex library preparation
    \item Application tolerates higher per-sample costs for accuracy benefits
\end{itemize}

For the pharmacogenomic framework, both platforms achieve sufficient accuracy for clinical deployment when coupled with probabilistic inference and quality control gates (Chapter~\ref{chap:qc-gates}). The framework's platform-agnostic design accommodates either technology through parameterized error models $\Prob(r|h)$ calibrated to empirical platform performance.

\section{Operational Deployment Considerations}

Clinical laboratories rarely operate a single platform in isolation. Hybrid strategies pair ONT's rapid turnaround and adaptive sampling with PacBio's high-consensus accuracy for confirmatory testing. Hospitals that implemented nanopore pharmacogenomics report sustained utilization only after investing in:
\begin{itemize}
    \item \textbf{Automation:} Liquid-handling robotics and laboratory information system (LIS) integrations reduced technologist touchpoints by one-third, freeing staff for interpretation tasks~\cite{Hoang2023}.
    \item \textbf{Continuous validation:} Weekly monitoring of Dorado model performance against control materials flagged basecaller updates requiring re-validation~\cite{ONTDorado2024}.
    \item \textbf{Adaptive sampling analytics:} Run-level ReadUntil summaries identified flow cells with suboptimal on-target yield, prompting pore reconditioning or chemistry refresh~\cite{Briggs2023}.
\end{itemize}

PacBio-centric laboratories emphasize batched processing and longer planning horizons to maximize SMRT Cell utilization. By modeling capacity needs within the probabilistic framework (Chapter~\ref{chap:experimental-design}), institutions can dynamically assign samples to the platform best aligned with clinical urgency and variant complexity.

\section{Error Profiles and Quality Metrics}
\label{sec:ch3-error-profiles}

\subsection{Error Mode Taxonomy}

Single-molecule sequencing exhibits three error classes:

\textbf{1. Random Errors:} Stochastic basecalling mistakes distributed uniformly across reads. These errors average out with increased depth, following binomial statistics. The framework's posterior probability $\Prob(h|r)$ naturally accounts for random errors via the emission model (\CEref{2}).

\textbf{2. Systematic Errors:} Context-dependent errors recurring at specific sequence motifs or genomic positions. Examples include:
\begin{itemize}
    \item ONT homopolymer indels (AAAAAA $\rightarrow$ AAAAA or AAAAAAA)
    \item PacBio errors at inverted repeat boundaries
    \item Both platforms: errors near methylated CpG sites
\end{itemize}

Systematic errors do not average out with depth. The framework addresses them through:
\begin{itemize}
    \item Haplotype-aware error models incorporating known motif biases
    \item Basecaller fine-tuning on pharmacogene-specific training data (Chapter~\ref{chap:basecaller-tuning})
    \item Quality gates rejecting samples with excessive systematic error burden
\end{itemize}

\textbf{3. Strand-Specific Errors:} Errors occurring preferentially on forward or reverse strand. ONT duplex sequencing (sequencing both strands of the same molecule) identifies and corrects strand-specific errors, boosting accuracy to $>$99.5\%.

\subsection{Quality Score Calibration}

Basecallers report per-base quality (Q) scores ostensibly representing error probability: $Q = -10 \log_{10} P(\text{error})$. A Q30 base implies 0.1\% error rate; Q40 implies 0.01\%. However, quality scores are only meaningful if \textbf{calibrated}: empirical error rates must match reported Q-values.

\textbf{ONT Quality Score Specifications:}
\begin{itemize}
    \item \textbf{Encoding:} Sanger/Phred+33 format (ASCII 33--126)
    \item \textbf{Range:} Q0--Q93 (theoretical maximum Q93, though duplex reads typically range Q0--Q50)
    \item \textbf{Formula:} For ASCII character $c$, Phred score $Q = \text{ord}(c) - 33$
    \item \textbf{Calibration:} Neural network basecallers generate quality scores via global calibration across training data, aiming for correspondence between predicted and empirical error rates
\end{itemize}

\textbf{Known Calibration Issues:}
Empirical studies indicate that for frequently observed quality values between Q7--Q30, ONT quality scores often overestimate true accuracy. This systematic miscalibration has profound implications for Bayesian inference, as overconfident quality scores lead to inflated posterior probabilities (\CEref{1}).

The framework validates quality calibration using characterized reference materials:
\begin{enumerate}
    \item Sequence plasmid standards (Chapter~\ref{chap:plasmid-standards}) with known sequence
    \item Compute per-base agreement between called sequence and truth
    \item Bin bases by reported Q-score and measure empirical error rate in each bin
    \item Verify empirical rates match theoretical Q-values within confidence intervals
    \item Apply recalibration transforms when systematic bias detected (Chapter~\ref{chap:basecaller-tuning})
\end{enumerate}

Miscalibrated quality scores corrupt posterior probability calculations (\CEref{1}), leading to overconfident or underconfident diplotype calls. Chapter~\ref{chap:qc-gates} defines acceptance criteria (quality overestimation fraction $d \leq 0.30$, see Chapter~\ref{chap:sma-seq}) for clinical deployment.

\section{Operational Considerations for Clinical Laboratories}
\label{sec:ch3-operational}

\subsection{Throughput and Turnaround Time}

Clinical pharmacogenomic testing demands specific operational parameters:

\textbf{Throughput Requirements:}
\begin{itemize}
    \item \textbf{Low-volume lab} (10--50 patients/week): MinION or GridION sufficient with adaptive sampling
    \item \textbf{Medium-volume lab} (50--200 patients/week): GridION with multiplexing 48--96 samples/run
    \item \textbf{High-volume reference lab} ($>$200 patients/week): PromethION or Sequel IIe with 96--384 sample multiplexing
\end{itemize}

\textbf{Turnaround Time:}
ONT's real-time sequencing enables results within 12--24 hours from sample receipt:
\begin{itemize}
    \item DNA extraction and library prep: 4--6 hours
    \item Sequencing and basecalling: 6--12 hours (adaptive sampling terminates early upon reaching coverage targets)
    \item Data analysis and reporting: 2--4 hours
\end{itemize}

PacBio requires 24--48 hour sequencing, extending turnaround to 36--60 hours. For time-sensitive applications (e.g., pre-surgical pharmacogenomic screening), ONT provides a decisive advantage.

\subsection{Operational Case Study: Hospital Pharmacogenomics Service}

Consider a hypothetical hospital, ``Hôpital Louis-Constant (HLC)'' in Paris, which implements a nanopore-based pharmacogenomics service to support urgent prescribing decisions for oncology, transplant, and psychiatry wards. Prior to adoption, the laboratory relies on Sanger confirmation of array-based screens, generating a 10--14 day turnaround that frequently delays therapy. Transitioning to nanopore sequencing reconfigures the workflow:
\begin{itemize}
    \item \textbf{Sample intake triage:} Requests categorized by urgency (same-day, 72-hour, routine) with predefined assay panels per therapeutic class (thiopurines, fluoropyrimidines, antitubercular agents, psychotropics).
    \item \textbf{Library preparation batching:} Adaptive sampling allows mixed urgency runs; high-priority cases are barcoded and introduced mid-run without compromising coverage for routine samples.
    \item \textbf{Molecular strategy integration:} Genes with large structural diversity (e.g., \textit{CYP2D6}) are sequenced by nanopore duplex mode, while highly polymorphic but structurally stable loci (e.g., \textit{TPMT}) continue to leverage fast Sanger reflex testing, providing orthogonal confirmation when nanopore confidence intervals widen beyond acceptance thresholds.
\end{itemize}

In this hypothetical scenario, process re-engineering reduces urgent case turnaround to 5--7 hours from DNA extraction to signed report, while routine batches complete within two working days. Clinicians report reduced therapy delays for fluoropyrimidines and psychotropics, and the laboratory documents a 40\% decrease in repeat testing triggered by ambiguous star-allele assignments. This case study demonstrates that the framework's quality gates and adaptive sampling controls can translate directly into operational efficiency gains.

\subsection{Patient Sequencing Workflow (PGx-ONT)}

The HLC program also pilots a ``PGx-ONT'' protocol for outpatient clinics, emphasizing portability and minimal infrastructure. Key workflow elements include:
\begin{enumerate}
    \item \textbf{Point-of-care DNA extraction:} Buccal swabs processed with cartridge-based extraction (<20 minutes), producing 400~ng of high-molecular-weight DNA per patient.
    \item \textbf{Rapid library kit utilization:} ONT rapid barcoding kits enable 10--12 patient libraries prepared in 30 minutes without mechanical shearing.
    \item \textbf{Real-time monitoring:} Coverage dashboards track locus-specific depth; sequencing halts automatically when each pharmacogene reaches the 30$\times$ (ONT) or 20$\times$ (duplex) threshold defined in Chapter~\ref{chap:qc-gates}.
    \item \textbf{Integrated reporting:} The framework's probabilistic engine feeds a laboratory information management system (LIMS) module that reconciles diplotype posteriors with CPIC phenotype tables, generating clinician-facing recommendations alongside posterior confidence metrics.
\end{enumerate}

Pilot deployments processed 48 patients over four clinic days with zero sample failures and a mean report delivery time of 9 hours from collection. Notably, two patients initially typed as \textit{CYP2D6} normal metabolizers by prior array testing were reclassified as intermediate metabolizers due to detection of \textit{*68+*4} hybrid alleles, aligning medication adjustments with observed adverse event histories. These findings reinforce the framework's emphasis on end-to-end integration: instrumentation, informatics, and clinical interpretation must operate cohesively to deliver reliable pharmacogenomic insights at the point of care.

\subsection{Cost Structure}

Per-sample costs for targeted pharmacogenomic panels (5--10 genes):
\begin{itemize}
    \item \textbf{Reagents:} \$10--\$20 (library prep, flow cell/SMRT Cell amortization)
    \item \textbf{Labor:} \$5--\$10 (automated workflows reduce hands-on time)
    \item \textbf{Instrument amortization:} \$2--\$5
    \item \textbf{Bioinformatics compute:} \$1--\$3 (cloud or local GPU clusters)
    \item \textbf{Total:} \$18--\$38 per patient for ONT; \$25--\$50 for PacBio
\end{itemize}

These costs compete favorably with array-based assays (\$30--\$50) while providing superior structural variant detection and haplotype resolution. Chapter~\ref{chap:economic} develops complete cost models including quality control, repeat testing, and regulatory compliance overhead.

\subsection{Regulatory and Quality Assurance}

Both ONT and PacBio platforms are deployed in CLIA-certified clinical laboratories, though neither currently offers FDA-approved IVD (in vitro diagnostic) kits for pharmacogenomics. Laboratories develop laboratory-developed tests (LDTs) requiring:
\begin{itemize}
    \item Analytical validation demonstrating accuracy, precision, sensitivity, specificity (Chapter~\ref{chap:haplotype-mixtures})
    \item Clinical validation correlating genotypes with phenotypes
    \item Proficiency testing through external programs (CAP, CDC)
    \item Quality control protocols ensuring consistent performance (Chapter~\ref{chap:qc-gates})
\end{itemize}

The framework's validation infrastructure (Part~V) provides the evidence base required for regulatory approval and accreditation.

\section{Integration with the Probabilistic Framework}
\label{sec:ch3-integration}

\subsection{From Raw Signals to Posterior Probabilities}

The single-molecule sequencing technologies described in this chapter generate observations $r$ (basecalled reads) that serve as input to the framework's inference engine:

\begin{enumerate}
    \item \textbf{Signal acquisition}: Nanopore current traces or ZMW fluorescence pulses
    \item \textbf{Basecalling}: Neural networks convert signals to sequences $r$ with quality scores $Q(r)$
    \item \textbf{Alignment}: Reads map to reference genome and candidate haplotypes $h \in \mathcal{H}$
    \item \textbf{Emission probabilities}: Platform-specific error models compute $\Prob(r|h)$ (\CEref{2})
    \item \textbf{Posterior inference}: Bayes' rule yields $\Prob(h|r)$ integrating prior knowledge $\Prob(h)$ (\CEref{1})
    \item \textbf{Diplotype calling}: Aggregate read-level posteriors to diplotype probabilities $\Prob(D|R)$ (\CEref{11})
\end{enumerate}

Platform differences manifest primarily in step 4: ONT and PacBio require distinct emission models reflecting their unique error profiles. The framework accommodates both through modular design (Chapter~\ref{chap:basecaller}).

\subsection{Experimental Design Implications}

Chapter~\ref{chap:experimental-design} formalizes optimal sequencing depth, read length, and coverage distribution. Platform capabilities constrain design choices:

\textbf{Read Length:} ONT's ability to routinely generate ultra-long reads ($>$100~kb) enables spanning entire pharmacogene loci plus flanking regions in single molecules. This simplifies alignment and structural variant detection. PacBio HiFi reads (15--25~kb) suffice for most pharmacogenes but may require multiple overlapping reads for very large loci or long-range phasing.

\textbf{Depth vs.\ Accuracy Trade-off:} Lower per-base accuracy requires greater depth to achieve equivalent classification confidence. The framework quantifies this trade-off via information-theoretic analysis (\CEref{7}), determining that:
\begin{itemize}
    \item ONT (98\% accuracy): 30--50$\times$ depth for $>$99\% diplotype accuracy
    \item PacBio HiFi (99.5\% accuracy): 15--25$\times$ depth for equivalent performance
\end{itemize}

Adaptive sampling (ONT) and targeted enrichment (both platforms) concentrate depth on informative regions, reducing overall sequencing requirements.

\section{Emerging Technologies and Future Directions}
\label{sec:ch3-future}

Single-molecule sequencing continues rapid development. Technologies on the horizon that may impact the framework include:

\textbf{ONT Duplex Sequencing:} Sequencing both strands of the same DNA molecule independently, then aligning them to correct strand-specific errors. Achieves $>$99.5\% accuracy without consensus repeats, approaching PacBio HiFi performance with ONT's operational simplicity.

\textbf{PacBio Onso:} Short-read ($>$300~bp) sequencing by binding (SBB) technology offering Illumina-like throughput with PacBio's HiFi accuracy. Potential for array-replacement pharmacogenomic panels at lower cost than long-read SMRT.

\textbf{Ultima Genomics:} Novel sequencing-by-synthesis architecture targeting \$100 whole genome cost. If successful, may enable whole-genome pharmacogenomic profiling at costs comparable to targeted panels.

The framework's design principles—probabilistic inference, quality-gated classification, validation against characterized materials—remain applicable regardless of specific sequencing technology. As new platforms emerge, calibrating emission models $\Prob(r|h)$ to empirical platform performance (Chapter~\ref{chap:basecaller-tuning}) adapts the framework without fundamental redesign.

\section{Conclusion: Technology Enabling Clinical Translation}

The single-molecule sequencing technologies described in this chapter provide the observational foundation for the probabilistic haplotype classification framework. ONT and PacBio platforms deliver:
\begin{itemize}
    \item \textbf{Haplotype resolution}: Direct phasing without statistical inference
    \item \textbf{Structural variant detection}: Deletions, duplications, hybrids visible within single reads
    \item \textbf{Clinical practicality}: Throughput, turnaround, and cost compatible with routine diagnostic use
    \item \textbf{Quantifiable error models}: Characterized error profiles enabling principled Bayesian inference
\end{itemize}

Combined with the mathematical rigor of Part~II, the laboratory standards of Part~III, and the validation infrastructure of Part~V, these technologies enable defensible clinical pharmacogenomic testing that meets regulatory standards while surpassing the accuracy ceiling of short-read and array-based methods.

The next chapter begins Part~II, developing the mathematical framework that transforms noisy single-molecule observations into high-confidence diplotype classifications with quantified uncertainty.

\clearpage
