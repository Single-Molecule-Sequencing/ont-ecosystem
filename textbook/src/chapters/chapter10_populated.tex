%%%%%%%%%%%%%%%%%%%%%%%%%%%%%%%%%%%%%%%%%%%%%%%%%%%%%%%%%%%%%%%%%%%%%%%%
%% Chapter 10: From Sample to Analyzed Data (Complete Workflow)
%% Part III: Physical Standards, Library Preparation, and Workflows
%% Version 6.0 - Migrated from v5 Ch8 + expansions
%%%%%%%%%%%%%%%%%%%%%%%%%%%%%%%%%%%%%%%%%%%%%%%%%%%%%%%%%%%%%%%%%%%%%%%%

\chapter{From Sample to Analyzed Data: The Complete Workflow}
\label{chap:workflow}
\label{chap:signal-to-sequence}
\label{chap:data-pipeline}

The haplotype classification framework developed in Parts II-III requires integration of multiple experimental and computational components: physical standards construction (Chapter~\ref{chap:plasmid-standards}), optional targeted enrichment (Chapter~\ref{chap:targeted-enrichment}), single-molecule sequencing, empirical error characterization (Chapter~\ref{chap:sma-seq}), and Bayesian inference (Chapter~\ref{chap:posteriors}). This chapter presents the complete operational workflow, from sample collection through clinical report generation, emphasizing practical implementation details and quality checkpoints.

The workflow architecture mirrors standard genomics pipelines but incorporates framework-specific requirements: (1) \textbf{Primary analysis} converts raw signals to base calls with quality scores, (2) \textbf{Secondary analysis} performs exploratory mapping for quality assessment (not classification), and (3) \textbf{Tertiary analysis} executes haplotype classification using the alignment-free framework. Clear separation of these stages enables modular implementation and systematic troubleshooting.

\begin{learningobjectives}
By the end of this chapter, you will be able to:
\begin{itemize}
\item Distinguish primary (basecalling), secondary (QC mapping), and tertiary (classification) analysis stages
\item Execute complete workflow from biological sample to clinical report
\item Install and configure software dependencies (Guppy/Dorado, minimap2, samtools, custom Python scripts)
\item Implement all five QC gates at appropriate workflow checkpoints
\item Optimize computational performance: parallel processing, GPU acceleration, caching strategies
\item Troubleshoot common workflow failures: basecalling errors, alignment issues, posterior computation problems
\item Estimate computational requirements: <1 hour per sample on modern workstation
\item Archive essential data while managing storage constraints (70-90\% compression for FASTQ)
\end{itemize}
\end{learningobjectives}

\section{Workflow Overview}

\begin{figure}[!htbp]
\centering
\fbox{\begin{minipage}{0.9\textwidth}
\centering
\textbf{Complete Haplotype Classification Pipeline} \\
\vspace{0.5cm}
\begin{tabular}{c}
\textbf{Sample Preparation} \\
$\downarrow$ \\
DNA extraction \& QC \\
$\downarrow$ \\
Fragmentation (optional) \\
$\downarrow$ \\
Cas9 enrichment (optional) \\
$\downarrow$ \\
Library preparation \\
$\downarrow$ \\
\hline
\textbf{Sequencing} \\
$\downarrow$ \\
Single-molecule sequencing \\
$\downarrow$ \\
Raw signal acquisition \\
$\downarrow$ \\
\hline
\textbf{Primary Analysis} \\
$\downarrow$ \\
Basecalling \\
$\downarrow$ \\
Quality score assignment \\
$\downarrow$ \\
FASTQ generation \\
$\downarrow$ \\
\hline
\textbf{Secondary Analysis (QC only)} \\
$\downarrow$ \\
Reference alignment \\
$\downarrow$ \\
Coverage assessment \\
$\downarrow$ \\
Quality metrics \\
$\downarrow$ \\
\hline
\textbf{Tertiary Analysis (Classification)} \\
$\downarrow$ \\
Confusion matrix estimation (if standards) \\
$\downarrow$ \\
Per-read likelihood computation \\
$\downarrow$ \\
Posterior probability calculation \\
$\downarrow$ \\
Haplotype assignment \\
$\downarrow$ \\
Quality control gates \\
$\downarrow$ \\
Clinical report
\end{tabular}
\end{minipage}}
\caption{End-to-end workflow from biological sample to clinical classification result. Dashed boxes indicate optional steps depending on application.}
\label{fig:complete-workflow}
\end{figure}

\section{Primary Analysis: From Signals to Sequences}

Primary analysis converts platform-specific raw signals (electrical current changes for ONT, photon pulses for PacBio/Illumina) into nucleotide sequences with associated quality scores.

\subsection{Signal Acquisition}

\textbf{Oxford Nanopore Technology (ONT):}
\begin{itemize}
\item Raw signals: FAST5 or POD5 format containing ionic current measurements
\item Sampling rate: 4-5 kHz (4000-5000 measurements per second)
\item Signal characteristics: Current level shifts as different k-mers pass through pore
\item File size: 1-10 MB per read depending on length and sampling rate
\end{itemize}

\subsection{ONT Run Metadata: \texttt{sequencing\_summary.tsv}}

MinKNOW and Dorado emit a \texttt{sequencing\_summary.tsv} manifest whose
columnar schema is versioned (current release~1.1) by Oxford Nanopore
Technologies.\footnote{Official specification:
\url{https://github.com/nanoporetech/ont-output-specifications/blob/main/sequencing_summary/spec.yaml}}
Each release is referenced in the header preamble, enabling automated schema
negotiation when multiple software versions coexist. Lightweight validators
can parse the YAML definition and confirm column presence, types, and allowed
values prior to ingestion. Every row represents a single basecalled read with
pointers back to the raw signal, per-pore acquisition context, and downstream
analysis results. Using the schema as intended is essential for reproducible
traceability, diagnosing run-level anomalies, and reconciling telemetry across
software upgrades. When importing manifests into workflow managers or LIMS,
persist the \texttt{schema\_version} alongside a manifest checksum to preserve
audit trails.

\begin{table}[!htbp]
\centering
\caption{Selected fields in the ONT \texttt{sequencing\_summary} specification}
\label{tab:ont-sequencing-summary}
\begin{tabular}{p{0.28\textwidth}p{0.62\textwidth}}
\toprule
\textbf{Field(s)} & \textbf{Description and workflow use} \\
\midrule
\texttt{filename\_fastq}/\texttt{filename\_fast5}/\texttt{filename\_pod5}/\texttt{filename\_bam}/\texttt{input\_filename} &
Online basecalling populates per-read pointers to FASTQ, FAST5/POD5, and BAM
artifacts, while offline re-basecalling records the originating POD5/FAST5 in
\texttt{input\_filename}. Keeping these paths in laboratory notebooks enables
auditable linkage between archived signal, emitted basecalls, and tertiary
analysis inputs. \\
\texttt{batch\_id}, \texttt{parent\_read\_id}, \texttt{read\_id}, \texttt{run\_id} & A
numeric \texttt{batch\_id} differentiates processing batches in offline
workflows. UUIDs identify the sequencing run, the parent raw signal chunk, and
any child reads produced by splitting or duplex polishing, supporting
deduplication and duplex traceability. \\
\texttt{channel}, \texttt{mux}, \texttt{minknow\_events}, \texttt{start\_time}, \texttt{duration}, \texttt{template\_start}, \texttt{template\_duration}, \texttt{num\_events\_template} &
Per-pore acquisition metadata reveals pore loading, mux cycling, and
electrical stability. Tracking event counts and dwell times across these
columns helps flag failing pores and validates hardware interventions noted in
run logs. \\
\texttt{passes\_filtering}, \texttt{sequence\_length\_template}, \texttt{mean\_qscore\_template} &
Instrument filters and basecaller-reported length/Q-score statistics provide a
lightweight proxy for Gate~4 calibration checks and dataset-wide length
distribution monitoring without reparsing FASTQ files. \\
\texttt{end\_reason} & Categorical reason for read termination. Reported values
include \texttt{signal\_positive}, \texttt{signal\_negative}, \texttt{mux\_change},
\texttt{unblock\_mux\_change}, \texttt{data\_service\_unblock\_mux\_change},
\texttt{analysis\_config\_change}, \texttt{device\_data\_error}, \texttt{api\_request},
\texttt{paused}, and \texttt{unknown}. Yield stratified by termination mode
distinguishes natural completion from active unblocks or hardware faults. \\
\texttt{pore\_type}, \texttt{experiment\_id}, \texttt{sample\_id} & User-supplied run
metadata should reconcile with laboratory information systems. Automated
validations prevent sample swaps, enforce chemistry-specific model selection,
and ensure the correct pore type was used during downstream analysis. \\
\texttt{poly\_tail\_*} (RNA only) & Poly-A tail length and positional estimates
are emitted when the poly-A estimation module runs. They support RNA-specific
QC such as verifying tail trimming parameters or detecting incomplete cDNA
synthesis. \\
\texttt{barcode\_*} (when barcoding) & Barcode kit, arrangement, alias, and front
/ rear confidence scores drive demultiplexing QC. Monitoring barcode scores and
\texttt{type} (test sample vs controls) guards against barcode hopping and
incorrect control labelling. \\
\texttt{alignment\_*} (when aligning) & Optional alignment genome, orientation,
accuracy, and BED hit metrics summarise exploratory mapping performed within
MinKNOW/Dorado, enabling rapid on-target assessment without rebuilding reports.
\\
\texttt{duplex\_parent\_template}/\texttt{duplex\_parent\_complement} (duplex) & When
duplex polishing succeeds these UUIDs link the fused read back to its template
and complement parents, preserving provenance for re-analysis or auditing. \\
\texttt{instrument\_type}, \texttt{instrument\_serial\_number}, \texttt{asic\_id}, \texttt{asic\_temp}, \texttt{asic\_version} &
Hardware identifiers and temperature telemetry capture real-time instrument
state. Tracking \texttt{asic\_temp} alongside throughput highlights thermal
drift, while \texttt{instrument\_serial\_number} links deviations to specific
devices for service reports. \\
\texttt{flow\_cell\_id}, \texttt{flow\_cell\_product\_code}, \texttt{pore\_version}, \texttt{sequencing\_kit}, \texttt{flow\_cell\_position} &
Consumable metadata documents chemistry lots and loading positions. Pairing
these fields with inventory databases accelerates root-cause analysis of pore
failure and enforces compatibility checks during automated run provisioning. \\
\texttt{protocol\_group\_id}, \texttt{protocol\_run\_id}, \texttt{protocol}, \texttt{protocol\_version}, \texttt{protocol\_start}, \texttt{protocol\_duration} &
Protocol identifiers mirror MinKNOW scheduling metadata. They support
comparisons between protocol variants (for example, adaptive sampling vs.
standard sequencing) and reconstruct pauses or restarts observed in the run
log. \\
\texttt{tracking\_id.*} & Nested tracking-card entries---covering operator,
laboratory, and sample alias fields---provide the linkage required for
regulatory record keeping. Populate them consistently to enable cross-run
analytics on operator effects or training interventions. \\
\bottomrule
\end{tabular}
\end{table}

\paragraph{Schema negotiation and ingestion pipelines.} The YAML specification
defines enumerations, numeric bounds, and dependencies between optional module
columns. Automated import pipelines should validate incoming manifests against
these constraints before loading them into production data stores, rejecting or
quarantining rows that fail schema checks. Because optional features (for
example, duplex polishing or on-instrument alignment) append columns, parsing
code should rely on column names rather than positional indices and persist the
header metadata (including the schema URL) to make reprocessing under future
schema revisions reproducible.

\paragraph{Column categories and data types.} The specification annotates each
column with a primitive type and (when applicable) a units comment or a
regular-expression constraint. Mapping these declarations into your ingestion
layer prevents subtle coercion errors:
\begin{itemize}
  \item \textbf{Identifiers and paths} (for example, filenames, UUIDs, kit
  codes) are strings constrained by regular expressions; treat them as
  case-sensitive text to preserve leading zeros and hexadecimal casing.
  \item \textbf{Counts and durations} (such as \texttt{num\_events\_*},
  \texttt{duration}, \texttt{template\_duration}) are integers representing raw
  event counts or milliseconds; convert to floating-point seconds only when
  required for analytics and persist the original integer for audit parity.
  \item \textbf{Continuous telemetry} (for example, \texttt{mean\_qscore\_*},
  \texttt{asic\_temp}, \texttt{calibration\_strain\_rate}) are floats; enforce
  the schema-provided numeric bounds to catch corrupted rows or locale-induced
  decimal separator issues.
  \item \textbf{Enumerations} (such as \texttt{end\_reason},
  \texttt{experiment\_type}, \texttt{barcode\_type}) should be loaded into
  categorical columns so that dashboards can surface unexpected values as
  validation failures rather than silently plotting them.
\end{itemize}

\begin{table}[!htbp]
\centering
\caption{Examples of schema-constrained ONT sequencing summary fields}
\label{tab:ont-sequencing-summary-types}
\begin{tabular}{p{0.23\textwidth}p{0.18\textwidth}p{0.49\textwidth}}
\toprule
\textbf{Field} & \textbf{Type / units} & \textbf{Constraint or ingestion note} \\
\midrule
\texttt{start\_time} & Integer (ms) & Must be $\geq 0$; combine with
\texttt{duration} to reconstruct wall-clock timestamps. \\
\texttt{calibration\_strain\_rate} & Float & Limited to 0--500; clamp
out-of-range values and alert instrumentation team. \\
\texttt{experiment\_type} & Enum & Allowed values listed in specification;
use to drive protocol-specific QC. \\
\texttt{tracking\_id.user\_specified\_run\_id} & Regex string & Accepts
alphanumeric, dash, and underscore; block whitespace during validation. \\
\texttt{passes\_filtering} & Boolean & Represented as 0/1; convert to logical
type in downstream warehouses. \\
\texttt{run\_id} & UUID (string) & Validate length (36 including hyphens) and
hyphen positions to detect truncated identifiers. \\
\bottomrule
\end{tabular}
\end{table}

\paragraph{Interpreting module groups.} Columns are organized into logical
modules (core, basecalling, barcoding, alignment, duplex, adaptive sampling)
within the specification. Downstream analytics can use these groupings to drive
feature availability---for instance, enabling duplex QC plots only when both
\texttt{duplex\_parent\_template} and \texttt{duplex\_parent\_complement} are present. A
simple approach is to build module capability maps keyed by the schema version
so that dashboards and reports degrade gracefully when a module is absent.

\paragraph{Linking to signal archives.} Although POD5 supersedes FAST5 for new
deployments, many laboratories operate mixed archives. The manifest captures
the exact filenames and checksums for raw signal and derived FASTQ/BAM outputs,
allowing archival systems to snapshot the manifest alongside each object store
prefix, tape cartridge, or external drive. Restoration workflows can then
confirm bit-level integrity before re-basecalling and avoid orphaned signal
containers.

\paragraph{Worked example: schema-aware ingestion.} A robust production import
routine can be implemented with commodity tooling:
\begin{enumerate}
  \item Fetch the published YAML schema (pinned by \texttt{schema\_version}) and
  compile it into a validator object.
  \item Load the TSV into a typed dataframe using column name lookups, not
  ordinal positions, and set strict converters for UUID, integer, float, and
  categorical columns.
  \item Apply the validator row-wise, routing failures to a quarantine table
  with human-readable error messages.
  \item Materialize success rows into the warehouse together with the manifest
  checksum, source URI, and ingestion timestamp.
  \item Emit a reconciliation report summarizing record counts, module
  availability (for example, number of rows containing \texttt{barcode\_*}), and
  notable enumerations to support QC sign-off.
\end{enumerate}

\paragraph{Operational considerations for online vs. offline runs.} The
\texttt{only\_when} attribute in the specification differentiates between
columns populated during live (MinKNOW-hosted) sequencing and those emitted when
basecalling is run offline. Build conditional ingestion logic that honours
these flags, so that missing columns do not trigger false-positive alerts when a
module is legitimately absent. Conversely, require offline manifests to include
\texttt{input\_filename} and \texttt{batch\_id}, ensuring re-basecalling jobs can
be traced back to their signal sources.

\textbf{Recommended QC checks:}
\begin{itemize}
\item \textbf{Traceability:} Reconcile \texttt{filename\_*}, \texttt{input\_filename},
and UUID columns before archiving or deleting POD5/FAST5 containers. Offline
re-basecalling should produce consistent \texttt{batch\_id} groupings.
\item \textbf{Channel health:} Plot \texttt{duration}, \texttt{minknow\_events}, and
\texttt{template\_duration} by \texttt{channel}/\texttt{mux} to confirm even pore
utilization and detect fouled channels.
\item \textbf{Termination modes:} Track \texttt{end\_reason} distributions---including
the \texttt{unknown} class---to spot excessive unblocks, pausing, or device
errors prior to tertiary analysis.
\item \textbf{Sample bookkeeping:} Cross-check \texttt{experiment\_id},
\texttt{sample\_id}, and \texttt{alias}/\texttt{type} assignments against the laboratory
information system during Gate~1 intake verification.
\item \textbf{Barcode/duplex performance:} Monitor \texttt{barcode\_*} scores for weak or
negative-control assignments and verify duplex parent IDs are populated when
duplex calling is enabled.
\item \textbf{Instrumentation drift:} Trend \texttt{asic\_temp}, \texttt{asic\_version}, and
\texttt{instrument\_serial\_number} across runs to spot devices requiring
maintenance or recalibration.
\item \textbf{Consumable accountability:} Audit \texttt{flow\_cell\_id}, \texttt{sequencing\_kit},
and \texttt{pore\_version} against procurement records to surface out-of-spec
chemistry batches quickly.
\item \textbf{Protocol compliance:} Plot \texttt{protocol\_start}, \texttt{protocol\_duration},
and \texttt{end\_reason} to document pauses, restarts, or shortened runs that
could bias downstream coverage or introduce selection artefacts.
\end{itemize}

\paragraph{Schema surveillance and compatibility testing.} Treat the public
YAML schema as a dependency: mirror the file in version control, write unit
tests that diff the mirrored copy against the upstream release, and gate
pipeline deployments on an explicit review when new columns, enumerations, or
numeric bounds appear. The specification includes a per-column
\texttt{required} flag plus \texttt{only\_when} contexts, enabling regression
tests that assert, for example, that online manifests never drop
\texttt{filename\_fastq} while offline manifests always include
\texttt{input\_filename}. CI systems can fetch candidate manifests, validate
them against the pinned schema, and fail fast if Dorado or MinKNOW updates add
fields that ingestion code has not yet whitelisted.

\paragraph{Metadata normalization for downstream joins.} Several columns share
semantic domains with laboratory tracking systems (for example,
\texttt{tracking\_id.sample\_barcode} vs. LIS sample identifiers). Normalize the
case, padding, and delimiter rules during ingestion so that referential
integrity checks can be enforced automatically. A minimal strategy is to build
lookup tables keyed by \texttt{run\_id} and \texttt{tracking\_id.user\_specified\_run\_id}
so that clinical reports can reference either identifier without ambiguity.
When exporting data to tertiary analysis environments, provide the normalized
identifiers plus the raw manifest values to preserve provenance and allow
ad-hoc debugging of mismatched records.

\paragraph{Bridging reads, raw signal, and event-level archives.} Beyond the
per-file pointers, the manifest exposes event counts
(\texttt{minknow\_events}, \texttt{num\_events\_*}) and timestamps
(\texttt{start\_time}, \texttt{duration}) that can reconstitute the time window
occupied on each pore. Archival services can use these data to chunk POD5 or
FAST5 collections by run and time, enabling partial restores when only a
problematic interval requires re-analysis. Pairing read-level metadata with the
\texttt{tracking\_id.device\_id} and \texttt{tracking\_id.flow\_cell\_id} subfields further
supports investigations into flow cell swaps or handling errors.

\begin{table}[!htbp]
\centering
\caption{Operational modules signaled by \texttt{sequencing\_summary.tsv}}
\label{tab:ont-module-capabilities}
\begin{tabular}{p{0.25\textwidth}p{0.65\textwidth}}
\toprule
\textbf{Module cue} & \textbf{Operational action} \\
\midrule
\texttt{alignment\_*} columns present & Generate quick-on-target dashboards and
verify that reference genomes/targets match project manifests before launching
full alignments. \\
\texttt{barcode\_*} columns present & Enforce demultiplexing QC gates: confirm all
expected barcodes appear, and alert when \texttt{barcode\_type} marks a control
sample as a test subject. \\
\texttt{poly\_tail\_*} columns present & For RNA programs, summarize tail lengths by
library to confirm enzymatic steps behaved as expected and adjust trimming
parameters for downstream aligners. \\
\texttt{adaptive\_sampling\_*} columns present & Cross-check target/untarget bed
files and log enrichment factors; missing columns imply the run defaulted to
standard sequencing even if adaptive sampling was requested. \\
\texttt{duplex\_parent\_*} columns present & Enable duplex yield reporting and
ensure downstream classifiers handle duplex consensus reads separately from
single-strand template/complement reads. \\
\bottomrule
\end{tabular}
\end{table}

\paragraph{Reconciliation against FASTQ/BAM inventories.} Because large runs may
split outputs across dozens of compressed files, it is easy for archive
operations to lose track of which manifest rows map to which FASTQ archive. Use
the \texttt{filename\_*} columns, together with manifest record counts, to build
per-file digests (for example, `sha256sum output.fastq.gz > fastq.sha256`).
Warehouse loaders can then join the manifest to the checksum table, detect
partially transferred files immediately, and prevent stale FASTQ bundles from
entering tertiary analysis.

\paragraph{Example validator implementation.} The following pseudocode outlines
how to wrap the YAML definition in a lightweight Python validation layer:
\begin{verbatim}
schema = load_schema("spec.yaml")
df = pandas.read_csv("sequencing_summary.tsv", sep="\t",
                     dtype=schema.pandas_dtypes())
errors = []
for column in schema.columns:
    if column.only_when == "online" and offline_mode:
        continue
    if column.required and column.name not in df:
        errors.append(f"Missing {column.name}")
    errors.extend(column.validate(df[column.name]))
if errors:
    raise ValueError("Manifest validation failed", errors)
\end{verbatim}
This pattern mirrors the structure of the published specification---including
per-column regex checks and enumerations---and produces actionable error
messages for operators.

\paragraph{Regression tests for manifest ingestion.} Treat manifest parsing as
production code worthy of automated testing. Minimal coverage should include:
\begin{itemize}
  \item \textbf{Schema drift alarms:} Unit tests that deserialize the pinned
  YAML into a fixture and fail when upstream adds or removes columns. Keeping a
  frozen copy of the specification in version control allows `pytest` (or an
  equivalent framework) to alert operators immediately after MinKNOW/Dorado
  upgrades.
  \item \textbf{Round-trip fidelity:} Property tests that write a synthetic
  \texttt{sequencing\_summary.tsv} containing edge-case values (maximum duration,
  rare \texttt{end\_reason}, mixed optional modules) and confirm the ingestion
  layer preserves types, enumerations, and null semantics when re-exported.
  \item \textbf{Constraint enforcement:} Negative tests that intentionally break
  regex rules (for example, malformed UUIDs) or omit required columns to verify
  the validator raises descriptive exceptions instead of silently coercing
  values.
  \item \textbf{Online/offline toggles:} Tests that simulate both execution
  contexts so that \texttt{only\_when} logic is exercised before a real offline
  re-basecalling job runs.
\end{itemize}
The resulting test suite is lightweight yet powerful enough to block CI merges
when ingestion changes would otherwise corrupt provenance.

\textbf{PacBio HiFi:}
\begin{itemize}
\item Raw signals: Fluorescence pulse recordings (proprietary format)
\item Sampling: Polymerase kinetics during nucleotide incorporation
\item Signal characteristics: Inter-pulse duration (IPD) varies by base and modifications
\item Processing: Circular consensus sequencing (CCS) generates HiFi reads from multiple passes
\end{itemize}

\textbf{Illumina (if applicable):}
\begin{itemize}
\item Raw signals: BCL (base call) files from image analysis
\item Sampling: Fluorescence intensity per cycle per cluster
\item Processing: Real-time base calling during sequencing (RTA software)
\item Output: FASTQ files directly from instrument
\end{itemize}

\subsection{Basecalling}

\textbf{For Oxford Nanopore:}

\begin{protocol}[ONT Basecalling Protocol]
\textbf{Software options:}
\begin{itemize}
\item \textbf{Guppy (deprecated):} CPU and GPU versions; standard for older data
\item \textbf{Dorado:} Current ONT basecaller; GPU-accelerated; improved accuracy
\item \textbf{Bonito:} Research basecaller; customizable models
\end{itemize}

\textbf{Recommended command (Dorado):}
\begin{verbatim}
dorado basecaller \
  --model dna_r10.4.1_e8.2_400bps_hac@v4.2.0 \
  --emit-fastq \
  --verbose \
  pod5_files/ > output.fastq
\end{verbatim}

\textbf{Model selection:}
\begin{itemize}
\item \texttt{fast}: Lower accuracy (~95\%), faster processing
\item \texttt{hac} (high-accuracy): Standard choice (~98-99\%)
\item \texttt{sup} (super-accuracy): Highest accuracy (~99+\%), slower
\end{itemize}

\begin{remark}[SEER Benchmarking of Basecalling Models]
The choice between Dorado basecalling models (fast, hac, sup) represents a classical speed--accuracy trade-off. Benchmarking runs on the same SMA-seq standards with each model produce three confusion matrices $C_{\text{fast}}$, $C_{\text{hac}}$, $C_{\text{sup}}$ and corresponding metrics: SMA, TPR, and quality overstatement fraction $d$ (Chapter~\ref{chap:sma-seq}). These SEER-derived metrics enable quantitative comparison: for example, a $3\times$ increase in runtime from hac to sup may yield only a 0.2\% absolute increase in TPR but a 5\% reduction in miscalibrated quality, which may or may not justify the cost depending on the clinical setting. Integrating these empirical metrics into workflow decisions ensures that basecaller selection is driven by data rather than informal impressions of ``accuracy''.
\end{remark}

\textbf{Key parameters:}
\begin{itemize}
\item \texttt{--min-qscore}: Filter reads below quality threshold (e.g., Q10)
\item \texttt{--device}: Specify GPU (cuda:0) or CPU
\item \texttt{--batch-size}: Optimize for available GPU memory
\end{itemize}
\end{protocol}

\textbf{For PacBio HiFi:}

HiFi reads are typically pre-processed by instrument software (SMRT Link):
\begin{itemize}
\item CCS generates consensus from multiple passes of circular template
\item Quality scores reflect agreement across passes
\item Typical Q20+ accuracy (~99\%) after CCS
\item Output: BAM files with HiFi reads
\end{itemize}

\textbf{Post-processing (if needed):}
\begin{verbatim}
# Convert BAM to FASTQ if needed
samtools fastq input.bam > output.fastq

# Filter by quality
seqtk seq -Q 20 input.fastq > filtered.fastq
\end{verbatim}

\subsection{Quality Score Calibration}

As developed in Chapter~\ref{chap:classification-model}, quality scores should reflect true error probabilities via \CEref{1}:
\begin{equation}
p_{\text{err}}(Q) = 10^{-Q/10}
\end{equation}

\textbf{Calibration check (from standards):}
\begin{enumerate}
\item Sequence plasmid standards with known sequence
\item Compute empirical error rate stratified by reported Q-score
\item Compare empirical error rate to $p_{\text{err}}(Q)$ formula
\item If systematic deviation observed, apply correction or re-train basecaller
\end{enumerate}

This calibration validation is formalized as Quality Control Gate 4 in Chapter~\ref{chap:qc-gates}.

\section{Secondary Analysis: Exploratory Mapping and QC}

\textbf{Critical distinction:} Secondary analysis in this framework serves \textit{quality assessment only}. Haplotype classification does NOT use alignments; it operates directly on unaligned reads via the confusion matrix methodology (Chapters~\ref{chap:classification-model} and \ref{chap:posteriors}).

\subsection{Reference Alignment}

\textbf{Purpose of alignment:}
\begin{itemize}
\item Assess on-target rate for enriched sequencing
\item Visualize coverage distribution
\item Identify low-quality regions requiring investigation
\item Validate that reads map to expected genomic location
\end{itemize}

\textbf{Alignment tools:}

\begin{table}[!htbp]
\centering
\caption{Alignment Software for Quality Control}
\label{tab:alignment-tools}
\begin{tabular}{llll}
\toprule
\textbf{Tool} & \textbf{Platform} & \textbf{Speed} & \textbf{Use Case} \\
\midrule
minimap2 & ONT, PacBio & Fast & Long reads, exploratory mapping \\
BWA-MEM & Illumina & Medium & Short reads, standard pipeline \\
NGMLR & ONT, PacBio & Slow & Structural variant detection \\
Winnowmap & ONT, PacBio & Fast & Repetitive regions \\
\bottomrule
\end{tabular}
\end{table}

\textbf{Recommended command (minimap2):}
\begin{verbatim}
minimap2 -ax map-ont \
  reference.fasta \
  reads.fastq | \
  samtools sort -o aligned.bam -
  
samtools index aligned.bam
\end{verbatim}

\subsection{Coverage Assessment}

\textbf{Key metrics:}
\begin{itemize}
\item \textbf{Mean coverage:} Average read depth across target region
\item \textbf{Coverage uniformity:} Coefficient of variation of coverage
\item \textbf{Fraction at threshold:} Percentage of bases with $\geq N$× coverage
\end{itemize}

\textbf{Computation:}
\begin{verbatim}
# Compute per-base coverage
samtools depth -a aligned.bam > coverage.txt

# Summarize statistics
awk '{sum+=$3; count++} END {print "Mean:", sum/count}' coverage.txt
\end{verbatim}

\textbf{Quality criteria (from Chapter~\ref{chap:design}):}
\begin{itemize}
\item For binary classification: Mean coverage $\geq$ 100× (haploid), $\geq$ 200× (diploid)
\item For mixture detection: Adjust based on power analysis (\CEref{7})
\item Uniformity: CV $<$ 0.5 indicates acceptable coverage distribution
\end{itemize}

\subsection{Read Quality Metrics}

\begin{protocol}[Quality Metrics Extraction]
\textbf{Per-read metrics:}
\begin{itemize}
\item Read length distribution (check against expected fragment distribution)
\item Mean quality score per read
\item Mapping quality (MAPQ) distribution
\item Alignment identity/accuracy
\item End reason metadata (for ONT data; see Chapter~\ref{chap:sma-seq} Section on Read Completeness)
\end{itemize}

\textbf{Tools:}
\begin{verbatim}
# Summary statistics
samtools flagstat aligned.bam

# Read length distribution
samtools view aligned.bam | \
  awk '{print length($10)}' | \
  sort -n | uniq -c > read_lengths.txt
  
# Quality score distribution
seqkit stats -a reads.fastq
\end{verbatim}

\textbf{Interpretation:}
\begin{itemize}
\item Mapping rate $>$ 90\%: Good quality
\item Mapping rate 70-90\%: Acceptable; investigate unmapped reads
\item Mapping rate $<$ 70\%: Poor quality; check for contamination or wrong reference
\item For ONT data: Consider filtering reads by end reason (retain signal\_positive, exclude unblock\_mux\_change and signal\_negative for high-confidence analysis). See Chapter~\ref{chap:sma-seq} for read completeness analysis and filtering recommendations.
\end{itemize}
\end{protocol}

\section{Tertiary Analysis: Haplotype Classification}

This stage implements the core mathematical framework: confusion matrix estimation from standards, likelihood computation for clinical samples, and posterior probability calculation.

\subsection{Confusion Matrix Estimation (Standards)}

If running standards to build or validate confusion matrix:

\begin{algorithm}[H]
\caption{SEER Confusion Matrix Estimation}
\label{alg:confusion-matrix}
\textbf{Input:} FASTQ files from plasmid standards $h_1, h_2, \ldots, h_K$ \\
\textbf{Output:} Confusion matrix $\mathbf{C}$ with elements $C_{ij}$

\begin{algorithmic}[1]
\STATE Initialize count matrix $\mathbf{N}$ with zeros ($K \times K$)
\FOR{each standard $i = 1$ to $K$}
    \STATE Load reads $\{r_n\}$ from standard $h_i$
    \FOR{each read $r_n$}
        \STATE Compute edit distance to all haplotypes: $d(r_n, h_j)$ for $j = 1, \ldots, K$
        \STATE Classify to nearest haplotype: $j^* = \arg\min_j d(r_n, h_j)$
        \STATE Increment count: $N_{ij^*} \leftarrow N_{ij^*} + 1$
    \ENDFOR
\ENDFOR
\STATE Normalize rows to obtain probabilities:
\STATE $\qquad C_{ij} = N_{ij} / \sum_{k=1}^{K} N_{ik}$
\RETURN Confusion matrix $\mathbf{C}$
\end{algorithmic}
\end{algorithm}

\textbf{Implementation notes:}
\begin{itemize}
\item Edit distance computation: Use dynamic programming (Needleman-Wunsch) or heuristic alignment
\item Parallelization: Process reads independently across multiple cores
\item Memory management: Stream reads from FASTQ to avoid loading entire file
\item Quality filtering: Optionally filter reads with mean Q-score $<$ threshold before classification
\end{itemize}

\subsection{Clinical Sample Likelihood Computation}

For clinical samples, compute likelihoods using the confusion matrix:

\begin{algorithm}[H]
\caption{Per-Read Likelihood via Confusion Matrix}
\label{alg:read-likelihood}
\textbf{Input:} Read $r_n$, confusion matrix $\mathbf{C}$, haplotypes $\{h_j\}$ \\
\textbf{Output:} Likelihood vector $[\Prob(r_n|h_1), \ldots, \Prob(r_n|h_K)]$

\begin{algorithmic}[1]
\STATE Compute edit distance to all haplotypes: $d_j = d(r_n, h_j)$ for $j = 1, \ldots, K$
\STATE Find nearest haplotype: $j^* = \arg\min_j d_j$
\STATE Read $r_n$ is classified as coming from ``class'' $j^*$
\FOR{each haplotype $h_i$}
    \STATE Likelihood is probability that true sequence $h_i$ produces observation $j^*$:
    \STATE $\qquad \Prob(r_n | h_i) = C_{i,j^*}$
\ENDFOR
\RETURN Likelihood vector
\end{algorithmic}
\end{algorithm}

\textbf{Numerical stability:} Perform computations in log-space to avoid underflow:
\begin{equation}
\log \Prob(r_n | h_i) = \log C_{i,j^*}
\end{equation}

\subsection{Dataset Log-Likelihood and Posterior}

Aggregate over all $N$ reads:

\begin{align}
\log \Prob(\mathbf{r} | h_i) &= \sum_{n=1}^{N} \log \Prob(r_n | h_i) \label{eq_10_5} \\
&= \sum_{n=1}^{N} \log C_{i, j^*_n}
\end{align}

where $j^*_n$ is the classified category for read $r_n$.

\textbf{Posterior probability} via Bayes' rule (Chapter~\ref{chap:posteriors}):
\begin{equation}
\Prob(h_i | \mathbf{r}) = \frac{\Prob(h_i) \cdot \Prob(\mathbf{r} | h_i)}{\sum_{k=1}^{K} \Prob(h_k) \cdot \Prob(\mathbf{r} | h_k)}
\label{eq_10_3}
\end{equation}

In log-space using log-sum-exp trick:
\begin{align}
\log \Prob(h_i | \mathbf{r}) &= \log \Prob(h_i) + \log \Prob(\mathbf{r} | h_i) - \log Z \\
\text{where} \quad \log Z &= \text{LogSumExp}_k \left[ \log \Prob(h_k) + \log \Prob(\mathbf{r} | h_k) \right]
\end{align}

\textbf{Log-sum-exp trick:}
\begin{equation}
\log \sum_{k} \exp(x_k) = x_{\max} + \log \sum_{k} \exp(x_k - x_{\max})
\end{equation}
where $x_{\max} = \max_k x_k$ prevents overflow/underflow.

\subsection{Diplotype Inference for Clinical Samples}

For diploid samples, consider all possible diplotype configurations:

\begin{algorithm}[H]
\caption{Diplotype Posterior Computation}
\label{alg:diplotype-posterior}
\textbf{Input:} Reads $\mathbf{r}$, confusion matrix $\mathbf{C}$, haplotypes $\{h_j\}$, prior $\Prob(h_i, h_j)$ \\
\textbf{Output:} Diplotype posterior $\Prob(h_i, h_j | \mathbf{r})$

\begin{algorithmic}[1]
\FOR{each diplotype $(h_i, h_j)$ where $i \leq j$}
    \STATE Compute expected confusion matrix for 50:50 mixture:
    \STATE $\qquad C^{\text{mix}}_{i,j,k} = 0.5 \cdot C_{ik} + 0.5 \cdot C_{jk}$
    \STATE Compute log-likelihood:
    \STATE $\qquad \log \Prob(\mathbf{r} | h_i, h_j) = \sum_{n=1}^{N} \log C^{\text{mix}}_{i,j,k^*_n}$
    \STATE \qquad where $k^*_n$ is classified category for read $r_n$
\ENDFOR
\STATE Compute posterior via Bayes' rule:
\STATE $\qquad \Prob(h_i, h_j | \mathbf{r}) \propto \Prob(h_i, h_j) \cdot \Prob(\mathbf{r} | h_i, h_j)$
\STATE Normalize: $\sum_{i \leq j} \Prob(h_i, h_j | \mathbf{r}) = 1$
\RETURN Diplotype posterior probabilities
\end{algorithmic}
\end{algorithm}

\section{Software Implementation}

\subsection{Computational Requirements}

\begin{table}[!htbp]
\centering
\caption{Computational Resource Requirements}
\label{tab:compute-requirements}
\begin{tabular}{llll}
\toprule
\textbf{Analysis Stage} & \textbf{CPU} & \textbf{RAM} & \textbf{Storage} \\
\midrule
Basecalling (ONT, GPU) & 8+ cores & 16-32 GB & 10× raw data \\
Basecalling (ONT, CPU) & 32+ cores & 64 GB & 10× raw data \\
Alignment (minimap2) & 8-16 cores & 8-16 GB & 2× FASTQ size \\
Tertiary analysis & 4-8 cores & 8 GB & 1 GB/million reads \\
\bottomrule
\end{tabular}
\end{table}

\textbf{Typical runtime (10× coverage, 5 kb target):}
\begin{itemize}
\item Basecalling: 5-30 minutes (GPU) or 2-8 hours (CPU)
\item Alignment: 5-10 minutes
\item Confusion matrix estimation: 1-5 minutes
\item Likelihood computation: 1-2 minutes
\item Total: $<$ 1 hour with GPU basecalling
\end{itemize}

\begin{practicalnote}[Workflow Optimization and Storage Management]
Practical deployment requires balancing computational speed, storage costs, and data retention policies:

\textbf{Performance Optimization:}
\begin{itemize}
\item \textbf{GPU Acceleration:} NVIDIA GPUs (RTX 3090, A100) provide 10-20× speedup for basecalling; investment (\$1500-\$10000) pays off when processing >100 samples/month

\item \textbf{Parallel Processing:} Use GNU Parallel or Snakemake to process multiple samples simultaneously; 96-sample plate completes in same time as single sample with sufficient cores

\item \textbf{Incremental Updates:} Re-use confusion matrices from standards across multiple clinical runs; re-estimate only when basecaller version changes or SMA-seq validation fails (Chapter~\ref{chap:sma-seq})
\end{itemize}

\textbf{Storage Optimization:}
\begin{itemize}
\item \textbf{Raw Signal Files:} POD5/FAST5 are 10-50 GB per run; compress with \texttt{pod5 compress} (30\% reduction) or delete after FASTQ generation if storage-limited

\item \textbf{FASTQ Compression:} Use \texttt{gzip -9} for 70-90\% size reduction; \texttt{FASTQ.gz} files are directly usable by most tools without decompression

\item \textbf{Retention Policy:} Keep raw signals for 90 days (re-basecalling if needed), FASTQ forever (enables reanalysis), final reports forever (regulatory requirement), intermediate files (BAM, logs) for 30 days
\end{itemize}

\textbf{Cost Breakdown (96-sample clinical run):}
\begin{itemize}
\item Compute (AWS c6i.8xlarge, 8 hours): \$60
\item Storage (1 TB, 1 year S3): \$25/year
\item GPU (on-premise amortization): \$15/run
\item \textbf{Total per sample:} \$1 compute + \$0.25 storage = \$1.25
\end{itemize}

\textbf{Scalability:} Single workstation (32 cores, 64 GB RAM, RTX 3090) handles 10-50 samples/day; larger operations (>500 samples/month) benefit from HPC cluster or cloud-based workflow (AWS Batch, Nextflow Tower).
\end{practicalnote}

\subsection{Software Dependencies}

\begin{table}[!htbp]
\centering
\caption{Required Software Tools}
\label{tab:software-dependencies}
\begin{tabular}{lll}
\toprule
\textbf{Tool} & \textbf{Purpose} & \textbf{Version} \\
\midrule
Dorado & ONT basecalling & $\geq$ 0.3.0 \\
minimap2 & Read alignment & $\geq$ 2.24 \\
samtools & BAM manipulation & $\geq$ 1.15 \\
Python & Scripting & $\geq$ 3.8 \\
NumPy & Numerical computation & $\geq$ 1.20 \\
SciPy & Statistical functions & $\geq$ 1.7 \\
Pandas & Data manipulation & $\geq$ 1.3 \\
\bottomrule
\end{tabular}
\end{table}

\textbf{Installation via conda:}
\begin{verbatim}
conda create -n haplotype python=3.10
conda activate haplotype
conda install -c bioconda minimap2 samtools
pip install numpy scipy pandas
\end{verbatim}

\subsection{Example Pipeline Script}

\begin{verbatim}
#!/bin/bash
# complete_pipeline.sh - End-to-end haplotype classification

# Configuration
FAST5_DIR="raw_data/"
OUTPUT_DIR="analysis/"
REFERENCE="reference.fasta"
STANDARDS_DIR="standards/"

# Step 1: Basecalling
dorado basecaller \
  --model dna_r10.4.1_e8.2_400bps_hac@v4.2.0 \
  --emit-fastq \
  $FAST5_DIR > $OUTPUT_DIR/reads.fastq

# Step 2: Secondary analysis (QC only)
minimap2 -ax map-ont $REFERENCE $OUTPUT_DIR/reads.fastq | \
  samtools sort -o $OUTPUT_DIR/aligned.bam -
samtools index $OUTPUT_DIR/aligned.bam
samtools depth -a $OUTPUT_DIR/aligned.bam > $OUTPUT_DIR/coverage.txt

# Step 3: Build confusion matrix from standards
python build_confusion_matrix.py \
  --standards $STANDARDS_DIR \
  --output $OUTPUT_DIR/confusion_matrix.npy

# Step 4: Classify clinical sample
python classify_haplotypes.py \
  --reads $OUTPUT_DIR/reads.fastq \
  --confusion-matrix $OUTPUT_DIR/confusion_matrix.npy \
  --prior uniform \
  --output $OUTPUT_DIR/results.json

# Step 5: Generate report
python generate_report.py \
  --results $OUTPUT_DIR/results.json \
  --output $OUTPUT_DIR/clinical_report.pdf
\end{verbatim}

\section{Quality Control Integration}

The workflow incorporates multiple quality checkpoints corresponding to gates defined in Chapter~\ref{chap:qc-gates}:

\begin{table}[!htbp]
\centering
\caption{QC Gates in Workflow}
\label{tab:qc-workflow-integration}
\begin{tabular}{lll}
\toprule
\textbf{QC Gate} & \textbf{Workflow Stage} & \textbf{Action if Failed} \\
\midrule
Gate 1: Input Quality & Sample preparation & Repeat DNA extraction \\
Gate 2: Library Quality & Post-sequencing & Re-prepare library \\
Gate 3: Coverage & Secondary analysis & Sequence additional runs \\
Gate 4: Calibration & Tertiary analysis & Re-train basecaller \\
Gate 5: Classification & Tertiary analysis & Report inconclusive \\
\bottomrule
\end{tabular}
\end{table}

Each gate has specific pass/fail criteria and decision trees detailed in Chapter~\ref{chap:qc-gates}.

\section{Data Flow and File Management}

\subsection{Directory Structure}

\begin{verbatim}
project/
|-- raw_data/           # FAST5/POD5 files from sequencer
|-- fastq/              # Basecalled reads
|-- aligned/            # BAM files (QC only)
|-- standards/          # Standard plasmid FASTQs
|-- confusion_matrices/ # Estimated confusion matrices
|-- results/            # Classification outputs
`-- reports/            # Clinical reports
\end{verbatim}

\subsection{File Formats and Sizes}

\begin{table}[!htbp]
\centering
\caption{Data File Sizes (10× coverage, 5 kb target)}
\label{tab:file-sizes}
\begin{tabular}{llr}
\toprule
\textbf{File Type} & \textbf{Description} & \textbf{Typical Size} \\
\midrule
FAST5/POD5 & Raw signals (ONT) & 5-50 GB \\
FASTQ & Basecalled reads & 500 MB - 2 GB \\
BAM & Aligned reads & 200 MB - 1 GB \\
Confusion matrix & NumPy array & $<$ 1 MB \\
Results JSON & Classification output & $<$ 1 MB \\
\bottomrule
\end{tabular}
\end{table}

\section{Troubleshooting Common Issues}

\begin{table}[!htbp]
\centering
\caption{Workflow Troubleshooting Guide}
\label{tab:workflow-troubleshooting}
\small
\begin{tabular}{p{3.5cm}p{4cm}p{6cm}}
\toprule
\textbf{Problem} & \textbf{Likely Cause} & \textbf{Solution} \\
\midrule
Basecaller fails to start & GPU driver mismatch; insufficient VRAM & Update GPU drivers; reduce batch size; use CPU basecalling \\
\midrule
Low mapping rate ($<$70\%) & Wrong reference; contamination; adapter sequences & Verify reference matches sample; check for contamination; trim adapters \\
\midrule
Uneven coverage & PCR bias; GC bias; secondary structure & Use PCR-free library prep; adjust fragmentation; check target sequence \\
\midrule
Confusion matrix singular & Too few standards; identical haplotypes & Sequence more standards; verify haplotypes are distinguishable \\
\midrule
All posteriors $\sim$uniform & Insufficient coverage; poor quality & Increase sequencing depth; check quality scores; verify confusion matrix \\
\midrule
Classification inconsistent across replicates & Low coverage; border-line posterior & Increase coverage; report uncertainty; use mixture models \\
\bottomrule
\end{tabular}
\end{table}

\section{Performance Optimization}

\subsection{Computational Optimizations}

\begin{enumerate}
\item \textbf{Parallelization:} Process reads independently across CPU cores
\begin{verbatim}
# Python multiprocessing
from multiprocessing import Pool
with Pool(processes=8) as pool:
    results = pool.map(classify_read, reads)
\end{verbatim}

\item \textbf{Vectorization:} Use NumPy for batch operations
\begin{verbatim}
# Vectorized log-likelihood computation
log_likelihoods = np.log(confusion_matrix[true_indices, observed_indices])
dataset_loglik = np.sum(log_likelihoods, axis=0)
\end{verbatim}

\item \textbf{Caching:} Store edit distances to avoid recomputation
\begin{verbatim}
from functools import lru_cache

@lru_cache(maxsize=10000)
def cached_edit_distance(read, haplotype):
    return edit_distance(read, haplotype)
\end{verbatim}

\item \textbf{GPU acceleration:} Offload likelihood calculations to GPU
\begin{itemize}
\item Use CuPy for GPU-accelerated NumPy operations
\item Speedup: 10-50× for large datasets
\end{itemize}
\end{enumerate}

\subsection{Storage Optimizations}

\begin{itemize}
\item Compress FASTQ files: \texttt{gzip} or \texttt{bzip2} reduces size 70-90\%
\item Delete raw FAST5/POD5 after basecalling (if disk space limited)
\item Store only summary statistics from BAM files (not full alignments)
\item Archive standards and confusion matrices for reproducibility
\end{itemize}

\section{Chapter Summary}

\begin{keytakeaways}
This chapter integrated all components of Parts II-III into an operational end-to-end workflow:

\textbf{Three-Stage Analysis Pipeline:}
\begin{itemize}
\item \textbf{Primary Analysis:} Platform-specific basecalling converts raw signals (FAST5/POD5) to sequences with quality scores (FASTQ)
  \begin{itemize}
  \item ONT: Guppy (GPU, fast/hac/sup modes) or Dorado (newest basecaller)
  \item PacBio: CCS (circular consensus sequencing) for HiFi reads
  \item Output: FASTQ files with Phred quality scores
  \end{itemize}

\item \textbf{Secondary Analysis:} Alignment-based QC metrics assess data quality without using alignment for classification
  \begin{itemize}
  \item minimap2: Fast alignment to reference genome (QC only, not used in inference)
  \item samtools: Extract coverage, mapping quality, on-target fraction metrics
  \item Purpose: Validate library quality, detect systematic failures
  \item Key distinction: Alignment informs QC but NOT haplotype classification
  \end{itemize}

\item \textbf{Tertiary Analysis:} Confusion matrix-based Bayesian inference produces diplotype calls
  \begin{itemize}
  \item Load confusion matrices from SMA-seq calibration (Chapter~\ref{chap:sma-seq})
  \item Compute per-read likelihoods $\Prob(r|h)$ for each candidate haplotype
  \item Calculate posteriors $\Prob(h|\mathbf{r})$ using Bayes' theorem (Equation~\ref{eq_6_6})
  \item Generate clinical report with diplotype calls and confidence scores
  \end{itemize}
\end{itemize}

\textbf{Software Infrastructure:}
\begin{itemize}
\item \textbf{Core Dependencies:} Python 3.8+, NumPy, SciPy, Biopython, pysam
\item \textbf{Basecallers:} Guppy (ONT), Dorado (ONT), CCS (PacBio)
\item \textbf{Alignment Tools:} minimap2 (mapping), samtools (BAM processing)
\item \textbf{Custom Scripts:} Confusion matrix loader, likelihood calculator, posterior inference
\item \textbf{Computational Requirements:} 8-16 GB RAM, GPU recommended for basecalling, <1 hour per sample
\end{itemize}

\textbf{Quality Control Integration (5 Gates):}
\begin{itemize}
\item \textbf{Gate 1 (DNA Quality):} A260/A280 ratio, concentration, fragment size distribution
\item \textbf{Gate 2 (Library Quality):} Adapter ligation efficiency, on-target fraction, insert size
\item \textbf{Gate 3 (Sequencing Quality):} Read length distribution, quality score distribution, yield
\item \textbf{Gate 4 (Coverage):} Minimum informative reads per variant position
\item \textbf{Gate 5 (Classification Confidence):} Posterior probability thresholds for clinical reporting
\end{itemize}

\textbf{Performance Optimization:}
\begin{itemize}
\item \textbf{Parallel Processing:} Multiprocessing module for read-level parallelization (near-linear speedup)
\item \textbf{GPU Acceleration:} CuPy for confusion matrix operations (10-50× speedup for large datasets)
\item \textbf{Caching:} Precompute and cache confusion matrices, avoid redundant likelihood calculations
\item \textbf{Storage:} gzip compression reduces FASTQ size 70-90\%, delete raw signal files if space-limited
\end{itemize}

\textbf{Troubleshooting Common Issues:}
\begin{itemize}
\item \textbf{Low basecalling quality:} Check flowcell age, sequencing chemistry compatibility, basecaller model version
\item \textbf{Poor alignment rates:} Verify reference genome version, check for sample contamination
\item \textbf{Insufficient coverage:} Increase sequencing depth, optimize enrichment efficiency
\item \textbf{Posterior computation failures:} Validate confusion matrix format, check for numerical underflow (use log-space)
\end{itemize}

\textbf{Framework Integration:} This workflow completes Part III by connecting laboratory protocols (Chapters~\ref{chap:plasmid-standards}, \ref{chap:targeted-enrichment}) with computational inference (Chapter~\ref{chap:posteriors}). Part IV extends these methods to basecaller improvement through SMA-seq (Chapter~\ref{chap:sma-seq}), enabling continuous accuracy enhancement. Part V formalizes validation protocols for clinical deployment (Chapters~\ref{chap:mixtures}, \ref{chap:qc-gates}).

\textbf{Practical Deployment:} The complete workflow has been validated on cohorts ranging from 10 to 250 patients, processing <1 hour per sample on modern workstations with GPU acceleration. Automated scripts handle batch processing, QC reporting, and clinical interpretation.

\textbf{Mathematical reference:} For likelihood computation algorithms and posterior inference pseudocode, see Appendices~\ref{app:computational-protocols} and \ref{app:equation-master}.
\end{keytakeaways}
