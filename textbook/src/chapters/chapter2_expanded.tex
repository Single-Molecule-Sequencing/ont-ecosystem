%%%%%%%%%%%%%%%%%%%%%%%%%%%%%%%%%%%%%%%%%%%%%%%%%%%%%%%%%%%%%%%%%%%%%%%%
%% Chapter 2: Genomic Complexity of Pharmacogenes
%% Part I: Clinical Motivation and Technical Background
%% Status: Expanded Outline
%%%%%%%%%%%%%%%%%%%%%%%%%%%%%%%%%%%%%%%%%%%%%%%%%%%%%%%%%%%%%%%%%%%%%%%%

\chapter{Genomic Complexity of Pharmacogenes}
\label{chap:genomic-complexity}
\label{chap:haplotype-graph}

\section{Introduction: The Challenge of Structural Complexity}

Pharmacogenes exhibit extraordinary structural diversity that fundamentally challenges conventional genotyping technologies. Unlike most genomic loci, many clinically critical pharmacogenes reside in regions harboring deletions, duplications, hybrid genes, pseudogene paralogs, and extensive copy number variation. These structural features directly determine drug metabolism phenotypes—ultrarapid, extensive, intermediate, and poor metabolizer status—yet remain largely invisible to short-read sequencing and array-based assays.

This chapter characterizes the genomic complexity that necessitates the single-molecule sequencing approach and probabilistic framework developed in subsequent parts. We focus on structural variation classes that confound phasing and haplotype determination, document the specific failure modes of conventional technologies, and establish the technical performance requirements that the framework must satisfy to enable clinical deployment.

\noindent\textbf{Chapter Objectives:}
\begin{itemize}
\item Characterize structural variation in clinically important pharmacogenes
\item Document short-read sequencing limitations with specific examples
\item Establish the necessity for haplotype-resolved, long-read approaches
\item Define technical requirements for clinical pharmacogenomic testing
\item Connect genomic complexity to framework design principles
\end{itemize}

\section{Structural Variation Landscape in Pharmacogenes}
\label{sec:ch2-structural-landscape}

\subsection{CYP2D6: The Flagship Complex Locus}

The cytochrome P450 2D6 gene (\textit{CYP2D6}, chromosome 22q13.2) metabolizes approximately 25\% of clinically used drugs including codeine, tramadol, venlafaxine, risperidone, tamoxifen, and most tricyclic antidepressants~\cite{Gaedigk2017}. \textit{CYP2D6} is the most structurally complex pharmacogene, with over 150 documented star alleles exhibiting diverse structural features~\cite{Gaedigk2018PharmVar,DelTredici2018}:

\textbf{Gene Deletions:} The \textit{*5} allele represents a complete gene deletion resulting in no functional enzyme (poor metabolizer phenotype). This deletion spans approximately 30~kb and cannot be reliably detected by short-read sequencing or most array platforms due to alignment ambiguity with flanking pseudogenes~\cite{Gaedigk2018CopyNumber}.

\textbf{Gene Duplications and Multiplications:} Tandem duplications of functional alleles (e.g., \textit{*1$\times$N}, \textit{*2$\times$N}) confer ultrarapid metabolizer phenotype. Patients with three or more active gene copies metabolize substrates so rapidly that standard dosing produces subtherapeutic exposure. Conversely, duplications of nonfunctional alleles (e.g., \textit{*4$\times$2}) do not alter phenotype. Distinguishing duplication of functional versus nonfunctional alleles requires haplotype resolution—precisely the capability that single-molecule sequencing provides and that short-read methods fundamentally lack~\cite{Gaedigk2018CopyNumber,Gordon2022}.

\textbf{Hybrid Alleles:} \textit{CYP2D6} shares 96\% sequence identity with downstream pseudogene \textit{CYP2D7}. Recombination events generate hybrid alleles such as~\cite{Gaedigk2018PharmVar}:
\begin{itemize}
    \item \textit{*13}: CYP2D6-2D7 hybrid with altered substrate specificity
    \item \textit{*36}: CYP2D6-2D7 hybrid converting gene to pseudogene
    \item \textit{*68}: CYP2D7-2D6-2D7 hybrid with complex rearrangement
\end{itemize}

Hybrid alleles pose dual challenges: (1) breakpoint positions vary among individuals, requiring base-level resolution of gene-pseudogene junctions; (2) distinguishing parental gene from pseudogene sequences demands long reads spanning entire exons or introns to anchor alignment unambiguously~\cite{Gordon2022}.

\textbf{Gene Conversions:} Localized sequence transfer between \textit{CYP2D6} and \textit{CYP2D7} creates mosaic haplotypes. The \textit{*4.013} suballele carries exon 9 conversion from \textit{CYP2D7}, while retaining \textit{CYP2D6} sequence elsewhere. Short-read assays misalign converted segments to the pseudogene, yielding false-negative variant calls and incorrect diplotype assignment.

\subsection{Additional Structurally Complex Pharmacogenes}

\textbf{CYP2A6 (Coumarin 7-Hydroxylase):} Metabolizes nicotine, tegafur, letrozole, and anesthetic agents. The \textit{*1$\times$2} duplication allele is common in African populations (15--20\% frequency), while deletion alleles \textit{*4} (entire gene deletion) and \textit{*12} (hybrid with \textit{CYP2A7}) occur at 1--3\% globally. Individuals with \textit{*4/*4} genotype exhibit dramatically reduced nicotine clearance with implications for smoking cessation pharmacotherapy. \textit{CYP2A6} shares extensive homology with \textit{CYP2A7} and \textit{CYP2A13}, creating alignment challenges parallel to \textit{CYP2D6/CYP2D7}.

\textbf{CYP2B6 (Efavirenz Metabolizer):} Critical for antiretroviral therapy metabolism. The \textit{*29} allele is a CYP2B6-2B7 hybrid resulting in absent protein expression. Copy number variants and hybrid alleles occur at appreciable frequencies in African ancestry populations, directly impacting efavirenz dosing and neuropsychiatric adverse event risk~\cite{Gaedigk2017}.

\textbf{UGT1A1 (Bilirubin Glucuronidation):} Harbors both copy number variation (duplications of the entire locus) and regulatory repeat expansions (TA repeat in the \textit{*28} promoter variant causing Gilbert syndrome). The UGT1A locus contains nine functional genes in tandem, sharing exon 1 sequences but differing in substrate specificity. Haplotype-resolved sequencing is essential to distinguish individual UGT1A gene variants and their combination on each chromosome.

\textbf{GSTT1 and GSTM1 (Glutathione S-Transferases):} Homozygous deletions occur at 10--50\% frequency depending on ancestry~\cite{Bolt2006}. Null genotypes confer altered risk for chemotherapy toxicity, environmental carcinogen susceptibility, and asthma. Array-based assays rely on presence/absence calls but cannot distinguish heterozygous deletion from homozygous presence without independent copy number assessment.

\subsection{Structural Variant Frequency and Clinical Impact}

Table~\ref{tab:ch2-sv-frequency} summarizes structural variant frequencies across populations. Structural variants account for a substantial proportion of pharmacogenetic diversity: in \textit{CYP2D6}, deletions, duplications, and hybrids collectively represent 15--30\% of alleles depending on ancestry. For African populations, this proportion exceeds 40\% when including all duplication and hybrid alleles. Ignoring or misclassifying structural variation directly translates to phenotype prediction errors with clinical consequences detailed in Section~\ref{sec:ch2-clinical-impact}.

\begin{table}[h]
\centering
\caption{Frequency of Structural Variants in Key Pharmacogenes by Ancestry}
\label{tab:ch2-sv-frequency}
\small
\begin{tabular}{lcccc}
\toprule
\textbf{Gene} & \textbf{SV Type} & \textbf{European} & \textbf{African} & \textbf{East Asian} \\
\midrule
\textit{CYP2D6} & Deletion (*5) & 2--5\% & 4--6\% & 6--8\% \\
                & Duplication & 1--3\% & 15--30\% & 0.5--1\% \\
                & Hybrid (*13, *36, *68) & 1--2\% & 3--5\% & 0.5--1\% \\
\midrule
\textit{CYP2A6} & Deletion (*4) & 1--2\% & 1--3\% & 0.5--1\% \\
                & Duplication (*1$\times$2) & 2--3\% & 15--20\% & 1--2\% \\
                & Hybrid (*12) & 1--2\% & 1--3\% & 0.5--1\% \\
\midrule
\textit{GSTT1}  & Homozygous deletion & 10--20\% & 20--40\% & 50--60\% \\
\textit{GSTM1}  & Homozygous deletion & 40--50\% & 20--30\% & 40--60\% \\
\midrule
\textit{UGT1A1} & Copy number variant & 1--3\% & 2--5\% & 1--3\% \\
\bottomrule
\end{tabular}
\end{table}

\subsection{Population-Level Haplotype Diversity}

Allele frequencies differ markedly across global populations, necessitating ancestry-aware priors in any probabilistic diplotype caller. PharmVar and gnomAD analyses show that more than 40\% of \textit{CYP2D6} alleles in African ancestry cohorts carry structural variants (deletions, duplications, hybrids), compared with less than 15\% in European cohorts~\cite{Gaedigk2017,DelTredici2018}. East Asian cohorts exhibit elevated frequencies of \textit{CYP2A6} deletions and reduced-function \textit{CYP2C19*2} alleles, reshaping predicted metabolizer distributions for nicotine cessation therapy and antiplatelet regimens~\cite{Chen2014}. Accurate modeling therefore requires:
\begin{itemize}
    \item Ancestry-specific haplotype frequency tables updated with long-read sequencing evidence.
    \item Dynamic prior re-weighting when laboratory intake demographics shift or when applied to admixed populations.
    \item Explicit uncertainty quantification when posterior probability mass is distributed across multiple plausible diplotypes.
\end{itemize}

Chapter~\ref{chap:population-priors} extends these insights into hierarchical Bayesian priors that condition diplotype probabilities on both global allele frequencies and patient-level covariates (ancestry estimates, phenotype measurements).

\section{Limitations of Short-Read Sequencing}
\label{sec:ch2-short-read-limits}

\subsection{Alignment Ambiguity in Repetitive Regions}

Short-read sequencing (Illumina, BGI) generates fragments typically 150--300~bp in length. When a read spans sequence shared between a gene and its pseudogene or paralog, the aligner faces ambiguity: does this read originate from \textit{CYP2D6} or \textit{CYP2D7}? Standard aligners (BWA, Bowtie2) either discard multiply-mapping reads (sacrificing sensitivity) or distribute them probabilistically (introducing systematic bias). For \textit{CYP2D6}, regions with $>$98\% identity to \textit{CYP2D7} span multiple exons; short reads from these regions are inherently unresolvable.

\textbf{Consequence:} Variant calls in shared regions are unreliable. Hybrid allele detection fails because the breakpoint—the boundary where gene sequence transitions to pseudogene sequence—cannot be localized with confidence. Copy number estimation algorithms falter because read depth cannot distinguish gene from pseudogene contributions.

\subsection{Phasing Limitations and Haplotype Ambiguity}

Even when individual variants are called correctly, short reads provide no information about phase—which variants reside together on the same chromosome. For pharmacogenes, phase determines function:
\begin{itemize}
    \item \textit{CYP2D6 *4/*10} (two reduced-function alleles in trans) yields intermediate metabolizer phenotype
    \item A hypothetical individual with duplication of \textit{*4} plus a single \textit{*10} allele (*4$\times$2/*10) also appears as heterozygous for \textit{*4} and \textit{*10} variants but carries \textbf{two} copies of nonfunctional \textit{*4}, conferring near-poor metabolizer status
\end{itemize}

Short-read phasing relies on read-pair linkage (variants within $<$500~bp inferred to be in cis) or population-based statistical phasing (e.g., SHAPEIT, Beagle). Neither approach resolves phase across entire genes spanning 10--50~kb, especially for structural variants that disrupt local linkage disequilibrium assumptions. Long-read sequencing provides \textbf{direct observation} of haplotypes, eliminating statistical inference uncertainty.

\subsection{Benchmarking: Short-Read vs.\ Long-Read Performance}

Table~\ref{tab:ch2-tech-comparison} summarizes accuracy benchmarks from published validation studies using characterized reference materials (GeT-RM, synthetic constructs). For simple, single-copy genes (\textit{CYP2C19}, \textit{CYP2C9}), short-read assays achieve $>$99\% diplotype concordance. Performance degrades sharply for structurally complex loci:

\begin{table}[h]
\centering
\caption{Diplotype Concordance: Short-Read vs.\ Long-Read Sequencing}
\label{tab:ch2-tech-comparison}
\begin{tabular}{lccc}
\toprule
\textbf{Gene (Complexity)} & \textbf{Short-Read NGS} & \textbf{Long-Read (PacBio)} & \textbf{Long-Read (ONT)} \\
\midrule
\textit{CYP2C19} (simple) & 99.2\% & 99.8\% & 99.5\% \\
\textit{CYP2C9} (simple) & 99.0\% & 99.7\% & 99.3\% \\
\midrule
\textit{CYP2D6} (complex) & 85--92\% & 96--98\% & 94--97\% \\
\textit{CYP2A6} (moderate) & 90--95\% & 97--99\% & 96--98\% \\
\textit{CYP2B6} (moderate) & 92--96\% & 98--99\% & 97--99\% \\
\bottomrule
\end{tabular}
\end{table}

\textbf{Key Observation:} For \textit{CYP2D6}, short-read error rates (8--15\%) are clinically unacceptable, particularly given the gene's high actionability. Long-read platforms achieve 96--98\% accuracy, meeting clinical deployment thresholds when coupled with rigorous quality control (Chapter~\ref{chap:qc-gates}). The framework developed in Part~II (Chapters~\ref{chap:basecaller}--\ref{chap:diplotypes}) formalizes this long-read advantage into a mathematically principled classification system with quantified uncertainty.

\section{Clinical Consequences of Misclassification}
\label{sec:ch2-clinical-impact}

\subsection{Case Example: Tamoxifen Therapy Failure}

A 52-year-old woman with estrogen receptor-positive breast cancer receives adjuvant tamoxifen~\cite{Gordon2022}. She undergoes array-based pharmacogenomic testing reported as \textit{CYP2D6 *1/*4} (extensive/reduced-function, intermediate metabolizer phenotype). Dosing proceeds with standard 20~mg daily tamoxifen.

\textbf{Genetic Reality:} Long-read resequencing reveals she carries \textit{*1/*5}—one functional allele and one complete deletion—also conferring intermediate metabolizer status. Her array result was correct by chance; the array platform interrogates specific SNPs but does not detect the \textit{*5} deletion.

\textbf{Clinical Risk:} Had she carried \textit{*4/*5} (poor metabolizer), the array would report \textit{*4/wt} or \textit{*1/*4} (failing to detect \textit{*5}), yielding falsely reassuring intermediate/extensive classification. Tamoxifen requires CYP2D6-mediated conversion to active endoxifen; poor metabolizers exhibit 50--70\% reduced endoxifen exposure, correlating with increased breast cancer recurrence risk. Array-based misclassification directly compromises treatment efficacy.

\subsection{Case Example: Codeine Overdose in Infant}

Discussed in detail in Chapter~\ref{chap:pharmacogenomics}, this case underscores the consequences of failing to detect \textit{CYP2D6} gene duplication. An ultrarapid metabolizer mother (genotype \textit{*2$\times$2/*1}) produces toxic morphine levels from standard codeine doses, resulting in lethal opioid exposure to her breastfed infant~\cite{Koren2006}. Short-read assays frequently misclassify duplications as simple heterozygosity, particularly when duplication involves alleles with multiple SNPs that confound read mapping~\cite{Gaedigk2018CopyNumber}.

\subsection{Thiopurine Toxicity in TPMT-Deficient Patients}

Thiopurine drugs (azathioprine, 6-mercaptopurine) used for inflammatory bowel disease and acute lymphoblastic leukemia require dose reduction by 90\% in patients with two nonfunctional \textit{TPMT} alleles to prevent life-threatening myelosuppression~\cite{Amstutz2018}. The \textit{TPMT*1/*3A} heterozygote requires 50\% dose reduction. However, distinguishing \textit{*3A} (two variants in cis: rs1800460 and rs1142345) from compound heterozygosity \textit{*3B/*3C} (same two variants in trans) is critical—both affect activity but \textit{*3A/*3A} homozygosity is more severe. Short-read assays observe both variants but cannot phase them, forcing reliance on population frequency assumptions that fail in admixed populations.

\subsection{Economic and Regulatory Implications}

Misclassification errors generate three cost categories:
\begin{enumerate}
    \item \textbf{Direct harm}: ADRs requiring hospitalization (\$30--130 billion annually in the US, Chapter~\ref{chap:pharmacogenomics})
    \item \textbf{Therapeutic failure}: Suboptimal treatment efficacy from incorrect dosing
    \item \textbf{Repeat testing}: When clinical phenotype contradicts genotype prediction, laboratories order confirmatory testing (TaqMan, Sanger sequencing of specific regions), adding \$200--500 per case
\end{enumerate}

From a regulatory perspective, CLIA and CAP require that clinical assays achieve defined accuracy thresholds, typically $\geq$95--99\% concordance with reference methods. Short-read assays for complex pharmacogenes \textbf{do not meet this standard} without supplemental orthogonal methods for structural variant detection, undermining the economic value proposition of panel-based testing.

\section{Star Allele Nomenclature and Reference Standards}
\label{sec:ch2-nomenclature}

\subsection{PharmVar and Standardized Nomenclature}

The Pharmacogene Variation Consortium (PharmVar, \url{https://www.pharmvar.org}) maintains the authoritative catalog of star allele definitions for CYP and other pharmacogenes. Each star allele (e.g., \textit{CYP2D6*4}) represents a specific haplotype defined by a set of variants (SNPs, indels, structural changes) relative to the reference allele \textit{*1}. As of 2024, PharmVar documents:
\begin{itemize}
    \item 158 \textit{CYP2D6} star alleles with suballele variants (e.g., \textit{*4.001}, \textit{*4.002})
    \item 35 \textit{CYP2C19} alleles
    \item 60 \textit{CYP2C9} alleles
    \item Structural variant annotations including exact breakpoint sequences for hybrid alleles
\end{itemize}

\textbf{Naming Conventions:}
\begin{itemize}
    \item \textit{*1}: Reference (wild-type) allele with normal function
    \item \textit{*2}, \textit{*3}, etc.: Variant alleles numbered chronologically by discovery
    \item Suballeles (e.g., \textit{*2.001}, \textit{*2.002}): Differ by silent or intronic variants not affecting function
    \item \textit{*N$\times$M}: Duplication/multiplication (e.g., \textit{*1$\times$2} = two copies of \textit{*1})
\end{itemize}

\subsection{Integration with Framework Nomenclature}

The framework developed in Part~II employs haplotype identifiers $h \in \mathcal{H}$ (Chapter~\ref{chap:basecaller}, \CEref{1}) that map directly to PharmVar star alleles. Appendix~\ref{app:notation} provides a comprehensive lookup table translating between:
\begin{itemize}
    \item Framework haplotype index (e.g., $h_{23}$)
    \item PharmVar star allele designation (e.g., \textit{*4.013})
    \item Functional impact category (normal, decreased, poor, increased function)
    \item Defining variants (rsIDs and structural features)
\end{itemize}

This bidirectional mapping ensures that posterior probabilities $\Prob(h|r)$ (\CEref{1}) generated by the basecaller are immediately interpretable in standardized clinical terminology. The diplotype caller (Chapter~\ref{chap:diplotypes}, \CEref{11}) outputs diplotype probabilities $\Prob(D|R)$ labeled with PharmVar allele pairs, enabling direct integration with clinical decision support systems (Chapter~\ref{chap:cyp2d6}).

\subsection{Graph-Based Representation of Pharmacogene Variation}

Traditional star-allele catalogs enumerate haplotypes as flat variant lists. For structurally complex genes, however, graph representations capture exon-level rearrangements, gene conversions, and copy-number polymorphisms more naturally~\cite{Klein2019,Mantere2019}. Our framework encodes each pharmacogene as a directed acyclic graph where:
\begin{itemize}
    \item Nodes correspond to contiguous sequence segments (exons, intronic motifs, promoter regions) annotated with variant sets.
    \item Edges capture allowable transitions, including recombination events that form hybrid alleles (e.g., \textit{CYP2D6*36}).
    \item Path multiplicity models copy-number changes; duplicated segments manifest as repeated traversals through the same subgraph.
\end{itemize}

This structure enables efficient dynamic programming for likelihood computation (Chapter~\ref{chap:diplotypes}) while preserving compatibility with PharmVar nomenclature. When PharmVar releases new alleles, we insert additional paths without refactoring the entire model, supporting rapid iterative updates.

\section{Framework Requirements Derived from Genomic Complexity}
\label{sec:ch2-framework-reqs}

The structural and phasing challenges documented in this chapter impose specific technical requirements that drive framework design:

\subsection{Requirement 1: Single-Molecule Haplotype Resolution}

Duplications, deletions, and hybrid alleles demand that sequencing reads span entire gene copies or definitively anchor to gene versus pseudogene loci. The framework requires:
\begin{itemize}
    \item \textbf{Read length}: $\geq$5--10 kb to span full \textit{CYP2D6} coding sequence plus flanking regions
    \item \textbf{Haplotype phasing}: Direct observation of variant combinations on individual molecules, not statistical inference
\end{itemize}

Chapter~\ref{chap:single-molecule} details Oxford Nanopore and PacBio platforms meeting these criteria.

\subsection{Requirement 2: Probabilistic Classification with Uncertainty Quantification}

Single-molecule sequencing trades reduced per-base accuracy (90--99\% depending on chemistry and base-calling) for haplotype information. The framework must:
\begin{itemize}
    \item Accept noisy reads $r$ with error rates 1--10\% (\CEref{2}, emission probability $\Prob(r|h)$)
    \item Compute posterior probabilities $\Prob(h|r)$ integrating prior knowledge $\Prob(h)$ (population frequencies) via Bayes' rule (\CEref{1})
    \item Propagate uncertainty through diplotype inference (\CEref{11}) to clinical phenotype prediction
\end{itemize}

This probabilistic framework (Part~II) transforms noisy long-read data into high-confidence diplotype calls by leveraging:
\begin{enumerate}
    \item Known haplotype structures (PharmVar database)
    \item Population allele frequencies (gnomAD, 1000 Genomes)
    \item Read depth and quality metrics (Chapter~\ref{chap:qc-gates})
\end{enumerate}

\subsection{Requirement 3: Structural Variant Detection and Classification}

The framework must explicitly model deletions, duplications, and hybrid structures:
\begin{itemize}
    \item \textbf{Deletion detection} (\textit{*5}): Absence of reads spanning the locus, confirmed by reads spanning the deletion breakpoint. Requires breakpoint-aware alignment (Chapter~\ref{chap:basecaller}).
    \item \textbf{Duplication detection} (\textit{*1$\times$2}): Read depth analysis combined with haplotype phasing to determine which allele is duplicated. The SMA-seq approach (Chapter~\ref{chap:sma-seq}, \CEref{9}) provides orthogonal copy number information.
    \item \textbf{Hybrid allele detection} (\textit{*13}, \textit{*36}): Identification of gene-pseudogene junction sequences within individual reads. Requires custom alignment strategies tolerating large indels and reference-switching.
\end{itemize}

\subsection{CYP2D6 Complex Structural Configurations}

The \textit{CYP2D6} locus exemplifies why long-read sequencing is indispensable. The functional gene resides on chromosome~22 flanked by the pseudogenes \textit{CYP2D7} and \textit{CYP2D8}. Gene conversion events can copy pseudogene sequence into the functional gene, yielding alleles such as \textit{*36} (gene-pseudogene hybrid) or \textit{*68} (duplication with hybrid component). Traditional amplicon or short-read assays often collapse these structures into simplified diploid models, obscuring whether the observed variant combination corresponds to a nonfunctional, reduced-function, or normal-function haplotype.

Long-read data disambiguate these structures by preserving contiguous haplotypes that extend across the gene, pseudogene, and intergenic spacer. Reconstructions from orthogonal methods (e.g., multiplex ligation-dependent probe amplification, MLPA) confirm three clinically critical scenarios that the framework must resolve:
\begin{enumerate}
    \item \textbf{Full gene deletion}: Complete loss of \textit{CYP2D6} sequence with intact flanking pseudogenes (\textit{*5}).
    \item \textbf{Duplication with tandem arrangement}: Two functional copies arranged head-to-tail, frequently sharing identical star-allele sequence (\textit{*1$\times$2}) but occasionally representing two different haplotypes (e.g., \textit{*1/*2}).
    \item \textbf{Hybrid tandems}: Mosaic haplotypes where one copy contains a pseudogene-derived segment, altering catalytic residues (\textit{*36+*10}, \textit{*13}).
\end{enumerate}

Each configuration produces distinct metabolizer phenotypes, yet generates overlapping copy-number signatures unless phase-aware analysis is applied. The framework therefore integrates structural variant breakpoints directly into the haplotype graph (Chapter~\ref{chap:haplotype-graph}) and uses molecule-spanning reads to anchor junctions unambiguously. Posterior inference treats structural configurations as categorical states, preventing hybrid alleles from being forced into nearest conventional star-allele definitions.

\subsection{Star-Allele Nomenclature and Functional Mapping}

Haplotype nomenclature, while standardized by PharmVar, remains incomplete for complex loci. Clinical laboratories often report composite designations (e.g., \textit{*68+*4}) that implicitly assume a particular arrangement of structural modules. The framework generalizes star-allele labeling by:
\begin{itemize}
    \item Representing each allele as an ordered list of functional modules (promoter, exons 1--9, downstream regulatory region) with explicit provenance (gene vs. pseudogene).
    \item Mapping modules to functional annotations (normal, decreased, no function) derived from curated biochemical and clinical evidence.
    \item Allowing hybrid or novel combinations to inherit functional priors from their constituent modules while retaining explicit uncertainty until validated.
\end{itemize}

This structured representation supports dynamic updates as new alleles are described, avoiding the brittleness of static star-allele catalogs. It also aligns with the long-read data model: posterior probabilities attach directly to module configurations observed in the sequencing reads, enabling transparent traceability from raw data to clinical phenotype assignments.

\subsection{Requirement 4: Validation Against Characterized Reference Materials}

Clinical deployment requires empirical demonstration of accuracy. The framework incorporates validation protocols (Part~V) using:
\begin{itemize}
    \item CDC Genetic Testing Reference Materials (GeT-RM) for \textit{CYP2D6}, \textit{CYP2C19}, \textit{CYP2C9}
    \item Coriell Institute DNA panels with known diplotypes
    \item Synthetic plasmid constructs for rare and hybrid alleles (Chapter~\ref{chap:plasmid-standards})
    \item Mixture experiments with defined diplotype combinations (Chapter~\ref{chap:haplotype-mixtures})
\end{itemize}

These validation datasets enable accuracy measurement across the diplotype space, including rare structural variants underrepresented in clinical samples. Quality gates (Chapter~\ref{chap:qc-gates}) operationalize validation results into per-sample accept/reject criteria, ensuring that only samples meeting accuracy thresholds receive clinical reports.

\section{Conclusion: Bridging Complexity and Clinical Solutions}

The genomic complexity of pharmacogenes—structural variation, pseudogene paralogs, copy number changes—is not an abstract bioinformatics challenge but a direct determinant of patient safety. Short-read sequencing technologies, while transformative for many applications, fundamentally cannot resolve this complexity at the accuracy required for clinical pharmacogenomics.

This chapter establishes the necessity for single-molecule, haplotype-resolved approaches and defines the technical performance envelope that guides framework development:
\begin{itemize}
    \item \textbf{Sensitivity}: Detect $>$95\% of structural variants including deletions, duplications, and hybrids
    \item \textbf{Specificity}: Minimize false-positive structural variant calls ($<$2\% error rate)
    \item \textbf{Diplotype accuracy}: Achieve $\geq$95\% concordance with reference methods for complex genes
    \item \textbf{Uncertainty quantification}: Report confidence metrics enabling quality-gated clinical decision-making
\end{itemize}

Part~II develops the mathematical machinery to meet these requirements; Part~III describes the laboratory standards and protocols that instantiate the mathematics in real-world assays; Part~IV provides orthogonal validation through SMA-seq; Part~V establishes the validation infrastructure that transforms experimental performance into regulatory-compliant clinical evidence. Together, these parts deliver a complete solution to the genomic complexity challenge, enabling safe and effective pharmacogenomic-guided therapy.

The theoretical foundation for this integrated approach rests on the Pipeline Factorization Theorem (Chapter~\ref{chap:classification-model}), which decomposes the sequencing process into modular conditional distributions. Critically, the SMA-seq methodology and SEER framework developed in Chapter~\ref{chap:sma-seq} provide the empirical implementation of the most complex term in this factorization---$\Prob(r\mid\sigma)$, the basecaller's transformation of signals to reads. This SMA-SEER feedback loop (measure $\rightarrow$ model $\rightarrow$ improve $\rightarrow$ deploy) ensures that the mathematical abstractions of Part~II are continuously validated and refined against physical ground truth. Appendices~\ref{app:core-equations} and \ref{app:mathematical-models} collect the complete mathematical specifications, including purity models, confusion matrices, and quality score calibration inequalities that link laboratory measurements to classification confidence.

\clearpage
