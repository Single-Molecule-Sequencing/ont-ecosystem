\begin{table}[!htbp]
\centering
\caption{Selected fields in the ONT \texttt{sequencing\_summary} specification}
\label{tab:ont-sequencing-summary}
\begin{tabular}{p{0.28\textwidth}p{0.62\textwidth}}
\toprule
\textbf{Field(s)} & \textbf{Description and workflow use} \\
\midrule
\texttt{filename\_fastq}/\texttt{filename\_fast5}/\texttt{filename\_pod5}/\texttt{filename\_bam}/\texttt{input\_filename} &
Online basecalling populates per-read pointers to FASTQ, FAST5/POD5, and BAM
artifacts, while offline re-basecalling records the originating POD5/FAST5 in
\texttt{input\_filename}. Keeping these paths in laboratory notebooks enables
auditable linkage between archived signal, emitted basecalls, and tertiary
analysis inputs. \\
\texttt{batch\_id}, \texttt{parent\_read\_id}, \texttt{read\_id}, \texttt{run\_id} & A
numeric \texttt{batch\_id} differentiates processing batches in offline
workflows. UUIDs identify the sequencing run, the parent raw signal chunk, and
any child reads produced by splitting or duplex polishing, supporting
deduplication and duplex traceability. \\
\texttt{channel}, \texttt{mux}, \texttt{minknow\_events}, \texttt{start\_time}, \texttt{duration}, \texttt{template\_start}, \texttt{template\_duration}, \texttt{num\_events\_template} &
Per-pore acquisition metadata reveals pore loading, mux cycling, and
electrical stability. Tracking event counts and dwell times across these
columns helps flag failing pores and validates hardware interventions noted in
run logs. \\
\texttt{passes\_filtering}, \texttt{sequence\_length\_template}, \texttt{mean\_qscore\_template} &
Instrument filters and basecaller-reported length/Q-score statistics provide a
lightweight proxy for Gate~4 calibration checks and dataset-wide length
distribution monitoring without reparsing FASTQ files. \\
\texttt{end\_reason} & Categorical reason for read termination. Reported values
include \texttt{signal\_positive}, \texttt{signal\_negative}, \texttt{mux\_change},
\texttt{unblock\_mux\_change}, \texttt{data\_service\_unblock\_mux\_change},
\texttt{analysis\_config\_change}, \texttt{device\_data\_error}, \texttt{api\_request},
\texttt{paused}, and \texttt{unknown}. Yield stratified by termination mode
distinguishes natural completion from active unblocks or hardware faults. \\
\texttt{pore\_type}, \texttt{experiment\_id}, \texttt{sample\_id} & User-supplied run
metadata should reconcile with laboratory information systems. Automated
validations prevent sample swaps, enforce chemistry-specific model selection,
and ensure the correct pore type was used during downstream analysis. \\
\texttt{poly\_tail\_*} (RNA only) & Poly-A tail length and positional estimates
are emitted when the poly-A estimation module runs. They support RNA-specific
QC such as verifying tail trimming parameters or detecting incomplete cDNA
synthesis. \\
\texttt{barcode\_*} (when barcoding) & Barcode kit, arrangement, alias, and front
/ rear confidence scores drive demultiplexing QC. Monitoring barcode scores and
\texttt{type} (test sample vs controls) guards against barcode hopping and
incorrect control labelling. \\
\texttt{alignment\_*} (when aligning) & Optional alignment genome, orientation,
accuracy, and BED hit metrics summarise exploratory mapping performed within
MinKNOW/Dorado, enabling rapid on-target assessment without rebuilding reports.
\\
\texttt{duplex\_parent\_template}/\texttt{duplex\_parent\_complement} (duplex) & When
duplex polishing succeeds these UUIDs link the fused read back to its template
and complement parents, preserving provenance for re-analysis or auditing. \\
\texttt{instrument\_type}, \texttt{instrument\_serial\_number}, \texttt{asic\_id}, \texttt{asic\_temp}, \texttt{asic\_version} &
Hardware identifiers and temperature telemetry capture real-time instrument
state. Tracking \texttt{asic\_temp} alongside throughput highlights thermal
drift, while \texttt{instrument\_serial\_number} links deviations to specific
devices for service reports. \\
\texttt{flow\_cell\_id}, \texttt{flow\_cell\_product\_code}, \texttt{pore\_version}, \texttt{sequencing\_kit}, \texttt{flow\_cell\_position} &
Consumable metadata documents chemistry lots and loading positions. Pairing
these fields with inventory databases accelerates root-cause analysis of pore
failure and enforces compatibility checks during automated run provisioning. \\
\texttt{protocol\_group\_id}, \texttt{protocol\_run\_id}, \texttt{protocol}, \texttt{protocol\_version}, \texttt{protocol\_start}, \texttt{protocol\_duration} &
Protocol identifiers mirror MinKNOW scheduling metadata. They support
comparisons between protocol variants (for example, adaptive sampling vs.
standard sequencing) and reconstruct pauses or restarts observed in the run
log. \\
\texttt{tracking\_id.*} & Nested tracking-card entries---covering operator,
laboratory, and sample alias fields---provide the linkage required for
regulatory record keeping. Populate them consistently to enable cross-run
analytics on operator effects or training interventions. \\
\bottomrule
\end{tabular}
\end{table}