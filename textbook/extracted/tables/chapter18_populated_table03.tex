\begin{table}[htbp]
\centering
\caption{Factors for Precision Endoxifen Prediction Algorithm}
\label{tab:ch18-endoxifen-factors}
\small
\begin{tabular}{p{0.25\textwidth}p{0.65\textwidth}}
\toprule
\textbf{Factor Category} & \textbf{Specific Variables and Rationale} \\
\midrule
\textbf{Primary Genetic} & CYP2D6 diplotype (fully resolved, including CNVs and hybrid alleles); activity score. \textit{Dominant genetic determinant, explains $\sim$30--50\% of endoxifen variability.} \\[6pt]

\textbf{Secondary Genetic} & CYP2C8/9/19 variants (alternative metabolic pathways); CYP3A4/5 variants (primary Tamoxifen metabolism to N-desmethyl-tamoxifen); SULT1A1, UGT2B15 variants (conjugation and clearance). \textit{Modulate endoxifen levels through secondary pathways; combined effect $\sim$10--20\% of variability.} \\[6pt]

\textbf{Clinical Covariates} & Co-medication with CYP2D6 inhibitors (SSRIs, antipsychotics, etc.); phenoconversion from extensive to poor metabolizer. Adherence to Tamoxifen regimen (pill counts, pharmacy refill records). Body mass index, age, menopausal status (affect distribution volume and clearance). \textit{Co-medication can reduce endoxifen by 50--75\%; adherence failures are common.} \\[6pt]

\textbf{Pharmacokinetic} & Direct measurement of (Z)-endoxifen plasma concentration via LC-MS/MS at steady state (4--8 weeks post-initiation). Therapeutic threshold: endoxifen $> 16$ nM associated with improved outcomes. \textit{Gold standard for dose adjustment; closes the loop between genotype prediction and actual exposure.} \\
\bottomrule
\end{tabular}
\end{table}