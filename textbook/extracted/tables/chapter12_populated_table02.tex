\begin{table}[!htbp]
\centering
\caption{Mapping ONT manifest fields into the signal database}
\label{tab:signal-db-manifest}
\begin{tabular}{p{0.3\textwidth}p{0.55\textwidth}}
\toprule
\textbf{Manifest fields} & \textbf{Signal database use} \\
\midrule
\texttt{run\_id}, \texttt{batch\_id} & Partition raw signals by production run
and offline re-basecalling batch; enable stratified sampling for drift studies.
\\
\texttt{instrument\_serial\_number}, \texttt{asic\_temp}, \texttt{pore\_type} &
Track instrumentation effects on standards; anomalous instruments can be
quarantined without deleting valuable signals. \\
\texttt{flow\_cell\_id}, \texttt{sequencing\_kit}, \texttt{pore\_version} & Link
consumable lots to training outcomes; regression tests can weight records when
chemistry transitions occur. \\
\texttt{filename\_pod5}/\texttt{filename\_fast5} & Persist pointers to the exact
signal container that produced a database row, enabling byte-for-byte recovery
years later. \\
\texttt{mean\_qscore\_template}, \texttt{passes\_filtering} & Store the initial
model assessment so that label-noise research can correlate downstream
confidence with basecaller self-reports. \\
\texttt{barcode\_*}, \texttt{sample\_id} & Automate multiplexed standard curation
and prevent mis-labelled reads from entering the training split. \\
\bottomrule
\end{tabular}
\end{table}