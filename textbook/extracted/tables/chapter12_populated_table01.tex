\begin{table}[!htbp]
\centering
\caption{Signal Database Schema}
\label{tab:signal-db-schema}
\begin{tabular}{p{3cm}p{3cm}p{6cm}}
\toprule
\textbf{Field} & \textbf{Type} & \textbf{Description} \\
\midrule
\texttt{signal\_id} & UUID & Unique identifier for this signal \\
\texttt{raw\_signal} & Binary blob & FAST5/POD5/BAM file or extracted array \\
\texttt{standard\_id} & String & Identifier for source physical standard \\
\texttt{true\_sequence} & String & Verified sequence from standard \\
\texttt{platform} & Enum & ONT R9.4.1, R10.4.1, PacBio Revio, etc. \\
\texttt{flowcell\_id} & String & Hardware identifier for traceability \\
\texttt{run\_date} & Timestamp & Sequencing date for temporal QC \\
\texttt{basecaller\_version} & String & Model used to generate comparison calls \\
\texttt{basecalled\_seq} & String & Sequence from initial basecalling \\
\texttt{quality\_scores} & Array & Per-base Q-scores from initial basecalling \\
\texttt{edit\_distance} & Integer & Levenshtein distance to true sequence \\
\texttt{qc\_flags} & Bitfield & Pass/fail flags for various QC criteria \\
\bottomrule
\end{tabular}
\end{table}