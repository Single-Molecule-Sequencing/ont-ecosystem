\documentclass[11pt]{article}
\usepackage{amsthm}
\newtheorem{theorem}{Theorem}

\usepackage[margin=1in]{geometry}
\usepackage{amsmath,amssymb,amsfonts,amsthm,bm}
\usepackage{graphicx}
\usepackage{booktabs}
\usepackage{hyperref}
\usepackage{enumitem}

\title{Mathematical Models for Single-Molecule Sequencing,\\
Error Characterization, Haplotype Classification, and Haplotagging}
\author{Pranjal Srivastava}
\date{November 2025}

\begin{document}
\maketitle
\tableofcontents

\section{Overview}

This document merges and harmonizes the mathematical content for:
\begin{itemize}[noitemsep]
    \item single-molecule sequencing and basecalling,
    \item Phred quality scores and alignment-based quality metrics,
    \item sequence classification and confusion-matrix error models,
    \item haplotype / diplotype classification (including polyploidy),
    \item read-level haplotagging with known haplotypes,
    \item plasmid replication and purity bounds, and
    \item dual Cas9 cutting efficiency.
\end{itemize}

Each section reintroduces its own notation so that variable definitions remain consistent with the original sources.

\section{Error Quantification}

In this section we formalize the objects used to quantify sequencing and
basecalling error.  We first introduce the mathematical preliminaries,
then define refined read- and sequence-level quality variables, and
finally give summary metrics that we use to evaluate a basecaller.

\subsection{Preliminaries (Mathematical)}

Let $\Sigma$ denote the nucleotide alphabet (e.g.\ $\Sigma=\{A,C,G,T\}$).
Let
\[
\mathcal{S} \subset \Sigma^{<\infty}
\qquad\text{and}\qquad
\mathcal{R} \subset \Sigma^{<\infty}
\]
be the sets of possible input sequences and observed reads, respectively.
For any sequence $x\in\mathcal{S}\cup\mathcal{R}$ we write $L(x)$ for its
length in bases and $x[\ell]\in\Sigma$ for the base at position
$\ell\in\{1,\dots,L(x)\}$.

Each read $r\in\mathcal{R}$ comes with basecaller-provided per–base Phred
quality scores
\[
Q_{r,\ell}\in\mathbb{R}, \qquad \ell=1,\dots,L(r),
\]
which are related to the basewise error probabilities by
$p_{r,\ell}=10^{-Q_{r,\ell}/10}$.

We will frequently consider \emph{pairs} $(s,r)\in \mathcal{S}\times\mathcal{R}$
consisting of a ``true'' input sequence $s$ and an observed read $r$
that has been aligned to $s$.  On such pairs we define the following
joint variables.
\begin{itemize}
  \item The (integer) Levenshtein edit distance
        \[
        d(s,r)\in\mathbb{N}_0,
        \]
        i.e.\ the minimum number of substitutions, insertions and
        deletions required to transform $s$ into $r$.
  \item An optimal alignment
        \[
        A(s,r),
        \]
        for example a global alignment in the sense of Needleman--Wunsch
        or an affine-gap variant, used to count how many of the edits in
        $d(s,r)$ are mismatches versus insertions or deletions.
\end{itemize}

We consider a finite collection of experiments
\[
\mathcal{E} = \{e_1,\dots,e_N\}.
\]
Each experiment $e_i\in\mathcal{E}$ consists of:
\begin{itemize}
  \item A finite set of input (``ground truth'') sequences
        \[
        \mathcal{S}_i=\{s_{ij}\}_{j=1}^{m_i}\subset\mathcal{S},
        \]
        with associated mixture proportions (or purities)
        \[
        \pi_{ij}>0,\qquad
        \sum_{j=1}^{m_i}\pi_{ij}=1.
        \]
  \item A finite multiset of reads
        \[
        \mathcal{R}_i=\{r_{ik}\}_{k=1}^{\ell_i}\subset\mathcal{R}.
        \]
        Each $r_{ik}$ carries basewise qualities
        $Q_{ik\ell}:=Q_{r_{ik},\ell}$ for $\ell=1,\dots,L(r_{ik})$.
\end{itemize}

Whenever a read $r_{ik}$ is aligned to an input sequence $s_{ij}$ in
experiment $e_i$, the pair $(s_{ij},r_{ik})$ inherits the joint
variables $d(s_{ij},r_{ik})$ and $A(s_{ij},r_{ik})$ defined above.

\subsection{Refined Quality Variables}
\label{subsec:refined-quality}

We now turn the edit distance into empirical error probabilities and
corresponding Phred-like quality scores for both reads and sequences.

\subsubsection*{Weighted edit distance}

Given an alignment $A(s,r)$ of $s$ to $r$, let
\[
M(s,r) \quad\text{and}\quad I(s,r)
\]
denote, respectively, the number of mismatch events and the total number
of insertion and deletion (indel) events in the alignment.

Often we wish to penalize indels more strongly than mismatches.  Let
$k\ge 0$ be the desired ratio of the indel weight to the mismatch
weight; that is, we want one indel to ``count'' as $k$ mismatches.  Set
\[
\alpha = \frac{1}{1+k}, \qquad 1-\alpha = \frac{k}{1+k},
\]
so that indeed $(1-\alpha)/\alpha = k$.  We define the \emph{weighted
edit count}
\[
E_\alpha(s,r)
  := \alpha\,M(s,r) + (1-\alpha)\,I(s,r)
\]
and the corresponding \emph{normalized weighted edit distance}
\[
\tilde d_\alpha(s,r)
  := \frac{E_\alpha(s,r)}{L(s)} \in [0,\infty).
\]

\paragraph{Example.}
If we want each indel to weigh three times as much as a mismatch, we
take $k=3$, hence
\[
\alpha = \frac{1}{1+3} = 0.25, \qquad 1-\alpha = 0.75.
\]

The unweighted normalized Levenshtein distance is recovered as
$\tilde d_{{1}/{2}}(s,r) = (M(s,r)+I(s,r))/L(s)$.

\subsubsection*{Per–read empirical quality}

For a fixed aligned pair $(s_{ij},r_{ik})$, we interpret
$\tilde d_\alpha(s_{ij},r_{ik})$ as an empirical estimate of the
per–base error probability for that read relative to that sequence.
To avoid degenerate values when the observed distance is extremely
small or zero, we truncate this estimate to lie in the interval
$\big[ L(s_{ij})^{-2},\,1 \big]$:
\begin{equation}
  \hat p^{\text{read}}_{ijk}
    := \min\!\Bigl\{1,\,
        \max\!\bigl(L(s_{ij})^{-2},\,
                    \tilde d_\alpha(s_{ij},r_{ik})\bigr)
      \Bigr\}.
\end{equation}
The corresponding empirical Phred-like quality of read $r_{ik}$ with
respect to $s_{ij}$ is
\begin{equation}
  Q^{\text{read}}_{ijk}
    := -10 \log_{10} \hat p^{\text{read}}_{ijk}.
\end{equation}
In particular, smaller normalized edit distances correspond to larger
quality values.

\subsubsection*{Sequence-level empirical quality}

Fix a sequence $s_{ij}$ in experiment $e_i$, and suppose that
$r_{ik}$, $k=1,\dots,\ell_i$, are the reads aligned to $s_{ij}$.  Under
a simple model in which each base in each read is independently wrong
with probability $p_{\mathrm{he}}$ (``per–base error''), the total
number of errors across all alignments is approximately
\[
X_{ij} \sim \mathrm{Binomial}\bigl(N_{ij}, p_{\mathrm{he}}\bigr),
\qquad
N_{ij} := \ell_i\,L(s_{ij}).
\]
An unbiased and maximum-likelihood estimator of $p_{\mathrm{he}}$ is
\begin{equation}
  \hat p^{\text{seq}}_{ij}
    := \frac{1}{\ell_i L(s_{ij})}
       \sum_{k=1}^{\ell_i} E_\alpha\bigl(s_{ij},r_{ik}\bigr)
     = \frac{X_{ij}}{N_{ij}}.
\end{equation}
We then define the empirical sequence-level quality of $s_{ij}$ as
\begin{equation}
  Q^{\text{seq}}_{ij}
    := -10 \log_{10} \hat p^{\text{seq}}_{ij}.
\end{equation}

\subsubsection*{Basecaller-derived read quality}

The basecaller provides per–base Phred scores $Q_{ik\ell}$ for read
$r_{ik}$.  Converting these to per–base error probabilities
$p_{ik\ell}=10^{-Q_{ik\ell}/10}$ and averaging over the read gives the
basecaller’s implied per–base error probability for that read,
\begin{equation}
  \hat p^{\text{bc}}_{ik}
    := \frac{1}{L(r_{ik})}
       \sum_{\ell=1}^{L(r_{ik})} 10^{-Q_{ik\ell}/10}.
\end{equation}
The corresponding read-level quality reported by the basecaller is
\begin{equation}
  Q^{\text{bc}}_{ik}
    := -10 \log_{10} \hat p^{\text{bc}}_{ik}.
\end{equation}
Here the $Q_{ik\ell}$ are the per–base $q$-values produced by the
basecaller.

\subsection{Quality Metrics of the Basecaller}

Finally, we define summary statistics that compare the empirical
qualities above and thereby quantify the performance of a basecaller.

For each aligned pair $(s_{ij},r_{ik})$ we have three associated
qualities:
\[
Q^{\text{seq}}_{ij},\qquad
Q^{\text{read}}_{ijk},\qquad
Q^{\text{bc}}_{ik}.
\]
We consider the following random differences over all such triples
$(i,j,k)$:
\begin{align}
  \Delta_{\text{seq-read}} &:= Q^{\text{seq}}_{ij}
                             - Q^{\text{read}}_{ijk}, \\
  \Delta_{\text{bc-read}}  &:= Q^{\text{bc}}_{ik}
                             - Q^{\text{read}}_{ijk}.
\end{align}

\paragraph{Precision.}
We define the \emph{precision} of our empirical quality estimates as
the empirical standard deviation of $\Delta_{\text{seq-read}}$:
\begin{equation}
  \mathrm{Precision}
    := \operatorname{sd}\!\bigl(\Delta_{\text{seq-read}}\bigr).
\end{equation}
Small values indicate that the per–read empirical qualities
$Q^{\text{read}}_{ijk}$ are tightly concentrated around the
sequence-level qualities $Q^{\text{seq}}_{ij}$.

\paragraph{Accuracy.}
We define the \emph{accuracy} (or calibration error) of the basecaller
as the empirical standard deviation of $\Delta_{\text{bc-read}}$:
\begin{equation}
  \mathrm{Accuracy}
    := \operatorname{sd}\!\bigl(\Delta_{\text{bc-read}}\bigr).
\end{equation}
A perfectly calibrated basecaller would yield both small variance and a
mean of $\Delta_{\text{bc-read}}$ close to zero; in practice we report
$\mathrm{Accuracy}$ together with its empirical mean to summarize both
dispersion and bias.

These definitions provide a mathematically consistent framework for
quantifying sequencing error and for comparing basecaller-reported
quality values with empirically observed error rates.


%=====================================================================


\section{Single-Molecule Sequencing and Basecalling Model}

\subsection{Raw signal and segmentation}

A single-molecule sequencing instrument outputs a raw signal
\begin{equation}
  T = (t_1,t_2,\dots,t_z),
\end{equation}
where $t_j$ is the measurement at time index $j$ and $t$ is the total
number of measurements in a run.  

A \emph{single-molecule signal} (one binding event) is a contiguous
subsequence of $X$:
\begin{equation}
  \mathbf{t}^{(i)} = (t_{a_i},t_{a_i+1},\dots,t_{b_i}),
\end{equation}
with length $L(\mathbf{t}^{(i)}) = b_i - a_i + 1$.

A segmentation model $\mathcal{W}$ is applied to $T$ to identify $n$
binding events:
\begin{equation}
  \mathcal{T} = \bigl\{ \mathbf{t}^{(1)},\dots,\mathbf{t}^{(n)} \bigr\}
  = \mathcal{W}(T), 
\end{equation}
where each $\mathbf{x}^{(i)}$ is intended to correspond to a single
molecule interacting with the instrument.

\subsection{Basecalling}

Let $\Sigma$ denote the nucleotide alphabet, typically
\begin{equation}
  \Sigma = \{A,C,G,T\}
  \quad\text{or}\quad
  \Sigma = \{A,C,G,T,N\}.
\end{equation}

A basecalling model $f$ maps each single-molecule signal to a sequence
of bases (a \emph{read})
\begin{equation}
  r_{ik} = f\bigl(\mathbf{t}^{(i)}\bigr)
  \in \Sigma^{L(\mathcal{T})},
\end{equation}
where $L(\mathcal{T})$ is the length of $r_{ik}$ which is $k$-th read in experiment $i$. For each base in read $r_{ik}$ , $f$ outputs a Phred
quality score $Q_{ikl}$ encoding an estimated error probability
$p_{ikl}$:
\begin{equation}
  Q_{ikl} = -10 \log_{10}\!\bigl( p_{ikl} \bigr),
  \qquad
  p_{ikl} = 10^{-Q_{ikl}/10}.
\end{equation}

Collecting all reads from a run gives
\begin{equation}
  \mathcal{R} = \bigl\{r_{i1},\dots,r_{ik}\bigr\}, \qquad
  Q = \bigl\{\mathbf{Q}^{(1)},\dots,\mathbf{Q}^{(z)}\bigr\},
\end{equation}
where $\mathbf{Q}^{(z)} = (Q_{iz1},\dots,Q_{izL(r_{iz})})$.


%=====================================================================

\section{Sequence Counts, Experiments, and Confusion Matrix}

\subsection{Individual experiments}

Consider a single sequencing experiment $e$ generating a set of reads
$R$.  Let
\begin{equation}
  U_e = \{u_{e1}, \dots, u_{ey}\}
\end{equation}
be the set of unique sequences observed in $R$ (after collapsing
identical reads).

Define the count vector $\mathbf{c}^{(E)}$ by
\begin{equation}
  c^{(e)}_j = 
  \bigl|\{ r_{ek} \in R \;:\; r_{ek} = u_j \}\bigr|,
  \qquad j=1,\dots, L(U_e).
\end{equation}
Then
\begin{equation}
  N_e = \sum_{j=1}^{U_e} c^{(e)}_j
\end{equation}
is the total number of reads in experiment $e$.


\paragraph{Average predicted and empirical quality scores:}
For each unique sequence $u_j \in U_E$, let  
$b_{\mathrm{avg},j}$ denote the average predicted Phred quality score over reads matching $u_j$,  
and let $q_{\mathrm{emp},j}$ denote the empirical read-level quality computed from Levenshtein error rates.

We collect these quantities into sets:
\[
Q_{\mathrm{pred}} =
\{Q^{bc}_{\mathrm{avg},1}, Q^{bc}_{\mathrm{avg},2}, \ldots, Q^{bc}_{\mathrm{avg},L(U_e)}\},
\tag{18}
\]
\[
Q_{\mathrm{emp}} =
\{Q^{read}_{\mathrm{avg},1}, Q^{read}_{\mathrm{avg},2}, \ldots, Q^{read}_{\mathrm{avg},L(U_e)}\}.
\tag{19}
\]

\paragraph{Percentage of overestimated accuracy:}
For each unique sequence $s_i$ with count $c^{(E)}_i$, define
\[
I_{c_i} =
\begin{cases}
\sigma\!\left(k\bigl[Q_{\mathrm{pred}}(c^{(e)}_i) - Q_{\mathrm{emp}}(c^{(e)}_i) - \alpha\bigr]\right),
& \text{if } Q_{\mathrm{pred}}(c^{(e)}_i) > Q_{\mathrm{emp}}(c^{(e)}_i), \\[4pt]
0, & \text{otherwise},
\end{cases}
\tag{20}
\]
where $\sigma(x) = \dfrac{1}{1 + e^{-x}}$, and $k > 0$ and $\alpha$ are tunable parameters controlling the slope
and offset, respectively.

The percentage of reads whose accuracy is overestimated by the basecaller is then
\[
d =
\frac{\displaystyle \sum_{i=1}^{L(U_e)} c^{(e)}_i I_{c_i}}{n}
\,\times\, 100.
\tag{21}
\]



\subsection{Set of experiments and global sequence index}

Let $\mathcal{E} = \{e_1,\dots,e_N\}$ be a set of experiments
performed using the same sequencing technology.

Let the global set of unique sequences across all experiments be
\begin{equation}
  S = \bigcup_{k=1}^K S_{e_k}
  = \{ s_1,\dots,s_M\}.
\end{equation}

For each experiment $e_k$, we can construct a count vector
$\mathbf{c}^{(k)}\in\mathbb{N}^M$ over the global index $1,\dots,M$,
where $c^{(k)}_j$ is the count of sequence $s_j$ in experiment $E_k$.
Let $N_k = \sum_{j=1}^M c^{(k)}_j$ be the size of experiment $e_k$.

\subsection{Confusion matrix and empirical error model}

Using high-purity standards with known ground-truth sequences, we can
construct a confusion matrix $C$ summarizing sequence-level
classification performance of the basecaller.

Let $S = \{s_1,\dots,s_M\}$ be the set of possible sequences used in
standards.  The confusion matrix is an $M\times M$ matrix $C$ with
entries
\begin{equation}
  C_{ij}
  = \text{number of times a molecule of true sequence $s_i$
         was classified as $s_j$}.
\end{equation}
Diagonal elements $C_{ii}$ correspond to correct classifications;
off-diagonal elements $C_{ij}$ ($i\neq j$) correspond to
misclassifications.

For a given true sequence $s_i$, define
\begin{equation}
  N_i = \sum_{j=1}^M C_{ij},
\end{equation}
the total number of molecules of type $s_i$ used in standards.

\paragraph{Correct classification (true positive rate).}
\begin{equation}
  \text{TPR}_i
  = P(\hat{s}=s_i \mid s_i)
  = \frac{C_{ii}}{N_i}.
\end{equation}

\paragraph{Misclassification probability for sequence $s_i$.}
\begin{equation}
  \epsilon_i
  = P(\hat{s}\neq s_i \mid s_i)
  = 1 - \text{TPR}_i
  = 1 - \frac{C_{ii}}{N_i}
  = \frac{\sum_{j\neq i}C_{ij}}{N_i}.
\end{equation}

\paragraph{Pairwise misclassification probability.}
For $i\neq j$,
\begin{equation}
  P(\hat{s}=s_j \mid s_i)
  = \frac{C_{ij}}{N_i}.
\end{equation}

These probabilities define a sequence-level empirical error model
that can be used inside higher-level haplotype and diplotype
classification models.

%=====================================================================

\section{Mean Phred Score vs.~Phred of Mean Error Probability}
\label{sec:phred-inequality}

Let $p_1,\dots,p_n\in(0,1]$ be error probabilities corresponding to
individual basecalls, and let the Phred quality for base $i$ be
\begin{equation}
  Q_i = -10 \log_{10}(p_i).
\end{equation}
Define the arithmetic means
\begin{equation}
  p = \frac{1}{n} \sum_{i=1}^n p_i,
  \qquad
  Q = \frac{1}{n} \sum_{i=1}^n Q_i.
\end{equation}

\begin{theorem}
The mean Phred score $Q$ is always greater than or equal to the Phred
score of the mean error probability:
\begin{equation}
  Q \;\ge\; -10 \log_{10}(p),
\end{equation}
with equality if and only if all $p_i$ are equal (or $n=1$).
\end{theorem}

\begin{proof}
The base-10 logarithm $\log_{10}(x)$ is concave on $(0,\infty)$.
By Jensen's inequality, for any concave $f$,
\begin{equation}
  f\!\left(\frac{1}{n} \sum_{i=1}^n x_i\right)
  \;\ge\;
  \frac{1}{n}\sum_{i=1}^n f(x_i).
\end{equation}
Take $f(x) = \log_{10}(x)$ and $x_i = p_i$:
\begin{equation}
  \log_{10}\!\left(\frac{1}{n}\sum_{i=1}^n p_i\right)
  \;\ge\;
  \frac{1}{n}\sum_{i=1}^n \log_{10}(p_i).
\end{equation}
Multiply both sides by $-10$, which reverses the inequality:
\begin{equation}
 -10\log_{10}\!\left(\frac{1}{n}\sum_{i=1}^n p_i\right)
 \;\le\;
 -10\cdot\frac{1}{n}\sum_{i=1}^n \log_{10}(p_i).
\end{equation}
Recognizing that
\(
  Q_i = -10\log_{10}(p_i)
\),
we obtain
\begin{equation}
  -10 \log_{10}(p) \;\le\;
  \frac{1}{n}\sum_{i=1}^n Q_i
  = Q.
\end{equation}
Equality in Jensen's inequality holds if and only if all $x_i$ are
equal, i.e.\ $p_1=\cdots=p_n$ (or if $n=1$).  This proves the claim.
\end{proof}

\paragraph{Interpretation.}
Averaging in log-space (Phred) is more optimistic than converting the
arithmetic mean error probability to a single Phred score, because the
logarithm is concave and therefore gives more weight to smaller
error probabilities.

%=====================================================================

\section{Alignment-Based Quality Metric}

Let $G=(g_1,\dots,g_N)$ be a ground-truth sequence and
$B=(b_1,\dots,b_N)$ a basecalled sequence aligned to $G$ (gaps are
allowed).  Each basecall $b_i$ has a Phred quality score $Q_i$, with
error probability
\begin{equation}
  p_i = 10^{-Q_i/10}.
\end{equation}

We define a per-alignment-column score $s_i$ as
\begin{equation}
  s_i =
  \begin{cases}
    1 - p_i, & \text{if } g_i = b_i, \\[4pt]
    p_i, & \text{if } g_i \neq b_i, \\[4pt]
    0, & \text{if } g_i = \text{``-''} \text{ or } b_i = \text{``-''}.
  \end{cases}
\end{equation}

\paragraph{Mean correctness score.}
Define
\begin{equation}
  M = \frac{1}{N} \sum_{i=1}^N s_i,
\end{equation}
so that $M\in[0,1]$ represents the average correctness score across
the alignment, weighted by both basecall accuracy and confidence.

\paragraph{Phred-like aggregate quality.}
Define an \emph{effective} error probability
\begin{equation}
  p_{\text{err}} = 1 - M,
\end{equation}
and a Phred-like score
\begin{equation}
  Q_{\text{new}} = -10 \log_{10}(p_{\text{err}})
  = -10 \log_{10}(1-M).
\end{equation}
Here, $M$ is the mean correctness, $1-M$ is the effective empirical
error probability, and $Q_{\text{new}}$ is a single Phred-scale
summary that incorporates both predicted and empirical accuracy.

%=====================================================================

\section{Per-Base Variant Likelihood From Basewise Error Rates}

Consider a particular haplotype (or reference sequence) $h$ with
sequence
\begin{equation}
  g = (g_1,\dots,g_L) \in \mathcal{A}^L,
\end{equation}
and a read
\begin{equation}
  r = (r_1,\dots,r_L)\in\mathcal{A}^L
\end{equation}
aligned to $g$.  Let $e_i\in[0,1]$ be the error rate at base $i$
(e.g.\ $e_i = p_i$ from its Phred score).

Assume that for each position $i$,
\begin{align}
  P(r_i = g_i \mid g_i, e_i) &= 1 - e_i, \\[4pt]
  P(r_i = b \neq g_i \mid g_i, e_i)
    &= \frac{e_i}{|\mathcal{A}|-1},
    \qquad b\in\mathcal{A},\, b\neq g_i,
\end{align}
and that different positions are conditionally independent given $g$.
Then the per-read likelihood is
\begin{equation}
  P(r\mid g, \mathbf{e})
  = \prod_{i=1}^L
      \Bigl[
        (1-e_i)\,\mathbf{1}\{r_i=g_i\}
        + \frac{e_i}{|\mathcal{A}|-1}\,\mathbf{1}\{r_i\neq g_i\}
      \Bigr],
\end{equation}
where $\mathbf{e}=(e_1,\dots,e_L)$ and $\mathbf{1}\{\cdot\}$ is the
indicator function.

This can be used as a sequence-specific likelihood term inside
haplotype or molecule-of-origin likelihood calculations.

%=====================================================================

\section{Haplotype Classification (Unknown Haplotype)}

\subsection{Sets and priors}

Let
\begin{equation}
  \mathcal{H} = \{h_1,\dots,h_p\}
\end{equation}
be a set of possible haplotypes for a sample (e.g.\ a set of known
haplotypes from a population).  Each $h_i$ is specified as a
population of DNA molecules (chromosomes) with known sequences and
stoichiometric ratios.

\begin{itemize}[noitemsep]
  \item $P(h_i)$ is the prior probability of haplotype $h_i$, estimated
        from population frequency data.
  \item For haplotype $h_i$, let
        \begin{equation}
          M(h_i) = \{ m_{i1},\dots,m_{iv_i} \}
        \end{equation}
        be the set of DNA molecules in that haplotype (e.g.\ chromosomes
        and plasmids).
\end{itemize}

\subsection{Cell population and derived sequences}

Let
\begin{itemize}[noitemsep]
  \item $C(h_i) = \{c_1,\dots,c_w\}$ be the set of cells in a sample
        derived from an original genome with haplotype $h_i$,
  \item $U(h_i) = \{u_1,\dots,u_x\}$ be the set of unique DNA sequences
        present in $C(h_i)$ after mutation accumulation.
\end{itemize}

Mutations accumulate over $n_{\text{div}}$ cell divisions at mutation
rate $\mu$ for a genome (or molecule) of length $L$.  Conceptually we
write
\begin{equation}
  P\bigl(u\mid m,\, \mu, n_{\text{div}}, L\bigr)
\end{equation}
for the probability of generating a particular sequence $u$ from an
ancestral molecule $m$.

\subsection{Fragmentation, labeling, and sequencing}

\paragraph{Fragmentation.}
Let
\begin{equation}
  D(h_i) = \{ d_1,\dots,d_y \}
\end{equation}
denote the set of DNA fragments after extraction and fragmentation
(from $U(h_i)$).  Let $\theta_{\text{frag}}$ be parameters describing
the fragmentation process.  Then
\begin{equation}
  P\bigl(d \mid u,\, \theta_{\text{frag}}\bigr)
\end{equation}
denotes the probability that a fragment $d$ is produced from a
sequence $u$.

\paragraph{Labeling and enrichment.}
Let
\begin{equation}
  L(h_i) = \{\ell_1,\dots,\ell_z\}
\end{equation}
be the set of sequences that successfully receive sequencing-specific
adaptors (i.e.\ are labeled and enriched).  Let $\theta_{\text{lab}}$
be parameters describing adapter ligation and enrichment.  Then
\begin{equation}
  P\bigl(\ell \mid d,\, \theta_{\text{lab}}\bigr)
\end{equation}
is the probability that fragment $d$ becomes labeled sequence $\ell$.

\paragraph{Sequencing and signal generation.}
From the labeled sequences, a single-molecule sequencing experiment
produces a set of events and signals
\begin{equation}
  X(h_i) = \{\mathbf{x}^{(1)},\dots,\mathbf{x}^{(n)}\}
\end{equation}
and corresponding basecalled reads
\begin{equation}
  R = \{r_1,\dots,r_n\}.
\end{equation}

Let $\theta_{\text{seq}}$ capture sequencing and basecalling error
parameters.  Then
\begin{equation}
  P\bigl(r \mid \ell,\, \theta_{\text{seq}}\bigr)
\end{equation}
is the probability of obtaining read $r$ from labeled sequence $\ell$,
incorporating the empirical error model (e.g.\ confusion matrix $C$).

\subsection{Likelihood for a haplotype}

Conditioning on haplotype $h_i$ and marginalizing over unobserved
stages (mutation, fragmentation, labeling), a generic factorization
for the likelihood of an observed read $r$ is
\begin{equation}
  P(r \mid h_i)
  = \sum_{u\in U(h_i)} \sum_{d\in D(h_i)} \sum_{\ell\in L(h_i)}
      P(r \mid \ell, \theta_{\text{seq}})
      P(\ell \mid d, \theta_{\text{lab}})
      P(d \mid u, \theta_{\text{frag}})
      P(u \mid h_i, \mu, n_{\text{div}}, L).
\end{equation}

Assuming independence of reads given haplotype $h_i$,
\begin{equation}
  P(R \mid h_i)
  = \prod_{r\in R} P(r \mid h_i).
\end{equation}

\subsection{Posterior probabilities and classification rule}

By Bayes' theorem, the posterior for haplotype $h_i$ given observed
reads $R$ is
\begin{equation}
  P(h_i \mid R)
  = \frac{P(R \mid h_i) P(h_i)}{\displaystyle
     \sum_{j=1}^p P(R \mid h_j) P(h_j)}.
\end{equation}

The posterior probability that the sample corresponds to \emph{any}
haplotype other than $h_i$ is
\begin{equation}
  P\bigl(\text{``not } h_i\text{''} \mid R\bigr)
  = 1 - P(h_i \mid R).
\end{equation}

Define the likelihood ratio (LR) for haplotype $h_i$ as
\begin{equation}
  \text{LR}_i(R)
  = \frac{P(h_i \mid R)}{P(\text{``not }h_i\text{''} \mid R)}
  = \frac{P(h_i \mid R)}{1 - P(h_i \mid R)}.
\end{equation}

Given a threshold $\tau > 0$, a simple decision rule is
\begin{align}
  \text{Accept } h_i \text{ for the sample}
  \quad &\text{if} \quad \text{LR}_i(R) \ge \tau, \\
  \text{otherwise} \quad &\text{declare the sample ``uncertain'' or consider resequencing}.
\end{align}

In a multiclass setting (multiple haplotypes), one may classify the
sample as
\begin{equation}
  \hat{h}
  = \arg\max_{1\le i \le p} P(h_i \mid R),
\end{equation}
optionally requiring that $P(\hat{h}\mid R)$ exceed a minimum
probability threshold before accepting the call.

%=====================================================================

\section{Diplotypes, Polyploidy, and Cost-Based Decision Rules}

\subsection{Additional notation for diplotypes and polyploidy}

Let
\begin{itemize}[noitemsep]
  \item $\mathcal{D}$: set of all possible diplotypes (pairs of haplotypes),
  \item $K$: number of sets of homologous chromosomes (e.g.\ $K=1$ for
        a single gene, larger for polyploid loci),
  \item $\gamma$: classification threshold on posterior probabilities,
  \item $N$: number of reads per sample used for classification,
  \item $\epsilon_{d\to d'}(N)$: empirical misclassification rate from
        true diplotype $d\in\mathcal{D}$ to diplotype $d'\in\mathcal{D}$ when $N$ reads are used,
  \item $C_{d\to d'}$: cost of misclassifying $d$ as $d'$,
  \item $C_{\text{res},d}$: cost of resequencing a sample whose true
        diplotype is $d$,
  \item $\psi_{d}(\gamma,N)$: probability that a sample with true
        diplotype $d$ is flagged for resequencing, given threshold
        $\gamma$ and $N$ reads,
  \item $\pi_d$: prior probability of diplotype $d$,
  \item $\eta_{d,s}$: enrichment efficiency for sequence $s$ in diplotype $d$,
  \item $L_d(R)$: likelihood of observing reads $R$ under diplotype $d$,
  \item $\ell_{r,s}$: sequence-specific likelihood for signal/read $r$
        and sequence $s$ (from basecalling error model).
\end{itemize}

\subsection{Expected cost}

For a diplotype $d$, the posterior probability given reads $R$ is
\begin{equation}
  P(d\mid R) = \frac{L_d(R)\,\pi_d}{\displaystyle 
    \sum_{d'\in\mathcal{D}} L_{d'}(R)\,\pi_{d'} }.
\end{equation}

A decision policy maps posterior probabilities to one of three
actions: \emph{call diplotype $d'$}, \emph{declare uncertain and
resequencing}, or \emph{no call}.  Under a given policy that depends
on $\gamma$ and $N$, define:
\begin{align}
  \epsilon_{d\to d'}(\gamma,N)
    &= P\bigl(\text{policy calls }d' \neq d
              \,\big|\,
              \text{true diplotype }d,\, N,\,\gamma\bigr),
  \\[4pt]
  \psi_{d}(\gamma,N)
    &= P\bigl(\text{policy chooses ``resequencing''}
              \,\big|\,
              \text{true diplotype }d,\, N,\,\gamma\bigr).
\end{align}

The expected cost for a given $(\gamma,N)$ is
\begin{equation}
  \mathcal{C}(\gamma,N)
  = \sum_{d\in\mathcal{D}} \pi_d
      \left[
        \sum_{d'\neq d} C_{d\to d'}\,\epsilon_{d\to d'}(\gamma,N)
        + C_{\text{res},d}\,\psi_d(\gamma,N)
      \right].
\end{equation}

An optimal pair $(\gamma^*,N^*)$ can be defined as
\begin{equation}
  (\gamma^*,N^*)
  = \arg\min_{\gamma,N} \mathcal{C}(\gamma,N),
\end{equation}
subject to constraints (e.g.\ minimal acceptable sensitivity, budget
limits on resequencing, etc.).

%=====================================================================

\section{Read-Level Haplotagging with Known Haplotype}

In this section, the haplotype of the genomic source is assumed known:
there is a single haplotype $h$ and
\begin{equation}
  P(h) = 1.
\end{equation}

The goal is to assign each read to its most likely \emph{molecule of
origin} (e.g.\ specific chromosome or plasmid) within $h$.

\subsection{Model and notation}

Let
\begin{itemize}[noitemsep]
  \item $M(h) = \{m_1,\dots,m_v\}$ be the set of DNA molecules in haplotype $h$,
  \item $P(m_j \mid h)$ be the prior probability that a randomly
        selected molecule in the sample is of type $m_j$ (stoichiometric ratio),
  \item $P(r\mid m_j,h)$ be the likelihood of read $r$ given that its
        molecule-of-origin is $m_j$ (this incorporates the
        basecalling error model, e.g.\ confusion matrix and
        per-base error rates).
\end{itemize}

\subsection{Posterior for molecule-of-origin of a single read}

By Bayes' theorem, the posterior probability that read $r$ originated
from molecule $m_j$ is
\begin{equation}
  P(m_j \mid r, h)
  = \frac{P(r\mid m_j,h)\,P(m_j\mid h)}{\displaystyle 
          \sum_{k=1}^v P(r\mid m_k,h)\,P(m_k\mid h)}.
\end{equation}

The probability that $r$ originated from any other molecule is
\begin{equation}
  P(\text{``other''} \mid r, h)
  = 1 - P(m_j \mid r,h).
\end{equation}

Define the per-read likelihood ratio
\begin{equation}
  \text{LR}_j(r)
  = \frac{P(m_j \mid r,h)}{P(\text{``other''} \mid r,h)}
  = \frac{P(m_j \mid r,h)}{1 - P(m_j\mid r,h)}.
\end{equation}

\subsection{Decision rule for haplotagging a read}

Given a threshold $\tau > 0$, a basic decision rule for assigning read
$r$ to molecule $m_j$ is
\begin{align}
  \text{Assign read $r$ to $m_j$}
    &\quad \text{if} \quad \text{LR}_j(r) \ge \tau, \\[4pt]
  \text{Declare $r$ ``unphased'' (unassigned)}
    &\quad \text{if} \quad \text{LR}_j(r) < \tau
    \quad \text{for all } j=1,\dots,v.
\end{align}

\subsection{Unphased reads and cost-based optimization}

Let $N$ be the total number of reads from the sample.  For a given
threshold $\tau$, define
\begin{itemize}[noitemsep]
  \item $P_{\text{unph}}(\tau)$: probability a randomly selected read is unphased
        (i.e.\ no molecule’s LR exceeds $\tau$),
  \item $P_{\text{mis}}(\tau)$: probability a read is assigned to the wrong molecule.
\end{itemize}
Then the expected number of unphased reads is
\begin{equation}
  U(\tau,N)
  = N \, P_{\text{unph}}(\tau).
\end{equation}

Let $C_{\text{mis}}$ be the cost of misclassifying a read and
$C_{\text{unph}}$ be the cost associated with leaving a read unphased.
An example cost function is
\begin{equation}
  \mathcal{C}(\tau,N)
  = C_{\text{mis}} \, N P_{\text{mis}}(\tau)
    + C_{\text{unph}} \, U(\tau,N).
\end{equation}
An optimal threshold $\tau^*$ may be defined as
\begin{equation}
  \tau^* = \arg\min_{\tau} \mathcal{C}(\tau,N).
\end{equation}

In practice, $P_{\text{mis}}(\tau)$ and $P_{\text{unph}}(\tau)$ are
estimated from standards or simulations, using the same empirical
error model (confusion matrix, per-base error rates) that defines
$P(r\mid m_j,h)$.

%=====================================================================

\section{Plasmid Replication, Mutation, and Purity Bounds}

\subsection{Basic assumptions}

\begin{itemize}[noitemsep]
  \item A plasmid is an extragenomic circular dsDNA molecule that can
        be maintained at (on average) single-copy per bacterium if it
        contains a functional single-copy origin of replication (ORI).
  \item An \emph{E.~coli} chromosome is a single circular dsDNA
        molecule of length on the order of $10^6$\,bp.
  \item A typical plasmid is two orders of magnitude shorter than the
        chromosome, often between $3\times 10^3$ and $10^4$\,bp.
  \item The per-base replication error rate of the \emph{E.~coli}
        replisome is on the order of $10^{-10}$ errors per base per
        replication.
\end{itemize}

Consider a single isolated \emph{E.~coli} cell containing:
\begin{itemize}[noitemsep]
  \item one genomic chromosome,
  \item one single-copy plasmid.
\end{itemize}
After one replication (cell division), both DNA molecules replicate and
segregate into two daughter cells, yielding two chromosomes and two
plasmids in total, with the possibility of replication errors.

\subsection{Theoretical purity under replication errors}

Let
\begin{itemize}[noitemsep]
  \item $r$ be the per-base replication error rate,
  \item $L$ be the plasmid length in bp,
  \item $k$ be the number of replication cycles.
\end{itemize}

The probability that a single base remains error-free in a single
replication is $(1-r)$.  After $k$ replications, the probability that
this base is still identical to the original is $(1-r)^k$.

Assuming independence across bases, the probability that \emph{all}
$L$ bases of the plasmid remain error-free after $k$ replications is
\begin{equation}
  P_{\text{pure}}(k)
  = (1-r)^{Lk}.
\end{equation}
This $P_{\text{pure}}(k)$ serves as an upper bound on the fraction of
plasmids that remain identical to the original sequence after $k$
replication cycles (i.e.\ an upper bound on purity).

For small $r$ and large $Lk$, the approximation
\begin{equation}
  P_{\text{pure}}(k) \approx \exp(-rLk)
\end{equation}
is often useful.

\subsection{Purity Q-value}

Let $P = P_{\text{pure}}(k)$ be the theoretical purity.  Define the
\emph{mutated fraction} (fraction of molecules that contain at least
one replication error) as
\begin{equation}
  P_{\text{mut}} = 1 - P.
\end{equation}

A Phred-like \emph{purity Q-value} can be defined by treating
$P_{\text{mut}}$ as an error probability:
\begin{equation}
  Q_{\text{pur}}
  = -10 \log_{10}(P_{\text{mut}})
  = -10 \log_{10}(1 - P_{\text{pure}}(k)).
\end{equation}

Higher $Q_{\text{pur}}$ indicates a smaller mutated fraction (higher
purity).  This quantity can be plotted as a function of $k$, $L$, and
$r$ to visualize how purity decays with replication cycles and
plasmid length.

\subsection{Lower-bound purity estimate from capillary electrophoresis}

Let
\begin{itemize}[noitemsep]
  \item $C_{\text{major}}$: concentration of the major (intended) plasmid sequence,
  \item $C_{\text{other}}$: concentration of all other sequences
        (variants) in the sample.
\end{itemize}
Capillary electrophoresis (CE) size distributions plus standards of
known concentration and length can be used to estimate
$C_{\text{major}}$ and $C_{\text{other}}$.

A lower bound on purity is
\begin{equation}
  P_{\text{low}}
  = \frac{C_{\text{major}}}{C_{\text{major}} + C_{\text{other}}}.
\end{equation}

\subsection{Empirical purity from clonal expansion and Sanger sequencing}

For a standard experiment $E$ with nominal sequence $s^\star$:
\begin{itemize}[noitemsep]
  \item Let $a$ be the number of single-colony expansions whose Sanger
        sequencing matches $s^\star$,
  \item Let $b$ be the total number of colonies sequenced.
\end{itemize}
Then an empirical purity estimate is
\begin{equation}
  \hat{P}_E = \frac{a}{b}.
\end{equation}

For all samples derived from a given \emph{E.~coli} strain with
original sequence $s_0$:
\begin{itemize}[noitemsep]
  \item Let $c$ be the number of colonies whose Sanger sequencing matches $s_0$,
  \item Let $d$ be the total number of colonies pooled across experiments.
\end{itemize}
Then an overall empirical purity estimate is
\begin{equation}
  \hat{P}_{\text{strain}} = \frac{c}{d}.
\end{equation}

Comparing $P_{\text{low}}$, $P_{\text{pure}}(k)$, and empirical
estimates $\hat{P}_E$ and $\hat{P}_{\text{strain}}$ provides a
consistency check between theoretical replication-based purity bounds
and experimental measurements.

%=====================================================================

\section{Dual Cas9 Cutting: Probability of Isolating a Gene}

We model the probability of successfully isolating a gene (or locus)
via dual Cas9 cutting, accounting for fragment length distribution and
cutting efficiencies.

\subsection{Definitions}

\begin{itemize}[noitemsep]
  \item $G$: length of the gene (or distance between two Cas9 target
        sites) in base pairs (bp).
  \item $L$: length of a DNA fragment in bp.
  \item $f_L(\ell)$: probability density function (pdf) of fragment lengths.
  \item $F_L(\ell) = P(L \le \ell)$: cumulative distribution function (cdf).
  \item $p_{\text{frag}}(G) = P(L \ge G)$: probability that a fragment
        is at least as long as $G$.
  \item $e_1,e_2$: base cutting efficiencies at Cas9 target sites 1
        and 2, respectively.
\end{itemize}

\subsection{Probability that a fragment can contain the full gene}

\begin{equation}
  p_{\text{frag}}(G)
  = P(L \ge G)
  = 1 - F_L(G^-),
\end{equation}
where $F_L(G^-)$ denotes the limit of $F_L(\ell)$ from the left at
$\ell=G$ (or simply $F_L(G)$ when the distribution is continuous).

\subsection{Dual Cas9 cutting probability}

Assuming:
\begin{itemize}[noitemsep]
  \item Fragmentation is random and independent of Cas9 cutting.
  \item Cas9 cutting at each of the two target sites is independent,
        with probabilities $e_1$ and $e_2$.
\end{itemize}

Then the probability that both cuts are successful \emph{given} that
the fragment is long enough is
\begin{equation}
  p_{\text{cut} \mid \text{frag}} = e_1 e_2.
\end{equation}

The overall probability of successfully isolating the gene via dual
Cas9 cutting is
\begin{equation}
  p_{\text{dual}}(G)
  = p_{\text{frag}}(G) \cdot e_1 e_2
  = \bigl[1 - F_L(G^-)\bigr] e_1 e_2.
\end{equation}

\subsection{Example: Exponential fragment size distribution}

If fragment lengths are exponentially distributed with rate $\lambda$
(so mean fragment length $= 1/\lambda$):
\begin{equation}
  f_L(\ell) = \lambda e^{-\lambda \ell}, \qquad \ell\ge 0,
\end{equation}
then
\begin{equation}
  F_L(\ell) = 1 - e^{-\lambda \ell}, \qquad
  P(L \ge G) = e^{-\lambda G}.
\end{equation}
So
\begin{equation}
  p_{\text{dual}}(G)
  = e^{-\lambda G} e_1 e_2.
\end{equation}
This explicitly shows the exponential decrease in success probability
with gene length $G$ under random fragmentation.

%=====================================================================

\section{Summary of Key Notation}

For convenience, we collect here several commonly used symbols (many
are also redefined locally in their sections):

\begin{itemize}[noitemsep]
  \item $X$ : raw single-molecule signal from an experiment (sequence of current levels).
  \item $\mathbf{x}^{(i)}$ : single-molecule signal (one binding event).
  \item $R = \{r_1,\dots,r_n\}$ : set of basecalled reads.
  \item $\mathcal{A}$ : nucleotide alphabet (e.g.\ $\{A,C,G,T\}$).
  \item $Q_i$ : Phred quality score for base $i$.
  \item $p_i$ : error probability for base $i$, $Q_i = -10\log_{10}(p_i)$.
  \item $M$ : mean correctness score in alignment-based metric.
  \item $Q_{\text{new}}$ : Phred-style summary from $M$:
        $Q_{\text{new}}=-10\log_{10}(1-M)$.
  \item $S$ : set of unique sequences across experiments.
  \item $C_{ij}$ : confusion-matrix entry (true $s_i$, predicted $s_j$).
  \item $\epsilon_i$ : misclassification probability for sequence $s_i$.
  \item $\mathcal{H} = \{h_1,\dots,h_p\}$ : set of possible haplotypes.
  \item $P(h_i)$ : prior probability of haplotype $h_i$.
  \item $M(h)$ : set of molecules (chromosomes, plasmids) in haplotype $h$.
  \item $P(r\mid h_i)$ : likelihood of reads under haplotype $h_i$.
  \item $P(h_i\mid R)$ : posterior probability of haplotype $h_i$ given reads $R$.
  \item $\text{LR}_i$ : likelihood ratio for haplotype or molecule $i$.
  \item $\mathcal{D}$ : set of possible diplotypes.
  \item $\pi_d$ : prior probability of diplotype $d$.
  \item $\epsilon_{d\to d'}$ : misclassification rate from diplotype $d$ to $d'$.
  \item $C_{d\to d'}$ : cost of misclassifying $d$ as $d'$.
  \item $C_{\text{res},d}$ : cost of resequencing when true diplotype is $d$.
  \item $r$ : per-base replication error rate (in plasmid model).
  \item $L$ : plasmid length (bp).
  \item $k$ : number of replication cycles.
  \item $P_{\text{pure}}(k) = (1-r)^{Lk}$ : theoretical purity after $k$ cycles.
  \item $P_{\text{low}}$ : lower-bound purity from CE and standards.
  \item $G$ : gene length (distance between dual Cas9 cut sites).
  \item $L$ (in Cas9 model) : fragment length (bp) with pdf $f_L$ and cdf $F_L$.
  \item $e_1,e_2$ : Cas9 cutting efficiencies at two sites.
\end{itemize}

\end{document}


\end{document}
