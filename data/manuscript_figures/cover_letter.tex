\documentclass[11pt,a4paper]{letter}
\usepackage[margin=1in]{geometry}
\usepackage[utf8]{inputenc}
\usepackage[T1]{fontenc}

\signature{[Corresponding Author Name]\\
Department of Computational Medicine and Bioinformatics\\
University of Michigan}

\address{Department of Computational Medicine and Bioinformatics\\
University of Michigan\\
Ann Arbor, MI 48109\\
USA}

\begin{document}

\begin{letter}{Editorial Office\\
[Journal Name]}

\opening{Dear Editor,}

We are pleased to submit our manuscript entitled ``\textbf{A Comprehensive Registry Framework for Oxford Nanopore Sequencing Experiments: Metadata Management, Quality Tracking, and Institutional Standardization}'' for consideration for publication in [Journal Name].

Long-read sequencing technologies, particularly Oxford Nanopore platforms, have transformed genomic research by enabling real-time, portable sequencing with reads spanning tens of kilobases. However, the rapid adoption of these technologies has outpaced the development of standardized metadata management practices, creating challenges for reproducibility and cross-study comparisons.

Our manuscript addresses this gap by presenting a comprehensive registry framework that:

\begin{itemize}
    \item Provides a standardized schema for capturing experimental metadata across 165 Oxford Nanopore sequencing experiments
    \item Achieves 100\% metadata completeness through systematic extraction, inference, and validation
    \item Documents the technological transition from R10.4 to R10.4.1 chemistry and from Guppy to Dorado basecallers
    \item Establishes quality control benchmarks with median Q-scores of 14.0 and N50 values of 4,828 bp
    \item Supports diverse applications including plasmid sequencing (48.5\%), research projects (23.6\%), and pharmacogenomics studies (7.9\%)
\end{itemize}

This work is particularly timely given the increasing adoption of nanopore sequencing in clinical and research settings, where metadata standardization is essential for regulatory compliance and scientific reproducibility. The registry framework and associated tools are freely available to the research community.

The manuscript has not been published previously and is not under consideration elsewhere. All authors have approved the manuscript and agree with its submission to [Journal Name].

We believe this work will be of significant interest to your readership in the fields of genomics, bioinformatics, and data management.

\closing{Sincerely,}

\end{letter}
\end{document}
